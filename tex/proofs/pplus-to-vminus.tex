\documentclass[]{article}
\usepackage{vmlmacros}
\usepackage{syntax}
\usepackage{relsize}
% \usepackage{palatino} % I don't love this, to be honest. 
\usepackage{amsmath}
\usepackage{booktabs}
\usepackage{simplebnf}
\setcounter{secnumdepth}{1}

\DeclareMathOperator{\dom}{dom}


% \setlength{\grammarparsep}{20pt plus 1pt minus 1pt} % increase separation between rules
\setlength{\grammarindent}{10em} % increase separation between LHS/RHS
\setlength{\parindent}{0cm}
\title{Translation from {\PPlus} to {\VMinus}}
\author{Roger Burtonpatel}
\begin{document}

\maketitle

\subsection{Domains}
We give the names and domains of our translation functions: 

\begin{align*}
    &\mathcal{P}: Pattern\; ->\; Name\; ->\; Guard \\
    % &\mathcal{P}: Pattern\; ->\; Name\; ->\; Exp\; ->\; Guarded\; Exp \\
    &\mathcal{E}: Exp\; ->\; Exp \\
    &\mathcal{B}: Pattern\; ->\; Name\; Set \\
    \end{align*}

The translation functions $\mathcal{B}$, $\mathcal{E}$, and $\mathcal{P}$
are defined case by case: 

\subsection{Binding names}

\begin{align*}
    &\Bindings{x} = \bracketed{x} \\ 
    &\Bindings{K} = \bracketed{} \\
    &\Bindings{K\; p_{1} \dots p_{n}} = \Bindings{p} \cup \dots \cup \Bindings{p_{n}} \\
    &\Bindings{\porp} = \Bindings{p_{1}} \cap \Bindings{p_{2}} \\
    &\Bindings{\pcommap} = \Bindings{p} \cup \Bindings{p'} \\
    &\Bindings{\parrowe} = \Bindings{p} \\
    &\Bindings{\whenexpr} = \bracketed{}
\end{align*}

\subsection{Translating Expressions}


\begin{align*}
    &\ptov[exp=x, result=x] \\
    &\ptov[exp={K\; \expr[1] \dots \expr[n]}, result={K \etran{\expr[1]} \dots \etran{\expr[n]}}] \\
    &\ptov[exp={\lambda x.\; \expr}, result={\lambda x.\; \etran{\expr}}] \\
    &\ptov[exp={\expr[1]\; \expr[2]}, result={\etran{\expr[1]}\; \etran{\expr[2]}}] \\
    &\ptov[exp={\tt{case}\; \expr\;  \emptyseq}, result={{\iffitt{\vexists{x}\; x = \etran{\expr};\; \iffitt{}}}}]\; \rm{, $x$ fresh }   \\
    &\etran{\tt{case}\; \expr\;  p_{1}\; \expr[1] \vert \dots \vert p_{n}\; \expr[n]} \\
    &\rightsquigarrow {\iffitt{\vexists{x}\; x = \etran{\expr};\; 
            \iffitt{\vexists{\btran{p_{1}}}\; \ptran{p_{1}}x \rightarrow \etran{\expr[1]};\;
            []\; \dots []\; \vexists {\btran{p_{n}}}\; \ptran{p_{n}}x \rightarrow \etran{\expr[n]}}}} \\
    &\rm{, $x$ fresh }
\end{align*}

% \rab{how do you format this so it's not terrible}

\subsection{Translating Patterns}

\begin{align*}
    &\pattov[pat=y, result={(y, [x = y])}] \\
    &\pattov[pat=K, result={([], [x = K])}] \\
    &\ptran{K p_{1} \dots p_{n}}x \rightsquigarrow \\
    & \rm{$\forall i.\; 1 \leq i \leq n:$} \\
    & \rm{ let } y_{i} \rm{ be a fresh name}  \\
    & (ns_{1}\dots ns_{i}, gs_{1}\dots gs_{i}) = \ptran{p_{1}}y_{1}\; \cdot\; \dots \cdot\; \ptran{p_{i}}y_{i} \\
    & \rm{ in } \\
    & ({ns_{1} \cdot \dots \cdot ns_{i}} \cdot {y_{1} \dots y_{i}}, x = K y_{1} \dots y_{i} \cdot \; gs_{1}\dots gs_{i}) \\
    &\pattov[pat=\mathit{when}\; e, result={([], [\etran{e}])}] \\
    &\pattov[pat=\pcommap, 
    result={\rm{let } 
    {(ns_{1}, gs_{1}) = \ptran{p}x}\; ; 
    {(ns_{2}, gs_{2}) = \ptran{p'}x} \rm{ in }
    (ns_{1} \cdot ns_{2}, gs_{1} \cdot gs_{2})}] \\
    &\pattov[pat=\porp, 
    result={\rm{let } (ns_{1}, gs_{1}) = \ptran{p}x\; ;
    (ns_{2}, gs_{2}) = \ptran{p'}x \rm{ in }
    (ns_{1} \cdot ns_{2}, [gs_{1} \choice gs_{2}])}]
\end{align*}


% \rab{The vconapp rule is a formatting disaster. How do I fix it.}

Let's make some claims: 

\begin{itemize}
    \item Equational translation of everything outside of case is purely equivalent 
    \item Inductive hypothesis could be over patterns? 
    \item Semantics: when \prun, \vmrung
\end{itemize}

\newcommand\translatesto\rightsquigarrow

\newcommand\ep{\ensuremath{e_{p}}}
\newcommand\ev{\ensuremath{e_{v}}}
\newcommand\nsgs{\ensuremath{(ns, gs)}}


Let \ep\ be a \PPlus\ expression, \ev\ be a \VMinus\ expression, and \Rho\ be a
specific but arbitrarily chosen environment. If: 
\begin{enumerate}
    \item $\ptov[exp=\ep,result=\ev]$
    \item \prun[exp=\ep, value=v_{1}]
    \item \vmrung[guard=\ev, result=v_{2}]
\end{enumerate}

then $v_{1} = v_{2}$


Let $p$ be a \PPlus\ pattern, \nsgs be a list of names and \VMinus\ guards,
\Rho\ be a specific but arbitrarily chosen environment, \context\ be a specific
but arbitrarily chosen context, and $x$ be a name that stands for the scrutinee
of a case expression $v$ such that $\Rho(x) = v$. If: 
\begin{enumerate}
    \item $\pattov[pat=p, result=\nsgs]$
    \item \pmatch[newenv=r_{1}]
    \item \vmgs[envext={\bracketed{\forall n \in ns, n \mapsto
    \bot}},result=r_{2}] 
\end{enumerate}

then $r_{1} = r_{2}$

NEXT STEPS:


Case: 

Name 

(1) \pattov[pat=y, result={(y, [x = y])}] (2) \pmatch[pat=y, newenv=r_{1}] (3) \vmgs[envext=\bracketed{y \mapsto \rho(x)}, guards={[x = y]}, result=r_{2}]

By rule BLAH, $r_{1}$ = BLAH1. 
By rule BLAH2, $r_{2}$ = BLAH2. 

Therefore, $r_{1} = r_{2}$. 

\vmsemantics


\end{document}

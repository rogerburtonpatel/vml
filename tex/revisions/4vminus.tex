\documentclass[manuscript,screen,review, 12pt, nonacm]{acmart}
\let\Bbbk\relax % Fix for amssymb clash 
\usepackage{vmlmacros}
\AtBeginDocument{%
  \providecommand\BibTeX{{%
    \normalfont B\kern-0.5em{\scshape i\kern-0.25em b}\kern-0.8em\TeX}}}
\usepackage{outlines}
\setlength{\headheight}{14.0pt}
\setlength{\footskip}{13.3pt}
\title{An Alternative to Pattern Matching, Inspired by Verse}

\author{Roger Burtonpatel}
\email{roger.burtonpatel@tufts.edu}
\affiliation{%
\institution{Tufts University}
\streetaddress{419 Boston Ave}
  \city{Medford}
  \state{Massachusetts}
  \country{USA}
  \postcode{02155}
  }
\begin{document}
  
\section{A compromise}
\label{compromise}
    
    To bridge the gap between pattern matching, equations, and decision trees, I
    have created and implemented a semantics for three core languages. To model
    pattern matching with extensions, I~introduce \PPlus. To model programming
    with equations, I~introduce \VMinus. And to provide an
    efficient cost model to which both \PPlus and \VMinus can be compiled, I
    introduce \D.

    Of these three languages, the most unusual is \VMinus. It has equations and
    choice, like \VC, but without multiple results or backtracking. To eliminate
    multiple results, expressions in \VMinus evaluate to at most one result,
    and choice is not a valid form of expression in the language. To eliminate
    backtracking, the compiler rejects a \VMinus program that would need to
    backtrack at runtime. 
    
    % They are \PPlus, which has pattern matching with all the extensions above,
    % \VMinus, which has equations without backtracking or multiple results, and
    % \D, which has decision trees. \PPlus and \VMinus serve to explore the
    % design space of pattern matching and equations, while \D serves as the
    % target of translation to reason about efficiency. 

    % TODO: dsemantics 
    In this section, I~present the three languages. Their semantics appear in
    Sections \ref{pplus}, \ref{vmsemantics}, and~\ref{dsemantics},
    respectively, and in Appendix~\ref{languagedefs}. In my design, I~took
    inspiration from Verse: each of \PPlus, \VMinus, and \D has a conventional
    sub-language that is the lambda calculus extended with named value
    constructors \it{K} applied to zero or more values. I~chose named value
    constructors over \VC's tuples because they look more like patterns. 
    
    Because they share a core, and to facilitate comparisons and proofs, I
    present \VMinus, \PPlus, and \D as three subsets of a single unifying
    language \U, whose abstract syntax is given in Table~\ref{fig:unilang}.
    Column “Unique To” indicates which components of \U\ belong to the
    sub-languages. For all intents and purposes, the three languages are
    distinct; it is because they all share the same core of lambdas, value
    constructors (\it{K}), names, and function application that I~have decided
    to house them in \U\ together. 
    
    Like in \VC, every lambda-calculus term is valid in our languages and has
    the same semantics. Also like the lambda calculus and \VC, all three
    languages are \it{strict}, meaning every expression is evaluated when it is
    bound to a variable, and they are all untyped. Creating a type system for
    \PPlus and \VMinus is a worthy effort but is the subject of another paper.
    Typing for low-level languages like \D is outside the scope of this thesis. 
    
    That the only form of constructed data in these languages is
    value-constructor application is an endeavor for simplicity. In full
    languages, other forms of data like numbers and strings have a similar
    status to value constructors, but their presence would complicate the
    development of semantics and code in these core languages. 
    
    Using just value constructors, though, a programmer can simulate more
    primitive data like strings. For example, \tt{Wow! That's A Tall Square} is
    a valid expression in any of the languages, because it is an application of
    constructor \tt{Wow!} to the arguments \tt{That's}, \tt{A}, \tt{Tall}, and
    \tt{Square}, all of which are value constructors themselves. Each name in
    this “sentence” is considered a value constructor by the programs because
    their names begin with a capital letter. A programmer can use a similar
    trick to simulate integers: \tt{One}, \tt{Two}, etc. Because the languages
    all also have lambda, Church Numerals~\citep{church1985calculi} are another
    viable\footnote{Not to mention meditative.} option. 

    In the subsections below, I~discuss each language in more detail.
    Section~\ref{futurework} goes further in analyzing how \PPlus and \VMinus
    relate to modern languages with pattern matching and to \VC respectively. 


\subsection{Introducing \PPlus}
\label{pplus}

    \PPlus offers a standardized core for pattern matching, enhanced by common
    extensions. In addition to bare pattern matching— names and applications of
    value constructors— the language includes pattern guards, or-patterns, and
    side conditions. Although pattern guards subsume side conditions, I~include
    side conditions in \PPlus and separate them from guards for purpose of
    study. Furthermore, \PPlus introduces a new form of pattern: \pcommap. This
    allows a pattern in \PPlus be a \it{sequence} of sub-patterns, allowing a
    programmer to stuff as many patterns as they want in the space of a single
    one. I~discuss the implications of this in Section~\ref{ppweird}.

\subsection{Formal Semantics of \PPlus}

    In this section, I~present a big-step operational semantics for \VMinus. The
    semantics describes how expressions in \VMinus are evaluated and how
    equations (and more generally, guards) work in the language. Instead of a
    rewrite semantics that makes substitutions within guards, \VMinus has a
    big-step semantics that directly describes how they are handled by the
    runtime core. Figure~\ref{fig:vmsyntax} contains the concrete syntax of
    \VMinus, Figure~\ref{fig:vmmetavars} provides the metavariables used in the
    judgement forms and~rules of the semantics, and Section~\ref{vmsemantics}
    contains the forms and rules. Since solving guards is the heart of \VMinus,
    I~also describe it in plain English.

\begin{figure}
\begin{center}
\ppcsyntax
\end{center}
\Description{The concrete syntax of PPlus.}
\caption{\PPlus: Concrete syntax}
\label{fig:ppsyntax}
\end{figure}

  % \subsection{Forms of Judgement for \PPlus:}

\begin{table}
  \begin{tabular}{ll}
  \toprule
      \multicolumn2{l}{\emph{\PPlus Metavariables}} \\
  \midrule
      $\expr$& expression \\
      $\v,\; \v'$& value \\
      $K$& value constructor \\ 
      $p$& pattern \\ 
      $x, y$& names \\ 
      $\dagger$& pattern match failure \\ 
      $s$& a solution, either $\v$ or $\dagger$ \\ 
      \Rho& environment: $name \rightarrow \mathcal{V}$ \\
      \Rho\:+\:\Rhoprime& extension \\
      $\Rho \uplus \Rho'$& disjoint union \\
      $\{ x \mapsto y \} $& environment with name $x$ mapping to $y$ \\
  \bottomrule
  \end{tabular}    
  \Description{Metavariables used in rules for PPlus.}
  \caption{\PPlus metavariables and their meanings}
  \label{fig:ppmetavars}
\end{table}

\subsubsection{Expressions in \PPlus}
    An expression in \PPlus evaluates to produce a single value, shown by the
    \textsc{Eval} judgement form. 
    \begin{align*}
        &(\textsc{Eval})\; \prun \quad   \rm{ Expression $e$ evaluates in 
                            environment \Rho\ to produce value $\v$.} \\
    \end{align*}

\subsubsection{Pattern matching in \PPlus}

Pattern matching in \PPlus is represented by a single judgement form, with two
possible outcomes: success (a refined environment \Rhoprime) and failure
($\dagger$). The metavariable \solution, a \it{solution} to a pattern match,
combines these outcomes. 
\raggedbottom

\begin{align*}
      &\pmatch \quad   \rm{ Pattern $p$ matches value $\v$ in environment \Rho, 
                              producing bindings \Rhoprime;} \\
      &(\textsc{Match-Success}) \\
      &\pfail  \quad\; \rm{ Pattern $p$ does not match value $\v$ in environment \Rho.} \\
      &(\textsc{Match-Fail})
\end{align*}

Pattern guards introduce a special case: if a pattern is bound to an expression
in the form $\pmatch[pat = {p \leftarrow \expr}, newenv = \solution]$, it will
match if the expression $e$ evaluates to value $v'$ and $p$ matches with $v'$.
When a pattern is standalone, as in all other cases, it will match on $v$, the
\it{original} scrutinee of the case expression. For example, in the \PPlus
expression \tt{case x of Square s, Big b <- mumble s -> b}, the result of
evaluating \tt{x} matches with \tt{Square s}, and the result of evaluating
\tt{mumble s} matches with \tt{Big b}. Generally: when $\pmatch[pat = {p_{1},
p_{2} \leftarrow \expr}, newenv = \solution]$, $\pmatch[pat=p_{1}, value=\v,
newenv=\solution{1}]$, $\ppeval[result=\v']$, and $\pmatch[pat=p_{2},
value=\v', newenv=\solution{2}]$.

Pattern matching is defined inductively: 
\begin{itemize}
    \item A name $x$ matches any value $\v$, and produces the environment 
    $\bracketed{x \mapsto \v}$. 
    \item A value constructor $K$ applied to atoms  matches 
    a value $\v$ if $\v$ is an application of $K$ to the same number of values,
    each of which matches the corresponding atom. Its match produces 
    the disjoint union of matching all internal atoms to all internal values. 
    \item A pattern \whenexpr\ matches when $\expr$ evaluates to a value other than 
    $\mathit{false}$, and produces the empty environment. 
    \item A wildcard pattern \wildcard\ matches any value \v, and produces the
    empty environment. 
    \item A pattern \parrowe\ matches when $\expr$ evaluates to 
          value $\v$, and $p$ matches $\v$. 
    \item A pattern \pcommap\ matches if both $p$ and $p_{2}$ match.
    \item A pattern \porp\ matches if either $p_{1}$ or $p_{2}$
    matches. 
\end{itemize}

When a pattern is of the form $K, p_{1}, \dots, p_{n}$, each sub-pattern $p_{i}$
may introduce new variables during pattern matching. Bindings for all these
variables must be combined in a single environment. \it{Disjoint union} is an
operation on two environments. Its definition borrowed, paraphrased, from
\citet{bpc}: 

\it{Disjoint union} is a way to capture the aggregate environment of matching
constructed data with a constructor-application pattern. The disjoint union of
environments $\rho_{1}$ and $\rho_{2}$ is written $\rho_{1} \uplus \rho_{12}$,
and it is defined if and only if $\dom \rho_{1} \cap \dom \rho_{2} = \emptyset$:

\begin{align*}
  \dom(\rho_{1} \uplus \rho_{2}) &= \dom \rho_{1} \uplus \dom \rho_{2}, \\
    (\rho_{1} \uplus \rho_{2})(x) &= 
  \begin{cases}
    \rho_{1}(x), & \text{if } x \in \dom  \rho_{1} \\
    \rho_{2}(x), & \text{if } x \in \dom \rho_{2} \\
\end{cases}
\end{align*}

For example, in the \PPlus expression \tt{case Pair 1 (Pair 2 3) of Pair x (Pair
y z) -> z}, the right-hand size \tt{z} is evaluated with the environment
$\bracketed{x \mapsto 1} \uplus \bracketed{y \mapsto 2} \uplus \bracketed{z
\mapsto 3}$. 

Disjoint union across multiple results, when any result is failure, still
represents failure: 

\begin{gather*}
  \dagger \uplus \rho = \dagger \quad
  \rho \uplus \dagger = \dagger \quad
  \dagger \uplus\, \dagger = \dagger
\end{gather*}


At runtime, disjoint union also fails if  $\dom \rho_{1} \cap \dom \rho_{2} \neq
\emptyset$, meaning a constructor-application pattern had duplicate names, like
in \tt{Pair x x}. This means \PPlus has only \it{linear} patterns under value
constructors, i.e., the same name cannot bind multiple components of a single
instance of constructed data. 

Finally, when a pattern in a branch in a \it{case} expression fully matches, its
corresponding right-hand side is evaluated with top-level $\rho$ extended with
the $\rho'$ produced by the pattern match. Environment extension is notated
$\rho + \rho'$. 

    \subsection{Rules (Big-step Operational Semantics) for \PPlus:}
    \label{ppsemantics}
    
    Some of these rules are a variation on the rules found in Ramsey~\cite{bpc}.   
    
      \ppsemantics 
      
      \bigskip 
    
    I~show how a programmer might utilize \PPlus to solve the previous problems
    (Section~\ref{extensions}) in Figure~\ref{fig:ppfuncs}. The examples in the
    figure all compile with the \tt{pplus} program. 

    \begin{figure}[ht] 
      \begin{minipage}[h]{0.54\linewidth}
        \pplst 
        \begin{lstlisting}[numbers=none, basicstyle=\tiny, xleftmargin=.2em,
          showstringspaces=false,
          frame=single]
val exclaimTall = \sh.
  case sh of Square s, when (> s) 100 -> 
            Wow! That's A Tall Square!  
    | Triangle _ h, when (> h) 100 ->
            Goodness! Towering Triangle!
    | Trapezoid _ _ h, when (> g) 100 -> 
            Zounds! Tremendous Trapezoid!
    | _ ->  Your Shape Is Small
  \end{lstlisting}
      \Description{exclaimTall in PPlus}
      \subcaption{\tt{exclaimTall} in \PPlus}
          \label{fig:ppexclaimtall} 
      %   \vspace{4ex}
      \end{minipage}%%
      \begin{minipage}[h]{0.5\linewidth}
        \pplst 
        \begin{lstlisting}[numbers=none, basicstyle=\tiny, xleftmargin=2em,
                      frame=single]
  val tripleLookup = \ rho. \x.
    case x of 
      Some w <- (lookup rho) x
    , Some y <- (lookup rho) w
    , Some z <- (lookup rho) y -> z
    | _ -> handleFailure x

   \end{lstlisting}
        \Description{tripleLookup in PPlus}
        \subcaption{\tt{tripleLookup} in \PPlus}
            \label{fig:pptriplelookup} 
        \vspace{4ex}
      \end{minipage} 
      \begin{minipage}[h]{\linewidth}
        \pplst 
        \begin{lstlisting}[numbers=none, basicstyle=\tiny, xleftmargin=9em,
          showstringspaces=false,
          frame=single]
val game_of_token = \t. 
  case t of  
    BattlePass f  | (ChugJug f | TomatoTown f) -> P (Fortnite f)
  | HuntersMark f | SawCleaver f               -> P (Bloodborne ((* 2) f))
  | MoghLordOfBlood f | PreatorRykard f        -> P (EldenRing ((* 3) f))
  | _                                          -> P (Irrelevant 0)
\end{lstlisting}
      \Description{game_of_token in PPlus}
      \subcaption{\tt{game\_of\_token} in \PPlus}
          \label{fig:ppgot}
      \vspace{4ex}
      \end{minipage}%% 
      \caption{The functions in \PPlus have the same desirable implementation as
      before. All the example compile with the \tt{pplus} program.}
  \label{fig:ppfuncs}
    \end{figure}        
    
    As mentioned in Section~\ref{compromise}, masquerading value constructors
    stand in for strings. \PPlus has no infix operators, so some expressions are
    parenthesized. 


\subsection{Introducing \VMinus }
\label{vminus}

        To fuel the pursuit of smarter decision-making, I~now draw inspiration
        from \VC. Equations in \VC look attractive, but the cost model of \VC is
        a challenge. 
        
        The elements of \VC that lead to unpredictable or costly run times are
        backtracking and multiple results. So, I~begin with a subset of \VC,
        which I~call \VMinus ("V minus"), with these elements removed. Removing
        them strips much of the identity of \VC, but it leaves its
        \it{equations} to build on top of in an otherwise-typical programming
        context of single results and no backtracking at runtime. 

        Having stripped out the functional logic programming elements of \VC
        (backtracking and multiple results), only the decision-making bits are
        left over. To wrap these, I~add a classic decision-making construct:
        guarded commands~\citep{dijkstra} The result is \VMinus. 

        \VMinus's concrete syntax is defined in Figure~\ref{fig:vmsyntax}.
        \VMinus takes several key concepts from \VC, with several key
        differences, illustrated in Table~\ref{tab:vmvsvc}. 


        \begin{table}[htbp]
          \centering
          \begin{tabular}{|p{0.5\linewidth}|p{0.48\linewidth}|}
              \hline
              \bfseries Like \VC & \bfseries Unlike \VC \\
              \hline
              \VMinus uses equations to guard computation.  & \VMinus solves an equation at most once at runtime and never backtracks. \\
              \hline
              \VMinus uses choice. & \VMinus's choice can only guard computation and its result is never returned. \\
              \hline
              \VMinus uses \textit{success} and \textit{failure} to make decisions. & An expression in \VMinus returns at most one value. \\
              \hline
          \end{tabular}
          \caption{Key similarities and differences between \VMinus and \VC}
          \label{tab:vmvsvc}
      \end{table}

      \begin{lemma}
        An expression in \VMinus can return at most one result. 
      \end{lemma}

      \newcommand\ev{\ensuremath{e_{v}}\xspace}
      \begin{proof}
        The syntax of \VMinus (Figure~\ref{fig:vmsyntax}) forbids any expression
        to return the result of evaluating choice. Let \ev be a \VMinus
        expression. If choice appears in \ev, it may only appear in a guard.
        Guards may only precede expressions and are never returned. Therefore,
        \ev\ may never return the result of evaluating choice, and there is no
        other syntactic form in \VMinus that allows for multiple results. 

        % The only values in \VMinus are value constructor application and lambda,
        % each of which is itself a single result. 
      \end{proof}

      \begin{lemma}
        An expression in \VMinus can never backtrack at runtime. 
      \end{lemma}
      % \newcommand\gs{\ensuremath{\mathit{gs}}\xspace}
      \begin{proof}
        In progress. 
      \end{proof}

        % Both backtracking and multiple results often manifest within \VC's
        % choice operator, which tempt its removal altogether. 
        
        % However, I~am drawn
        % to harnessing the expressive potential of this operator, particularly
        % when paired with \VC's 'one' keyword. When employed with choice as a
        % condition, 'one' elegantly signifies 'proceed if any branch of the
        % choice succeeds. 
        
        % To this end, I~propose a new language, \VMinus, in which choice is
        % permitted with several modifications:

        % \begin{enumerate}
        % \item Choice may only appear as a condition or 'guard', not as a result
        % or the right-hand side of a binding.
        % \item If any branch of the choice succeeds, the choice succeeds,
        % producing any bindings found in that branch. The program examines the
        % branches in a left-to-right order.
        % \item The existential $\exists$ may not appear under choice.
        % \end{enumerate}

        % I~introduce one more crucial modification to the \VC runtime: a name in
        % \VMinus is an \it{expression} rather than a \it{value}. This alteration,
        % coupled with my adjustments to choice, eradicates backtracking. Our
        % rationale behind this is straightforward: if an expression returns a
        % name, and subsequently, the program imposes a new constraint on that
        % name, it may necessitate the reevaluation of the earlier expression—- a
        % scenario I~strive to avoid. 
        
        % An example in \VC illustrates this precise
        % scenario: 

        % FIRST PAPER EXAMPLE 

        % IMPORTANTLY, \VMinus REALLY ISN'T A FUNCTIONAL LOGIC PROGRAMMING LANGUAGE ANYMORE. 
        
        %  Doesn't backtrack
        
        %  No multiple results 
        
        %  Doesn't evaluate functions backwards, have top-level patterns like verse, the list goes on 

        % If you're well-versed in functional logic programming, may perceive that
        % imposing such restrictions on choice and names effectively strips away
        % much of Verse's essence as a functional logic programming language. With
        % these constraints enforced, there can be no backtracking, multiple
        % results, backward function evaluation, or top-level patterns, among
        % other classic functional logic programming features. But do not fear:
        % our intent is not to recklessly strip Verse of its meticulously crafted
        % core tenets. Instead, our aim is to extract a select few—namely, its
        % equations, existentials, and nondeterministic evaluation order—and
        % juxtapose them with pattern matching.    

        % \bf{Choice forces "name knowing." Names must be known in all branches 
        % if bound to unknown variable or bound to known variable if any name 
        % is unknown}.
        
        % With these modifications in place, alongside a few additional
        % adjustments to the placement of existentials and the timing of choice
        % evaluation (footnote: we discuss these later), I~unveil a redefined
        % decision-making core. To complete this transformation, we incorporate a
        % venerable decision-making construct: Guarded commands [cite: Dijkstra].
        % The result is \VMinus. 
        \begin{figure}[H]
          \begin{center}
              \begin{bnf}
              $P$ : \textsf{Programs} ::=
              $\bracketed{d}$ : definition
              ;;
              $d$ : \textsf{Definitions} ::=
              | $\tt{val}\; x\; \tt{=}\; \expr$ : bind name to expression
              ;;
              $\expr$ : \textsf{Expressions} ::=
              | $v$ : literal values 
              | $x, y, z$ : names
              | $\tt{if}\; [\; B\; \bracketed{\tt{[]}\; B}\;]\; \tt{fi}$ : if-fi 
              | $K \bracketed{\expr}$ : value constructor application 
              | $\expr[1]\; \expr[2]$ : function application 
              | $\ttbackslash x\tt{.}\; \expr$ : lambda declaration 
              ;;
              $B$ : \textsf{Branches} ::=  
              $[\tt{E }{\bracketed{x}}{\tt{.}}] \bracketed{g} \;\ttrightarrow\; \expr$ : names, guards, and body
              ;;
              $g$ : \textsf{Guards} ::=  
              | $\expr$ : intermediate expression 
              | $x \;\tt{=}\; \expr$ : equation 
              | $ g \bracketed{\tt{;} g} \; \pbar \; g \bracketed{\tt{;} g}$ : choice 
              ;;
              $\v$ : Values ::= $K\bracketed{\v}$ : value constructor application 
              | $\ttbackslash x\tt{.}\; \expr$ : lambda value
              \end{bnf}
          \end{center}
          \Description{The concrete syntax of VMinus.}
              \caption{\VMinus: Concrete syntax}
              \label{fig:vmsyntax}
              \end{figure}      

    \subsection{Programming in \VMinus}

    Even with multiple modifications, \VMinus still allows for many of the same
    pleasing computations as full Verse. A programmer can\dots
        \begin{enumerate}
            \item Introduce a set of equations, to be solved in a
            nondeterministic order 
            \item Guard expressions with those equations 
            \item Flexibly express “proceed when any of these operations
            succeeds” with the new semantics of choice
            (Section~\ref{vmsemantics}). 
        \end{enumerate}

    Figure~\ref{fig:vminusfuncs} provides an example of how a programmer can
    utilize \VMinus to solve the previous problems (Section~\ref{extensions}): 
    
    
    \begin{figure}[ht] 
      \begin{minipage}[h]{0.54\linewidth}
        \vmlst 
        \begin{lstlisting}[numbers=none, basicstyle=\tiny, xleftmargin=.2em,
          showstringspaces=false,
          frame=single]
val exclaimTall = \sh.
  if E s. sh = Square s; (> s) 100 -> 
            Wow! That's A Tall Square!  
    [] E w h. sh = Triangle w h; (> h) 100 ->
            Goodness! Towering Triangle!
    [] E b1 b2 h. sh = Trapezoid b1 b2 h; 
     (> h) 100 -> Zounds! Tremendous Trapezoid!
    [] ->  Your Shape Is Small
  fi 
  \end{lstlisting}
      \Description{exclaimTall in VMinus}
      \subcaption{\tt{exclaimTall} in \VMinus}
          \label{fig:vmexclaimtall} 
      %   \vspace{4ex}
      \end{minipage}%%
      \begin{minipage}[h]{0.5\linewidth}
        \vmlst 
        \begin{lstlisting}[numbers=none, basicstyle=\tiny, xleftmargin=2em,
                      frame=single]
val tripleLookup = \rho. \x.
  if E w y z v1 v2 v3. 
    v1 = (lookup rho) x; v1 = Some w; 
    v2 = (lookup rho) w; v2 = Some y; 
    v3 = (lookup rho) y; v3 = Some z 
       -> z 
    [] -> handleFailure x
  fi 
   \end{lstlisting}
        \Description{tripleLookup in VMinus}
        \subcaption{\tt{tripleLookup} in \VMinus}
            \label{fig:vmtriplelookup} 
        \vspace{4ex}
      \end{minipage} 
      \begin{minipage}[h]{\linewidth}
        \vmlst 
        \begin{lstlisting}[numbers=none, basicstyle=\tiny, xleftmargin=9em,
          showstringspaces=false,
          frame=single]
val game_of_token = \t. 
  if E f. (t = BattlePass f  | (t = ChugJug f | t = TomatoTown f)) -> 
        P (Fortnite f)
  [] E f. (t = HuntersMark f | t = SawCleaver f)                   -> 
        P (Bloodborne ((* 2) f))
  [] E f. (t = MoghLordOfBlood f | t = PreatorRykard f)            -> 
        P (EldenRing ((* 3) f))
  [] -> P (Irrelevant 0)
  fi 
\end{lstlisting}
      \Description{game_of_token in VMinus}
      \subcaption{\tt{game\_of\_token} in \VMinus}
          \label{fig:vmgot}
      \vspace{4ex}
      \end{minipage}%% 
      \caption{The functions in \VMinus have a desirably concise
      implementation, as before.}
  \label{fig:vminusfuncs}
    \end{figure}   

    \VMinus looks satisfyingly similar to both \PPlus and \VC. The \VMinus
    examples in Figure~\ref{fig:vminusfuncs} have the same number of cases as
    the \PPlus examples in Figure~\ref{fig:ppfuncs}, and share the existential
    and equations with the \VC examples in Figure~\ref{fig:verseextfuncs}. 
   
% \subsection{Concrete Syntax of \VMinus}

% \bigskip

% \begin{center}
%     \begin{bnf}
%     $P$ : \textsf{Programs} ::=
%     $\bracketed{d}$ : definition
%     ;;
%     $d$ : \textsf{Definitions} ::=
%     | $\tt{val}\; x\; \tt{=}\; \expr$ : bind name to expression
%     ;;
%     $\expr$ : \textsf{Expressions} ::=
%     | $v$ : literal values 
%     | $x, y, z$ : names
%     | $\tt{if}\; \tt{[}\; G\; \bracketed{\square\; G}\; \tt{]}\; \tt{fi}$ : if-fi 
%     | $K \bracketed{\expr}$ : value constructor application 
%     | $\expr[1]\; \expr[2]$ : function application 
%     | $\lambda x.\; \expr$ : lambda declaration 
%     ;;
%     $G$ : \textsf{Guarded Expressions} ::=  
%     $[\vexists{\bracketed{x}}] \bracketed{g} \boldsymbol{\rightarrow}\expr$ : names, guards, and body
%     ;;
%     $g$ : \textsf{Guards} ::=  
%     | $\expr$ : intermediate expression 
%     | $x = \expr$ : equation 
%     | $ g \bracketed{; g} \; \choice \; g \bracketed{; g}$ : choice 
%     ;;
%     $\v$ : Values ::= $K\bracketed{\v}$ : value constructor application 
%     | $\lambda x.\; \expr$ : lambda value
%     ;;
%     \end{bnf}
% \end{center}

% A \it{name} is any token that is not an integer literal, 
% does not contain whitespace, a bracket, or parenthesis, 
% and is not a value constructor name or a reserved word.

\subsection{Formal Semantics of $\VMinus$:}

In this section, I~present a big-step operational semantics for \PPlus. The
semantics describes how expressions in \PPlus are evaluated and how pattern
matching works in the language. Instead of a rewrite semantics that desugars
extensions into \it{case} expressions, \PPlus has a big-step semantics that
directly describes how they are handled by the runtime core.
Figure~\ref{fig:ppsyntax} contains the concrete syntax of \PPlus,
Figure~\ref{fig:ppmetavars} provides the metavariables used in the judgement
forms and~rules of the semantics, and Section~\ref{ppsemantics} contains the
forms and rules. Since pattern matching is the heart of \PPlus, I~also
describe it in plain English.

\begin{table}
\begin{tabular}{ll}
\toprule
    \multicolumn2{l}{\emph{\VMinus Metavariables}} \\
\midrule
    $\expr$& An expression \\ 
    $\v,\; \v'$& value \\
    $\fail$& expression failure \\
    $\result$& $\v \vert \fail$ : expressions produce \it{results}: values or
    failure. \\
    \Rho& environment: $name \rightarrow {\mathcal{V}}_{\bot}$ \\
    $\rho\{ x \mapsto y \} $& environment extended with name $x$ mapping to $y$ \\
    $g$& A guard \\
    $eq$& equation \\ 
    % $\tempstuck$& a temporarily-stuck equation \\
    $\dagger$& when solving guards is rejected \\
    $\result$& $\rhohat \mid \dagger$ : guards produce \it{solutions}: a
    refined environment \rhohat\; or rejection\\
    $\mathcal{T}$& Context of all temporarily stuck guards (a sequence) \\ 
    % $G$& A guarded expression \\
    % \uppsidown& Inability to compile to a decision tree; a compile time error \\
\bottomrule
\end{tabular}    
\Description{Metavariables used in rules for VMinus.}
\caption{\VMinus metavariables and their meanings}
\label{fig:vmmetavars}
\end{table}

\bigskip
    
    \subsubsection{Expressions}
    
    \newcommand\GNoTree{\vmrung \rightsquigarrow \uppsidown} 
    
    An expression in \VC evaluates to produce possibly-empty sequence of values,
    where an empty sequence of values is treated as a special failure 'value'
    \fail. In \VMinus, an expression never returns multiple values, but it can
    produce \fail. Specifically, in \VMinus, an expression evaluates to produce
    a single \bf{result}. A result is either a single value $\v$ or \fail. 
    
    \[\result\; \rm{::=}\; \v\; \vert \; \fail \]
    
    \showvjudgement{Eval}{\vmeval}
    \subsubsection{Guards}

    For example, the \VMinus expression {\tt{((}\ttbackslash\tt{x. x) K)}}
    succeeds and returns the value \tt{K}. The \VMinus expression \tt{if  fi},
    the empty \tt{if-fi}, always produces \fail. 

    Like \VC, \VMinus has a nondeterministic semantics. Guards are solved in \VMinus
    similarly to how equations are solved in Verse: the program nondeterministically
    picks one out of a context (\context), attempts to solve it, and moves on. 

In our semantics, this process occurs over a \it{list} of guards \gs in a
context \context: the program picks a guard from \context, attempts to solve it
to refine the environment or fail, and repeats. \VMinus can only pick a guard
out of the context \context\ that it "knows" it can solve. "Knowing" a guard can
be solved can be determined in \VMinus at compile time. If \VMinus can't pick a
guard and there are guards left over, the semantics gets stuck at compile time. 

\showvjudgement{Solve-Guards}{\vmgs}


The environment $\rho$ maps from a name to a value {\v}s or $\bot$. $\bot$ means
a name has been introduced with the existential, $\exists$, but is not yet bound
to a value. Given any such $\rho$, a guard \g\ will either refine $\rho$
($\Rhoprime$) or be \bf{rejected}. We use the metavariable $\dagger$ to
represent rejection, and a guard producing $\dagger$ will cause the top-level
list of guards to also produce $\dagger$. 

    \showvjudgement{Guard-Refine}{\grefine[solution=\rho']}
    \showvjudgement{Guard-reject}\gfail

  For example, in the \VMinus expression \tt{if E x. x = K; x = K2 -> x fi}, the
  existential (in concrete syntax, \tt{E}) introduces \tt{x} to \Rho bound to
  $\bot$, producing the environment $\bracketed{x \mapsto \bot}$. The guard
  \tt{x = K} successfully unifies \tt{x} with \tt{K}, producing the environment
  $\bracketed{x \mapsto \tt{K}}$. The guard \tt{x = K2} attempts to unify \tt{K}
  with \tt{K2} and fails with $\dagger$. 

    \subsubsection{Refinement ordering on environments}

    \begin{align*}
    \rho \subseteq \Rhoprime \text{ when }&\dom\rho  \subseteq \dom \Rhoprime\\
    \text{ and } &\forall x \in \dom \rho: \rho(x) \subseteq \Rhoprime(x)
    \end{align*}
    
    \medskip
        
    Success only refines the environment; that~is, when ${\vtuple{\rho, \expr}}
    \rightarrowtail{} {\Rhoprime}$, we~expect $\rho \subseteq \Rhoprime$.
    
    \vfilbreak
    
    \subsection{Rules (Big-step Operational Semantics) for $\VMinus$:}
    \label{vmsemantics}
    \vmsemantics

\subsection{Introducing \D}
\label{d}

While \PPlus and \VMinus exist for writing programs, \D exists as the target
of translation. 

This is because the decision-making construct in \D, the \it{decision tree}, has
a deterministic and efficient cost model. Specifically, a decision tree is an
automaton that implements pattern matching with the key property that no part of
the scrutinee of a \it{case} expression is examined more than once at runtime.
This means that the worst-case cost of evaluating a decision tree is linear in
its depth. This desirable property of decision trees is half of a space-time
tradeoff. When a decision tree is produced by compiling a \it{case} expression,
there are pathological cases in which the total size of the tree is exponential
in the size of the source code (from \it{case}). Run time remains linear, but
code size may not be. 

Decision trees are classically used as an intermediate representation for
compiling \it{case} expressions. In this work, I~will use them as a target for
compiling \it{if-fi} in \VMinus. I~do so to prove that equations, at least given
their restrictions in \VMinus, can be compiled to a decision tree. 

\D is a generalization of the trees found in~\citet{maranget}. I~discuss the
specifics of \D's decision trees in the following section. 

\end{document}
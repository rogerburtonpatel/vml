\documentclass[manuscript,screen 12pt, nonacm]{acmart}
\let\Bbbk\relax % Fix for amssymb clash 
\usepackage{vmlmacros}
\AtBeginDocument{%
  \providecommand\BibTeX{{%
    \normalfont B\kern-0.5em{\scshape i\kern-0.25em b}\kern-0.8em\TeX}}}
\usepackage{outlines}
\setlength{\headheight}{14.0pt}
\setlength{\footskip}{13.3pt}
\title{An Alternative to Pattern Matching, Inspired by Verse}
\author{Roger Burtonpatel}
\email{roger.burtonpatel@tufts.edu}
\affiliation{%
  \institution{Tufts University}
  \streetaddress{419 Boston Ave}
  \city{Medford}
  \state{Massachusetts}
  \country{USA}
  \postcode{02155}
}
\begin{document}
    
    \section{\VC's equations}
    \label{verseoverobservers}

    An intriguing alternative to pattern matching exists in~\it{equations}, from
    the Verse Calculus (\VC), a core calculus for the functional logic
    programming language~\it{Verse}~\citep{antoy2010functional,
    hanus2013functional, verse}. (For the remainder of this thesis, I~use “\VC”
    and “Verse” interchangeably.)

    % Equations present a different, yet powerful, way to write code that makes
    % decisions and deconstructs data. In this section, I~will introduce you to
    % equations similarly to I~how I~introduced to you pattern matching. Once
    % you're familiar with equations, you'll be ready to compare their strengths
    % and weaknesses with those of pattern matching, and judge the compromise I
    % propose. 

    % Even if you read~\citet{verse},~\VC's equations and the paradigms of
    % functional logic programming might look unfamiliar. To help ease you into
    % familiarity with equations, I~ground explanations and examples in how
    % they relate to pattern matching. 

    \VC's~\it{equations} test for structural equality and create bindings. Like
    pattern matching, equations scrutinize and deconstruct data at runtime by
    testing for structural equality and unifying names with values. Unlike
    pattern matching,~\VC's equations can unify names on both left-~\it{and}
    right-hand sides. 

    Every equation in~\VC takes the form~\it{x = e}, where~\it{x} is a name and
    \it{e} is an expression\footnote{To make programmers happy, full Verse
    allows an equation to take the form $e_{1} = e_{2}$, which desugars to
    $\vexists{x} x = e_{1};\; x = e_{2}$, with $x$ fresh.}. During runtime,~\VC
    relies on a process called~\it{unification}~\citep{robinson} to attempt to
    bind~\it{x} and any unbound names in~\it{e} to values. Unification is the
    process of finding a substitution that makes two different logical atomic
    expressions identical. Much like pattern matching, unification can fail if
    the runtime attempts to bind incompatible values or structures (i.e., finds
    no substitution). 

    An equation offers a form of binding that looks like a single pattern match.
    What about a list of many patterns and right-hand sides, as in a case
    expression? For this,~\VC has~\it{choice} ($\choice$). The full semantics of
    choice are too complex to cover here, but choice, when combined with the
    \one operator, has a very similar semantics to case; that is, “proceed
    and create bindings if any one of these computations succeed.” 

    Let's look at what equations,~\one, and choice look like in~\VC
    (Figure~\ref{fig:versearea}). I've written the~\tt{area} function in~\VC
    extended with a~\tt{float} type and a multiplication operator~\tt{*}.

    \begin{figure}[]
        \verselst
        \begin{lstlisting}[numbers=none]
$\exists$ vc_area. vc_area = $\lambda$ sh. 
  one {  $\exists$ s. sh   = $\langle$SQUARE, s$\rangle$; s * s
       $\choice$ $\exists$ w h. sh = $\langle$TRIANGLE, w, h$\rangle$; 
              0.5 * w * h
       $\choice$ $\exists$ b1 b2 h. sh = $\langle$TRAPEZOID, b1, b2, h$\rangle$; 
              0.5 * (b1 + b2) * h}
        \end{lstlisting}

        \small
\begin{verbatim}


                        Laws for area: 
                        
                        area (SQUARE s)              == s * s 
                        area (TRIANGLE (w, h))       == 0.5 * w * h
                        area (TRAPEZOID (b1, b2, h)) == 0.5 * (b1 + b2) * h
        \end{verbatim}
    \caption{\tt{vc\_area} in~\VC uses existentials and equations rather than
    pattern matching. Below are the algebraic laws for the original~\tt{area}
    function.}~\Description{An implementation of area in~\VC.}
    \label{fig:versearea}
    \end{figure}

    In the figure, the name~\tt{vc\_area} is bound to a lambda ($\lambda$) that
    takes a single argument,~\tt{sh}. The body of the function is a~\one
    expression over three choices (separated by~$\choice$). If any of the
    choices succeeds,~\one~ensures evaluation of the other choices halts and the
    succeeding expression's result is returned. In each branch, the existential
    $\exists$ introduces names~\tt{s, w, h, b1, b2}, and is followed by an
    \it{equation} that unifies them with~\tt{sh}, along with the familiar value
    constructors~\tt{SQUARE},~\tt{TRIANGLE}, and~\tt{TRAPEZOID}. After the
    equation is a semicolon followed by an expression, which is evaluated if the
    equation succeeds. As in~\hyperref[fig:pmarea]{\tt{pm\_area}}, the
    right-hand sides of~\tt{vc\_area} are~\it{guarded} by a “check;” now, the
    check is successful unification in an equation rather than a successful
    match on a pattern. Similarly,~\one with a list of choices represents
    matching on any~\it{one} pattern to evaluate a single right-hand side. 

    Why use equations? I~begin with a digestible claim:~\VC's equations are
    preferable to observer functions. This claim mirrors my argument for pattern
    matching, and to support it I~appeal to the Nice~Properties: 

    \begin{enumerate}
      \item~\hyperref[p1]{\it{Lawlike.}} 
      \tt{vc\_area} makes only one addition to the algebraic laws: the explicit
      $\exists$. This makes~\tt{vc\_area} look more like mathematical notation
      than pure algebraic laws, but that might not be a bad thing: while it less
      resembles the algebraic laws a programmer would write, it more resembles
      the equations that a mathematician would. 
      \item~\hyperref[p2]{\it{Single copy.}}~\tt{vc\_area} does not duplicate any code. 
      \item~\hyperref[p5]{\it{Exhaustiveness.}}  By writing the equations that
      unify~\tt{sh} with the value-constructor forms first, it is easy in this
      example to see that the code is exhaustive. Creating a static analysis
      tool for~\VC that ensures exhaustiveness on all expressions may or may not
      be a significant challenge; full Verse has a tool that can verify if a
      terminating expression on the right-hand side of a function will
      \it{always succeed} or not~\citep{peyton-jones2024verification}.
      \item~\hyperref[p3]{\it{Call-free.}}~\tt{vc\_area} deconstructs
      user-defined types as easily as~\hyperref[fig:pmarea]{\tt{pm\_area}} does
      with pattern matching. 
      \item~\hyperref[p4]{\it{Signposting.}}~\tt{vc\_area} has all important
      internal values named explicitly.
      
      % This Property may not hold, because~\VC on its own is untyped.
      % Without a type system, a compiler cannot help me with non-exhaustive
      % cases. However, there is no published compiler, type system or no, for
      % Verse, and only when one is made available can I~make this assertion. For
      % this reason, and for the sake of focusing on more important details, I
      % choose to proceed in my analysis of equations in~\VC without considering
      % this Property. 
    \end{enumerate}

    You understand why I claim programmers prefer equations to observer
    functions. Now I~make a stronger claim: equations are~\it{at least as good
    as} pattern matching with popular extensions. How can I~claim this? By
    appealing again to the Nice~Properties! In~Section~\ref{pmandextensions},
    I~demonstrated how pattern matching had to resort to extensions to regain
    the Properties when challenging examples stole them away. In
    Figure~\ref{fig:verseextfuncs}, I've implemented those examples in~\VC (this
    time extended with strings, floats, and~\tt{*}) using choice and equations.
    Take a look for yourself! 

    \begin{figure}[ht] 
        \begin{minipage}[h]{0.54\linewidth}
          \verselst
          \begin{lstlisting}[numbers=none, basicstyle=\tiny, xleftmargin=.2em,
                            showstringspaces=false,
                            frame=single]
$\exists$ exclaimTall. exclaimTall = $\lambda$ sh. 
  one { 
      $\exists$ s. sh = $\langle$Square s$\rangle$; 
      s > 100.0; "Wow! That's a sizeable square!"
    | $\exists$ w h. sh = $\langle$Triangle, w, h$\rangle$; 
      h > 100.0; "Goodness! Towering triangle!"
    | $\exists$ b1 b2 h. sh = $\langle$Trapezoid, b1, b2, h$\rangle$;
      h > 100.0; "Zounds! Tremendous trapezoid!"
    | "Your shape is small." }
            \end{lstlisting}
                \Description{exclaimTall}
        \subcaption{\tt{exclaimTall} in~\VC }
            \label{fig:verseexclaimtall} 
        %   \vspace{4ex}
        \end{minipage}%%
        \begin{minipage}[h]{0.5\linewidth}
          \verselst
          \begin{lstlisting}[numbers=none, basicstyle=\tiny, xleftmargin=2em,
                        frame=single]
$\exists$ tripleLookup. tripleLookup = $\lambda$ rho x. 
    one { $\exists$ w. lookup rho x = $\langle$Just w$\rangle$; 
          $\exists$ y. lookup rho w = $\langle$Just y$\rangle$; 
          $\exists$ z. lookup rho y = $\langle$Just z$\rangle$;
          z
        | handleFailure x }
          \end{lstlisting}
                \Description{exclaimTall}
            \subcaption{\tt{tripleLookup} in~\VC }
            \label{fig:versetriplelookup} 
        \vspace{4ex}
        \end{minipage} 
        \begin{minipage}[h]{\linewidth}
          \verselst
          \begin{lstlisting}[numbers=none, basicstyle=\tiny, xleftmargin=9em, 
                            frame=single, showstringspaces=false]
$\exists$ game_of_token. game_of_token = $\lambda$ token. 
  $\exists$ f. one {
         token = one { $\langle$BattlePass, f$\rangle$ | $\langle$ChugJug, f$\rangle$ | $\langle$TomatoTown, f$\rangle$}; 
           $\langle$"Fortnite", f$\rangle$
       | token = one { $\langle$HuntersMark, f$\rangle$ | $\langle$SawCleaver, f$\rangle$}; 
           $\langle$"Bloodborne", 2 * f$\rangle$
       | token = one { $\langle$MoghLordOfBlood, f$\rangle$ | $\langle$PreatorRykard, f$\rangle$}; 
           $\langle$"Elden Ring", 3 * f$\rangle$
       |   $\langle$"Irrelevant", 0$\rangle$ }
          \end{lstlisting}
                \Description{game_of_token in~\VC}
        \subcaption{\tt{game\_of\_token} in~\VC }
            \label{fig:versegot} 
        \vspace{4ex}
        \end{minipage}%% 
    \caption{Code for the~\ref{pmandextensions} functions with equations looks
    similar, and it doesn't need extensions.}
    \label{fig:verseextfuncs}
      \end{figure}
        
    The code in Figure~\ref{fig:verseextfuncs} has all the Nice~Properties. This
    is promising for~\VC. If it rivals pattern matching with popular extensions
    in desirable properties, and~\VC does everything using only equations and
    choice, it seems like the language is an intriguing option for writing code! 

    \subsection{\VC has a challenging cost model}
    \label{vcbadcost}

    So what's the catch? In~\VC, names (logical variables) are~\it{values}, and
    they can just as easily be the result of evaluating an expression as an
    integer or tuple. To bind these names,~\VC, like other functional logic
    languages, relies on~\it{unification} of its logical variables and
    \it{search} at runtime to meet a set of program
    constraints~\citep{antoy2010functional, hanus2013functional}. Combining
    unifying logical variables with search requires backtracking, which can lead
    to exponential runtime cost~\citep{hanus2013functional, wadler1985replace,
    clark1982introduction}. 

    % Todo: An example would be nice. Backtracking/evaluating backwards, then
    % done. 

    Pattern matching, by contrast, can be compiled to a~\it{decision tree}, a
    data structure that enforces linear runtime performance by guaranteeing no
    part of the scrutinee will be examined more than once~\citep{maranget}. A
    decision tree does not backtrack: once a program makes a decision based on
    the form of a value, it does not re-test it later with new information. 
    

\end{document}
\documentclass[manuscript,screen,review, 12pt, nonacm]{acmart}
\let\Bbbk\relax % Fix for amssymb clash 
\usepackage{vmlmacros}
\AtBeginDocument{%
  \providecommand\BibTeX{{%
    \normalfont B\kern-0.5em{\scshape i\kern-0.25em b}\kern-0.8em\TeX}}}
\usepackage{outlines}
\setlength{\headheight}{14.0pt}
\setlength{\footskip}{13.3pt}
\title{An Alternative to Pattern Matching, Inspired by Verse}
\author{Roger Burtonpatel}
\email{roger.burtonpatel@tufts.edu}
\affiliation{%
  \institution{Tufts University}
  \streetaddress{419 Boston Ave}
  \city{Medford}
  \state{Massachusetts}
  \country{USA}
  \postcode{02155}
}
\begin{document}



\section{Pattern Matching and its Extensions}
\label{pmandextensions}

In this section, I~expand on the definitions, forms, and tradeoffs of pattern
matching and equations. These tradeoffs inform the compromises I~make in
\VMinus (Section~\ref{vminus}).


% \1 What is pattern matching 

Pattern matching lets programmers examine and deconstruct data by matching them
against patterns. When a pattern $p$ matches with a value $v$, it can produce
bindings for sub-values of $v$. For example, pattern $x::xs$ matches any 
application of the value constructor \it{cons} (\it{::}), and it binds the first 
element of the cons cell to $x$ and the second to $xs$. 

Why use pattern matching? What could programmers use instead? Before pattern
matching was invented, a programmer had to deconstruct data using \it{observers}
\citep{liskov:abstraction}: functions that explicitly examine a piece of data
and extract its components. Examples of observers in functional programming
languages include Scheme's \tt{null?}, \tt{car}, and \tt{cdr}, and ML's
\tt{null}, \tt{hd}, and \tt{tl}. Given the option of pattern matching, however,
many functional programmers favor it over observers. I~demonstrate with an
example and a claim. 

Consider a \tt{standard\_shape} datatype in Standard ML, which represents shapes
by their dimensions\footnote{For sake of simplicity, \tt{standard\_shape}s
always have an area that can be obtained by standard area formulas: $\frac{1}{2}
* base * height$ for triangles; $\frac{1}{2} * (base_{1} + base_{2}) * height$
for trapezoids.}: 

\medskip 
\begin{minipage}[t]{\textwidth}
    \begin{verbatim}
        datatype standard_shape = SQUARE    of real 
                                | TRIANGLE  of real * real 
                                | TRAPEZOID of real * real * real
\end{verbatim}
\end{minipage}
\medskip 

I~define an \tt{area} function on \it{standard\_shape}s, with this type and these
algebraic laws: 

\medskip 
\begin{minipage}[t]{\textwidth}
    \begin{verbatim}
        area : standard_shape -> real 
           area (SQUARE s)              == s * s 
           area (TRIANGLE (w, h))       == 0.5 * w * h
           area (TRAPEZOID (b1, b2, h)) == 0.5 * (b1 + b2) * h
        
\end{verbatim}
\end{minipage}
\medskip 

Now compare two implementations of \tt{area}, one with observers and one with
pattern matching (Figure~\ref{fig:area}).

    \begin{figure}[H]
      \begin{minipage}[t]{0.7\textwidth}
        \begin{verbatim}
fun observers_area sh =
  if isSquare sh
  then sqSide sh * sqSide sh
  else if isTriangle sh 
  then 0.5 * triW sh * triH sh
  else 0.5 * traB1 sh * traB2 sh * traH sh
            \end{verbatim}
            \Description{An implementation of area observers}
        \subcaption{\tt{area} with observers}
            \label{fig:observersarea} 
      \end{minipage}
      \vfill
      \begin{minipage}[t]{0.7\textwidth}
        % \centering 
        \begin{verbatim}
fun pm_area sh =
  case sh 
    of SQUARE s              => s * s
     | TRIANGLE (w, h)       => 0.5 * w * h
     | TRAPEZOID (b1, b2, h) => 0.5 * (b1 + b2) * h
                \end{verbatim}
       \Description{An implementation of area using implicit deconstruction via
       patterns.}
       \vspace{2.2em}
       \subcaption{\tt{area} with pattern matching}
       \label{fig:pmarea}
      \end{minipage}
      \caption{Implementing \tt{area} using observers is tedious, and the code
      doesn't look like the algebraic laws. Using pattern matching makes an
      equivalent implementation more appealing.}
      \label{fig:area}
    \end{figure}

    Implementing the observers \tt{isSquare}, \tt{isTriangle}, \tt{sqSide},
    \tt{triW}, \tt{traB1}, \tt{traB2}, and \tt{traH} is left as an
    (excruciating) exercise to the reader. 

    In general, pattern matching is preferred over observers for five reasons. 

    \begin{enumerate}
      \item [\textbf{A.}]
      \begin{enumerate}[label=\arabic*]
        \item  \it{Lawlike.} With pattern matching, code more closely
        resembles algebraic laws. 
        \label{p1}
        \item \it{Single copy} With pattern matching, it's easier to avoid
        duplicating code.
        \label{p2}
        \item \it{Exhaustiveness.} With pattern matching, a programmer can
        easily tell if they've covered all cases, and a compiler can verify this
        through \it{exhaustiveness analysis}.
        \label{p5}
    \end{enumerate}

    \item [\textbf{B.}]
    \begin{enumerate}[start=4, label=\arabic*]
      \item \it{Call-free.} Pattern matching does not need function
      calls to deconstruct data. \nolinebreak
      \label{p3}
      \item \it{Signposting.} With pattern matching, important
      intermediate values are always given names. 
      \nolinebreak
      \label{p4}
    \end{enumerate}
  \end{enumerate}

    I~refer to these as Nice~Properties for the rest of the paper. They are broken
    into two groups: Group~A, which contains properties of pattern matching that
    programmers enjoy in general, and Group~B,  which contains properties strictly
    to do with pattern matching's specific strengths over observers. 

    \hyperref[p1]{Lawlike}~and~\hyperref[p2]{Single copy} are the most important
    of the Nice~Properties: they allow programmers to write code that looks like
    what they write at the whiteboard, with flexible laws and minimal
    duplication. Upholding these two properties is the primary responsibility of
    \it{extensions} to pattern matching (Section~\ref{extensions}). 
    
    Let's see how each of our Nice~Properties holds up in \tt{area}: 
    
    \begin{enumerate}
      \item [\textbf{A.}]
      \begin{enumerate}[label=\arabic*]
        \item \hyperref[p1]{\it{Lawlike.}}
        \hyperref[fig:pmarea]{\tt{pm\_area}} more closely resembles the
        algebraic laws for \tt{area}. 
        \item \hyperref[p2]{\it{Single copy.}}
        \hyperref[fig:observersarea]{\tt{observers\_area}} had to call observers
        like \tt{squareSide} multiple times, and each observer needs \tt{sh} as
        an argument. \hyperref[fig:pmarea]{\tt{pm\_area}} was able to extract
        the \tt{standard\_shape}s' internal values with a single pattern, and
        the name \tt{sh} is not duplicated anywhere in its body. 
        \item \hyperref[p5]{\it{Exhaustiveness.}} 
        If the user adds another value
        constructor to \tt{standard\_shape}--- say, \tt{CIRCLE}, the compiler
        will warn the user of the possibility of a \tt{Match} exception in
        \hyperref[fig:pmarea]{\tt{pm\_area}}, and even tell them that they must
        add a pattern for \tt{CIRCLE} to rule out this possibility.
        \hyperref[fig:observersarea]{\tt{observers\_area}} will not cause the
        compiler to complain, and if it's passed a \tt{CIRCLE} at runtime, the
        program will likely crash! 
    \end{enumerate}
      
    \item [\textbf{B.}]
      \begin{enumerate}[start=4, label=\arabic*]
        \item \hyperref[p3]{\it{Call-free.}} 
        Where did \tt{isSquare},
        \tt{sqSide}, and all the other observers come from? To even
        \it{implement} \hyperref[fig:observersarea]{\tt{observers\_area}}, a
        programmer has to define a whole new set of observers for
        \tt{standard\_shape}s!\footnote{Sometimes the compiler throws
        programmers a bone: with some constructed data, i.e., Scheme's records,
        the compiler provides observers automatically. In others, like with
        algebraic datatypes in ML, it does not.} Most programmers find this
        tiresome indeed. \hyperref[fig:pmarea]{\tt{pm\_area}} did not have to do
        any of this.
        \item \hyperref[p4]{\it{Signposting.}} 
        To extract the internal values,
        \hyperref[fig:pmarea]{\tt{pm\_area}} had to name them, and their names
        serve as documentation. 
      \end{enumerate}
    \end{enumerate}


    Having had the chance to compare pattern matching and observers, if you
    moderately prefer pattern matching, that's good: most functional
    programmers---in fact, most \it{programmers}---likely do as well. 

    \hyperref[fig:pmarea]{\tt{pm\_area}} provides an opportunity to introduce a
    few terms that are common in pattern matching.
    \hyperref[fig:pmarea]{\tt{pm\_area}}~is a classic example of where pattern
    matching most commonly occurs: within a \it{case} expression. A \it{case}
    expression tests a \it{scrutinee} (\tt{sh}) against a list of \it{branches}.
    Each branch contains a pattern on the left-hand side (\tt{SQUARE s}, etc.)
    and an expression on the right-hand side (\tt{s * s}, etc.). When the result
    of evaluating the scrutinee matches with a pattern, the program evaluates
    the right-hand side of the respective branch.\footnote{OCaml, which you'll
    see in future sections, calls \it{case} \tt{match}. Some
    literature~\citep{guardproposal} calls this a \it{head expression}. I~follow
    the example of~\citet{bpc} and \citet{maranget} in calling the things
    \it{case} and \it{scrutinee}. Any of these terms does the job.} 
% \1 Pattern matching has these properties, observers do not. 

\subsection{Popular extensions to pattern matching}
\label{extensions}

    Extensions to pattern matching simplify cases that are otherwise
    troublesome. Specifically, extensions help restore Nice~Properties
    \hyperref[p1]{Lawlike} and~\hyperref[p2]{Single copy} in cases where pattern
    matching fails to satisfy them. 
    
    In this section, I~illustrate several such cases, and I~demonstrate how
    extensions help programmers write code that adheres to the Nice~Properties.
    The three extensions I~describe are those commonly found in the literature
    and implemented in compilers: side conditions, pattern guards, and
    or-patterns. 
    
    To denote pattern matching \it{without} extensions, I~coin the term \it{bare
    pattern matching}. In bare pattern matching, pattern have only two syntactic
    forms: a name or an application of a value constructors to zero or more
    patterns. 


% Subject: Extensions to pattern matching. 

\subsubsection{Side conditions}

    First, I~illustrate why programmers want \it{side conditions}, an extension
    to pattern matching common in most popular functional programming languages,
    including OCaml, Erlang, Scala, and~Haskell\footnote{I~use the term \it{side
    conditions} to refer to a pattern followed by a Boolean expression. Some
    languages call this a \it{guard}, which I~use to describe a different, more
    powerful extension to pattern matching in Section~\ref{guards}. Haskell has
    \it{only} guards, but a Boolean guard \underline{is} a side condition.}. 
    
    I~define a (rather silly) function \tt{exclaimTall} in OCaml on
    \tt{standard\_shape}s. I~have to translate our \tt{standard\_shape} datatype to OCaml, and
    while I'm at it, I~write the type and algebraic laws for
    \tt{exclaimTall}:

    \begin{minipage}[t]{\textwidth}        
        \centering 
        \begin{verbatim}

type standard_shape = Square of float 
           | Triangle of float * float
           | Trapezoid of float * float * float

exclaimTall : standard_shape -> string 

exclaimTall (Square s)              == "Wow! That's a sizeable square!", 
                                        where s > 100.0
exclaimTall (Triangle (w, h))       == "Goodness! Towering triangle!",
                                        where h > 100.0
exclaimTall (Trapezoid (b1, b2, h)) == "Zounds! Tremendous trapezoid!", 
                                        where h > 100.0
exclaimTall sh                      == "Your shape is small.", 
                                        otherwise
    \end{verbatim}
    \end{minipage}

    Armed with pattern matching, I~implement \tt{exclaimTall} in OCaml
    (Figure~\ref{fig:ifexclaimtall}).

    \begin{figure}[ht]
        \begin{verbatim}
            let exclaimTall sh =
            match sh with 
              | Square s -> if s > 100.0 
                            then "Wow! That's a sizeable square!"
                            else "Your shape is small." 
              | Triangle (_, h) -> 
                            if h > 100.0 
                            then "Goodness! Towering triangle!"
                            else "Your shape is small." 
              | Trapezoid (_, _, h) -> 
                            if h > 100.0
                            then "Zounds! Tremendous trapezoid!"
                            else "Your shape is small." 
            \end{verbatim}    
        \caption{An invented function \tt{exclaimTall} in OCaml combines pattern
        matching with an \tt{if} expression, and is not very pretty.}   
        \Description{An implementation of a function exclaimTall in OCaml, which
        uses pattern matching and an if statement.}
        \label{fig:ifexclaimtall}
    \end{figure}
    
    Here, I'm using the special variable \tt{\_}---that's the underscore
    character, a wildcard pattern---to indicate that I~don't care about the
    bindings of a pattern. 

    I~am not thrilled with this code. It gets the job done, but it fails to
    adhere to the Nice~Properties of~\hyperref[p1]{Lawlike}
    and~\hyperref[p2]{Single copy}: the code does not look like the algebraic
    laws, and it duplicates right-hand side, \tt{"Your shape is small"}, three
    times. I~find the code unpleasant to read, too: the actual “good” return
    values of the function, the exclamatory strings, are gummed up in the middle
    of the \tt{if-then-else} expressions.
    
    Fortunately, this code can be simplified by using the \tt{standard\_shape} patterns
    with a \it{side condition}, i.e., a syntactic form for “match a pattern
    \it{and} a Boolean condition.” The \tt{when} keyword in OCaml provides such
    a form, as seen in Figure~\ref{fig:whenexclaimtall}.
        
        \begin{figure}[]
            \begin{verbatim}
                let exclaimTall sh =
                  match sh with 
                    | Square s when s > 100.0 ->
                            "Wow! That's a sizeable square!"
                    | Triangle (_, h) when h > 100.0 ->
                            "Goodness! Towering triangle!"
                    | Trapezoid (_, _, h) when h > 100.0 -> 
                            "Zounds! Tremendous trapezoid!"
                    | _ ->  "Your shape is small." 
                \end{verbatim}
            \caption{With a side condition, \tt{exclaimTall} in OCaml becomes
            simpler and more adherent to the Nice~Properties.} 
            \Description{An implementation of a function exclaimTall in OCaml,
            which uses pattern matching and a side condition.}
            \label{fig:whenexclaimtall}
        \end{figure}

    A side condition streamlines pattern-and-Boolean cases and minimizes
    overhead, restoring \hyperref[p1]{Lawlike} and~\hyperref[p2]{Single copy}.
    And a side condition can exploit bindings that emerge from the preceding
    pattern match. For instance, the \tt{when} clauses in
    Figure~\ref{fig:whenexclaimtall} exploit names \tt{s} and \tt{h}, which are
    bound in the match of \tt{sh} to \tt{Square s}, \tt{Triangle (\_, h)}, and
    \tt{Trapezoid (\_, \_, h)}, respectively. 

    Importantly, side conditions come at a cost: their inclusion means that a
    compiler enforcing the~\hyperref[p5]{Exhaustiveness} Nice~Property becomes
    an NP-hard problem, because it must now perform exhaustiveness analysis not
    only on patterns, but on arbitrary expressions. Modern compilers give a
    weaker form of exhaustiveness that only deals with patterns, and side
    conditions are worth the tradeoff for restoring the two most important of
    the Nice~Properties: \hyperref[p1]{Lawlike} and~\hyperref[p2]{Single copy}.

    A side condition adds an extra “check”---in this case, a Boolean
    expression---to a pattern. But side conditions have a limitation: the check
    can make a decision based off of an expression evaluating to \tt{true} or
    \tt{false}, but not if an expression matches with a non-Boolean pattern. In
    the next section, I~showcase when this limitation matters, and how another
    extension addresses it. 

    \subsubsection{Pattern guards}
    \label{guards}

    To highlight a common use of pattern guards to address such a limitation, I
    modify an example from \citet{guardproposal}, the proposal for pattern
    guards in GHC. Suppose I~have an OCaml abstract data type of finite maps,
    with a lookup operation: 

    \begin{minipage}[t]{\textwidth}
        \centering 
        \begin{verbatim}
            lookup : finitemap -> int -> int option
        \end{verbatim}
    \end{minipage}
    Let's say I~want to perform three successive lookups, and call a “fail”
    function if \it{any} of them returns \tt{None}. Specifically, I~want a
    function with this type and algebraic laws: 

    \begin{minipage}[t]{\textwidth}
        \centering 
        \begin{verbatim}
          

            tripleLookup : finitemap -> int -> int

              tripleLookup rho x == z, where 
                                        lookup rho x == Some w
                                        lookup rho w == Some y
                                        lookup rho y == Some z
              tripleLookup rho x == handleFailure x, otherwise
            
              handleFailure : int -> int 

              handleFailure's implementation is unimportant.
                handleFailure (x : int) = ... some error-handling ... -> x  

        \end{verbatim}
    \end{minipage}

    To express this computation succinctly, the program needs to make decisions
    based on how successive computations match with patterns, but neither bare
    pattern matching nor side conditions give that flexibility. 
    
    Side conditions don't appear to help here, so I~try with bare pattern
    matching. Figure~\ref{fig:pmtriplelookup} shows how I~might implement
    \tt{tripleLookup} as such. 

    \begin{figure}[ht]
        \begin{verbatim}

          let tripleLookup (rho : finitemap) (x : int) =
              match lookup rho x with Some w -> 
                (match lookup rho w with Some y -> 
                  (match lookup rho y with Some z -> z
                  | _ -> handleFailure x)
                | _ -> handleFailure x)
              | _ -> handleFailure x
            \end{verbatim}
        \caption{\tt{tripleLookup} in OCaml with bare pattern matching breaks
         the Nice~Property of~\hyperref[p2]{Single copy}: avoiding duplicated
         code.} 
                    
        \Description{An implementation of the tripleLookup with three levels of
        nested pattern matching.}
        \label{fig:pmtriplelookup}
    \end{figure}

    Once again, the code works, but it's lost
    the~\hyperref[p1]{Lawlike}~and~\hyperref[p2]{Single copy} Nice~Properties by
    duplicating three calls to \tt{handleFailure} and stuffing the screen full
    of syntax that distracts from the algebraic laws. Unfortunately, it's not
    obvious how a side condition could help us here, because we need pattern
    matching to extract and name internal values \tt{w}, \tt{y}, and~\tt{z}
    from constructed data.

    To restore the Nice~Properties, I~introduce a more powerful extension to
    pattern matching: \it{pattern guards}, a form of “smart pattern” in which
    intermediate patterns bind to expressions within a single branch of a
    \tt{case}. Pattern guards can make \tt{tripleLookup} appear \it{much}
    simpler, as shown in Figure~\ref{fig:guardtriplelookup}--- which, since
    pattern guards aren't found in OCaml, is written in Haskell.

    \begin{figure}
        \begin{center}
        \begin{verbatim}
                        tripleLookup rho x
                           | Just w <- lookup rho x
                           , Just y <- lookup rho w
                           , Just z <- lookup rho y
                           = z
                        tripleLookup _ x = handleFailure x
        \end{verbatim}
        \end{center}    
    \caption{Pattern guards swoop in to restore the Nice~Properties, and all is
    right again.} 
    \Description{An implementation of nameOf using pattern guards.}
    \label{fig:guardtriplelookup}
    \end{figure}

    Guards appear as a comma-separated list between the \tt{|} and the \tt{=}.
    Each guard has a pattern, followed by \tt{<-}, then an expression. The guard
    is evaluated by evaluating the expression and testing if it matches with the
    pattern. If it does, the next guard is evaluated in an environment that
    includes the bindings introduced by evaluating guards before it. If the
    match fails, the program skips evaluating the rest of the branch and falls
    through to the next one. As a bonus, a guard can simply be a Boolean
    expression which the program tests the same way it would a side condition,
    so guards subsume side conditions! 
    
    If you need further convincing on why programmers want for guards, look no
    further than Erwig \& Peyton Jones's \it{Pattern Guards and Transformational
    Patterns}~\citep{guardproposal}, the proposal for pattern guards in GHC: the
    authors show several other examples where guards drastically simplify
    otherwise-maddening code. 
    
    Pattern guards enable expressions within guards to utilize names bound in
    preceding guards, enabling imperative pattern-matched steps with expressive
    capabilities akin to Haskell's \tt{do} notation. It should come as no
    surprise that pattern guards are built in to GHC. \raggedbottom
\subsubsection{Or-patterns}
    I~conclude our tour of extensions to pattern matching with or-patterns,
    which are built in to OCaml. Let's consider a final example. I~have a type
    \tt{token} which represents an item or location in a video game and how much
    fun it is, and I~need to quickly know what game it's from and how much fun
    I'd have playing it. To do so, I'm going to write a function
    \tt{game\_of\_token} in OCaml. Here are the \tt{token} type and the type and
    algebraic laws for \tt{game\_of\_token}. 
  
\begin{minipage}[t]{\textwidth}
% \begin{figure}[t]
    \centering 
    \begin{footnotesize}
      \begin{verbatim}  
type funlevel = int

type token = BattlePass  of funlevel | ChugJug    of funlevel | TomatoTown of funlevel
           | HuntersMark of funlevel | SawCleaver of funlevel
           | MoghLordOfBlood of funlevel | PreatorRykard of funlevel
           ... other tokens ...
                    

game_of_token : token -> string * funlevel

game_of_token t == ("Fortnite", f),  where t is any of 
                                      BattlePass f, 
                                      ChugJug f, or
                                      TomatoTown f
game_of_token t == ("Bloodborne", 2 * f), 
                                      where t is any of 
                                        HuntersMark f or 
                                        SawCleaver f
game_of_token t == ("Elden Ring", 3 * f), 
                                      where t is any of 
                                        MoghLordOfBlood f or  
                                        PreatorRykard f
game_of_token _ == ("Irrelevant", 0), otherwise
      \end{verbatim}
    \end{footnotesize}
    \Description{Type and laws for game_of_token.} 
  \end{minipage}  
%     \caption{Type and laws for \tt{game\_of\_token}, which make great use of
%     "where."}
%     \ref{fig:gottypelaws}
% \end{figure}        
        
        I~can write code for \tt{game\_of\_token} in OCaml using bare patterns
        (Figure~\ref{fig:baregot}), but I'm dissatisfied with how it fails the
        \hyperref[p1]{Lawlike} and~\hyperref[p2]{Single copy} Nice~Properties:
        it is visually different from the algebraic laws, and it has many
        duplicated right-hand sides.
        
        \begin{figure}
            \begin{center}
                \begin{verbatim}
        let game_of_token token = match token with 
            | BattlePass f      -> ("Fortnite", f)
            | ChugJug f         -> ("Fortnite", f)
            | TomatoTown f      -> ("Fortnite", f)
            | HuntersMark f     -> ("Bloodborne", 2 * f)
            | SawCleaver f      -> ("Bloodborne", 2 * f)
            | MoghLordOfBlood f -> ("Elden Ring", 3 * f)
            | PreatorRykard f   -> ("Elden Ring", 3 * f)
            | _                 -> ("Irrelevant", 0)
                \end{verbatim}
            \end{center}    

        \caption{\tt{game\_of\_token}, with redundant right-hand sides,
        should raise a red flag.} 
        \Description{An implementation of game_of_token using bare patterns.}
        \label{fig:baregot}
        \end{figure}

        I~could try to use a couple of helper functions to reduce clutter, and
        write something like Figure~\ref{fig:helpergot}. It looks ok, but I'm
        still hurting for Nice~Property \hyperref[p2]{Lawlike}, and now I've
        lost the \hyperref[p3]{Call-free} Property, as well.

        \begin{figure}
            \begin{center}
                \begin{verbatim}
                  let fortnite   f = ... complicated   ... in
                  let bloodborne f = ... complicated'  ... in
                  let eldenring  f = ... complicated'' ... in
                  match token with
                  | BattlePass f -> fortnite f
                  ... and so on ...                
                \end{verbatim}
            \end{center}    
        \caption{\tt{game\_of\_token} with helpers is somewhat better, but I'm
        not satisfied with it.} 
        \Description{An implementation of
        game_of_token using helper functions.}
        \label{fig:helpergot}
        \end{figure}

        Once again, an extension comes to the rescue. \it{Or-patterns} condense
        multiple patterns that share a right-hand side, and when any one of the
        patterns matches with the scrutinee, the right-hand side is evaluated
        with the bindings created by that pattern. I~exploit or-patterns in
        Figure~\ref{fig:orgot} to restore the Nice~Properties and eliminate much
        of the uninteresting code that appeared in
        \ref{fig:baregot}~and~\ref{fig:helpergot}. 

    \begin{figure}
    \begin{center}
    \begin{verbatim}
let game_of_token token = match token with 
    | BattlePass f | ChugJug f | TomatoTown f  -> ("Fortnite", f)
    | HuntersMark f | SawCleaver f             -> ("Bloodborne", 2 * f)
    | MoghLordOfBlood f | PreatorRykard f      -> ("Elden Ring", 3 * f)
    | _                                        -> ("Irrelevant", 0)
    \end{verbatim}
    \end{center}    
    \caption{Or-patterns condense \tt{game\_of\_token}
    significantly, and it is easier to read line-by-line.} 
    \Description{An
    implementation of game_of_token using or-patterns.}
    \label{fig:orgot}
    \end{figure}

    In addition to the inherent appeal of brevity, or-patterns serve to
    concentrate complexity at a single juncture and create single points of
    truth, restoring the~\hyperref[p1]{Lawlike} and~\hyperref[p2]{Single copy}
    properties. 
    
    And by eliminating helper functions, like the ones in
    Figure~\ref{fig:helpergot}, or-patterns actually boost performance.
      
    \subsubsection{Wrapping up pattern matching and extensions}
    
    I~have presented three popular extensions that make pattern matching more
    expressive and how to use them effectively. Earlier, though, you might have
    noticed a problem. Say I~face a decision-making problem that obliges me to
    use \emph{all} of these extensions. When picking a language to do so, I am
    stuck! No major functional language has all three of these extensions.
    Remember when I~had to switch from OCaml to Haskell to use guards, and back
    to OCaml for or-patterns? The two extensions are mutually exclusive in
    Haskell and OCaml, and also Scala, Erlang/Elixir, Rust, F\#, and
    Agda~\citep{haskell, ocaml, scala, erlang, elixir, rust, fsharp, agda}. 


    I~find the extension story somewhat unsatisfying. At the very least, I~want
    to be able to use pattern matching, with the extensions I~want, in a single
    language. Or, I~want an alternative that gives me the expressive power of
    pattern matching with these extensions. 

    
    \section{\VC's equations}
    \label{verseoverobservers}

    An intriguing alternative to pattern matching exists in \it{equations}, from
    the Verse Calculus (\VC), a core calculus for the functional logic
    programming language \it{Verse}~\citep{antoy2010functional,
    hanus2013functional, verse}. (For the remainder of this thesis, I~use “\VC”
    and “Verse” interchangeably.)

    % Equations present a different, yet powerful, way to write code that makes
    % decisions and deconstructs data. In this section, I~will introduce you to
    % equations similarly to I~how I~introduced to you pattern matching. Once
    % you're familiar with equations, you'll be ready to compare their strengths
    % and weaknesses with those of pattern matching, and judge the compromise I
    % propose. 

    % Even if you read~\citet{verse}, \VC's equations and the paradigms of
    % functional logic programming might look unfamiliar. To help ease you into
    % familiarity with equations, I~ground explanations and examples in how
    % they relate to pattern matching. 

    \VC's \it{equations} test for structural equality and create bindings. Like
    pattern matching, equations scrutinize and deconstruct data at runtime by
    testing for structural equality and unifying names with values. Unlike
    pattern matching, \VC's equations can unify names on both left- \it{and}
    right-hand sides. 

    Every equation in \VC takes the form \it{x = e}, where \it{x} is a name and
    \it{e} is an expression\footnote{To make programmers happy, full Verse
    allows an equation to take the form $e_{1} = e_{2}$, which desugars to
    $\vexists{x} x = e_{1};\; x = e_{2}$, with $x$ fresh.}. During runtime, \VC
    relies on a process called \it{unification} to attempt to bind \it{x} and
    any unbound names in \it{e} to values. Unification is the process of finding
    a substitution that makes two different logical atomic expressions
    identical. Much like pattern matching, unification can fail if the runtime
    attempts to bind incompatible values or structures (i.e., finds no
    substitution). 

    An equation offers a form of binding that looks like a single pattern match.
    What about a list of many patterns and right-hand sides, as in a case
    expression? For this, \VC has \it{choice} ($\choice$). The full semantics of
    choice are too complex to cover here, but choice, when combined with the
    \one operator, has a very similar semantics to case; that is, “proceed
    and create bindings if any one of these computations succeed.” 

    Let's look at what equations, \one, and choice look like in \VC
    (Figure~\ref{fig:versearea}). I've written the \tt{area} function in \VC
    extended with a \tt{float} type and a multiplication operator \tt{*}.

    \begin{figure}[]
        \verselst
        \begin{lstlisting}[numbers=none]
$\exists$ vc_area. vc_area = $\lambda$ sh. 
  one {  $\exists$ s. sh   = $\langle$SQUARE, s$\rangle$; s * s
       $\choice$ $\exists$ w h. sh = $\langle$TRIANGLE, w, h$\rangle$; 
              0.5 * w * h
       $\choice$ $\exists$ b1 b2 h. sh = $\langle$TRAPEZOID, b1, b2, h$\rangle$; 
              0.5 * (b1 + b2) * h}
        \end{lstlisting}

        \small
\begin{verbatim}


                        Laws for area: 
                        
                        area (SQUARE s)              == s * s 
                        area (TRIANGLE (w, h))       == 0.5 * w * h
                        area (TRAPEZOID (b1, b2, h)) == 0.5 * (b1 + b2) * h
        \end{verbatim}
    \caption{\tt{vc\_area} in \VC uses existentials and equations rather than
    pattern matching. Below are the algebraic laws for the original \tt{area}
    function.} \Description{An implementation of area in \VC.}
    \label{fig:versearea}
    \end{figure}

    In the figure, the name \tt{vc\_area} is bound to a lambda ($\lambda$) that
    takes a single argument, \tt{sh}. The body of the function is a \one
    expression over three choices (separated by~$\choice$). If any of the
    choices succeeds, \one~ensures evaluation of the other choices halts and the
    succeeding expression's result is returned. In each branch, the existential
    $\exists$ introduces names \tt{s, w, h, b1, b2}, and is followed by an
    \it{equation} that unifies them with \tt{sh}, along with the familiar value
    constructors \tt{SQUARE}, \tt{TRIANGLE}, and~\tt{TRAPEZOID}. After the
    equation is a semicolon followed by an expression, which is evaluated if the
    equation succeeds. As in~\hyperref[fig:pmarea]{\tt{pm\_area}}, the
    right-hand sides of \tt{vc\_area} are \it{guarded} by a “check;” now, the
    check is successful unification in an equation rather than a successful
    match on a pattern. Similarly, \one with a list of choices represents
    matching on any \it{one} pattern to evaluate a single right-hand side. 

    Why use equations? I~begin with a digestible claim: \VC's equations are
    preferable to observer functions. This claim mirrors my argument for pattern
    matching, and to support it I~appeal to the Nice~Properties: 

    \begin{enumerate}
      \item \hyperref[p1]{\it{Lawlike.}} 
      \tt{vc\_area} makes only one addition to the algebraic laws: the explicit
      $\exists$. This makes \tt{vc\_area} look more like mathematical notation
      than pure algebraic laws, but that might not be a bad thing: while it less
      resembles the algebraic laws a programmer would write, it more resembles
      the equations that a mathematician would. 
      \item \hyperref[p2]{\it{Single copy.}} \tt{vc\_area} does not duplicate any code. 
      \item \hyperref[p5]{\it{Exhaustiveness.}}  By writing the equations that
      unify \tt{sh} with the value-constructor forms first, it is easy in this
      example to see that the code is exhaustive. Creating a static analysis
      tool for \VC that ensures exhaustiveness on all expressions may or may not
      be a significant challenge; full Verse has a tool that can verify if a
      terminating expression on the right-hand side of a function will
      \it{always succeed} or not~\citep{peyton-jones2024verification}.
      \item \hyperref[p3]{\it{Call-free.}} \tt{vc\_area} deconstructs
      user-defined types as easily as~\hyperref[fig:pmarea]{\tt{pm\_area}} does
      with pattern matching. 
      \item \hyperref[p4]{\it{Signposting.}} \tt{vc\_area} has all important
      internal values named explicitly.
      
      % This Property may not hold, because \VC on its own is untyped.
      % Without a type system, a compiler cannot help me with non-exhaustive
      % cases. However, there is no published compiler, type system or no, for
      % Verse, and only when one is made available can I~make this assertion. For
      % this reason, and for the sake of focusing on more important details, I
      % choose to proceed in my analysis of equations in \VC without considering
      % this Property. 
    \end{enumerate}

    You understand why I claim programmers prefer equations to observer
    functions. Now I~make a stronger claim: equations are \it{at least as good
    as} pattern matching with popular extensions. How can I~claim this? By
    appealing again to the Nice~Properties! In~Section~\ref{pmandextensions},
    I~demonstrated how pattern matching had to resort to extensions to regain
    the Properties when challenging examples stole them away. In
    Figure~\ref{fig:verseextfuncs}, I've implemented those examples in \VC (this
    time extended with strings, floats, and \tt{*}) using choice and equations.
    Take a look for yourself! 

    \begin{figure}[ht] 
        \begin{minipage}[h]{0.54\linewidth}
          \verselst
          \begin{lstlisting}[numbers=none, basicstyle=\tiny, xleftmargin=.2em,
                            showstringspaces=false,
                            frame=single]
$\exists$ exclaimTall. exclaimTall = $\lambda$ sh. 
  one { 
      $\exists$ s. sh = $\langle$Square s$\rangle$; 
      s > 100.0; "Wow! That's a sizeable square!"
    | $\exists$ w h. sh = $\langle$Triangle, w, h$\rangle$; 
      h > 100.0; "Goodness! Towering triangle!"
    | $\exists$ b1 b2 h. sh = $\langle$Trapezoid, b1, b2, h$\rangle$;
      h > 100.0; "Zounds! Tremendous trapezoid!"
    | "Your shape is small." }
            \end{lstlisting}
                \Description{exclaimTall}
        \subcaption{\tt{exclaimTall} in \VC }
            \label{fig:verseexclaimtall} 
        %   \vspace{4ex}
        \end{minipage}%%
        \begin{minipage}[h]{0.5\linewidth}
          \verselst
          \begin{lstlisting}[numbers=none, basicstyle=\tiny, xleftmargin=2em,
                        frame=single]
$\exists$ tripleLookup. tripleLookup = $\lambda$ rho x. 
    one { $\exists$ w. lookup rho x = $\langle$Just w$\rangle$; 
          $\exists$ y. lookup rho w = $\langle$Just y$\rangle$; 
          $\exists$ z. lookup rho y = $\langle$Just z$\rangle$;
          z
        | handleFailure x }
          \end{lstlisting}
                \Description{exclaimTall}
            \subcaption{\tt{tripleLookup} in \VC }
            \label{fig:versetriplelookup} 
        \vspace{4ex}
        \end{minipage} 
        \begin{minipage}[h]{\linewidth}
          \verselst
          \begin{lstlisting}[numbers=none, basicstyle=\tiny, xleftmargin=9em, 
                            frame=single, showstringspaces=false]
$\exists$ game_of_token. game_of_token = $\lambda$ token. 
  $\exists$ f. one {
         token = one { $\langle$BattlePass, f$\rangle$ | $\langle$ChugJug, f$\rangle$ | $\langle$TomatoTown, f$\rangle$}; 
           $\langle$"Fortnite", f$\rangle$
       | token = one { $\langle$HuntersMark, f$\rangle$ | $\langle$SawCleaver, f$\rangle$}; 
           $\langle$"Bloodborne", 2 * f$\rangle$
       | token = one { $\langle$MoghLordOfBlood, f$\rangle$ | $\langle$PreatorRykard, f$\rangle$}; 
           $\langle$"Elden Ring", 3 * f$\rangle$
       |   $\langle$"Irrelevant", 0$\rangle$ }
          \end{lstlisting}
                \Description{game_of_token in \VC}
        \subcaption{\tt{game\_of\_token} in \VC }
            \label{fig:versegot} 
        \vspace{4ex}
        \end{minipage}%% 
    \caption{Code for the \ref{pmandextensions} functions with equations looks
    similar, and it doesn't need extensions.}
    \label{fig:verseextfuncs}
      \end{figure}
        
    The code in Figure~\ref{fig:verseextfuncs} has all the Nice~Properties. This
    is promising for \VC. If it rivals pattern matching with popular extensions
    in desirable properties, and \VC does everything using only equations and
    choice, it seems like the language is an intriguing option for writing code! 

    \subsection{\VC has a challenging cost model}
    \label{vcbadcost}

    So what's the catch? In \VC, names (logical variables) are \it{values}, and
    they can just as easily be the result of evaluating an expression as an
    integer or tuple. To bind these names, \VC, like other functional logic
    languages, relies on \it{unification} of its logical variables and
    \it{search} at runtime to meet a set of program
    constraints~\citep{antoy2010functional, hanus2013functional}. Combining
    unifying logical variables with search requires backtracking, which can lead
    to exponential runtime cost~\citep{hanus2013functional, wadler1985replace,
    clark1982introduction}. 

    % Todo: An example would be nice. Backtracking/evaluating backwards, then
    % done. 

    Pattern matching, by contrast, can be compiled to a \it{decision tree}, a
    data structure that enforces linear runtime performance by guaranteeing no
    part of the scrutinee will be examined more than once~\citep{maranget}. A
    decision tree does not backtrack: once a program makes a decision based on
    the form of a value, it does not re-test it later with new information. 
    

\end{document}


% \subsection{Introducing \PPlus}
% \label{pplus}

%     \PPlus offers a standardized core for pattern matching, enhanced by common
%     extensions. In addition to bare pattern matching— names and applications of
%     value constructors— the language includes pattern guards, or-patterns, and
%     side conditions. Although pattern guards subsume side conditions, I~include
%     side conditions in \PPlus and separate them from guards for purpose of
%     study. Furthermore, \PPlus introduces a new form of pattern: \pcommap. This
%     allows a pattern in \PPlus be a \it{sequence} of sub-patterns, allowing a
%     programmer to stuff as many patterns as they want in the space of a single
%     one. I~discuss the implications of this in Section~\ref{ppweird}.

% \subsection{Formal Semantics of \PPlus}


% In this section, I~present a big-step operational semantics for \PPlus. The
% semantics describes how expressions in \PPlus are evaluated and how pattern
% matching works in the language. Instead of a rewrite semantics that desugars
% extensions into \it{case} expressions, \PPlus has a big-step semantics that
% directly describes how they are handled by the runtime core.
% Figure~\ref{fig:ppsyntax} contains the concrete syntax of \PPlus,
% Figure~\ref{fig:ppmetavars} provides the metavariables used in the judgement
% forms and~rules of the semantics, and Section~\ref{ppsemantics} contains the
% forms and rules. Since pattern matching is the heart of \PPlus, I~also
% describe it in plain English.

% \begin{figure}
% \begin{center}
% \ppcsyntax
% \end{center}
% \Description{The concrete syntax of PPlus.}
% \caption{\PPlus: Concrete syntax}
% \label{fig:ppsyntax}
% \end{figure}

%   % \subsection{Forms of Judgement for \PPlus:}

% \begin{table}
%   \begin{tabular}{ll}
%   \toprule
%       \multicolumn2{l}{\emph{\PPlus Metavariables}} \\
%   \midrule
%       $\expr$& expression \\
%       $\v,\; \v'$& value \\
%       $K$& value constructor \\ 
%       $p$& pattern \\ 
%       $x, y$& names \\ 
%       $\dagger$& pattern match failure \\ 
%       $s$& a solution, either $\v$ or $\dagger$ \\ 
%       \Rho& environment: $name \rightarrow \mathcal{V}$ \\
%       \Rho\:+\:\Rhoprime& extension \\
%       $\Rho \uplus \Rho'$& disjoint union \\
%       $\{ x \mapsto y \} $& environment with name $x$ mapping to $y$ \\
%   \bottomrule
%   \end{tabular}    
%   \Description{Metavariables used in rules for PPlus.}
%   \caption{\PPlus metavariables and their meanings}
%   \label{fig:ppmetavars}
% \end{table}

% \subsubsection{Expressions in \PPlus}
%     An expression in \PPlus evaluates to produce a single value, shown by the
%     \textsc{Eval} judgement form. 
%     \begin{align*}
%         &(\textsc{Eval})\; \prun \quad   \rm{ Expression $e$ evaluates in 
%                             environment \Rho\ to produce value $\v$.} \\
%     \end{align*}

% \subsubsection{Pattern matching in \PPlus}

% Pattern matching in \PPlus is represented by a single judgement form, with two
% possible outcomes: success (a refined environment \Rhoprime) and failure
% ($\dagger$). The metavariable \solution, a \it{solution} to a pattern match,
% combines these outcomes. 
% \raggedbottom

% \begin{align*}
%       &\pmatch \quad   \rm{ Pattern $p$ matches value $\v$ in environment \Rho, 
%                               producing bindings \Rhoprime;} \\
%       &(\textsc{Match-Success}) \\
%       &\pfail  \quad\; \rm{ Pattern $p$ does not match value $\v$ in environment \Rho.} \\
%       &(\textsc{Match-Fail})
% \end{align*}

% Pattern guards introduce a special case: if a pattern is bound to an expression
% in the form $\pmatch[pat = {p \leftarrow \expr}, newenv = \solution]$, it will
% match if the expression $e$ evaluates to value $v'$ and $p$ matches with $v'$.
% When a pattern is standalone, as in all other cases, it will match on $v$, the
% \it{original} scrutinee of the case expression. For example, in the \PPlus
% expression \tt{case x of Square s, Big b <- mumble s -> b}, the result of
% evaluating \tt{x} matches with \tt{Square s}, and the result of evaluating
% \tt{mumble s} matches with \tt{Big b}. Generally: when $\pmatch[pat = {p_{1},
% p_{2} \leftarrow \expr}, newenv = \solution]$, $\pmatch[pat=p_{1}, value=\v,
% newenv=\solution{1}]$, $\ppeval[result=\v']$, and $\pmatch[pat=p_{2},
% value=\v', newenv=\solution{2}]$.

% Pattern matching is defined inductively: 
% \begin{itemize}
%     \item A name $x$ matches any value $\v$, and produces the environment 
%     $\bracketed{x \mapsto \v}$. 
%     \item A value constructor $K$ applied to atoms  matches 
%     a value $\v$ if $\v$ is an application of $K$ to the same number of values,
%     each of which matches the corresponding atom. Its match produces 
%     the disjoint union of matching all internal atoms to all internal values. 
%     \item A pattern \whenexpr\ matches when $\expr$ evaluates to a value other than 
%     $\mathit{false}$, and produces the empty environment. 
%     \item A wildcard pattern \wildcard\ matches any value \v, and produces the
%     empty environment. 
%     \item A pattern \parrowe\ matches when $\expr$ evaluates to 
%           value $\v$, and $p$ matches $\v$. 
%     \item A pattern \pcommap\ matches if both $p$ and $p_{2}$ match.
%     \item A pattern \porp\ matches if either $p_{1}$ or $p_{2}$
%     matches. 
% \end{itemize}

% When a pattern is of the form $K, p_{1}, \dots, p_{n}$, each sub-pattern $p_{i}$
% may introduce new variables during pattern matching. Bindings for all these
% variables must be combined in a single environment. \it{Disjoint union} is an
% operation on two environments. Its definition borrowed, paraphrased, from
% \citet{bpc}: 

% \it{Disjoint union} is a way to capture the aggregate environment of matching
% constructed data with a constructor-application pattern. The disjoint union of
% environments $\rho_{1}$ and $\rho_{2}$ is written $\rho_{1} \uplus \rho_{12}$,
% and it is defined if and only if $\dom \rho_{1} \cap \dom \rho_{2} = \emptyset$:

% \begin{align*}
%   \dom(\rho_{1} \uplus \rho_{2}) &= \dom \rho_{1} \uplus \dom \rho_{2}, \\
%     (\rho_{1} \uplus \rho_{2})(x) &= 
%   \begin{cases}
%     \rho_{1}(x), & \text{if } x \in \dom  \rho_{1} \\
%     \rho_{2}(x), & \text{if } x \in \dom \rho_{2} \\
% \end{cases}
% \end{align*}

% For example, in the \PPlus expression \tt{case Pair 1 (Pair 2 3) of Pair x (Pair
% y z) -> z}, the right-hand size \tt{z} is evaluated with the environment
% $\bracketed{x \mapsto 1} \uplus \bracketed{y \mapsto 2} \uplus \bracketed{z
% \mapsto 3}$. 

% Disjoint union across multiple results, when any result is failure, still
% represents failure: 

% \begin{gather*}
%   \dagger \uplus \rho = \dagger \quad
%   \rho \uplus \dagger = \dagger \quad
%   \dagger \uplus\, \dagger = \dagger
% \end{gather*}


% At runtime, disjoint union also fails if  $\dom \rho_{1} \cap \dom \rho_{2} \neq
% \emptyset$, meaning a constructor-application pattern had duplicate names, like
% in \tt{Pair x x}. This means \PPlus has only \it{linear} patterns under value
% constructors, i.e., the same name cannot bind multiple components of a single
% instance of constructed data. 

% Finally, when a pattern in a branch in a \it{case} expression fully matches, its
% corresponding right-hand side is evaluated with top-level $\rho$ extended with
% the $\rho'$ produced by the pattern match. Environment extension is notated
% $\rho + \rho'$. 

%     \subsection{Rules (Big-step Operational Semantics) for \PPlus:}
%     \label{ppsemantics}
    
%     Some of these rules are a variation on the rules found in Ramsey~\cite{bpc}.   
    
%       \ppsemantics 
      
%       \bigskip 
    
%     I~show how a programmer might utilize \PPlus to solve the previous problems
%     (Section~\ref{extensions}) in Figure~\ref{fig:ppfuncs}. The examples in the
%     figure all compile with the \tt{pplus} program. 

%     \begin{figure}[ht] 
%       \begin{minipage}[h]{0.54\linewidth}
%         \pplst 
%         \begin{lstlisting}[numbers=none, basicstyle=\tiny, xleftmargin=.2em,
%           showstringspaces=false,
%           frame=single]
% val exclaimTall = \sh.
%   case sh of Square s, when (> s) 100 -> 
%             Wow! That's A Tall Square!  
%     | Triangle _ h, when (> h) 100 ->
%             Goodness! Towering Triangle!
%     | Trapezoid _ _ h, when (> g) 100 -> 
%             Zounds! Tremendous Trapezoid!
%     | _ ->  Your Shape Is Small
%   \end{lstlisting}
%       \Description{exclaimTall in PPlus}
%       \subcaption{\tt{exclaimTall} in \PPlus}
%           \label{fig:ppexclaimtall} 
%       %   \vspace{4ex}
%       \end{minipage}%%
%       \begin{minipage}[h]{0.5\linewidth}
%         \pplst 
%         \begin{lstlisting}[numbers=none, basicstyle=\tiny, xleftmargin=2em,
%                       frame=single]
%   val tripleLookup = \ rho. \x.
%     case x of 
%       Some w <- (lookup rho) x
%     , Some y <- (lookup rho) w
%     , Some z <- (lookup rho) y -> z
%     | _ -> handleFailure x

%    \end{lstlisting}
%         \Description{tripleLookup in PPlus}
%         \subcaption{\tt{tripleLookup} in \PPlus}
%             \label{fig:pptriplelookup} 
%         \vspace{4ex}
%       \end{minipage} 
%       \begin{minipage}[h]{\linewidth}
%         \pplst 
%         \begin{lstlisting}[numbers=none, basicstyle=\tiny, xleftmargin=9em,
%           showstringspaces=false,
%           frame=single]
% val game_of_token = \t. 
%   case t of  
%     BattlePass f  | (ChugJug f | TomatoTown f) -> P (Fortnite f)
%   | HuntersMark f | SawCleaver f               -> P (Bloodborne ((* 2) f))
%   | MoghLordOfBlood f | PreatorRykard f        -> P (EldenRing ((* 3) f))
%   | _                                          -> P (Irrelevant 0)
% \end{lstlisting}
%       \Description{game_of_token in PPlus}
%       \subcaption{\tt{game\_of\_token} in \PPlus}
%           \label{fig:ppgot}
%       \vspace{4ex}
%       \end{minipage}%% 
%       \caption{The functions in \PPlus have the same desirable implementation as
%       before. All the example compile with the \tt{pplus} program.}
%   \label{fig:ppfuncs}
%     \end{figure}        
    
%     As mentioned in Section~\ref{compromise}, masquerading value constructors
%     stand in for strings. \PPlus has no infix operators, so some expressions are
%     parenthesized. 

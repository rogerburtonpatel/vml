\documentclass[manuscript,screen,review, 12pt, nonacm]{acmart}
\let\Bbbk\relax % Fix for amssymb clash 
\usepackage{vmlmacros}
\AtBeginDocument{%
  \providecommand\BibTeX{{%
    \normalfont B\kern-0.5em{\scshape i\kern-0.25em b}\kern-0.8em\TeX}}}
\usepackage{outlines}
\setlength{\headheight}{14.0pt}
\setlength{\footskip}{13.3pt}
\title{An Alternative to Pattern Matching, Inspired by Verse}

\author{Roger Burtonpatel}
\email{roger.burtonpatel@tufts.edu}
\affiliation{%
\institution{Tufts University}
\streetaddress{419 Boston Ave}
  \city{Medford}
  \state{Massachusetts}
  \country{USA}
  \postcode{02155}
  }
\begin{document}
\section{Equations subsume pattern matching with popular extensions}
\label{pplustovminus}
    In my introduction I~stated that \VMinus can be compiled to a decision tree,
    and that \VMinus subsumes pattern matching with popular extensions. Having
    shown the former, I~now show the latter. I~do so by presenting an algorithm
    \PtoVTran\ which translates \PPlus to \VMinus. 
    
    % , and proving that the
    % translation preserves semantics. 

    \subsection{Domains}

    I~give the names and domains of the translation functions: 
    
    \begin{align*}
        &\PtoVTran: \PPlus Exp\; \rightarrow\; \VMinus Exp \\
        &\PTran: Pattern\; \rightarrow\; Name\; \rightarrow\; Name\ list\ *\ Guard\ list \\
        % &\mathcal{B}: Pattern\; ->\; Name\; Set \\
    \end{align*}
    
    The translation functions \PtoVTran\ and \PTran\ are defined case by case: 
    
    % \subsection{Binding names}
    
    % \begin{align*}
        %     &\Bindings{x} = \bracketed{x} \\ 
        %     &\Bindings{K} = \bracketed{} \\
        %     &\Bindings{K\; p_{1} {\dots} p_{n}} = \Bindings{p} \cup {\dots} \cup \Bindings{p_{n}} \\
        %     &\Bindings{\porp} = \Bindings{p_{1}} \cap \Bindings{p_{2}} \\
        %     &\Bindings{\pcommap} = \Bindings{p} \cup \Bindings{p'} \\
        %     &\Bindings{\parrowe} = \Bindings{p} \\
        %     &\Bindings{\whenexpr} = \bracketed{}
        % \end{align*}
        
        \subsection{Translating Expressions}
        
        \newcommand\btran[1]{\mathcal{B}[\![#1]\!]}
        
        \begin{align*}
            &\ptov[exp=x, result=x] \\
            &\ptov[exp={K\; \expr[1] {\dots} \expr[n]}, result={K\; \ptovtran{\expr[1]} {\dots} \ptovtran{\expr[n]}}] \\
            &\ptov[exp={\lambda x.\; \expr}, result={\lambda x.\; \ptovtran{\expr}}] \\
            &\ptov[exp={\expr[1]\; \expr[2]}, result={\ptovtran{\expr[1]}\; \ptovtran{\expr[2]}}] \\
            % &\ptov[exp={\tt{case}\; \expr\;  \emptyseq}, result={{\iffitt{\vexists{x}\; x = \ptovtran{\expr};\; \iffitt{}}}}]\; \rm{, $x$ fresh }   \\
            &\ptovtran{\tt{case}\; \expr\;  p_{1}\; \expr[1] \vert {\dots} \vert p_{n}\; \expr[n]} \rightsquigarrow \\
            &\hspace{2em} \rm{$\forall i.\; 1 \leq i \leq n:$} \\
            &\hspace{2em} \tt{if } {\vexists{x}\; x \tt{ = } \ptovtran{\expr};}\; \\
            &\hspace{2em} \rm{ let } (\mathit{ns}_{1}, \mathit{gs}_{1}) {\dots} (\mathit{ns}_{i}, \mathit{gs}_{i}) = \ptran{p_{1}}x\; \cdot\; {\dots} \cdot\; \ptran{p_{i}}x \rm { in } \\
            &\hspace{2em} \iffitt{\vexists{\mathit{ns}_{1}}\; {\mathit{gs}_{1}} \rightarrow \ptovtran{\expr[1]};\;
                       \square\; {\dots} \square\; \vexists {\mathit{ns}_{i}}\; {\mathit{gs}_{i}} \rightarrow \ptovtran{\expr[i]}} \\
            &\hspace{2em} \tt{fi} \\
            &\hspace{2em} \rm{, $x$ fresh }
        \end{align*}
        
        \subsection{Translating Patterns}
        
        \begin{align*}
            &\pattov[pat=y, result={(y, [x = y])}] \\
            &\pattov[pat=K, result={([], [x = K])}] \\
            &\ptran{K\; p_{1}\; {\dots}\; p_{n}}x \rightsquigarrow \\
            &\hspace{2em} \rm{$\forall i.\; 1 \leq i \leq n:$} \\
            &\hspace{2em} \rm{ let } y_{i} \rm{ be a fresh name, }  \\
            &\hspace{2em} (\mathit{ns}_{1}, \mathit{gs}_{1}) {\dots} (\mathit{ns}_{i}, \mathit{gs}_{i}) = \ptran{p_{1}}y_{1} \cdot {\dots} \cdot \ptran{p_{i}}y_{i} \\
            &\hspace{2em} \rm{ in } \\
            &\hspace{2em} ({\mathit{ns}_{1} \cdot {\dots} \cdot \mathit{ns}_{i}} \cdot {y_{1} {\dots} y_{i}}, x = K\; y_{1}\; {\dots}\; y_{i} \cdot \; \mathit{gs}_{1} \cdot {\dots} \cdot \mathit{gs}_{i}) \\
            &\pattov[pat=\mathit{when}\; e, result={([], [\ptovtran{e}])}] \\
            &\pattov[pat=\pcommap, 
            result={\rm{let } 
            {(\mathit{ns}_{1}, \mathit{gs}_{1}) = \ptran{p}x}\; , 
            {(\mathit{ns}_{2}, \mathit{gs}_{2}) = \ptran{p'}x} \rm{ in }
            (\mathit{ns}_{1} \cdot \mathit{ns}_{2}, \mathit{gs}_{1} \cdot \mathit{gs}_{2})}] \\
            &\pattov[pat=\porp, 
            result={\rm{let } (\mathit{ns}_{1}, \mathit{gs}_{1}) = \ptran{p}x\; ,
            (\mathit{ns}_{2}, \mathit{gs}_{2}) = \ptran{p'}x \rm{ in }
            (\mathit{ns}_{1} \cdot \mathit{ns}_{2}, [\mathit{gs}_{1} \choice \mathit{gs}_{2}])}]
        \end{align*}

    \subsubsection{Significance of the translation}

    In Section~\ref{extensions}, I~showed how extensions to pattern matching
    uphold Nice Properties \ref{p1}~and~\ref{p2}, and how with them, programmers
    can write more concise code. \bf{The translation aims to show that if a
    programmer can code with desirable properties in \PPlus, they can write code
    with the same properties in \VMinus.} Proving this claim formally is a goal
    for future work. Informally, \PTran\ does not duplicate code except for
    introducing new names when translating a constructor-application pattern,
    and I~believe eliminating this redundancy is possible through a desugaring
    optimization based off of the laws presented in Section~\ref{ppweird}. 

    \PtoVTran\ is largely uninteresting, except for the translation from
    \it{case} to \it{if-fi}.
        
    To compile \it{case} expressions to decision trees like Maranget does,
    translate \PPlus to \D using $(\DTran\; o\; \PtoVTran)$.
    
    Finally, I~claim that the translation from \it{case} expressions to decision
    trees, $(\DTran\; o\; \PtoVTran)$, is consistent with Maranget and
    others~\citep{maranget,scottramsey}. Proving this claim is a good goal for
    future work; it is not the main focus of this paper. 

    \section{Implementations}

    I~have full implementations of \PPlus, \VMinus, and~\D at
    \url{https://github.com/rogerburtonpatel/vml}. The languages are complete,
    from parsers to evaluation to unparsers. In the same repository lives the
    \tt{dtran} program, which translates from \PPlus to \VMinus and \VMinus to
    \D. Translating \PPlus to \D is also possible by composing these two
    translations. With the implementations, I~have been able to gather
    satisfying empirical evidence of the functionality of the translations, and
    that they (empirically) preserve semantics. 


    \section{Related Work}

    The dual foundations of this paper are Augustsson et al.'s Verse
    Calculus~\citep{verse} and Maranget's decision trees~\citep{maranget}.
    Augustsson et al. give the formal rewrite semantics for the Verse Calculus;
    Maranget gives an elegant formalism of decision trees. The big-step
    semantics in this paper is based off of the rewrite semantics; proving their
    equivalence is the subject of future work. I~chose a big-step semantics
    because it is the style of semantics I~am most comfortable with; writing the
    formalisms this way facilitated writing the code. Maranget's formalism was
    the foundation off of which I~built \D. 
    
    Extensions to pattern matching, and how they appeal to language designers,
    find an excellent example in Erwig \& Peyton Jones~\citep{guardproposal}.
    The authors describe pattern guards and transformational patterns, both of
    which allow a Haskell programmer to write more concise code using pattern
    matching. Or-patterns are documented in the OCaml Language Reference
    Manual~\citep{ocaml}.
    
    Compiling Pattern Matching~\citep{augustsson1985compiling} by Augustsson
    gives a foundation in exactly what it says. Ramsey and Scott have a crisp
    example of a match-compilation algorithm (pattern matching to decision
    trees) in When Do Match-Compilation Heuristics Matter?~\citep{scottramsey}.
    Scott and Ramsey's algorithm structurally inspired mine, and I~was privy to
    source code from the algorithm whose study aided my implementation. 
    
    
    \section{Conclusion}

    I~have introduced the languages \PPlus and \VMinus to demonstrate the
    viability of equations as an alternative to pattern matching, and \D,
    \PtoVTran, and \DTran\ to show how both languages can be compiled to
    efficient code. I~have shown with equivalent examples how programs written
    in \VMinus has the same desirable properties as equivalent programs written
    in \PPlus, and I~have also shown that translating from pattern matching to
    equations preserves the desirable properties. Finally, I~have shown how
    \VMinus, like pattern matching, can be compiled to efficient code. In doing
    so, I~have demonstrated that programming with equations is a promising
    alternative to pattern matching. 

    I~have also fully implemented the languages. They exist for use and
    experimentation: they are syntactically simple and have conceptually
    accessible operational semantics. I~hope that programmers will explore and
    develop their own opinions of these languages, which are publically
    available at \url{https://github.com/rogerburtonpatel/vml}. 

    Finally, and in particular with \VMinus, I~hope to provide a stepping stone
    between pattern matching and equations that a new programmer to Verse will
    find illuminating. 
    
    % hope to have paved a small
    % segment of the path that the curious programmer or language enthusiast who
    % wishes to better understand Verse will take. Be they transitioning to
    % equations from pattern matching to equations or curious about how those
    % equations might be compilable to decision trees, I~hope they find the
    % languages, and this document, illuminating. 
    
    \section{Discussion and Future Work}        
    \label{futurework}

        TODO: What to discuss about V- syntax not closed under subst? 

        \subsubsection{Desirable formalisms about translations}
        First and foremost, to flush out the proofs of semantics preservation of
        \PTran\ and \DTran\ is my top priority. Making these as airtight as
        possible will greatly strengthen the argument that \VMinus is not only a
        viable alternative to \PPlus syntactically; it is also formally
        equivalent. 

        \subsubsection{Desirable formalisms about \VMinus}
        Two formalisms would strengthen the viability of \VMinus, and are
        essential targets for future work: \bf{Proving \VMinus~is
        deterministic}, and \bf{proving the big-step semantics of \VMinus~is
        consistent with the published semantics of \VC.} As the authors of the
        Verse paper proved that the rewrite semantics of Verse is
        skew-confluent, I~must prove that big-step semantics of \VMinus is
        deterministic, despite the nondeterminism of choosing a guard. Second,
        \VMinus~is designed to be Verse-like, and formalizing the relationship
        between the two strengthens \VMinus's viability as an intermediate
        language between pattern matching and equations. 


        \subsubsection{Exhaustiveness analysis \PPlus and \VMinus}
        \label{typingppandvm}

        Exhaustiveness analysis can help restore Nice Property 5: with it,
        the compiler can warn programmers of a missing or extraneous match
        condition in a \it{case} expression, and potentially an \it{if-fi}.
        Owing to its significantly more flexible structure, however, \it{if-fi}
        may prove trickier to analyze for missing match conditions than
        \it{case}.

        % \subsubsection{Type-agnostic decision-making: the $\alpha$}
        % \label{alphas}

        % The three languages look similar: they each have value constructors and
        % a 'decision-making construct' to deal with constructed data. In \PPlus, the
        % construct is pattern-matching; in \VMinus, it is the guarded expression; in \D,
        % it is the decision tree. 

        % Because of this, it might be possible to make all three languages
        % \it{higher order} in right-hand sides; that is, to parameterize the
        % expressions that occur \it{after} all decision-making. Imagine an
        % abstract expression $\alpha$ that appears on the right-hand side of a
        % \it{case} branch, the right-hand side of a guarded expression, or in a
        % \it{match} node. The abstract syntax of the new \it{case}, \it{if-fi},
        % and decision tree might look like this: 
        % \begin{center}
        %     \begin{bnf}
        %         $\VMinus_{\alpha}$ : \VMinus with $\alpha$ ::=
        %         $\mathit{if}\; \mathit{[}\; g\; \bracketed{[] g}\; \mathit{]}\; \mathit{fi}$ : if-fi with $\alpha$
        %         ;;
        %         $G_{\alpha}$ : Guarded Expressions with $\alpha$ ::=
        %         $[\vexists{\bracketed{x}}] \bracketed{g} \boldsymbol{\rightarrow}\alpha$ : 
        %         ;;
        %         $\PPlus_{\alpha}$ : \PPlus with $\alpha$::=
        %         $\tt{case}\; \expr\; \bracketed{p\; \ttrightarrow\; \alpha}$ : \it{case} expression with $\alpha$ 
        %         ;;
        %         $t_{\alpha}$ : Decision Trees with $\alpha$ ::= 
        %         | \dots : other forms of tree 
        %         | $\alpha$ : match node 
        %     \end{bnf}
        % \end{center}

        % Why would one want to do this? Well, recall that \VMinus had to be
        % stripped of multiple results and other \VC-like constructs in order to
        % retain its desirable efficiency properties. $\alpha$ lets \VMinus and
        % the other languages do efficient decision-making without worrying about
        % the form of the result. If the result is a complex multi-value or a
        % computation that involves backtracking, it will be agnostic of the
        % decision-making. $\alpha$ makes right-hand sides polymorphic and
        % abstract, so a programmer could potentially insert expressions from \VC
        % in their place and know that the decision-making before the $\alpha$
        % will still be efficient. This would allow for fuller interoperability
        % between \VC and \VMinus. Section~\ref{vminusandvc} further describes
        % why bridging the gap between the two languages might be a worthwhile
        % exercise. 

        % However, the language designer must take special care to ensure no
        % $\alpha$ finds its way into the decision-making itself, or the whole
        % idea falls apart. I~am developing an implementation that enforces this
        % invariant, and may include it in a future publication. 
        
        % Other alpha material: 
% The decision-making construct that gets us there. Whether
% it's a single value (ML-style) a sequence of values (Verse-style), or
% even something else, the $\alpha$ represents \it{any} ultimate result of
% "making a decision," and it's the ways in which we make decisions that
% we truly care about examining. By making the return result both
% polymorphic and abstract, we eschew the need to worry about its type and
% compatibility with other results of otherwise-equivalent trees. 

% An expression in core Verse evaluates to produce possibly-empty sequence of
% values. In \VMinus, values depend on the form of abstract expression $\alpha.$
% If $\alpha$ is a Verse-like expression, \valpha\ will be a value sequence. If it
% is an ML-like expression, it will be a single value. 
            
%         A guarded expression evaluates to produce a \bf{result}. A result is either
%         a metavalue \valpha\ or reject. 
            
%         \[\it{r}\; \rm{::=}\; \vartheta\; \vbar \; \reject \]
            
%         \showvjudgement{Eval-Alpha}{\veval{\alpha}{\valpha}}

% Of note in both \VMinus and \D is that the 'decision-making construct'
% is annotated with an $\alpha$. This annotation gives us type flexibility on the
% right-hand side of the \it{terminating} case for each construct
% (\tt{$\rightarrow$ exp} in \VMinus and the match node in \D.) 


        % \subsubsection{Or-patterns and pattern guards}
        % \label{pplusindependently}
        % \PPlus has side conditions, guards, and or-patterns. No major
        % functional language has all three of these extensions. Back
        % Section~\ref{extensions}'s examples, I~had to switch from OCaml to
        % Haskell to use guards, and back to OCaml for or-patterns. The two
        % extensions are mutually exclusive in Haskell, OCaml, Scala,
        % Erlang/Elixir, Rust, F\#, Agda.~\citep{haskell, ocaml, scala, erlang,
        % elixir, rust, fsharp, agda}

        % Why is this the case? In my research, I~have yet to encounter a
        % substantial justification for this. I~have several theories: one,
        % reengineering the Haskell parser to integrate or-patterns into the
        % language may be considered too great an effort; two, the lesser
        % popularity of functional programming in comparison to other paradigms
        % has meant there are not enough voices in any one language's community
        % claiming that theirs needs all known extensions to pattern matching;
        % three, Haskell, the only language with guards, is lazy, and lazy
        % semantics may not operate well with or-patterns. Future work may explore
        % if a lazy semantics of \PPlus is possible; such a study may answer this
        % question. 

        \subsubsection{Using \VMinus to inform programming in Verse}
        \label{vminusandvc}
        
        % At ICFP last year, Tim Sweeney said that he wanted Verse to be an
        % accessible programming language to write a scalable, collaborative
        % metaverse~\citep{timtalk}. Can \VMinus be aid in this goal? I~can imagine
        % two ways in which it might:
        
        % \begin{enumerate}
        %     \item \VMinus could be  a tool to help ease programmers who are
        %     more familiar with pattern matching into the realm of functional
        %     logic programming with equations. 
        %     \item Programs written in Verse using ideas from \VMinus might have
        %     a friendlier cost model (depending on the compiler)
        % \end{enumerate}
        
        % To point~1, \VMinus sits both syntactically and semantically in between
        % \PPlus and \VC, which might help a new programmer to Verse bridge the
        % conceptual gap between pattern matching and equations. Also, \PtoVTran,
        % the $\PPlus \rightarrow \VMinus$ translation, could help a programmer 
        % who wishes to write code using pattern matching see how their ideas 
        % can be expressed in Verse. 

        % To point~2, 
        
        \DTran\ and the proof that \DTran\ preserves semantics help show that
        certain computations that use equations for decision-making can be
        compiled to efficient code. A future project could be to extend \VMinus
        to include \it{all} of \VC, and use \DTran to eliminates as much
        backtracking at runtime as possible, falling back to the VC's fully
        general evaluation mechanism only when necessary. My hope is that, using
        these ideas, both the Verse programmer and language designer might make
        any discovery that allows them to increase the efficiency of full-Verse
        programs. 


\end{document}
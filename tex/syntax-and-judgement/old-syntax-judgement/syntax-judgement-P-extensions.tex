\documentclass[]{article}
\usepackage{vmlmacros}
\usepackage{syntax}
\usepackage{relsize}

% \setlength{\grammarparsep}{20pt plus 1pt minus 1pt} % increase separation between rules
\setlength{\grammarindent}{10em} % increase separation between LHS/RHS

\setlength{\parindent}{0cm}
\title{A Syntax of Or-patterns and side conditions in $P^{+}$}
\title{A Syntax of Or-patterns and side conditions in $P^{+}$}
\author{rab}
\date{October 21, 2023}
\begin{document}

\maketitle


We extend an example grammar of patterns within uML with 
or-patterns and side conditions: 

\bigskip

\begin{grammar}
    <case-expression> ::= \tt{(case} <expr> \{ <case-branch> \} \tt{)}
    <case-expression> ::= \tt{(case} <expr> \{ <case-branch> \} \tt{)}
    
    <case-branch> ::= \tt{(}<pattern> <expr>\tt{)}
    
    <pattern> ::= <value-variable-name>
    \alt <value-constructor-name>
    \alt \tt{(}<value-constructor-name> \{<pattern>\}\tt{)}
    \alt \tt{(}<pattern> \tt{when} <expr> \tt{)} 
    % <pattern> ::= (<name> \{ <name> | \tt{_} \})
    \alt \tt{(oneof} \{<pattern>\}\tt{)}
    \alt \tt{_}
    
\end{grammar}

\bigskip

\section{Side conditions with \tt{when}}

The \tt{when} keyword may optionally appear on the rightmost side of a \tt{case}
branch in \it{P}, within a set of parentheses also containing an expression.
If the scrutinee matches the pattern, the expression is evaluated. If it 
evaluates to produce a truthy value, the match succeeds and the right-hand
side expression is evaluated with the new $\rho'$ produced by the pattern.

\medskip

\bf{General concrete syntax of \tt{when}: }

\begin{verbatim}
    (case scrutinee
        [pattern (when condition) rhs-exp])
\end{verbatim}


Example: 
\begin{verbatim}
    (case v
        ['() 0]
        [(cons x xs) (when (= 0 (mod 2 x))) (+ 1 (count-evens xs)) ])
\end{verbatim}

% \bf{A question about types:}
% \medskip

% I had a blurb like this: 

\medskip

Note: the \tt{exp} in a \tt{when} is not limited to be a boolean expression, 
and there is no static type system to assert that it will evaluate to a boolean.
As in the rest of \it{P}, when an expression evaluates to \tt{\#f}, it
is considered falsey; otherwise, it is considered truthy. 

\medskip

% As I was writing this, I realized uML \it{does}, obviously, have a type system
% to do exactly this. At the same time, I remember you saying we won't have static
% types in our languages- which do you want to go off of? 

\section{Or-patterns with \tt{oneof}}

\medskip

---------


The \tt{oneof} keyword may optionally appear on the leftmost side of a \tt{case}
branch in \it{P}, within a set of parentheses also containing the set of 
patterns for that branch. The set of patterns $S$ is defined as such: if $S$ 
contains a pattern $p$ and the scrutinee matches $p$, that branch is evaluated
if the pattern-matching algorithm reaches it. 
When the match succeeds and the right-hand
side expression is evaluated with the new $\rho'$ produced by a pattern, only 
that pattern's fresh variables are introduced into $\rho'$. 

\medskip

General concrete syntax of \tt{oneof}: 


\begin{verbatim}
    (case scrutinee
        [(oneof pattern-1 pattern-2 ... pattern-k) rhs-exp])
\end{verbatim}


Example: 
\begin{verbatim}
    (case light
        [RED 'stop]
        [(oneof GREEN YELLOW) 'keep-on-goin])
\end{verbatim}



\medskip

Typed languages with or-patterns, like OCaml, often have the restriction that
logical variables introduced within a section of an or-pattern must represesent
values of the same type within all parts of the or-pattern. Because $P$ has no
static type system, we don't make this assertion: whichever pattern in the
or-pattern matches will introduce its variables and bindings into the $\rho'$
with which the right-hand side is evaluated. 

In addition, a fresh variable on the right-hand side of the or-pattern must
appear in ALL branches of the or-pattern. 

In V-, you can have defaults to your 'or-patterns' in a way you can't so much in
P+, i.\expr. x can be a literal at the end of a list of unmatched patterns (or we
can fail).

Example: 
\begin{verbatim}
    (case (list2 1 #f)
        ['() 0]
        [(oneof (cons 4 x) (cons x 3) (cons x #f)) x]) ;; returns #f
\end{verbatim}

% \bf{A question about or-patterns and types: }


% The ocaml description of or-patterns is as follows: 

% \medskip

% The pattern $pattern_1 \mid pattern_2$ represents the logical “or” of the two
% patterns $pattern_1$ and $pattern_2$. A value matches $pattern_1 \mid pattern_2$
% if it matches $pattern_1$ or $pattern_2$. The two sub-patterns $pattern_1$ and
% $pattern_2$ must bind exactly the same identifiers to values having the same
% types. Matching is performed from left to right. More precisely, in case some
% value v matches $pattern_1 \mid pattern_2$, the bindings performed are those of
% $pattern_1$ when v matches $pattern_1$. Otherwise, value v matches $pattern_2$
% whose bindings are performed.

% \medskip

% \bf{This is a restriction at the level of the type system.
% Again, do we want strict static types in $P$?}


\end{document}
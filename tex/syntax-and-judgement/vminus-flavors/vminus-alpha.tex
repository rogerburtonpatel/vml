\documentclass[]{article}
\usepackage{vmlmacros}
\usepackage{syntax}
\usepackage{relsize}
\usepackage{amsmath}
\usepackage{booktabs}
\usepackage{simplebnf}
\setcounter{secnumdepth}{1}

\DeclareMathOperator{\dom}{dom}



% \setlength{\grammarparsep}{20pt plus 1pt minus 1pt} % increase separation between rules
\setlength{\grammarindent}{10em} % increase separation between LHS/RHS
\setlength{\parindent}{0cm}
\title{Syntax and Semantics of \VMinus with $\alpha$}
\author{Roger Burtonpatel}
\begin{document}

\maketitle

\section{Syntax}

\subsection{\VMinus with $\alpha$}

We present a grammar of \VMinus, a variant of Verse with a defined
decision-making construct: 

\bigskip


\begin{center}
    \begin{bnf}
    $P$ : \textsf{Programs} ::=
    $\bracketed{d}$ : definition
    ;;
    $d$ : \textsf{Definitions} ::=
    | $\tt{val} \; x \; \tt{=} \; \ealpha$ : bind name to expression
    ;;
    $\ealpha$ : \textsf{Expressions} ::=
    | $\alpha$ : terminating alpha
    | $x, y, z$ : names
    % Question: ebnf braces vs. regular brackets
    | $\tt{if} \; \tt{[}\; \galpha \; \bracketed{[] \galpha} \;\tt{]} \; \tt{fi}$ : if-fi 
    | $K \bracketed{\ealpha}$ : value constructor application 
    | $\ealpha[1] \; \ealpha[2]$ : function application 
    ;;
    $\galpha$ : \textsf{Guarded Expressions} ::=  
    $\boldsymbol{\rightarrow}\ealpha$ : terminating experession
    | $\ealpha; \; \galpha$ : intermediate expression 
    | $\exists \bracketed{x} \tt{.} \galpha$ : existential 
    | $\ealpha[1] = \ealpha[2]; \; \galpha$ : equation 
    | $\lambda x. \; \ealpha$ : lambda 
    ;;
    $\v$ : Values ::= $K\bracketed{\v}$ : value constructor application 
    | $\lambda x. \; \ealpha$ : lambda 
    ;;
    $K$ : \textsf{Value Constructors} ::=
    | \tt{true} $\vert$ \tt{false} : booleans
    | $\tt{\#}x$ : name beginning with \tt{\#}
    | \tt{A-Z}$x$ : name beggining with capital letter
    | $[\tt{-}\vert\tt{+}](\tt{0}-\tt{9})+$ : signed integer literal 
    \end{bnf}
\end{center}

A \it{name} is any token that is not an integer literal, 
does not contain whitespace, a bracket, or parenthesis, 
and is not a value constructor name or a reserved word.
        
\section{Refinement ordering on environments}

\begin{align*}
\rho \subseteq \Rhoprime \text{ when }&\dom\rho  \subseteq \dom \Rhoprime\\
\text{ and } &\forall x \in \dom \rho: \rho(x) \subseteq \Rhoprime(x)
\end{align*}




\vfilbreak



\section{Forms of Judgement for $\VMinus$:}
\begin{tabular}{ll}
\toprule
    \multicolumn2{l}{\emph{Metavariables}} \\
\midrule
    $\v, \; \v'$& value \\
    $eq$& equation \\ 
    % $\tempstuck$& a temporarily-stuck equation \\
    $\reject$& equation rejection \\
    $\result$& $\v \mid$ \reject : a result of \v \; or
    rejection\\
    \Rho& environment: $name \rightarrow \mathcal{\v}_{\bot}$ \\
    $\rho\{ x \mapsto y \} $& environment extended with name $x$ mapping to $y$ \\
    $\mathcal{T}$& Context of all temporarily stuck equations (a sequence) \\ 
    $\ealpha$& An expression \\ 
    $\galpha$& A guarded expression \\
    % \uppsidown& Inability to compile to a decision tree; a compile time error \\
\bottomrule
\end{tabular}    

\bigskip

\begin{tabular}{ll}
    \toprule
        \multicolumn2{l}{\emph{Sequences}} \\
    \midrule
        $\emptyseq$& the empty sequence \\
        $S_1 \cdot S_2 $&  Concatenate sequence $S_1$ and sequence $S_2$ \\
        $x \cdot S_2 $& Cons $x$ onto sequence $S_2$ \\
    \bottomrule
    \end{tabular}    
    
    \medskip
    
    % \mkjudgementcmd{EquationSuccess}{\vtuple{\rho, eq}}{\rhohat REMOVE THIS}
    % \mkjudgementcmd{EquationTempStuck}{\vtuple{\rho, eq}}{\tempstuck}
    % \mkjudgementcmd{EquationReject}{\vtuple{\rho, eq}}{reject}

    % \showvjudgement{EquationSuccess}{\EquationSuccess}
    % \showvjudgement{EquationTempStuck}{\EquationTempStuck}
    % \showvjudgement{EquationReject}{\EquationReject}
    
    
    
    
    Success only refines the environment; that~is, when
    ${\vtuple{\rho, \ealpha}} \rightarrowtail{} {\Rhoprime}$, we~expect $\rho \subseteq \Rhoprime$.
    
    

    
    \subsection{Expressions}
    
    \newcommand\GNoTree{\vmrung \rightsquigarrow \uppsidown} 
    
    An expression in core Verse evaluates to produce possibly-empty sequence of
    values. In \VMinus, expressions evaluate to a single value. We will explore
    a Verse variant in which expressions may evaluate to multiple values. 

    

    A guarded expression evaluates to produce a \bf{result}. A result is either
    a value $\v$ or reject. 
    
    \[\it{r} \; \rm{::=} \; \v \;|\; \reject \]
    
    \showvjudgement{Eval}{\veval{\ealpha}{\v}}
    \showvjudgement{Eval-Guarded-Expr}{\vmrung[guard=\galpha]}
    
    % \bigskip
    % If a guarded expression cannot be evaluated without producing logical 
    % variables at runtime, it cannot be expressed as a decision tree. 
    % This notation indicates this failure (think of \uppsidown as a fallen 
    % tree), which results in a compile-time error. 
    % \showvjudgement{NoTree}{\GNoTree}
    
    \bigskip
\subsection{Equations}

In \VMinus, we solve equations (intermediate computations in a guarded
expression of the form \\$\ealpha[1] = \ealpha[2]$) in a similar way to the
authors of the original Verse paper: we pick one, attempt to solve for it, and
move on. 

\rab{How do we express that \VMinus semantics preclude backtracking/logical 
variables at runtime?}

Given an environment from names to metavalues {\v}s \Rho, an equation \eq
~will either refine the environment ($\Rhoprime$) or lead to failure. We use
the metavariable $\dagger$ to represent failure, and an equation failing will
cause the top-level guarded expression to evaluate to \reject. 

\showvjudgement{Eq-Refine}\eqrefine
\showvjudgement{Eq-Fail}\eqfail
    
    
    \section{Sequences}
    
    The trivial sequence is \emptyseq. Sequences can be concatenated with infix 
$\cdot$. In an appropriate context, a value like $x$ stands for 
the singleton sequence containing $x$. 

\begin{align*}
    \emptyseq \cdot \ys &\equiv \ys \\
    \ys \cdot \emptyseq &\equiv \ys \\
    (\xs \cdot \ys) \cdot \zs &\equiv \xs \cdot (\ys \cdot \zs)
\end{align*}

\section{Rules (Big-step Operational Semantics) for $\VMinus$:}
    
\subsection{Evaluating Guarded Expressions}


\subsubsection{Evaluating simple parts of guarded expressions}

\[
\inferrule*[Left=\textsc{ (Eval-ArrowExpr) }]
    {\vmrung[context=\emptyseq, guard=\ealpha]}
    {\vmrung[context=\emptyseq, guard=\rightarrow \ealpha,result=\v]}
\]

\[
\inferrule*[Left=\textsc{ (Eval-Exists) }]
    {\vmrung[env=\rho\{x \mapsto \bot \}]}
    {\vmrung[guard=\exists x.\; \galpha]}
\]

\[
\inferrule*[Left=\textsc{ (Eval-Expseq-Succ) }]
    {\inferrule* {}
    {
    \veval{\ealpha}{\v}
    \and
    \v \neq \tt{false}
    \and
    \veval{\galpha}{\result}
    }}
    {\vmrung[guard=\ealpha;\; \galpha]}
\]

\[
\inferrule*[Left=\textsc{ (Eval-Expseq-Fail) }]
    {\inferrule* {}
    {
    \veval{\ealpha}{\tt{false}}
    }}
    {\vmrung[guard=\ealpha;\; \galpha, result=\reject]}
\]


\subsubsection{Shifting an equation to the context}
\[
\inferrule*[Left=\textsc{ (G-Move-To-Ctx) }]
    {\inferrule*{\vmrung[context=eq \cdot \context]}
    {}
    }
    {\vmrung[guard=eq; \; \galpha]}
\]

\subsubsection{Choosing and solving an equation}

\[
\inferrule*[Left=\textsc{ (Eval-Eq-Refine) }]
    {\eqrefine
    \and
    \vmrung[env=\Rhoprime,context=\TT]}
    {\vmrung[context=\TeqT{\eq},result=\v]}
\]

\[
\inferrule*[Left=\textsc{ (Eval-Eq-Fail) }]
    {\eqfail}
    {\vmrung[context=\TeqT{\eq},result=\reject]}
\]
\subsubsection{Properties of equations}

\[
\inferrule*[Left=\textsc{ (Single-Eq-Assoc) }]
    {\vmrung[context=\TeqT{\ealpha[2] = \ealpha[1]}]}
    {\vmrung[context=\TeqT{\ealpha[1] = \ealpha[2]}]}
\]

\[
\inferrule*[Left=\textsc{ (Multi-Eq-Assoc) }]
    {\vmrung[context=\TeqT{\ealpha[2] = \ealpha[1] \cdot \ealpha[1] = \ealpha[2]}]}
    {\vmrung[context=\TeqT{\ealpha[1] = \ealpha[2] \cdot \ealpha[2] = \ealpha[1]}]}
\]

\subsubsection{Desugaring of Complex Equations}
\[
\inferrule*[Left=\textsc{ (Desugar-EqExps) }]
    {\inferrule* {}
    {
    x,\;y \; \rm{are distinct and fresh}
    \\\\
    \vmrung[envext=\bracketed{x \mapsto \bot,\; y \mapsto \bot},
          context=\eqTT{x = \ealpha[1] \cdot y = \ealpha[2] \cdot x = y}]}}
    {\vmrung[context=\TeqT{\ealpha[1] = \ealpha[2]}]}
\]

\[
\inferrule*[Left=\textsc{ (Desugar-Vcon-Multi) }]
    {
    \vmrung[context=\TeqT{\lbrack \ealpha[i]=\ealpha[i]' \; 
           \vert \; 1 \leq i \leq n \rbrack}]}
    {\vmrung[context=\TeqT{K(\ealpha[1], \dots 
            \ealpha[n]) = K(\ealpha[1]', \dots \ealpha[n]')}]}
\]

\subsubsection{Refinement with different types of equations}

\[
\inferrule*[Left=\textsc{ (G-NameExp-Bot) }]
    {\inferrule* {}
    {
    x \in \dom \rho
    \\\\
    \veval{\ealpha}{\v}
    \\\\
    \rho(x) = \bot
    }}
    {\eqrefine[eq={x = \ealpha},newenv=\Rho\bracketed{x \mapsto \v}]}
\]

\[
\inferrule*[Left=\textsc{ (G-NameExp-Eq) }]
    {\inferrule* {}
    {
    x \in \dom \rho
    \\\\
    \veval{\ealpha}{\v}
    \\\\
    \rho(x) = \v
    }}
    {\eqrefine[eq={x = \ealpha},newenv=\Rho\bracketed{x \mapsto \v}]}
\]

\[
\inferrule*[Left=\textsc{ (G-NameExp-Fail) }]
    {\inferrule* {}
    {
    x \in \dom \rho
    \\\\
    \veval{\ealpha}{\v}
    \\\\ 
    \rho(x) = \v'
    \\\\
    \v \neq \v'
    }}
    {\eqrefine[eq={x = \ealpha},newenv=\dagger]}
\]

\[
\inferrule*[Left=\textsc{ (G-NameNotFound-Fail) }]
    {\inferrule* {}
    {
    x \not\in \dom \rho
    }}
    {\eqrefine[eq={x = \ealpha},newenv=\dagger]}
\]

\[
\inferrule*[Left=\textsc{ (G-EqNames-Vals-Succ) }]
    {\inferrule* {}
    {
    x,\;y \in \dom \rho
    \\\\
    \rho(x) = \v, \; \rho(y) = \v
    }}
    {\eqrefine[eq={x = y},newenv=\rho]}
\]

\[
\inferrule*[Left=\textsc{ (G-EqNames-Vals-Fail) }]
    {\inferrule* {}
    {
    x,\;y \in \dom \rho
    \\\\
    \rho(x) = \v, \; \rho(y) = \v'
    \\\\
    \v \neq \v'}}
    {\eqrefine[eq={x = y},newenv=\dagger]}
\]

% \[
% \inferrule*[Left=\textsc{ (G-EqNames-Bots-Fail) }]
%     {\inferrule* {}
%     {
%     x,\;y \in \dom \rho
%     \\\\
%     \rho(x) = \bot, \; \rho(y) = \bot
%     \\\\
%     x,\;y \; \rm{do not appear in} \; \context, \; \context'}}
%     {\vmrung[context=\TeqT{x = y}] 
%     \rightsquigarrow \uppsidown}
% \]

\[
\inferrule*[Left=\textsc{ (G-EqNames-BotVal-Succ) }]
    {\inferrule* {}
    {
    x,\;y \in \dom \rho
    \\\\
    \rho(x) = \bot, \; \rho(y) = \v
    }}
    {\eqrefine[eq={x = y},newenv=\rho\bracketed{x \mapsto \v}]}
\]

\[
\inferrule*[Left=\textsc{ (G-Vcon-Single-Fail) }]
    {K \neq K'}
    {\eqrefine[eq={K = K'}, newenv=\dagger]}
\]

\[
\inferrule*[Left=\textsc{ (G-Vcon-Single-Succ) }]
    {\ }
    {\eqrefine[eq={K = K}, newenv=\Rho]}
\]


\[
\inferrule*[Left=\textsc{ (G-Vcon-Multi-Fail) }]
    {K \neq K'}
    {\eqrefine[eq={K(\ealpha[1], \dots 
            \ealpha[n]) = K'(\ealpha[1]', \dots \ealpha[n]')},
            newenv=\dagger]}
\]

\[
\inferrule*[Left=\textsc{ (G-Vcon-Multi-Arity-Fail) }]
    {n \neq m}
    {\eqrefine[eq={K(\ealpha[1], \dots 
            \ealpha[n]) = K(\ealpha[1]', \dots \ealpha[m]')},
            newenv=\dagger]}
\]


\subsection{Evaluating General Expressions}


% \[
% \inferrule*[Left=\textsc{ (If-Fi-Fail) }]
%     {\ }
%     {\veval{\iffi{\ }}{\emptyseq}}
% \]

\[
\inferrule*[Left=\textsc{ (If-Fi-Success) }]
    {\vmrung[] \Downarrow \v}
    {\veval{\iffi{\galpha \; \square \; \dots}}{\v}}
\]

\[
\inferrule*[Left=\textsc{ (If-Fi-Reject) }]
    {\inferrule* {}
    {\vmrung[result=\reject]}
    \and 
    \veval{\iffi{\dots}}{\v}}
    {\veval{\iffi{\galpha \; \square \; \dots }}{\v}}
\]

\[
\inferrule*[Left=\textsc{ (Vcon-Empty) }]
    {\ }
    {\veval{K}{K}}
\]

\[
\inferrule*[Left=\textsc{ (Vcon-Multi) }]
    {\inferrule* {}
    {
    \veval{\ealpha[i]}{v_{i}}
    \and 
    1 \leq i \leq n
    }}
    {\veval{K(\ealpha[1], \dots \ealpha[n])}{K(v_{1}, 
    \dots v_{i})}}
\]

\[
\inferrule*[Left=\textsc{ (Eval-Name) }]
    {x \in \dom \rho 
    \\\\
    \rho(x) = \v}
    {\veval{x}{\v}}
\]

\[
\inferrule*[Left=\textsc{ (Eval-Funapp) }]
    {\veval{\ealpha[1]}{\lambda x.\ealpha}
    \\\\
    \veval{\ealpha[2]}{\v'}
    \\\\
    \vevalext{\bracketed{x \mapsto \v'}}{\ealpha}{\v}}
    {\veval{\ealpha[1]\; \ealpha[2]}{\v}}
\]

\end{document}

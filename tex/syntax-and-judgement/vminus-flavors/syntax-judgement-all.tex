\documentclass[]{article}
\usepackage{vmlmacros}
\usepackage{syntax}
\usepackage{relsize}
% \usepackage{palatino} % I don't love this, to be honest. 
\usepackage{amsmath}
\usepackage{booktabs}
\usepackage{simplebnf}
\setcounter{secnumdepth}{1}

\DeclareMathOperator{\dom}{dom}



% \setlength{\grammarparsep}{20pt plus 1pt minus 1pt} % increase separation between rules
\setlength{\grammarindent}{10em} % increase separation between LHS/RHS
\setlength{\parindent}{0cm}
\title{Syntax and Semantics of \VMinus with \alpha}
\author{Roger Burtonpatel}
\date{December 3, 2023}
\begin{document}

\maketitle

\section{Syntax}

\subsection{The Core Language}

We present a grammar of a base language, with no decision-making constructs: 

\bigskip

% I attempted to use the grammar environment you provided. But there was either
% something missing or something I overlooked in the example code and it would
% not compile, despite many reducing changes I made. So I went with the
% simplebnf package, which I quite like.  
\begin{center}
    \begin{bnf}
    $P$ : \textsf{Programs} ::=
    $\bracketed{d}$ : definition
    ;;
    $d$ : \textsf{Definitions} ::=
    | $\tt{val} \; x \; \tt{=} \; \expr$ : bind name to expression
    ;;
    $\expr$ : Core expressions ::= 
    | $x, y, z$ : names
    | $K\bracketed{\expr}$ : value constructor application 
    | $\tt{if} \; \expr[1] \; \tt{then} \; \expr[2] \; \tt{else} \; e_{3} $ : if
    | $\lambda x. \; \expr$ : lambda 
    | $\expr[1] \; \expr[2]$ : function application 
    ;;
    $v$ : Values ::= $K\bracketed{v}$ : value constructor application 
    ;;
    $K$ : \textsf{Value Constructors} ::=
    % \cons : cons 
    % | \tt{[]} : empty list 
    | \tt{true} $\vert$ \tt{false} : booleans
    | $\tt{\#}x$ : name beginning with \tt{\#}
    | \tt{A-Z}$x$ : name beggining with capital letter
    | $[\tt{-}\vert\tt{+}](\tt{0}-\tt{9})+$ : signed integer literal 

    \end{bnf}
\end{center}


A \it{name} is any token that is not an integer literal, 
does not contain whitespace, a bracket, or parenthesis, 
and is not a value constructor name or a reserved word.


We then present three language extensions that build off of this core: 
\PPlus, the language of patterns, \VMinus, the language of 
verse-like equations, and $D$, the language of decision trees. 

\subsection{Three Language Extensions}

\subsubsection{\VMinus:}

\begin{center}
    \begin{bnf}
    $\ealpha$ : \textsf{$\alpha$-Expressions} ::=
    | $\alpha$ : terminating alpha
    | $x, y, z$ : names
    % Question: ebnf braces vs. 
    | $\tt{if} \; \tt{[}\; \galpha \; \bracketed{[] \galpha} \;\tt{]} \; \tt{fi}$ : if-fi 
    | $K \bracketed{\ealpha}$ : value constructor application 
    | $\ealpha[1] \; \ealpha[2]$ : function application 
    ;;
    $\galpha$ : \textsf{Guarded Expressions} ::=  
    $\boldsymbol{\rightarrow}\ealpha$ : terminating $\ealpha$ 
    | $\ealpha; \; \galpha$ : intermediate expression 
    | $\exists \bracketed{x} \tt{.} \galpha$ : existential 
    | $\ealpha[1] = \ealpha[2]; \; \galpha$ : equation 
    % ;;
    \end{bnf}
\end{center}

\bigskip 

\subsubsection{\PPlus:}
\begin{center}
    \begin{bnf}
$\expr$ : \textsf{Expressions} ::=
    | $\ttbraced{\tt{case} \; \expr \; \bracketed{\tt{[} p \; \expr \tt{]}}}$ : case expression 
    ;;
    $p$ : \textsf{Patterns} ::= $x$ : name 
    | $K$ : value constructor 
    | $\ttbraced{K \; \bracketed{p}}$ : value constructor application 
    % | $\ttbraced{p \; \tt{when} \; \expr }$ : side condition
    | $\ttbraced{\tt{oneof} \; p_{1} \;, p_{2} }$ : or-pattern 
    | $\ttbraced{p \tt{;} \bracketed{\expr \vert \ttbraced{p  <- \expr}}}$ : pattern guard
    \end{bnf}
\end{center}


\bigskip 

\subsubsection{$D$:}

\begin{center}
    \begin{bnf}
        \Dalpha : \textsf{Decision Tree} ::= 
        $\tt{case} \; x \; \tt{of} \; 
        \bracketed{\vert \; K\bracketed{x} \; \tt{=>} \; \Dalpha}
        [\vert \; x \; \tt{=>} \Dalpha]$ : test node 
        | $\alpha$ : match node 
        | $\tt{if} \; x \; \tt{then} \; \Dalpha \; \tt{else} \; \Dalpha$ : condition with two children 
        | $\tt{let} \; x \; \tt{=} \; \expr \; \tt{in} \; \Dalpha$ : let-bind a name
        ;;
        $\expr$ : \textsf{Expressions} ::=
        | $\mathcal{D}_{\expr}$ : decision trees 
    \end{bnf}
\end{center}

        
\section{Refinement ordering on environments}

\begin{align*}
\rho \subseteq \Rhoprime \text{ when }&\dom\rho  \subseteq \dom \Rhoprime\\
\text{ and } &\forall x \in \dom \rho: \rho(x) \subseteq \Rhoprime(x)
\end{align*}



\vfilbreak



\section{Forms of Judgement for $V^{-}$:}
\begin{tabular}{ll}
\toprule
    \multicolumn2{l}{\emph{Metavariables}} \\
\midrule
    % $v, \; v'$& value \\
    \valpha& a metavalue produced from evaluating $\alpha$. \\
    $eq$& equation \\ 
    % $\tempstuck$& a temporarily-stuck equation \\
    $\reject$& equation rejection \\
    $\result$& $\vartheta \mid$ \reject : a result of \valpha \; or
    rejection\\
    \Rho& environment: $name \rightarrow \mathcal{V}_{\bot}$ \\
    $\rho\{ x \mapsto y \} $& environment extended with name $x$ mapping to $y$ \\
    $\mathcal{T}$& Context of all temporarily stuck equations (a sequence) \\ 
    $\ealpha$& An expression \\ 
    $\galpha$& A guarded expression \\
    % \uppsidown& Inability to compile to a decision tree; a compile time error \\
\bottomrule
\end{tabular}    

\bigskip

\begin{tabular}{ll}
    \toprule
        \multicolumn2{l}{\emph{Sequences}} \\
    \midrule
        $\emptyseq$& the empty sequence \\
        $S_1 \cdot S_2 $&  Concatenate sequence $S_1$ and sequence $S_2$ \\
        $x \cdot S_2 $& Cons $x$ onto sequence $S_2$ \\
    \bottomrule
    \end{tabular}    
    
    \medskip
    
    % \mkjudgementcmd{EquationSuccess}{\vtuple{\rho, eq}}{\rhohat REMOVE THIS}
    % \mkjudgementcmd{EquationTempStuck}{\vtuple{\rho, eq}}{\tempstuck}
    % \mkjudgementcmd{EquationReject}{\vtuple{\rho, eq}}{reject}

    % \showvjudgement{EquationSuccess}{\EquationSuccess}
    % \showvjudgement{EquationTempStuck}{\EquationTempStuck}
    % \showvjudgement{EquationReject}{\EquationReject}
    
    
    
    
    % Success only refines the environment; that~is, when
    % ${\vtuple{\rho, \expr}} \rightarrowtail{} {\Rhoprime}$, we~expect $\rho \subseteq \Rhoprime$.
    
    

    
    \subsection{Expressions}
    
    \newcommand\GNoTree{\vmrun \rightsquigarrow \uppsidown} An expression in
    core Verse evaluates to produce possibly-empty sequence of values. In
    \VMinus, values depend on the form of abstract expression $\alpha.$ If
    $\alpha$ is a Verse-like expression, \valpha ~will be a value sequence. If
    it is an ML-like expression, it will be a single value. 
    
    A guarded expression evaluates to produce a \bf{result}. A result is either
    a metavalue \valpha ~or reject. 
    
    \[\it{r} \; \rm{::=} \; \vartheta \;|\; \reject \]
    
    \showvjudgement{Eval-Alpha}{\veval{\alpha}{\valpha}}
    \showvjudgement{Eval-AlphaExpr}{\veval{\ealpha}{\valpha}}
    \showvjudgement{Eval-Guarded-Expr}{\vmrun}
    
    % \bigskip
    % If a guarded expression cannot be evaluated without producing logical 
    % variables at runtime, it cannot be expressed as a decision tree. 
    % This notation indicates this failure (think of \uppsidown as a fallen 
    % tree), which results in a compile-time error. 
    % \showvjudgement{NoTree}{\GNoTree}
    
    \bigskip
\subsection{Equations}

In \VMinus, we solve equations (intermediate computations in a guarded
expression of the form \\$\ealpha[1] = \ealpha[2]$) in a similar way to the
authors of the original Verse paper: we pick one, attempt to solve for it, and
move on. 

\rab{How do we express that \VMinus semantics preclude backtracking/logical 
variables at runtime?}

Given an environment from names to metavalues {\valpha}s \Rho, an equation \eq
~will either refine the environment ($\Rhoprime$) or lead to failure. We use
the metavariable $\dagger$ to represent failure, and an equation failing will
cause the top-level guarded expression to evaluate to \reject. 

\showvjudgement{Eq-Refine}\eqrefine
\showvjudgement{Eq-Fail}\eqfail
    
    
    \section{Sequences}
    
    The trivial sequence is \emptyseq. Sequences can be concatenated with infix 
$\cdot$. In an appropriate context, a value like $x$ stands for 
the singleton sequence containing $x$. 

\begin{align*}
    \emptyseq \cdot \ys &\equiv \ys \\
    \ys \cdot \emptyseq &\equiv \ys \\
    (\xs \cdot \ys) \cdot \zs &\equiv \xs \cdot (\ys \cdot \zs)
\end{align*}

\section{Rules (Big-step Operational Semantics) for $V^{-}$:}
    
\subsection{Evaluating Guarded Expressions}


\subsubsection{Evaluating simple parts of guarded expressions}

\[
\inferrule*[Left=\textsc{ (Eval-ArrowExpr) }]
    {\vmrun[context=\emptyseq,term=\alpha,result=\vartheta]}
    {\vmrun[context=\emptyseq, term=\rightarrow \alpha,result=\vartheta]}
\]

\[
\inferrule*[Left=\textsc{ (Eval-Exists) }]
    {\vmrun[env=\rho\{x \mapsto \bot \}]}
    {\vmrun[term=\exists x.\; \galpha]}
\]

\[
\inferrule*[Left=\textsc{ (Eval-Expseq-Succ) }]
    {\inferrule* {}
    {
    \veval{\ealpha}{\valpha}
    \and
    \valpha \neq \tt{false}
    \and
    \veval{\guard}{\result}
    }}
    {\vmrun[term=\ealpha;\; \galpha]}
\]

\[
\inferrule*[Left=\textsc{ (Eval-Expseq-Fail) }]
    {\inferrule* {}
    {
    \veval{\ealpha}{\tt{false}}
    }}
    {\vmrun[term=\ealpha;\; \galpha, result=\reject]}
\]

\subsubsection{Shifting an equation to the context}
\[
\inferrule*[Left=\textsc{ (G-Move-To-Ctx) }]
    {\inferrule*{\vmrun[context=eq \cdot \context]}
    {}
    }
    {\vmrun[term=eq; \; \galpha]}
\]

\subsubsection{Choosing and solving an equation}

\[
\inferrule*[Left=\textsc{ (Eval-Eq-Refine) }]
    {\eqrefine
    \and
    \vmrun[env=\Rhoprime,context=\TT]}
    {\vmrun[context=\TeqT{\eq},result=\vartheta]}
\]

\[
\inferrule*[Left=\textsc{ (Eval-Eq-Fail) }]
    {\eqfail}
    {\vmrun[context=\TeqT{\eq},result=\reject]}
\]
\subsubsection{Properties of equations}

\[
\inferrule*[Left=\textsc{ (Single-Eq-Assoc) }]
    {\vmrun[context=\TeqT{\ealpha[2] = \ealpha[1]}]}
    {\vmrun[context=\TeqT{\ealpha[1] = \ealpha[2]}]}
\]

\[
\inferrule*[Left=\textsc{ (Multi-Eq-Assoc) }]
    {\vmrun[context=\TeqT{\ealpha[2] = \ealpha[1] \cdot \ealpha[1] = \ealpha[2]}]}
    {\vmrun[context=\TeqT{\ealpha[1] = \ealpha[2] \cdot \ealpha[2] = \ealpha[1]}]}
\]

\subsubsection{Desugaring of Complex Equations}
\[
\inferrule*[Left=\textsc{ (Desugar-EqExps) }]
    {\inferrule* {}
    {
    x,\;y \; \rm{are distinct and fresh}
    \\\\
    \vmrun[envext=\bracketed{x \mapsto \bot,\; y \mapsto \bot},
          context=\eqTT{x = \ealpha[1] \cdot y = \ealpha[2] \cdot x = y}]}}
    {\vmrun[context=\TeqT{\ealpha[1] = \ealpha[2]}]}
\]

\[
\inferrule*[Left=\textsc{ (Desugar-Vcon-Multi) }]
    {
    \vmrun[context=\TeqT{\lbrack \ealpha[i]=\ealpha[i]' \; 
           \vert \; 1 \leq i \leq n \rbrack}]}
    {\vmrun[context=\TeqT{K(\ealpha[1], \dots 
            \ealpha[n]) = K(\ealpha[1]', \dots \ealpha[n]')}]}
\]

\subsubsection{Refinement with different types of equations}

\[
\inferrule*[Left=\textsc{ (G-NameExp-Bot) }]
    {\inferrule* {}
    {
    x \in \dom \rho
    \\\\
    \veval{\ealpha}{\valpha}
    \\\\
    \rho(x) = \bot
    }}
    {\eqrefine[eq={x = \ealpha},newenv=\Rho\bracketed{x \mapsto \valpha}]}
\]

\[
\inferrule*[Left=\textsc{ (G-NameExp-Eq) }]
    {\inferrule* {}
    {
    x \in \dom \rho
    \\\\
    \veval{\ealpha}{\valpha}
    \\\\
    \rho(x) = \valpha
    }}
    {\eqrefine[eq={x = \ealpha},newenv=\Rho\bracketed{x \mapsto \valpha}]}
\]

\[
\inferrule*[Left=\textsc{ (G-NameExp-Fail) }]
    {\inferrule* {}
    {
    x \in \dom \rho
    \\\\
    \veval{\ealpha}{\valpha}
    \\\\ 
    \rho(x) = \valpha'
    \\\\
    \valpha \neq \valpha'
    }}
    {\eqrefine[eq={x = \ealpha},newenv=\dagger]}
\]

\[
\inferrule*[Left=\textsc{ (G-NameNotFound-Fail) }]
    {\inferrule* {}
    {
    x \not\in \dom \rho
    }}
    {\eqrefine[eq={x = \ealpha},newenv=\dagger]}
\]

\[
\inferrule*[Left=\textsc{ (G-EqNames-Vals-Succ) }]
    {\inferrule* {}
    {
    x,\;y \in \dom \rho
    \\\\
    \rho(x) = \valpha, \; \rho(y) = \valpha
    }}
    {\eqrefine[eq={x = y},newenv=\rho]}
\]

\[
\inferrule*[Left=\textsc{ (G-EqNames-Vals-Fail) }]
    {\inferrule* {}
    {
    x,\;y \in \dom \rho
    \\\\
    \rho(x) = \valpha, \; \rho(y) = \valpha'
    \\\\
    \valpha \neq \valpha'}}
    {\eqrefine[eq={x = y},newenv=\dagger]}
\]

% \[
% \inferrule*[Left=\textsc{ (G-EqNames-Bots-Fail) }]
%     {\inferrule* {}
%     {
%     x,\;y \in \dom \rho
%     \\\\
%     \rho(x) = \bot, \; \rho(y) = \bot
%     \\\\
%     x,\;y \; \rm{do not appear in} \; \context, \; \context'}}
%     {\vmrun[context=\TeqT{x = y}] 
%     \rightsquigarrow \uppsidown}
% \]

\[
\inferrule*[Left=\textsc{ (G-EqNames-BotVal-Succ) }]
    {\inferrule* {}
    {
    x,\;y \in \dom \rho
    \\\\
    \rho(x) = \bot, \; \rho(y) = \valpha
    }}
    {\eqrefine[eq={x = y},newenv=\rho\bracketed{x \mapsto \valpha}]}
\]

\[
\inferrule*[Left=\textsc{ (G-Vcon-Single-Fail) }]
    {K \neq K'}
    {\eqrefine[eq={K = K'}, newenv=\dagger]}
\]

\[
\inferrule*[Left=\textsc{ (G-Vcon-Single-Succ) }]
    {\ }
    {\eqrefine[eq={K = K}, newenv=\Rho]}
\]


\[
\inferrule*[Left=\textsc{ (G-Vcon-Multi-Fail) }]
    {K \neq K'}
    {\eqrefine[eq={K(\ealpha[1], \dots 
            \ealpha[n]) = K'(\ealpha[1]', \dots \ealpha[n]')},
            newenv=\dagger]}
\]

\[
\inferrule*[Left=\textsc{ (G-Vcon-Multi-Arity-Fail) }]
    {n \neq m}
    {\eqrefine[eq={K(\ealpha[1], \dots 
            \ealpha[n]) = K(\ealpha[1]', \dots \ealpha[m]')},
            newenv=\dagger]}
\]


\subsection{Evaluating General Expressions}


% \[
% \inferrule*[Left=\textsc{ (If-Fi-Fail) }]
%     {\ }
%     {\veval{\iffi{\ }}{\emptyseq}}
% \]

\[
\inferrule*[Left=\textsc{ (If-Fi-Success) }]
    {\vmrun[] \Downarrow \vartheta}
    {\veval{\iffi{\galpha \; \square \; \dots}}{\vartheta}}
\]

\[
\inferrule*[Left=\textsc{ (If-Fi-Reject) }]
    {\inferrule* {}
    {\vmrun[result=\reject]}
    \and 
    \veval{\iffi{\dots}}{\vartheta}}
    {\veval{\iffi{\galpha \; \square \; \dots }}{\vartheta}}
\]

\[
\inferrule*[Left=\textsc{ (Vcon-Empty) }]
    {\ }
    {\veval{K}{K}}
\]

\[
\inferrule*[Left=\textsc{ (Vcon-Multi) }]
    {\inferrule* {}
    {
    \veval{\ealpha[i]}{\vartheta_{i}}
    \and 
    1 \leq i \leq n
    }}
    {\veval{K(\ealpha[1], \dots \ealpha[n])}{K(\vartheta_{1}, 
    \dots \vartheta_{i})}}
\]

\[
\inferrule*[Left=\textsc{ (Eval-Name) }]
    {x \in \dom \rho 
    \\\\
    \rho(x) = \valpha}
    {\veval{x}{\valpha}}
\]

\[
\inferrule*[Left=\textsc{ (Eval-Funapp) }]
    {\veval{\ealpha[1]}{\lambda x.\ealpha}
    \\\\
    \veval{\ealpha[2]}{\valpha'}
    \\\\
    \vevalext{\bracketed{x \mapsto \valpha'}}{\ealpha}{\valpha}}
    {\veval{\ealpha[1]\; \ealpha[2]}{\valpha}}
\]

\section{Forms of Judgement for \PPlus}
\begin{tabular}{ll}
    \toprule
        \multicolumn2{l}{\emph{Metavariables}} \\
    \midrule
        % $v, \; v'$& value \\
        \valpha& a metavalue produced from evaluating $\alpha$. \\
        $eq$& equation \\ 
        % $\tempstuck$& a temporarily-stuck equation \\
        $\reject$& equation rejection \\
        $\result$& $\vartheta \mid$ \reject : a result of \valpha \; or
        rejection\\
        \Rho& environment: $name \rightarrow \mathcal{V}_{\bot}$ \\
        $\rho\{ x \mapsto y \} $& environment extended with name $x$ mapping to $y$ \\
        $\mathcal{T}$& Context of all temporarily stuck equations (a sequence) \\ 
        $\ealpha$& An expression \\ 
        $\galpha$& A guarded expression \\
        % \uppsidown& Inability to compile to a decision tree; a compile time error \\
    \bottomrule
    \end{tabular}    
    
    \bigskip
    
    \begin{tabular}{ll}
        \toprule
            \multicolumn2{l}{\emph{Sequences}} \\
        \midrule
            $\emptyseq$& the empty sequence \\
            $S_1 \cdot S_2 $&  Concatenate sequence $S_1$ and sequence $S_2$ \\
            $x \cdot S_2 $& Cons $x$ onto sequence $S_2$ \\
        \bottomrule
        \end{tabular}    
        
        \medskip
        
        % \mkjudgementcmd{EquationSuccess}{\vtuple{\rho, eq}}{\rhohat REMOVE THIS}
        % \mkjudgementcmd{EquationTempStuck}{\vtuple{\rho, eq}}{\tempstuck}
        % \mkjudgementcmd{EquationReject}{\vtuple{\rho, eq}}{reject}
    
        % \showvjudgement{EquationSuccess}{\EquationSuccess}
        % \showvjudgement{EquationTempStuck}{\EquationTempStuck}
        % \showvjudgement{EquationReject}{\EquationReject}
        
        
        
        
        % Success only refines the environment; that~is, when
        % ${\vtuple{\rho, \expr}} \rightarrowtail{} {\Rhoprime}$, we~expect $\rho \subseteq \Rhoprime$.
        
        
    
        
        \subsection{Expressions}
        
        % \newcommand\GNoTree{\vmrun \rightsquigarrow \uppsidown} An expression in
        core Verse evaluates to produce possibly-empty sequence of values. In
        \VMinus, values depend on the form of abstract expression $\alpha.$ If
        $\alpha$ is a Verse-like expression, \valpha ~will be a value sequence. If
        it is an ML-like expression, it will be a single value. 
        
        A guarded expression evaluates to produce a \bf{result}. A result is either
        a metavalue \valpha ~or reject. 
        
        \[\it{r} \; \rm{::=} \; \vartheta \;|\; \reject \]
        
        \showvjudgement{Eval-Alpha}{\veval{\alpha}{\valpha}}
        \showvjudgement{Eval-AlphaExpr}{\veval{\ealpha}{\valpha}}
        \showvjudgement{Eval-Guarded-Expr}{\vmrun}
        
        % \bigskip
        % If a guarded expression cannot be evaluated without producing logical 
        % variables at runtime, it cannot be expressed as a decision tree. 
        % This notation indicates this failure (think of \uppsidown as a fallen 
        % tree), which results in a compile-time error. 
        % \showvjudgement{NoTree}{\GNoTree}
        
        \bigskip
    \subsection{Equations}
    
    In \VMinus, we solve equations (intermediate computations in a guarded
    expression of the form \\$\ealpha[1] = \ealpha[2]$) in a similar way to the
    authors of the original Verse paper: we pick one, attempt to solve for it, and
    move on. 
    
    \rab{How do we express that \VMinus semantics preclude backtracking/logical 
    variables at runtime?}
    
    Given an environment from names to metavalues {\valpha}s \Rho, an equation \eq
    ~will either refine the environment ($\Rhoprime$) or lead to failure. We use
    the metavariable $\dagger$ to represent failure, and an equation failing will
    cause the top-level guarded expression to evaluate to \reject. 
    
    \showvjudgement{Eq-Refine}\eqrefine
    \showvjudgement{Eq-Fail}\eqfail
        
        
        \section{Sequences}
        
        The trivial sequence is \emptyseq. Sequences can be concatenated with infix 
    $\cdot$. In an appropriate context, a value like $x$ stands for 
    the singleton sequence containing $x$. 
    
    \begin{align*}
        \emptyseq \cdot \ys &\equiv \ys \\
        \ys \cdot \emptyseq &\equiv \ys \\
        (\xs \cdot \ys) \cdot \zs &\equiv \xs \cdot (\ys \cdot \zs)
    \end{align*}


\section{Semantics of \PPlus}

\begin{itemize}
    \item Need a new judgement form for top-level patterns?
    \item Or, when, guards
\end{itemize}

\[
\inferrule*[Left=\textsc{ (Case-Match) }]
    {\vmrun[] \Downarrow \vartheta}
    {\veval{\iffi{\galpha \; \square \; \dots}}{\vartheta}}
\]

\[
\inferrule*[Left=\textsc{ (Case-Recurse) }]
    {\inferrule* {}
    {\vmrun[result=\reject]}
    \and 
    \veval{\iffi{\dots}}{\vartheta}}
    {\veval{\iffi{\galpha \; \square \; \dots }}{\vartheta}}
\]

\end{document}

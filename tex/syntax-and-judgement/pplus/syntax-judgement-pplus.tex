\documentclass[]{article}
\usepackage{vmlmacros}
\usepackage{syntax}
\usepackage{relsize}
% \usepackage{palatino} % I don't love this, to be honest. 
\usepackage{amsmath}
\usepackage{booktabs}
\usepackage{simplebnf}
\setcounter{secnumdepth}{1}

\DeclareMathOperator{\dom}{dom}


% \setlength{\grammarparsep}{20pt plus 1pt minus 1pt} % increase separation between rules
\setlength{\grammarindent}{10em} % increase separation between LHS/RHS
\setlength{\parindent}{0cm}
\title{Syntax and Semantics of \PPlus}
\author{Roger Burtonpatel}
\date{February 18th, 2024}
\begin{document}

\maketitle

\section{Syntax}

\subsection{The Core Language}

We present a grammar of \PPlus: 

\bigskip

% I attempted to use the grammar environment you provided. But there was either
% something missing or something I overlooked in the example code and it would
% not compile, despite many reducing changes I made. So I went with the
% simplebnf package, which I quite like.  
\begin{center}
    \begin{bnf}
    $P$ : \textsf{Programs} ::=
    $\bracketed{d}$ : definition
    ;;
    $d$ : \textsf{Definitions} ::=
    | $\tt{val} \; x \; \tt{=} \; e$ : bind name to expression
    ;;
    $e$ : Expressions ::= 
    | $x, y, z$ : names
    | $K\bracketed{e}$ : value constructor application 
    | $\lambda x. \; e$ : lambda declaration  
    | $e_{1} \; e_{2}$ : function application 
    | $\tt{case} \; e \; \bracketed{\tt{[} p \; e \tt{]}}$ : case expression 
    ;;
    \tlp : \textsf{Top-level Patterns} ::= $p$ : pattern 
    | $\ttbraced{p \; \tt{when} \; e }$ : side condition
    | $\tt \; p_{1} \; \vert \dots \vert \; p_{n}$ : or-pattern 
    | $\ttbraced{\tlp \; \bracketed{, \; p <- e}}$ : pattern guard
    ;;
    $p$ : \textsf{Patterns} ::= $x$ : name 
    | $K \; \bracketed{p}$ : value constructor application 
    ;;
    $v$ : Values ::= $K\bracketed{v}$ : value constructor application 
    | $\lambda x. \; e$ : lambda value 
    ;;
    $K$ : \textsf{Value Constructors} ::=
    | \tt{true} $\vert$ \tt{false} : booleans
    | $\tt{\#}x$ : name beginning with \tt{\#}
    | \tt{A-Z}$x$ : name beggining with capital letter
    | $[\tt{-}\vert\tt{+}](\tt{0}-\tt{9})+$ : signed integer literal 
    \end{bnf}
\end{center}


A \it{name} is any token that is not an integer literal, 
does not contain whitespace, a bracket, or parenthesis, 
and is not a value constructor name or a reserved word.


\vfilbreak

\section{Forms of Judgement for \PPlus:}
\begin{tabular}{ll}
\toprule
    \multicolumn2{l}{\emph{Metavariables}} \\
\midrule
    $e$& expression \\
    $v, \; v'$& value \\
    $K$& value constructor \\ 
    $p$& pattern \\ 
    $x, y$& names \\ 
    $g$& pattern guard \\ 
    $gs$& list of pattern guards \\ 
    $\dagger$& pattern match failure \\ 
    $r$& a result, either $v$ or $\dagger$ \\ 
    \Rho& environment: $name \rightarrow \mathcal{V}$ \\
    \Rho\:+\:\Rhoprime& extension \\
    $\Rho \uplus \Rho'$& disjoint union \\
    $\rho\{ x \mapsto y \} $& environment extended with name $x$ mapping to $y$ \\
\bottomrule
\end{tabular}    

\bigskip

\begin{tabular}{ll}
    \toprule
        \multicolumn2{l}{\emph{Sequences}} \\
    \midrule
        $\emptyseq$& the empty sequence \\
        $S_1 \cdot S_2 $&  Concatenate sequence $S_1$ and sequence $S_2$ \\
        $x \cdot S_2 $& Cons $x$ onto sequence $S_2$ \\
    \bottomrule
    \end{tabular}    
    
    \medskip
    
    For the forms of judgement for pattern matching, we borrow from Ramsey
    in our formalization [Ramsey 2023] XXX TO LINK:

    \begin{align*}
        &\pmatch \quad   \rm{ Pattern $p$ matches value $v$, 
                              producing bindings $\rho'$;} \\
        &\pfail  \quad\; \rm{ Pattern $p$ does not match value $v$.} 
    \end{align*}

    We invent a formalization for the extensions to pattern-matching, which
    we call \Tlp\it{s}, based off of work by XXX FILL IN:

    \begin{align*}
        &\tlpmatch \quad   \rm{ \Tlp\, \tlp\thinspace matches value 
                                $v$, producing bindings $\rho'$;} \\
        &\tlpfail  \quad\; \rm{ \Tlp\, \tlp\thinspace does not match value $v$.} 
    \end{align*}

    N.B. Matching a \Tlp may evaluate an arbitrary number of
    expressions. Matching a standard pattern may never evaluate an expression.     

    We may use the metavariable $r$ as a substitute for the result of 
    pattern matching, leaving us with two revised forms of judgement: 

    \begin{align*}
        &\pmatch[newenv=r] \\
        &\tlpmatch[newenv=r] 
    \end{align*}


    \subsection{Expressions}
    
        An expression in \PPlus evaluates to produce a single value. 

        \begin{align*}
            &\prun
        \end{align*}
    

    \rab{nr here includes the trick of evaluating the scrutinee in a case
    expression to a literal and THEN beginning pattern matching. Worth
    including? Probably. How much to borrow from BPC?}
    
    \bigskip 

    \rab{Same thought re: +, environments, disjoint union, et al.}


%     \section{Sequences}
    
%     The trivial sequence is \emptyseq. Sequences can be concatenated with infix 
% $\cdot$. In an appropriate context, a value like $x$ stands for 
% the singleton sequence containing $x$. 

% \begin{align*}
%     \emptyseq \cdot \ys &\equiv \ys \\
%     \ys \cdot \emptyseq &\equiv \ys \\
%     (\xs \cdot \ys) \cdot \zs &\equiv \xs \cdot (\ys \cdot \zs)
% \end{align*}

\section{Rules (Big-step Operational Semantics) for \PPlus:}
    
\subsection{Evaluating Expressions}


\subsubsection{Evaluating expressions other than \tt{case}}

\[
\inferrule*[Left=\textsc{ (Eval-Vcon-Empty) }]
    {\ }
    {\veval{K}{K}}
\]

\[
\inferrule*[Left=\textsc{ (Eval-Vcon-Multi) }]
    {\inferrule* {}
    {
    \veval{e_{i}}{v_{i}}
    \and 
    1 \leq i \leq n
    }}
    {\veval{K(e_{1}, \dots e_{n})}{K(v_{1}, 
    \dots v_{i})}}
\]

\[
\inferrule*[Left=\textsc{ (Eval-Name) }]
    {x \in \dom \rho 
    \\\\
    \rho(x) = v}
    {\prun[exp={\textsc{name} \;x}]}
\]

\[
\inferrule*[Left=\textsc{ (Eval-Lambda-Decl) }]
    {\ }
    {\prun[exp={\lambda x.e}, value={\lambda x.e}]}
\]


\[
\inferrule*[Left=\textsc{ (Eval-Funapp) }]
    {\veval{e_{1}}{\lambda x.e}
    \\\\
    \veval{e_{2}}{v'}
    \\\\
    \prun[envext=\bracketed{x \mapsto v'}]}
    {\veval{e_{1}\; e_{2}}{v}}
\]

\[
\inferrule*[Left=\textsc{ (CaseScrutinee) }]
    {\inferrule* {}
    {\prun[value=v]}
    \and 
    \prun[exp=
        {\textsc{case} \; (\textsc{literal}(v), 
        [p_{1} \; e_{1}], \dots, [p_{n} \; e_{n}])},
        value=v']}    
    {\prun[exp=
    {\textsc{case} \; (e, 
    [p_{1} \; e_{1}], \dots, [p_{n} \; e_{n}])},
    value=v']}
\]

\[
\inferrule*[Left=\textsc{ (CaseMatch) }]
    {\inferrule* {}
    {\tlpmatch[tlp={\tlp_{1}}, newenv=\rho']}
    \and 
    \prun[exp=e_{1}, env=\rho + \rho']}    
    {\prun[exp=
    {\textsc{case} \; (\textsc{literal}(v), 
    [\tlp_{1} \; e_{1}], \dots, [\tlp_{n} \; e_{n}])},
    value=v']}
\]

\[
\inferrule*[Left=\textsc{ (CaseFail) }]
    {\inferrule* {}
    {\tlpmatch[tlp={\tlp_{1}}, newenv=\dagger]}
    \and 
    \prun[exp={\textsc{case} \; (\textsc{literal}(v), 
    [\tlp_{2} \; e_{2}], \dots, [\tlp_{n} \; e_{n}])},
    value=v']}    
    {\prun[exp=
    {\textsc{case} \; (\textsc{literal}(v), 
    [\tlp_{1} \; e_{1}], \dots, [\tlp_{n} \; e_{n}])},
    value=v']}
\]



\subsection{Rules for pattern matching}


\[
\inferrule*[Left=\textsc{ (MatchVcon) }]
    {\inferrule* {}
    {
    \pmatch[pat=p_{i}, value=v_{i}, newenv=r_{i}], \and 1 \leq i \leq m
    \\\\
    r = r_{1} \uplus \dots \uplus r_{m}
    }}
    {\pmatch[pat={K\; p_{1}, \dots 
            p_{m}}, value={\textsc{vcon} \; (K, [v_{1}', \dots v_{m}'])},
            newenv=r]}
\]

\[
\inferrule*[Left=\textsc{ (FailVcon) }]
    {v \rm{ does not have the form } \textsc{vcon} \; (K, [v_{1}', \dots v_{m}'])}
    {\pmatch[pat={K\; p_{1}, \dots 
            p_{m}}, 
            newenv=\dagger]}
\]

\[
\inferrule*[Left=\textsc{ (MatchBareVcon) }]
    {\ }
    {\pmatch[pat={K}, value={\textsc{vcon} \; (K, [])},
            newenv=\bracketed{}]}
\]

\[
\inferrule*[Left=\textsc{ (FailBareVcon) }]
    {v \neq {\textsc{vcon} \; (K, [])}}
    {\pmatch[pat={K},
            newenv=\dagger]}
\]

\[
\inferrule*[Left=\textsc{ (MatchVar) }]
    {\ }
    {\pmatch[pat={x},
            newenv=\bracketed{x \mapsto v}]}
\]


\section{Rules for matching Top-level Patterns}

\[
\inferrule*[Left=\textsc{ (OrSucc) }]
    {\pmatch[pat=p_{1}]}
    {\tlpmatch[tlp={\textsc{or} \; (p_{1}, \dots p_{n})}]}
\]

\[
\inferrule*[Left=\textsc{ (OrRec) }]
    {\pmatch[pat=p_{1}, newenv=\dagger]
    \and 
    \tlpmatch[tlp={\textsc{or} \; (p_{2}, \dots p_{n})}]}
    {\tlpmatch[tlp={\textsc{or} \; (p_{1}, \dots p_{n})}]}
\]


\[
\inferrule*[Left=\textsc{ (OrFail) }]
    {\ }
    {\tlpmatch[tlp={\textsc{or} \; (\emptyseq)}, newenv=\dagger]}
\]


\[
\inferrule*[Left=\textsc{ (WhenMatch) }]
    {\pmatch
    \\\\
    \prun[env=\Rho \uplus \Rhoprime, value=v' \and v' \neq \tt{false}]}
    {\tlpmatch[tlp={\textsc{when} \; (p, e)}]}
\]

\[
\inferrule*[Left=\textsc{ (WhenNoMatch) }]
    {\pfail}
    {\tlpmatch[tlp={\textsc{when} \; (p, e)}, newenv=\dagger]}
\]

\[
\inferrule*[Left=\textsc{ (WhenFalse) }]
    {\pmatch
    \\\\
    \prun[env=\Rho \uplus \Rhoprime, value=v']  \and v' = \tt{false}}
    {\tlpmatch[tlp={\textsc{when} \; (p, e)}, newenv=\dagger]}
\]

\[
\inferrule*[Left=\textsc{ (GuardNoMatch) }]
    {\tlpmatch[newenv=\dagger]}
    {\tlpmatch[tlp={\textsc{guard} \; (\tlp, gs)}, newenv=\dagger]}
\]

\[
\inferrule*[Left=\textsc{ (GuardEmpty) }]
    {\tlpmatch[newenv=r]}
    {\tlpmatch[tlp={\textsc{guard} \; (\tlp, \emptyseq)}, newenv=r]}
\]

\[
\inferrule*[Left=\textsc{ (GuardFail) }]
    {\inferrule* {}
    {\tlpmatch[newenv=\Rhoprime]}
    \\\\
    \prun[env=\Rho \uplus \Rhoprime, value=v']
    \\\\
    \pmatch[value=v', newenv=\dagger]
    }
    {\tlpmatch[tlp={\textsc{guard} \; (\tlp, (p <- e) \cdot gs)}, newenv=\dagger]}
\]


\[
\inferrule*[Left=\textsc{ (GuardRec) }]
    {\inferrule* {}
    {\tlpmatch[newenv=\Rhoprime]}
    \\\\
    \prun[env=\Rho \uplus \Rhoprime, value=v']
    \\\\
    \pmatch[value=v', newenv=\rho'']
    \\\\
    \tlpmatch[env=\Rho \uplus \rho'', tlp={\textsc{guard} \; (\tlp, gs)},
    newenv=r]
    }
    {\tlpmatch[tlp={\textsc{guard} \; (\tlp, (p <- e) \cdot gs)}, newenv=r]}
\]


\end{document}

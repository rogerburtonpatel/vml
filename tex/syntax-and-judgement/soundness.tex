\documentclass{article}
\usepackage{amsmath}
\usepackage{vmlmacros}
\title{On Soundness Proofs}

\parindent=0pt
\parskip=0.7\baselineskip

\begin{document}

\maketitle

\noindent
\emph{Soundness}: Whenever
$$\vmrun[env=\emptyenv,context=\emptycontext]\text,$$
then
$$\vrun[vterm=\translation{\galpha},vresult=\vartheta]\text.$$


\newcommand\closefun{\ensuremath{\mathit{close}}}

\newcommand\close[1][]{% run Verse code
{\setkeys{state}{#1}%
\let\c=\component
\ensuremath{\closefun(\c{env}\c{envext}, \c{context}, \c{term})}}}


Idea for the proof: define function $\closefun$ that converts an
environment, context, and guarded expression to a closed guarded
expression.
We expect
$$\emptyenv; \emptyseq \vdash \close \Downarrow r$$ % vmrun -> tex capacity err
if and only if
$$\vmrun[]\text.$$
Given \closefun, 
the induction hypothesis for our soundness proof can be something like
this:
\begin{quote}
Whenever 
$$\vmrun[]\text,$$
then
$$\vrun[vterm=\translation{\close},vresult=\vartheta]\text.$$
\end{quote}

To~prove:
\begin{itemize}
\item
The \closefun\ function is sound.
\item
Evaluation in \VMinus\ is deterministic.
\end{itemize}

\section{Important sections}

\begin{quote}
    Whenever 
    $$\vmrun[]\text,$$
    then
    $$\vmrun[context=\emptyseq,term=\context; \galpha]\text.$$
\end{quote}

\begin{quote}
    Whenever 
    $$\vmrun[envext=\bracketed{x \mapsto \bot}, context=\emptyseq]\text,$$
    then
    $$\vmrun[context=\emptyseq,term={\exists x.\; \galpha}]\text.$$
\end{quote}

\begin{quote}
    Whenever 
    $$\vmrun[envext=\bracketed{x \mapsto \valpha'}, context=\emptyseq]\text,$$
    then
    $$\vmrun[envext=\bracketed{x \mapsto \bot}, context=\emptyseq,
            term={x=\valpha'; \; \galpha}]\text.$$
\end{quote}

\bigskip

Translating these ideas to laws for \closefun: 

\bigskip

\begin{align}
    \close[env=\emptyenv, context=\emptyseq]   &\rightarrow \galpha \\
    % \close[context=\TeqT{\expr[1] = \expr[2]}]      &\rightarrow \close[context=\TT, 
    %        term = {\expr[1] = \expr[2];\;\galpha}] \\
    \close &\rightarrow \close[context=\emptyseq, term={\context; \galpha}] 
    \rm{ when \context is non-empty} \\
    \close[envext=\bracketed{x \mapsto \bot}, context=\emptyseq] &\rightarrow 
    \close[term={\exists x.\; \galpha}] \\
    \close[envext=\bracketed{x \mapsto \valpha'}, context=\emptyseq] &\rightarrow
    \close[envext=\bracketed{x \mapsto \bot}, context=\emptyseq,
    term={x=\valpha'; \; \galpha}]
\end{align}

\bigskip

Question: how can we prove soundness of \closefun when order of the $\rho$
transformations is nondeterministic? I.\expr. how will we know that expanding a
guarded expression into a $\rho$ and a \context will and then closing 
that guarded expression with that $\rho$ and \context will produce equivalent 
results? Basically, I'm worried an $\exists$ will be put before where it should
be. 

\end{document}


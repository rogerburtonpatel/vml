\documentclass[]{article}
\usepackage{vmlmacros}
\usepackage{syntax}
\usepackage{relsize}
% \usepackage{palatino} % I don't love this, to be honest. 
\usepackage{amsmath}
\usepackage{booktabs}
\usepackage{simplebnf}
\setcounter{secnumdepth}{1}

\DeclareMathOperator{\dom}{dom}



% \setlength{\grammarparsep}{20pt plus 1pt minus 1pt} % increase separation between rules
\setlength{\grammarindent}{10em} % increase separation between LHS/RHS
\setlength{\parindent}{0cm}
\title{Syntax and Semantics of $V^{-}$}
\author{Roger Burtonpatel}
\date{October 22, 2023}
\begin{document}

\maketitle

\section{Syntax}

We present a grammar of $V^{-}$: 

\bigskip

% Integers \hspace{1cm} $k$ \\
% Variables \hspace{.45cm} $x, y, z, f, \galpha$ \\
% Programs \hspace{1cm} $p ::= \bf{one} \{ e \}$, where \ppl{fvs} $e = \emptyset$ \\
% Expressions $e ::= v \; \vert \; eq \ppl{;} \; e$ 
% Values \hspace{1cm} $v ::= x \;\vert\; hnf$ 
% Head Values \hspace{1cm} $hnf ::= k \;\vert\; op \;\vert\; \lambda x.\;e$ 


% I attempted to use the grammar environment you provided. But there was either
% something missing or something I overlooked in the example code and it would
% not compile, despite many reducing changes I made. So I went with the
% simplebnf package, which I quite like.  
\begin{center}
    \begin{bnf}
    $P$ : \textsf{Programs} ::=
    $\bracketed{d}$ : definition
    ;;
    $d$ : \textsf{Definitions} ::=
    | $\tt{val} \; x \; \tt{=} \; \alpha$ : bind name to expression
    ;;
    $\alpha$ : \textsf{Expressions} ::=
    | $x, y, z$ : names
    % Question: ebnf braces vs. 
    | $\tt{if} \; \tt{[}\; \galpha \; \bracketed{[] \galpha} \;\tt{]} \; \tt{fi}$ : if-fi 
    | $K \bracketed{\alpha}$ : value constructor application 
    | $\alpha_{1} \; \alpha_{2}$ : function application 
    ;;
    $\galpha$ : \textsf{Guarded Expressions} ::=  
    $\boldsymbol{\rightarrow}\alpha$ : terminating $\alpha$ 
    | $\alpha; \; \galpha$ : intermediate expression 
    | $\exists \bracketed{x} \tt{.} \galpha$ : existential 
    | $\alpha_{1} = \alpha_{2}; \; \galpha$ : equation 
    ;;
    $K$ : \textsf{Value Constructors} ::=
    \cons : cons 
    | \tt{[]} : empty list 
    | $\tt{\#}x$ : name beginning with \tt{\#}
    | \tt{A-Z}$x$ : name beggining with capital letter
    | $[\tt{-}\vert\tt{+}](\tt{0}-\tt{9})+$ : signed integer literal 

    \end{bnf}
\end{center}


A \it{name} is any token that is not an integer literal, 
does not contain whitespace, a bracket, or parenthesis, 
and is not a value constructor name or a reserved word.

\bigskip 

% \rab{Would like help cleaning up the format on this, specifically with regards 
% to the regex. The one downside of this nicer package is that descriptions will 
% not wrap, so describing an integer literal in full english isn't an option as 
% far as I can tell.}
% I'll ask this later. 

% The old grammar, with the \syntax package 

% \begin{grammar}
%     <program> ::= \{{<def>}\}
    
%     <def> ::= \tt{val} <name> <exp>
%         \alt <exp>
    
        
        
%         <exp> ::= <integer-literal>
%         \alt <name>
%         \alt <guarded-if>
%         \alt <lambda>
%         \alt <value-constructor-name> \{<exp>\}
%         \alt <exp> <exp>
%         \alt <exp> $\choice$ <exp>
        
        
%         <lambda> ::= $\lambda$\{<name>\}\tt{.} <exp>
        
%         <guarded-if> ::= \tt{if}  [<guarded-exp> \{ [] <guarded-exp> \}] \tt{fi}
        
        
%         <guarded-exp> ::= $\boldsymbol{\rightarrow}$<exp>
%         \alt  \tt{E} \{<name>\}\tt{.} <guarded-exp>${}_{\alpha}$
%         \alt  <exp>\tt{;} <guarded-exp>
%         \alt   <name> \tt{=} <exp>\tt{;} <guarded-exp>
%         \alt   <exp> \tt{=} <exp>\tt{;} <guarded-exp>
%         \alt   <guarded-exp> $\choice$ <guarded-exp> 
%         \alt   \tt{one}(\{<guarded-exp>\})
%         \alt   \tt{all}(\{<guarded-exp>\})
        
%         <name> ::= any token that is not an \textit{int-lit}, does not contain 
%         whitespace,
%         % a \texttt{'}, bracket, or parenthesis, 
%         and is not a <value-constructor-name> or a reserved word.

%         <value-constructor-name> ::= \cons | \tt{[]} | any token that begins
%         with a capital letter or a colon
    
%         <integer-literal> ::= token composed only of digits, possibly prefixed with a \texttt{+} or \texttt{-}.
    
%     \end{grammar}



% A $\star$ indicates syntactic sugar. 

% The desugaring of a multi-name \tt{E}
% is:


% $$\tt{E} \; lv_1 \tt{.} \; \dots \; \langle guarded\rm{-}exp \rangle \;
% \triangleq \; \tt{E} \; lv_1 \tt{.} \; \tt{E} \tt{.} \; \dots \; \langle
    % guarded\rm{-}exp \rangle  $$
        
\section{Refinement ordering on environments}

\begin{align*}
\rho \subseteq \rho' \text{ when }&\dom\rho  \subseteq \dom \rho'\\
\text{ and } &\forall x \in \dom \rho: \rho(x) \subseteq \rho'(x)
\end{align*}



\vfilbreak



\section{Forms of Judgement for $V^{-}$:}
\begin{tabular}{ll}
\toprule
    \multicolumn2{l}{\emph{Metavariables}} \\
\midrule
    % $v, \; v'$& value \\
    \valpha& a value produced from evaluating $\alpha$. \\
    $eq$& equation \\ 
    % $\tempstuck$& a temporarily-stuck equation \\
    $\reject$& equation rejection \\
    $\result$& $\vartheta \mid$ \reject : a result of \valpha \; or
    rejection\\
    $\rho$& environment: $name \rightarrow \mathcal{V}_{\bot}$ \\
    $\rho\{ x \mapsto y \} $& environment extended with name $x$ mapping to $y$ \\
    $\mathcal{T}$& Context of all temporarily stuck equations (a sequence) \\ 
    $\alpha$& An expression \\ 
    $\galpha$& A guarded expression \\
    \uppsidown& Inability to compile to a decision tree; a compile time error \\
\bottomrule
\end{tabular}    

\bigskip

\begin{tabular}{ll}
    \toprule
        \multicolumn2{l}{\emph{Sequences}} \\
    \midrule
        $\emptyseq$& the empty sequence \\
        $S_1 \cdot S_2 $&  Concatenate sequence $S_1$ and sequence $S_2$ \\
        $x \cdot S_2 $& Cons $x$ onto sequence $S_2$ \\
    \bottomrule
    \end{tabular}    
    
    \medskip
    
    % \mkjudgementcmd{EquationSuccess}{\vtuple{\rho, eq}}{\rhohat REMOVE THIS}
    % \mkjudgementcmd{EquationTempStuck}{\vtuple{\rho, eq}}{\tempstuck}
    % \mkjudgementcmd{EquationReject}{\vtuple{\rho, eq}}{reject}

    % \showvjudgement{EquationSuccess}{\EquationSuccess}
    % \showvjudgement{EquationTempStuck}{\EquationTempStuck}
    % \showvjudgement{EquationReject}{\EquationReject}
    
    
    
    
    % Success only refines the environment; that~is, when
    % ${\vtuple{\rho, e}} \rightarrowtail{} {\rho'}$, we~expect $\rho \subseteq \rho'$.
    
    
\subsection{Expressions}

    \newcommand\EvalE{\vmrun[term=\alpha]}
    \newcommand\EvalGe{\vmrun}
    \newcommand\GNoTree{\vmrun \rightsquigarrow \uppsidown} An expression
    in core Verse evaluates to produce possibly-empty sequence of values. In
    \Vminus, values depend on the form of abstract expression $\alpha.$ If
    $\alpha$ is a Verse-like expression, \valpha will be a value sequence. If it
    is an ML-like expression, it will be a single value. 

    A guarded expression evaluates to produce a \bf{result}. A result is either
    a possibly-empty sequence of values or reject. 

    \[\it{r} \; \rm{::=} \; \vartheta \;|\; \reject \]

    \showvjudgement{Eval-Expr}{\EvalE}
    \showvjudgement{Eval-Guarded-Expr}{\EvalGe}
    
    \bigskip
    If a guarded expression cannot be evaluated without producing logical 
    variables at runtime, it cannot be expressed as a decision tree. 
    This notation indicates this failure (think of \uppsidown as a fallen 
    tree), which results in a compile-time error. 
    \showvjudgement{NoTree}{\GNoTree}

\bigskip


\section{Sequences}

The trivial sequence is \emptyseq. Sequences can be concatenated with infix 
$\cdot$. In an appropriate context, a value like $x$ stands for 
the singleton sequence containing $x$. 

\begin{align*}
    \emptyseq \cdot \ys &\equiv \ys \\
    \ys \cdot \emptyseq &\equiv \ys \\
    (\xs \cdot \ys) \cdot \zs &\equiv \xs \cdot (\ys \cdot \zs)
\end{align*}

\section{Rules (Big-step Operational Semantics) for $V^{-}$:}
    
\subsection{Evaluating Guarded Expressions}
\subsubsection{Evaluating simple parts of guarded expressions}

\[
\inferrule*[Left=\textsc{ (Eval-ArrowExpr) }]
    {\vmrun[context=\emptyseq,term=e,result=\vartheta]}
    {\vmrun[context=\emptyseq, term=\arrowe,result=\vartheta]}
\]

\[
\inferrule*[Left=\textsc{ (Eval-Exists) }]
    {\vmrun[env=\rho\{x \mapsto \bot \}]}
    {\vmrun[term=\exists x.\; \galpha]}
\]

\[
\inferrule*[Left=\textsc{ (Eval-Expseq) }]
    {\inferrule* {}
    {
    \veval{\alpha}{\valpha}
    \and
    \veval{\galpha}{\result}
    }}
    {\vmrun[term=\alpha;\; \galpha]}
\]
\subsubsection{Shifting an equation to the context}
\[
\inferrule*[Left=\textsc{ (G-Move-To-Ctx) }]
    {\inferrule*{\vmrun[context=eq \cdot \context]}
    {}
    }
    {\vmrun[term=eq; \; \galpha]}
\]

\subsubsection{Evaluating with different types of equations}

\[
\inferrule*[Left=\textsc{ (G-EqExps) }]
    {\inferrule* {}
    {
    x,\;y \; \rm{are distinct and fresh}
    \\\\
    \vmrun[envext=\bracketed{x \mapsto \bot,\; y \mapsto \bot},
          context=\eqTT{x = \alpha_{1} \cdot y = \alpha_{2} \cdot x = y}]}}
    {\vmrun[context=\TeqT{\alpha_{1} = \alpha_{2}}]}
\]

\[
\inferrule*[Left=\textsc{ (G-EqNameExp) }]
    {\inferrule* {}
    {
    \veval{\alpha}{\valpha}
    \and 
    \vmrun[envext=\bracketed{x \mapsto \valpha},
           context=\context \cdot \context',result=\result']}}
    {\vmrun[context=\TeqT{x = \alpha},result=\result']}
\]

\[
\inferrule*[Left=\textsc{ (G-EqNames-Vals-Succ) }]
    {\inferrule* {}
    {
    x,\;y \in \dom \rho
    \\\\
    \rho(x) = \valpha, \; \rho(y) = \valpha
    \\\\ 
    \vmrun[context=\TT]}}
    {\vmrun[context=\TeqT{x = y}]}
\]

\[
\inferrule*[Left=\textsc{ (G-EqNames-Vals-Fail) }]
    {\inferrule* {}
    {
    x,\;y \in \dom \rho
    \\\\
    \rho(x) = \valpha, \; \rho(y) = \valpha'
    \\\\
    \valpha \neq \valpha'}}
    {\vmrun[context=\TeqT{x = y},result=\reject]}
\]

\[
\inferrule*[Left=\textsc{ (G-EqNames-Bots-Fail) }]
    {\inferrule* {}
    {
    x,\;y \in \dom \rho
    \\\\
    \rho(x) = \bot, \; \rho(y) = \bot
    \\\\
    x,\;y \; \rm{do not appear in} \; \context, \; \context'}}
    {\vmrun[context=\TeqT{x = y}] 
    \rightsquigarrow \uppsidown}
\]

\[
\inferrule*[Left=\textsc{ (G-EqNames-BotVal-Succ) }]
    {\inferrule* {}
    {
    x,\;y \in \dom \rho
    \\\\
    \rho(x) = \bot, \; \rho(y) = \valpha
    \\\\
    \vmrun[envext=\bracketed{x \mapsto \valpha},
           context=\context \cdot \context',result=\result']}}
    {\vmrun[context=\TeqT{x = y},result=\result']}
\]

\[
\inferrule*[Left=\textsc{ (G-Vcon-Single-Fail) }]
    {K \neq K'}
    {\vmrun[context=\TeqT{K = K'},result=\reject]}
\]

\[
\inferrule*[Left=\textsc{ (G-Vcon-Single-Succ) }]
    {\vmrun[context=\TT]}
    {\vmrun[context=\TeqT{K= K}]}
\]


\[
\inferrule*[Left=\textsc{ (G-Vcon-Multi-Fail) }]
    {K \neq K'}
    {\vmrun[context=\TeqT{K(\alpha_{1}, \dots 
            \alpha_{n}) = K'(\alpha_{1}', \dots \alpha_{n}')},
            result=\reject]}
\]

\[
\inferrule*[Left=\textsc{ (G-Vcon-Multi-Arity-Fail) }]
    {n \neq m}
    {\vmrun[context=\TeqT{K(\alpha_{1}, \dots 
            \alpha_{n}) = K(\alpha_{1}', \dots \alpha_{m}')},
            result=\reject]}
\]

\[
\inferrule*[Left=\textsc{ (G-Vcon-Multi-Succ) }]
    {
    \vmrun[context=\eqTT{\lbrack \alpha_{i}=\alpha_{i}' \; 
           \vert \; 1 \leq i \leq n \rbrack}]}
    {\vmrun[context=\TeqT{K(\alpha_{1}, \dots 
            \alpha_{n}) = K(\alpha_{1}', \dots \alpha_{n}')}]}
\]

\subsection{Evaluating General Expressions}


% \[
% \inferrule*[Left=\textsc{ (If-Fi-Fail) }]
%     {\ }
%     {\veval{\iffi{\ }}{\emptyseq}}
% \]

\[
\inferrule*[Left=\textsc{ (If-Fi-Success) }]
    {\vmrun[] \Downarrow \vartheta}
    {\veval{\iffi{\galpha \; \square \; \dots}}{\vartheta}}
\]

\[
\inferrule*[Left=\textsc{ (If-Fi-Reject) }]
    {\inferrule* {}
    {\vmrun[result=\reject]}
    \and 
    \veval{\iffi{\dots}}{\vartheta}}
    {\veval{\iffi{\galpha \; \square \; \dots }}{\vartheta}}
\]

\[
\inferrule*[Left=\textsc{ (Vcon-Empty) }]
    {\ }
    {\veval{K}{K}}
\]

\[
\inferrule*[Left=\textsc{ (Vcon-Multi) }]
    {\inferrule* {}
    {
    \veval{\alpha_{i}}{\vartheta_{i}}
    \and 
    1 \leq i \leq n
    }}
    {\veval{K(\alpha_{1}, \dots \alpha_{n})}{K(\vartheta_{1}, 
    \dots \vartheta_{i})}}
\]
\end{document}

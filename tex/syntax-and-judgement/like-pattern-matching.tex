\documentclass{article}
\usepackage{vmlmacros}

\begin{document}

1. An equation can be solved if it has a side with no unevaluated (i.e.,
unbound) logical variables. 

2. Once an equation is solved, all its logical variables are evaluated (bound,
known). 

3. A guard can be evaluated "like pattern matching" if there is an order in
which all its equations can be solved. 

Now consider a graph of equations.  There is an edge from
equation A to equation B if solving A makes B solvable.  If this graph can 
be topologically sorted to create a plain DAG, traversal of the DAG represents
an order in which those equations can be solved. 

so: 

4. If the equations in a guard can be topologically sorted, the guard can be 
   evaluated in order, making it "like pattern matching." 

guard ::= exp list 

exp ::= value $\vert$ eq; e $\vert$ $\exists$ x. exp $\vert$ fail $\vert$ e1
\choice $\;$ e2 $\vert$ v1 v2 $\vert$ one{exp} $\vert$ all{exp}

eq ::= exp $\vert$ equation 

equation ::= x = exp 


A term eq is either an ordinary expression e, or an equation v = e; this syntax
ensures that equations can only occur to the left of a “; ”. 


BUILDING A GUARD IS THE TOPOLOGICAL SORT. 

the guard is the decision tree. 

BIG NOTE: 

In verse, 

1 ; 2 will give you 2 results. we can't have 2 results in ml. 
so- can you eliminate the form e1 ; e2 ? 

No- we just return the result of e2. 

% ntegers 𝑘
% Variables 𝑥, ~, 𝑧, 𝑓 , 𝑔
% Programs 𝑝 ::= one{e} where fvs(𝑒) = ∅
% Expressions 𝑒 ::= v | 𝑒𝑞; e | ∃x. e | fail | e1 e2 | v1 v2 | one{e} | all{e}
% 𝑒𝑞 ::= e | v = e Note: “𝑒𝑞” is pronounced “expression or equation”
% Values v ::= 𝑥 | hnf
% Head values hnf ::= 𝑘 | 𝑜𝑝 | ⟨v1, ···, vn⟩ | 𝜆x. e
% Primops 𝑜𝑝 ::= gt | add
% Concrete syntax: “ ” and “;” are right-associative.
% “=” binds more tightly than “;”.
% “𝜆” and “∃” each scope as far to the right as possible.
% For example, (𝜆y. ∃x. x = 1; x + y) means (𝜆y. (∃x. ((x = 1); (x + y)))).
% Parentheses may be used freely to aid readability and override default precedence.
% fvs(e) means the free variables of e; in VC, 𝜆 and ∃ are the only binders.

\end{document}
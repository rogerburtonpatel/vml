\documentclass[]{article}
\usepackage{vmlmacros}
\usepackage{syntax}
\usepackage{relsize}
% \usepackage{palatino} % I don't love this, to be honest. 
\usepackage{amsmath}
\usepackage{booktabs}
\usepackage{simplebnf}
\setcounter{secnumdepth}{1}

\DeclareMathOperator{\dom}{dom}



% \setlength{\grammarparsep}{20pt plus 1pt minus 1pt} % increase separation between rules
\setlength{\grammarindent}{10em} % increase separation between LHS/RHS
\setlength{\parindent}{0cm}
\title{Syntax and Semantics of $V^{-}$}
\author{Roger Burtonpatel}
\date{October 22, 2023}
\begin{document}

\maketitle

\section{Syntax}

We present a grammar of $V^{-}$: 

\bigskip

% Integers \hspace{1cm} $k$ \\
% Variables \hspace{.45cm} $x, y, z, f, g$ \\
% Programs \hspace{1cm} $p ::= \bf{one} \{ e \}$, where \ppl{fvs} $e = \emptyset$ \\
% Expressions $e ::= v \; \vert \; eq \ppl{;} \; e$ 
% Values \hspace{1cm} $v ::= x \;\vert\; hnf$ 
% Head Values \hspace{1cm} $hnf ::= k \;\vert\; op \;\vert\; \lambda x.\;e$ 

\begin{center}
    \begin{bnf}

    $P$ : \textsf{Programs} ::=
    $\bracketed{d}$ : definition
    ;;
    $d$ : \textsf{Definitions} ::=
    | $\tt{val} \; x  \tt{=} \; e$ : bind name to expression
    ;;
    $e$ : \textsf{Expressions} ::=
    | $x$ : name
    | $\tt{if} \; \galpha \; \bracketed{[] \galpha} \; \tt{fi}$ : if-fi 
    | $K \bracketed{e}$ : value constructor application 
    | $e_1 \; e_2$ : function application 
    ;;
    $\galpha$ : \textsf{Guarded Expressions} ::=  
    $\boldsymbol{\rightarrow}\alpha$ : terminating $\alpha$ 
    | $e; \; \galpha$ : intermediate expression 
    | $\tt{E} \bracketed{x} \tt{.} \galpha$ : existential 
    | $e_1 = e_2; \; \galpha$ : equation 
    ;;
    $K$ : \textsf{Value Constructors} ::=
    \cons : cons 
    | \tt{[]} : empty list 
    | $\tt{\#}x$ : name beginning with \tt{\#}
    | \tt{A-Z}$x$ : name beggining with capital letter
    | $[\tt{-}\vert\tt{+}](\tt{0}-\tt{9})+$ : signed integer literal 

    \end{bnf}
\end{center}

\bigskip 

A \it{name} is any token that is not an integer literal, 
does not contain whitespace, a bracket, or parenthesis, 
and is not a value constructor name or a reserved word.

\bigskip 

\rab{Would like help cleaning up the format on this, specifically with regards 
to the regex. The one downside of this nicer package is that descriptions will 
not wrap, so describing an integer literal in english isn't an option as 
far as I can tell.}
% \begin{grammar}
%     <program> ::= \{{<def>}\}
    
%     <def> ::= \tt{val} <name> <exp>
%         \alt <exp>
    
        
        
%         <exp> ::= <integer-literal>
%         \alt <name>
%         \alt <guarded-if>
%         \alt <lambda>
%         \alt <value-constructor-name> \{<exp>\}
%         \alt <exp> <exp>
%         \alt <exp> $\choice$ <exp>
        
        
%         <lambda> ::= $\lambda$\{<name>\}\tt{.} <exp>
        
%         <guarded-if> ::= \tt{if}  [<guarded-exp> \{ [] <guarded-exp> \}] \tt{fi}
        
        
%         <guarded-exp> ::= $\boldsymbol{\rightarrow}$<exp>
%         \alt  \tt{E} \{<name>\}\tt{.} <guarded-exp>${}_{\alpha}$
%         \alt  <exp>\tt{;} <guarded-exp>
%         \alt   <name> \tt{=} <exp>\tt{;} <guarded-exp>
%         \alt   <exp> \tt{=} <exp>\tt{;} <guarded-exp>
%         \alt   <guarded-exp> $\choice$ <guarded-exp> 
%         \alt   \tt{one}(\{<guarded-exp>\})
%         \alt   \tt{all}(\{<guarded-exp>\})
        
%         <name> ::= any token that is not an \textit{int-lit}, does not contain 
%         whitespace,
%         % a \texttt{'}, bracket, or parenthesis, 
%         and is not a <value-constructor-name> or a reserved word.

%         <value-constructor-name> ::= \cons | \tt{[]} | any token that begins
%         with a capital letter or a colon
    
%         <integer-literal> ::= token composed only of digits, possibly prefixed with a \texttt{+} or \texttt{-}.
    
%     \end{grammar}



% A $\star$ indicates syntactic sugar. 

% The desugaring of a multi-name \tt{E}
% is:


% $$\tt{E} \; lv_1 \tt{.} \; \dots \; \langle guarded\rm{-}exp \rangle \;
% \triangleq \; \tt{E} \; lv_1 \tt{.} \; \tt{E} \tt{.} \; \dots \; \langle
    % guarded\rm{-}exp \rangle  $$
        
\section{Refinement ordering on environments}

\begin{align*}
\rho \subseteq \rho' \text{ when }&\dom\rho  \subseteq \dom \rho'\\
\text{ and } &\forall x \in \dom \rho: \rho(x) \subseteq \rho'(x)
\end{align*}



\vfilbreak



\section{Forms of Judgement for $V^{-}$:}
\begin{tabular}{ll}
\toprule
    \multicolumn2{l}{\emph{Metavariables}} \\
\midrule
    % $v, \; v'$& value \\
    $\valbar$& a (possibly empty) sequence of values. \\
    $eq$& equation \\ 
    % $\tempstuck$& a temporarily-stuck equation \\
    $\reject$& equation rejection \\
    $\result$& $\valbar \mid$ \reject : a result of a sequence of values or
    rejection\\
    $\rho$& environment: $name \rightarrow \mathcal{V}_{\bot}$ \\
    $\rho\{ x \mapsto y \} $& environment extended with name $x$ mapping to $y$ \\
    $\mathcal{T}$& Context of all temporarily stuck equations (a sequence) \\ 
    $e$& An expression \\ 
    $g$& A guarded expression \\
\bottomrule
\end{tabular}    

\bigskip

\begin{tabular}{ll}
    \toprule
        \multicolumn2{l}{\emph{Sequences}} \\
    \midrule
        $\emptyseq$& the empty sequence \\
        $S_1 \cdot S_2 $&  Concatenate sequence $S_1$ and sequence $S_2$ \\
        $x \cdot S_2 $& Cons $x$ onto sequence $S_2$ \\
    \bottomrule
    \end{tabular}    
    
    \medskip
    
    \mkjudgementcmd{EquationSuccess}{\vtuple{\rho, eq}}{\rhohat REMOVE THIS}
    \mkjudgementcmd{EquationTempStuck}{\vtuple{\rho, eq}}{\tempstuck}
    \mkjudgementcmd{EquationReject}{\vtuple{\rho, eq}}{reject}

    % \showvjudgement{EquationSuccess}{\EquationSuccess}
    % \showvjudgement{EquationTempStuck}{\EquationTempStuck}
    % \showvjudgement{EquationReject}{\EquationReject}
    
    
    
    
    % Success only refines the environment; that~is, when
    % ${\vtuple{\rho, e}} \rightarrowtail{} {\rho'}$, we~expect $\rho \subseteq \rho'$.
    
    
\subsection{Expressions}

    
    \newcommand\EvalE{\vctx[Expr=\alpha] \Downarrow \valbar}
    \newcommand\EvalGe{\vctx[Expr=g] \Downarrow \result}
    An expression evaluates to produce possibly-empty sequence of values.

    A guarded expression evaluates to produce a \bf{result}. A result is either
    a possibly-empty sequence of values or reject. 

    \[\it{r} \; \rm{::=} \; \valbar \;|\; \reject \]

    \showvjudgement{Eval-Expr}{\EvalE}
    \showvjudgement{Eval-Guarded-Expr}{\EvalGe}

\bigskip


\section{Sequences}

The trivial sequence is \emptyseq. Sequences can be concatenated with infix 
$\cdot$. In an appropriate context, a value like $x$ stands for 
the singleton sequence containing $x$. 

\begin{align*}
    \emptyseq \cdot \ys &\equiv \ys \\
    \ys \cdot \emptyseq &\equiv \ys \\
    (\xs \cdot \ys) \cdot \zs &\equiv \xs \cdot (\ys \cdot \zs)
\end{align*}

\section{Rules (Big-step Operational Semantics) for $V^{-}$:}
    
\subsection{Evaluating Guarded Expressions}

\[
\inferrule*[Left=\textsc{ (Eval-ArrowExpr) }]
    {\vctx[Tempstucks=\emptyseq] \Downarrow \valbar}
    {\vgctx[Tempstucks=\emptyseq, Expr=\arrowe] 
    \Downarrow \valbar}
\]

\[
\inferrule*[Left=\textsc{ (Eval-Exists) }]
    {\vgctx[Rho=\rho\{x \mapsto \bot \}] \Downarrow \result}
    {\vgctx[Expr=\exists x.\; g] 
    \Downarrow \result}
\]

\[
\inferrule*[Left=\textsc{ (G-Eval-With-Ctx) }]
    {\ }
    {\vgctx[Tempstucks=\TeqT] 
    \Downarrow \result}
\]

\[
\inferrule*[Left=\textsc{ (G-Move-To-Ctx) }]
    {\inferrule*{\vgctx[Tempstucks=eq \cdot \tempstucks] 
                 \Downarrow \result}{}
    }
    {\vctx[Expr=eq; \; g] \Downarrow \result}
\]

\subsection{Evaluating General Expressions}


% \[
% \inferrule*[Left=\textsc{ (If-Fi-Fail) }]
%     {\ }
%     {\veval{\iffi{\ }}{\emptyseq}}
% \]

\[
\inferrule*[Left=\textsc{ (If-Fi-Success) }]
    {\vgctx[] \Downarrow \valbar}
    {\veval{\iffi{g \; \square \; \dots}}{\valbar}}
\]

\[
\inferrule*[Left=\textsc{ (If-Fi-Reject) }]
    {\inferrule*[] {}
    {\vgctx[] \Downarrow \reject}
    \and 
    \veval{\iffi{\dots}}{\valbar}}
    {\veval{\iffi{g \; \square \; \dots }}{\valbar}}
\]

\[
\inferrule*[Left=\textsc{ (Vcon-Empty) }]
    {\ }
    {\veval{K}{K}}
\]

\[
\inferrule*[Left=\textsc{ (Vcon-Multi) }]
    {\inferrule*[] {}
    {
    \veval{e_{i}}{\valbar_{i}}
    \and 
    1 \leq i \leq n
    }}
    {\veval{K(e_{1}, \dots e_{n})}{K(\valbar_{1}, \dots \valbar_{i})}}
\]

% \[
% \inferrule*[Left=\textsc{ (Apply) }]
%     {\inferrule*[] {}
%     {
%     \veval{e_{i}}{\valbar_{i}}
%     \and 
%     1 \leq i \leq n
%     }}
%     {\veval{\lambda \valbar_{1}, \dots \valbar_{i}. e }{K(\valbar_{1}, \dots \valbar_{i})}}
% \]

% % \begin{mathpar}
% %     \inferrule*[Left=\textsc{ (G-Ctx-Stuck) }]
% %     {\inferrule*{\EquationTempStuck}{\CtxToRho}
% %     \and 
% %     \inferrule*{}{\EvalGeSucc}
% %     }
% %     {\GeEqCtxEval}
% % \end{mathpar}

% % Which of the above two do you prefer? 

% \nr{The notation below has multiple horizontal lines.  That makes it a
% derivation, not a rule.  I'm having trouble figuring out what's being
% said here.}



% \mkevaljudgementcmd{GeCtxEqEval}{\vtuple{\rho, \tempstucks, eq; g}}{v}

% \newcommand\GeEqStuckRule{%
% \inferrule*[Left=\textsc{ (G-Eq-Succ) }]
% {\GeCtxStuckRule[Right]}
% {\GeCtxEqEval}
% }

% \mpar{\GeEqStuckRule}

% \nr{If you're trying to pluck an equation out of a list of things, try
% ``$\tempstucks \cdot \eq \cdot \tempstucks'$.''}

% % Gotta fix the stacking. 

% \newcommand\GeCtxSuccRule[1][Left]{%
% \inferrule*[#1=\textsc{ (G-Ctx-Succ) }]
%     {\inferrule*{}{\EquationSuccess} 
%     \and 
%     \inferrule*{}{\EvalGeSucc[\rhohat]}
%     }
%     {\GeEqCtxEval}
% }

% \mpar{\GeCtxSuccRule}


% \newcommand\GeEqSuccRule{%
% \inferrule*[Left=\textsc{ (G-Eq-Succ) }]
%     {\inferrule*{}{\GeCtxSuccRule[Right]} 
%     }
%     {\GeCtxEqEval}
% }

% \mpar{\GeEqSuccRule}


% \mkevaljudgementcmd{GeEqCtxEvalFail}{\vtuple{\rho, eq \cdot \tempstucks, g}}{\fail}
% \mkevaljudgementcmd{GeCtxEqEvalFail}{\vtuple{\rho, \tempstucks, eq; g}}{\fail}

% \newcommand\GeCtxFailRule[1][Left]{%
% \inferrule*[#1=\textsc{ (G-Ctx-Fail) }]
%     {\inferrule*{}{\EquationReject} 
%     }
%     {\GeEqCtxEvalFail}
% }

%     \mpar{\GeCtxFailRule}

% \newcommand\GeEqFailRule{%
% \inferrule*[Left=\textsc{ (G-Eq-Fail) }]
%     {\inferrule*{}{\GeCtxFailRule[Right]} 
%     }
%     {\GeCtxEqEvalFail}
% }

% \mpar{\GeEqFailRule}

% \mkevaljudgementcmd{GeExpSucc}{\vtuple{\rho, \tempstucks, e; g}}{v}
% \mkevaljudgementcmd{GeExpFail}{\vtuple{\rho, \tempstucks, e; g}}{\fail}

% \newcommand\GeExpFailRule{%
% \inferrule*[Left=\textsc{ (G-Exp-Fail) }]
%     {\EvalSucc[\fail]}
%     {\GeExpFail}
% }

% \mpar{\GeExpFailRule}

% \newcommand\GeExpSuccRule{%
% \inferrule*[Left=\textsc{ (G-Eq-Succ) }]
%     {\EvalSucc[v']
%     \and 
%     \EvalGeSucc}
%     {\GeExpSucc}
% }

% \mpar{\GeExpSuccRule}




% % \begin{mathpar}
% % \inferrule*[Left=\textsc{G-Ctx-Stuck}]
% %     {\inferrule*[Left=\textsc{FormalVar}]
% %       {\texttt{x} \in dom \rho}
% %       {\evalr{\textsc{var}(\texttt{x})} \Downarrow \evalr{1}}
% %      \and 1 \neq 0 \and
% %      \inferrule*[Right=\textsc{Literal}]
% %        {\EvalGeSucc}
% %        {\evalr{\textsc{lit}(\texttt{2})} \Downarrow \evalr{2}} 
% %     }
% %     {\GeCtxStuck}
% % \end{mathpar}

% % \begin{mathpar}
% %   \inferrule*[Left=\textsc{IfTrue}]
% %     {\inferrule*[Left=\textsc{FormalVar}]
% %       {\texttt{x} \in dom \rho}
% %       {\evalr{\textsc{var}(\texttt{x})} \Downarrow \evalr{1}}
% %      \and 1 \neq 0 \and
% %      \inferrule*[Right=\textsc{Literal}]
% %        {\ }
% %        {\evalr{\textsc{lit}(\texttt{2})} \Downarrow \evalr{2}} 
% %     }
% %     {\evalr{\textsc{if}(\textsc{var}(\texttt{x}), 
% %                         \textsc{lit}(\texttt{2}),
% %                         \textsc{lit}(\texttt{3}))}
% %      \Downarrow \evalr{2} 
% %     }
% %   \end{mathpar}

% % A series of equations is either is solved to produce a value or gets stuck. 
% % A single sub-equation of the form x = e , where eq is an equation or an
% % expression, either is solved to produce an environment or gets stuck. 

% % An equation x = e can only be solved if e contains no unbound logical variables. 

% % \it{In progress}
% % $$
% % (\tt{if} \; \tt{fi}) \triangleq (\tt{wrong})
% % $$

\end{document}

\documentclass[manuscript,screen 12pt, nonacm]{acmart}
\let\Bbbk\relax % Fix for amssymb clash 
\usepackage{vmlmacros}
\AtBeginDocument{%
  \providecommand\BibTeX{{%
    \normalfont B\kern-0.5em{\scshape i\kern-0.25em b}\kern-0.8em\TeX}}}
\usepackage{outlines}
\setlength{\headheight}{14.0pt}
\setlength{\footskip}{13.3pt}
\title{An Alternative to Pattern Matching, Inspired by Verse}

\author{Roger Burtonpatel}
\email{roger.burtonpatel@tufts.edu}
\affiliation{%
\institution{Tufts University}
\streetaddress{419 Boston Ave}
  \city{Medford}
  \state{Massachusetts}
  \country{USA}
  \postcode{02155}
  }
\begin{document}
  
\section{A compromise}
\label{compromise}
    
    To bridge the gap between pattern matching, equations, and decision trees, I
    have created and implemented a semantics for a new core language:~\VMinus
     (``V-minus''). 
    
    \VMinus has equations and choice, like~\VC, but it does not have multiple results or
    backtracking. To eliminate multiple results, expressions in~\VMinus evaluate
    to at most one result, and choice only~\it{guards} computation; it~is not a
    valid form of expression. To~eliminate backtracking, the
    compiler rejects a~\VMinus program that would need to backtrack at runtime.
    To provide an efficient and backtracking-free cost model to which~\VMinus
    can be compiled, I~introduce a core language of decision trees,~\D,
    in Section~\ref{d}. 

    \subsection{Two languages of a kind}
    
    In this section, I~present~\VMinus. Its semantics appears in Section
    \ref{vmsemantics}, and for reference in~Appendix~\ref{languagedefs}. In my
    design, I~took inspiration from Verse:~\VMinus has a conventional
    sub-language that is the lambda calculus extended with named value
    constructors~\it{K} applied to zero or more values. I~chose named value
    constructors over~\VC's tuples because they look more like patterns.~\D, the
    language of decision trees and target of translation from~\VMinus, has the
    same lambda-calculus-plus-value-constructors core, with decision trees
    substituted for~\VMinus's~\iffibf. Because they share a core, and to
    facilitate comparisons and proofs, I present~\VMinus and~\D as two subsets
    of a single unifying language~\U, whose abstract syntax appears in
    Table~\ref{fig:unilang1}. Forms in black are present in both languages,
    forms in~\red{red} are specific to~\VMinus, and forms in~\blue{blue} are
    specific to~\D. 

    \begin{table}[ht]
      \utable
      \caption{Abstract Syntax of~\VMinus and~\D. Forms in black are present in
              both languages, forms in~\red{red} are specific to~\VMinus, and
              forms in~\blue{blue} are specific to~\D.}
      \label{fig:unilang1}
    \end{table}

    As in~\VC, every lambda-calculus term is valid in~\VMinus and~\D has the
    same semantics. Also like the lambda calculus and~\VC,~\VMinus and~\D are
    \it{strict}, meaning every expression is evaluated when it is bound to a
    variable.~\VMinus and~\D are also untyped. 
    
    The only form of constructed data in~\VMinus and~\D is value-constructor
    application, represented by the metavariable~\it{K}. In full languages,
    other forms of data like numbers and strings have a similar status to value
    constructors, but their presence would complicate the development of
    semantics and code. 
    
    Using just value constructors, though, a programmer can simulate more
    primitive data like strings. For example,~\tt{Wow! That's A Sizeable Square} is
    a valid expression in~\VMinus and~\D, because it is an application of
    constructor~\tt{Wow!} to the arguments~\tt{That's},~\tt{A},~\tt{Tall}, and
    \tt{Square}, all of which are value constructors themselves. Each name in
    this “sentence” is considered a value constructor because it begins with a
    capital letter. To simulate integers, a programmer can use Peano numbers,
    they can use value constructors to implement binary numbers, or they can
    cheat with singletons:~\tt{One},~\tt{Two}, etc. Because the languages all
    also have lambda, Church Numerals~\citep{church1985calculi} are another
    option. 

    In the subsection below, I~discuss~\VMinus in more detail. I discuss~\D in
    more detail in Section~\ref{d}. In Section~\ref{futurework}, I discuss how
    \VMinus relates to~\VC. 

\subsection{Introducing~\VMinus}
\label{vminus}

        To fuel the pursuit of smarter decision-making, I~now draw inspiration
        from~\VC. Equations in~\VC look attractive, but the cost model of~\VC is
        a challenge. 
        
        The elements of~\VC that lead to unpredictable or costly run times are
        backtracking and multiple results. So, I~begin with a subset of~\VC,
        which I~call~\VMinus ("V minus"), with these elements removed. Removing
        them strips much of the identity of~\VC, but it leaves its
        \it{equations} to build on top of in an otherwise-typical programming
        context of single results and no backtracking at runtime. 

        Having stripped out the functional logic programming elements of~\VC
        (backtracking and multiple results), only the decision-making bits are
        left over. To wrap these, I~add a classic decision-making construct:
        guarded commands~\citep{dijkstra} The result is~\VMinus. 

        % , illustrated in Table~\ref{tab:vmvsvc}. 


      %   \begin{table}[htbp]
      %     \centering
      %     \begin{tabular}{|p{0.5\linewidth}|p{0.48\linewidth}|}
      %         \hline
      %         \bfseries Like~\VC &~\bfseries Unlike~\VC~\\
      %         \hline
      %         \VMinus uses equations to guard computation.  &~\VMinus solves an equation at most once at runtime and never backtracks.~\\
      %         \hline
      %         \VMinus uses choice. &~\VMinus's choice can only guard computation and its result is never returned.~\\
      %         \hline
      %         \VMinus uses~\textit{success} and~\textit{failure} to make decisions. & An expression in~\VMinus returns at most one value.~\\
      %         \hline
      %     \end{tabular}
      %     \caption{Key similarities and differences between~\VMinus and~\VC}
      %     \label{tab:vmvsvc}
      %~\end{table}

      %~\begin{lemma}
      %   An expression in~\VMinus can return at most one result. 
      %~\end{lemma}

      %~\newcommand\ev{\ensuremath{e_{v}}\xspace}
      %~\begin{proof}
      %   The syntax of~\VMinus (Figure~\ref{fig:vmsyntax}) forbids any expression
      %   to return the result of evaluating choice. Let~\ev be a~\VMinus
      %   expression. If choice appears in~\ev, it may only appear in a guard.
      %   Guards may only precede expressions and are never returned. Therefore,
      %   \ev\ may never return the result of evaluating choice, and there is no
      %   other syntactic form in~\VMinus that allows for multiple results. 

      %   % The only values in~\VMinus are value constructor application and lambda,
      %   % each of which is itself a single result. 
      %~\end{proof}

      %~\begin{lemma}
      %   An expression in~\VMinus can never backtrack at runtime. 
      %~\end{lemma}
      % %~\newcommand\gs{\ensuremath{\mathit{gs}}\xspace}
      %~\begin{proof}
      %   In progress. 
      %~\end{proof}

        % Both backtracking and multiple results often manifest within~\VC's
        % choice operator, which tempt its removal altogether. 
        
        % However, I~am drawn
        % to harnessing the expressive potential of this operator, particularly
        % when paired with~\VC's 'one' keyword. When employed with choice as a
        % condition, 'one' elegantly signifies 'proceed if any branch of the
        % choice succeeds. 
        
        % To this end, I~propose a new language,~\VMinus, in which choice is
        % permitted with several modifications:

        %~\begin{enumerate}
        %~\item Choice may only appear as a condition or 'guard', not as a result
        % or the right-hand side of a binding.
        %~\item If any branch of the choice succeeds, the choice succeeds,
        % producing any bindings found in that branch. The program examines the
        % branches in a left-to-right order.
        %~\item The existential $\exists$ may not appear under choice.
        %~\end{enumerate}

        % I~introduce one more crucial modification to the~\VC runtime: a name in
        %~\VMinus is an~\it{expression} rather than a~\it{value}. This alteration,
        % coupled with my adjustments to choice, eradicates backtracking. Our
        % rationale behind this is straightforward: if an expression returns a
        % name, and subsequently, the program imposes a new constraint on that
        % name, it may necessitate the reevaluation of the earlier expression—- a
        % scenario I~strive to avoid. 
        
        % An example in~\VC illustrates this precise
        % scenario: 

        % FIRST PAPER EXAMPLE 

        % IMPORTANTLY,~\VMinus REALLY ISN'T A FUNCTIONAL LOGIC PROGRAMMING LANGUAGE ANYMORE. 
        
        %  Doesn't backtrack
        
        %  No multiple results 
        
        %  Doesn't evaluate functions backwards, have top-level patterns like verse, the list goes on 

        % If you're well-versed in functional logic programming, may perceive that
        % imposing such restrictions on choice and names effectively strips away
        % much of Verse's essence as a functional logic programming language. With
        % these constraints enforced, there can be no backtracking, multiple
        % results, backward function evaluation, or top-level patterns, among
        % other classic functional logic programming features. But do not fear:
        % our intent is not to recklessly strip Verse of its meticulously crafted
        % core tenets. Instead, our aim is to extract a select few—namely, its
        % equations, existentials, and nondeterministic evaluation order—and
        % juxtapose them with pattern matching.    

        %~\bf{Choice forces "name knowing." Names must be known in all branches 
        % if bound to unknown variable or bound to known variable if any name 
        % is unknown}.
        
        % With these modifications in place, alongside a few additional
        % adjustments to the placement of existentials and the timing of choice
        % evaluation (footnote: we discuss these later), I~unveil a redefined
        % decision-making core. To complete this transformation, we incorporate a
        % venerable decision-making construct: Guarded commands [cite: Dijkstra].
        % The result is~\VMinus. 
        \begin{figure}[H]
          \begin{center}
            \begin{bnf}
            $P$ :~\textsf{Programs} ::=
            $\bracketed{d}$ : definition
            ;;
            $d$ :~\textsf{Definitions} ::=
            | $\it{val}\; x\;~\it{=}\;~\expr$ : bind name to expression
            ;;
            $\expr$ :~\textsf{Expressions} ::=
            | $\v$ : literal values 
            | $x, y, z$ : names
            | $\bf{if}\;~\left[\; G\;~\bracketed{\dbar\; G}\;~\right]\;~\bf{fi}$ : if-fi 
            | $K~\bracketed{\expr}$ : value constructor application 
            | $\expr[1]\;~\expr[2]$ : function application 
            | $\lambda x.\;~\expr$ : lambda declaration 
            ;;
            $G$ :~\textsf{Guarded Expressions} ::=  
            $[\vexists{\bracketed{x}}]~\bracketed{g}~\boldsymbol{\rightarrow}\expr$ : names, guards, and body
            ;;
            $g$ :~\textsf{Guards} ::=  
            | $\expr$ : intermediate expression 
            | $\expr[1] =~\expr[2]$ : equation 
            | $ g~\bracketed{; g}~\;~\choice~\; g~\bracketed{; g}$ : choice 
            ;;
            $\v$ : Values ::= $K\bracketed{\v}$ : value constructor application 
            | $\lambda x.\;~\expr$ : lambda value
            ;;
            \end{bnf}
            \medskip
          * Desugars to~\tt{x = e\_{1}; x = e\_{2}}, where~\tt{x} is fresh. 
          \end{center}
          \Description{The concrete syntax of VMinus.}
              \caption{\VMinus: Concrete syntax}
              \label{fig:vmsyntax}
              \end{figure}      

        \VMinus takes several key concepts from {\VC}---namely, equations and
        choice---with several key modifications. Like~\VC,~\VMinus has
        equations and choice, but unlike in~\VC, choice only guards computations, 
        there are no multiple results, and all decision-making appears in the 
        \iffibf construct, inspired by Dijkstra~\citep{dijkstra}. 
      

    \subsection{Programming in~\VMinus}

    Even with multiple modifications,~\VMinus still allows for many of the same
    pleasing computations as full Verse. A programmer can\dots
        \begin{enumerate}
            \item Introduce a set of equations, to be solved in a
            nondeterministic order 
            \item Guard expressions with those equations 
            \item Flexibly express “proceed when any of these operations
            succeeds” with the new semantics of choice
            (Section~\ref{vmsemantics}). 
        \end{enumerate}

    Figure~\ref{fig:vminusfuncs} provides an example of how a programmer can
    utilize~\VMinus to solve the previous problems (Section~\ref{extensions}): 
    
    
    \begin{figure}[ht] 
      \begin{minipage}[h]{0.54\linewidth}
        \vmlst 
        \begin{lstlisting}[numbers=none, basicstyle=\tiny, xleftmargin=.2em,
          showstringspaces=false,
          frame=single]
val exclaimTall =~\sh.
  if E s. sh = Square s; (> s) 100 -> 
            Wow! That's A Sizeable Square!  
    [] E w h. sh = Triangle w h; (> h) 100 ->
            Goodness! Towering Triangle!
    [] E b1 b2 h. sh = Trapezoid b1 b2 h; 
     (> h) 100 -> Zounds! Tremendous Trapezoid!
    [] ->  Your Shape Is Small
  fi 
  \end{lstlisting}
      \Description{exclaimTall in VMinus}
      \subcaption{\tt{exclaimTall} in~\VMinus}
          \label{fig:vmexclaimtall} 
      %   \vspace{4ex}
      \end{minipage}%%
      \begin{minipage}[h]{0.5\linewidth}
        \vmlst 
        \begin{lstlisting}[numbers=none, basicstyle=\tiny, xleftmargin=2em,
                      frame=single]
val tripleLookup =~\rho.~\x.
  if E w y z v1 v2 v3. 
    v1 = (lookup rho) x; v1 = Some w; 
    v2 = (lookup rho) w; v2 = Some y; 
    v3 = (lookup rho) y; v3 = Some z 
       -> z 
    [] -> handleFailure x
  fi 
   \end{lstlisting}
        \Description{tripleLookup in VMinus}
        \subcaption{\tt{tripleLookup} in~\VMinus}
            \label{fig:vmtriplelookup} 
        \vspace{4ex}
      \end{minipage} 
      \begin{minipage}[h]{\linewidth}
        \vmlst 
        \begin{lstlisting}[numbers=none, basicstyle=\tiny, xleftmargin=9em,
          showstringspaces=false,
          frame=single]
val game_of_token =~\t. 
  if E f. (t = BattlePass f  | (t = ChugJug f | t = TomatoTown f)) -> 
        P (Fortnite f)
  [] E f. (t = HuntersMark f | t = SawCleaver f)                   -> 
        P (Bloodborne ((* 2) f))
  [] E f. (t = MoghLordOfBlood f | t = PreatorRykard f)            -> 
        P (EldenRing ((* 3) f))
  [] -> P (Irrelevant 0)
  fi 
\end{lstlisting}
      \Description{game_of_token in VMinus}
      \subcaption{\tt{game\_of\_token} in~\VMinus}
          \label{fig:vmgot}
      \vspace{4ex}
      \end{minipage}%% 
      \caption{The functions in~\VMinus have a desirably concise
      implementation, as before.}
  \label{fig:vminusfuncs}
    \end{figure}   

    \VMinus looks satisfyingly similar to pattern matching and to~\VC.
    The~\VMinus examples in Figure~\ref{fig:vminusfuncs} have the same number of
    cases as the pattern-matching examples, and share the existential and
    equations with the~\VC examples in Figure~\ref{fig:verseextfuncs}. 
   
% \subsection{Syntax of~\VMinus}

% \bigskip



% A~\it{name} is any token that is not an integer literal, 
% does not contain whitespace, a bracket, or parenthesis, 
% and is not a value constructor name or a reserved word.

\subsection{Formal Semantics of $\VMinus$:}
    In this section, I~present a big-step operational semantics for~\VMinus. The
    semantics describes how an expression in~\VMinus is evaluated and how an
    equation (and more generally, a guard) works in the language. Instead of a
    rewrite semantics that makes substitutions within guards,~\VMinus has a
    big-step semantics that directly describes how they are handled by the
    runtime core. Figure~\ref{fig:vmsyntax} contains the syntax of \VMinus,
    Figure~\ref{fig:vmmetavars} provides the metavariables used in the judgement
    forms and~rules of the semantics, and Section~\ref{vmsemantics} contains the
    forms and rules. Since solving guards is the heart of~\VMinus, I~also
    describe it in plain English.


\begin{table}
\begin{tabular}{ll}
\toprule
    \multicolumn2{l}{\emph{\VMinus Metavariables}}~\\
\midrule
    $\expr$& An expression~\\ 
    $\v,\;~\v'$& A value~\\
    $\fail$& An expression failure~\\
    $\result$& $\v~\vert~\fail$ : expressions produce~\it{results}: values or
    failure.~\\
    \Rho& An environment: $name~\rightarrow {\mathcal{V}}_{\bot}$~\\
    $\rho\{ x~\mapsto y~\} $& An environment extended with name $x$ mapping to $y$~\\
    $\g$& A guard~\\
    $\gs$& Zero or more guards, separated by~\it{;}~\\
    $\G$& A guarded expression~\\
    $\Gs$& Zero or more guarded expressions, separated by~\dbar~\\
    $eq$& An equation~\\ 
    % $\tempstuck$& a temporarily-stuck equation~\\
    $\dagger$& when solving guards is rejected~\\
    $s$& $\rhohat~\mid~\dagger$ : guards produce~\it{solutions}: a
    refined environment~\rhohat\; or rejection\\
    $\mathcal{T}$& A context of all temporarily stuck guards (a sequence)~\\ 
    % $G$& A guarded expression~\\
    %~\uppsidown& Inability to compile to a decision tree; a compile time error~\\
\bottomrule
\end{tabular}    
\Description{Metavariables used in rules for VMinus.}
\caption{\VMinus metavariables and their meanings}
\label{fig:vmmetavars}
\end{table}

\bigskip
    
    \subsubsection{Expressions}
    
    \newcommand\GNoTree{\vmrung~\rightsquigarrow~\uppsidown} 
    
    An expression in~\VC evaluates to produce possibly-empty sequence of values,
    where an empty sequence of values is identical to the syntactic form~\fail.
    In~\VMinus, an expression never returns multiple values, but it can~\fail.
    Specifically, in~\VMinus, an expression evaluates to produce a single~\it{result}.
    A result is either a single value $\v$ or~\fail. 
    
    \[\result\;~\rm{::=}\;~\v\;~\vert~\;~\fail~\]
    
    \showvjudgement{Eval}{\vmeval}

    \subsubsection{Names and refinement of environments}

    In~\VMinus, like in~\VC, names can be introduced with the existential
    {$\exists$} before they are given a binding. Bindings are given by equations
    in guards~(\ref{guards}). When a name is introduced with {$\exists$}, it is
    bound in the environment~\Rho to $\bot$ (pronounced “bottom”). Success of a
    guard only~\it{refines} the environment; that~is, when
    $\grefine[solution=\Rhoprime]$, we~expect $\rho~\subseteq~\Rhoprime$. The
    definition of $\subseteq$ on environments is given below. 
    
    \begin{align*}
    \rho~\subseteq~\Rhoprime~\text{ when }&\dom\rho  \subseteq~\dom~\Rhoprime\\
    \text{ and } &\forall x~\in~\dom~\rho:~\rho(x)~\subseteq~\Rhoprime(x)
    \end{align*}
    
    \medskip
        
    \subsubsection{Guards}
    \label{guards}

    For example, the~\VMinus expression {\tt{((}\ttbackslash\tt{x. x) K)}}
    succeeds and returns the value~\tt{K}. The~\VMinus expression~\tt{if  fi},
    the empty~\tt{if-fi}, always {\fail}s. 

    Like~\VC,~\VMinus has a nondeterministic semantics. Guards are solved in~\VMinus
    similarly to how equations are solved in Verse: the program nondeterministically
    picks one out of a context (\context), attempts to solve it, and moves on. 

In my semantics, this process occurs over a~\it{list} of guards~\gs in a guarded
expression \G: the program picks a guard from~\G, attempts to solve it to refine
the environment or fail, and repeats.~\VMinus can only pick a guard out of \G\
that it “knows” it can solve. “Knowing” a guard can be solved can be determined
in~\VMinus before code is executed. If~\VMinus can't pick a guard and there are
guards left over, the semantics gets stuck before code is executed. 

\showvjudgement{Solve-Guards}{\vmgs}


The environment $\rho$ maps from a name to a value {\v} or $\bot$. $\bot$ means
a name has been introduced with the existential, $\exists$, but is not yet bound
to a value. Given any such $\rho$, a guard~\g\ eithers refine $\rho$
($\Rhoprime$) or is~\bf{rejected}. We use the metavariable $\dagger$ to
represent rejection, and if any guard in a list is rejected, the entire list of
guards is rejected.

    \showvjudgement{Guard-Refine}{\grefine[solution=\rho']}
    \showvjudgement{Guard-reject}\gfail

  For example, in the~\VMinus expression~\tt{if E x. x = K; x = K2 -> x fi}, the
  existential (in concrete syntax,~\tt{E}) introduces~\tt{x} to~\Rho\ bound to
  $\bot$, producing the environment $\bracketed{x~\mapsto~\bot}$. The guard
  \tt{x = K} successfully unifies~\tt{x} with~\tt{K}, producing the environment
  $\bracketed{x~\mapsto~\tt{K}}$. The guard~\tt{x = K2} attempts to unify~\tt{K}
  with~\tt{K2} and is rejected with $\dagger$. 

  \subsection{Notable rules in the \VMinus semantics}

  Worth discussing in the~\VMinus semantics are those in which a name in \Rho\
  is bound to a value or to $\bot$. Notable among these include
  \textsc{Guard-Name-Exp-Bot}, \textsc{Guard-Name-Exp-Eq},
  \textsc{Guard-Name-Exp-Fail}, \textsc{Guard-Names-Bot-Succ},
  and~\textsc{Guard-Names-Bot-Succ-Rev}. This section discusses each of these
  rules in short detail. The full set of rules for \VMinus is in
  Section~\ref{vmsemantics}. 

  In this section, I use the terms~\it{known} and~\it{unknown} to denote a
  name's status in \Rho. If a name $x$ is bound to a value \v in \Rho, then $x$
  is \it{known}. If $x$ is bound to $\bot$ in \Rho, then $x$ is \it{unknown}. I
  use this terminology again when compiling~\VMinus to \D, since it describes
  the same status in both evaluation in compilation. 

  \bigskip 

\guardnameexpbot

When $\rho(x) = \bot$ ($x$ is \it{unknown}) and $\vmeval$, the program refines
\Rho by making a binding of $x$ to \v. Here,~\it{=} is treated like a
let-binding in ML-like languages. 

\guardnameexpeq

When $\rho(x) = \v$ ($x$ is \it{known}) and $\vmeval$, the program proceeds
without refining the environment, since no new information is gained.
Here,~\it{=} is treated like a \tt{==} in C-like languages. 

\guardnameexpfail

When $\rho(x) = \v'$ ($x$ is \it{known}) and $\vmeval$ and $\v \neq \v'$,
unification fails, so the guarded expression fails. Here,~\it{=} is treated like
a \tt{==} in C-like languages. 

\guardnamesbotsucc

When $\rho(x) = \v$ ($x$ is \it{known}) and $rho(y) = \bot$ ($y$ is
\it{unknown}), the program can always bind $y$ to \v. Here,~\it{=} is treated
like a let-binding in ML-like languages. 


\guardnamesbotsuccrev

This rule is simply an application of \textsc{Guard-Names-Bot-Succ} in reverse. 
    
    \vfilbreak
    
    \subsection{Rules (Big-step Operational Semantics) for~\U, shared by~\VMinus and~\D}
    \label{usemantics1}
    \usemantics 
    \subsection{Rules (Big-step Operational Semantics) specific to~\VMinus}
    \label{vmsemantics}
    \vmsemantics
\end{document}

\documentclass{article}
\usepackage{vmlmacros}
\usepackage{simplebnf}
\setcounter{secnumdepth}{1}

% \setlength{\grammarparsep}{20pt plus 1pt minus 1pt} % increase separation between rules
\setlength{\parindent}{0cm}
\title{Syntax and Semantics of $D$}
\author{Roger Burtonpatel}
\date{November 24, 2023}
\begin{document}

\maketitle

\section{Syntax}

We present a grammar of $D$, the language of decision trees: 


\begin{center}
  \begin{bnf}
    \Dalpha : \textsf{Decision Tree} ::= 
    $\tt{case}\; x\; \tt{of}\; 
    \bracketed{\vert\; K\bracketed{x}\; \tt{=>}\; \Dalpha}
    [\vert\; x\; \tt{=>} \Dalpha]$ : test node 
    | $\alpha$ : match node 
    | $\tt{if}\; x\; \tt{then}\; \Dalpha\; \tt{else}\; \Dalpha$ : condition with two children 
    | $\tt{let}\; x\; \tt{=}\; \expr\; \tt{in}\; \Dalpha$ : let-bind a name
    ;;
    $\expr$ : \textsf{Expressions} ::=
    | $x$ : name
    | $\mathcal{D}_{\expr}$ : decision trees 
    | $K \bracketed{\expr}$ : value constructor application 
    | $e_1\; e_2$ : function application 
    ;;
    % $K$ : \textsf{Value Constructors} ::=
    % \cons : cons 
    % | \tt{[]} : empty list 
    % | $\tt{\#}x$ : name beginning with \tt{\#}
    % | \tt{A-Z}$x$ : name beggining with capital letter
    % | $[\tt{-}\vert\tt{+}](\tt{0}-\tt{9})+$ : signed integer literal 
  \end{bnf}
\end{center}

\section{What is a decision tree?}
Scott, Ramsey 2000: 

A decision tree is a pattern-matching automaton in which every state except the
initial state has a unique predecessor. 

\bigskip

\it{More details will go here as needed.}

% Patterns and subject trees: 

% In functional languages, a pattern matcher takes a value and identifies the
% first of a list of patterns that matches the value. We call the value a subject
% tree or term. Patterns and subject trees are defined recursively. A subject tree
% is a constructor applied to a (possibly empty) list of subject trees. A pattern
% is either a variable or a constructor applied to a (possibly empty) list of
% patterns. A linear pattern is one in which no variable appears more than once.
% ML requires that patterns be linear, and our match compiler works only with
% linear patterns. 


% Let's define an ML datatype for our trees. 
% \begin{verbatim}
%     datatype 'a tree = MATCH of 'a * (path * name) list 
%                      | TEST of path * 'a edge list * 'a tree option 
%     withtype 'a edge = constructor * 'a tree 

%     (* From 106: *)

%     type register
%     type arity = int
%     type labeled_constructor = Pattern.vcon * arity
%     type pat = Pattern.pat
%     datatype 'a tree = TEST      of register * 'a edge list * 'a tree option
%                      | LET_CHILD of (register * int) * (register -> 'a tree)
%                      | MATCH     of 'a * register Env.env
%     and      'a edge = E of labeled_constructor * 'a tree
  
%     datatype path = REGISTER of register | CHILD of register * int
%       (* in order to match block slots, children should be numbered from 1 *)
  
%     type constraint = path * pat
%       (* (pi, p) is satisfied if the subject subtree at path pi matches p *)
  
%     datatype 'a frontier = F of 'a * constraint list
%       (* A frontier holds a set of constraints that apply to the scrutinee.
  
%          A choice's initial frontier has just one contraint: [(root, p)],
%          where root is the scrutinee register and p is the original pattern
%          in the source code.
  
%          A choice is known to match the scrutinee if its frontier
%          contains only constraints of the form (REGISTER t, VAR x).
%          These constraints show in which register each bound name is stored.
      
%          The key operation on frontiers is *refinement* (called `project`
%          in the paper).  Refing revises the constraints under the assumption
%          that a given register holds an application of a given labeled_constructor 
%        *)
  
  
% \end{verbatim}
\end{document}
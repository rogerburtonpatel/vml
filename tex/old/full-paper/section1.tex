\documentclass[manuscript,screen,review, 12pt, nonacm]{acmart}
\let\Bbbk\relax % Fix for amssymb clash 
\usepackage{vmlmacros}

\begin{document}

\section{Introduction}
% Subjects: Pattern matching and Equations 

Pattern matching is a beloved tool among functional programmers for examining
and deconstructing data. 
% It does so implicitly by matching constructed data
% directly against a number of possible forms.
% Maybe combine? 
It is also an established and well-researched topic \citep{wadler1987views, macqueen1985tree, burton1993pattern,
palao1996new, maranget, bpc}. Pattern matching is appreciated by programmers and
researchers alike for two main reasons: It enables \it{implicit} data
deconstruction, and it has a desirable cost model. Specifically (regarding the
latter), pattern matching can be compiled to a \it{decision tree}, a data
structure that enforces linear runtime performance by guaranteeing no part of
the data will be examined more than once.~\citep{maranget}

However, pattern matching cannot express certain common computations succinctly,
forcing programmers who wish to express these computations to duplicate code,
nest \it{case} expressions, and create multiple points of truth. To mitigate
this, designers of popular programming languages have introduced \it{extensions}
to pattern matching (Section~\ref{extensions}). 

Extensions strengthen pattern matching, but they are not standardized, so each
popular programming language with pattern matching features its own unique suite
of extensions. This makes extensions less of a ubiquitous feature of pattern
matching and more subject to the discretion of the individual language designer.
Rather than continuing to extend pattern matching across various frontiers
\it{ad hoc}, a worthwhile goal could be to find an alternative to pattern
matching that doesn't need extensions to succinctly solve the problems
programmers face. A tempting possibility for such an alternative was introduced
last year by the programming language Verse~\citep{verse}. In Verse, a
programmer can implicitly deconstruct data without pattern matching using a
different tool the language offers: \it{equations}. Equations are expressive and
flexible, and it appears that they can express everything that pattern matching
can, including with popular extensions. 

But a full implementation of Verse is complicated, cost-wise. Verse is a
functional logic programming language, and expressions can backtrack at runtime
and return multiple results, both of which are hard to predict in their costs. 

% A worthwhile goal could be to harmonize the expressive quality of Verse's
% equations with the decision tree property of patterns.

\bf{My central contribution in this thesis} is to show that the expressive
quality of Verse's equations and the decision-tree property of patterns can be
combined in a single language. Since the language is a streamlined adaptation of
Verse with a reduced feature set, I call it \VMinus (“V minus”). In this thesis,
I also show that \VMinus subsumes pattern matching with popular extensions. 

To support my central contribution, 
I have formalized pattern matching in a core language \PPlus (“P plus”) with a
big-step operational semantics (Section~\ref{pplus}),
I have formalized a subset of Verse into a core language \VMinus with a big-step
operational semantics (Section~\ref{vminus}), 
I have formalized decision trees into a core language \D (“D”) with a big-step
operational semantics (Section~\ref{d}), 
I have formalized a translation between
the languages (Sections \ref{vminustod} and~\ref{pplustovminus}), 
and I have implemented each language in Standard ML. 

\end{document}
\documentclass[manuscript,screen,review, 12pt, nonacm]{acmart}
\let\Bbbk\relax % Fix for amssymb clash 
\usepackage{vmlmacros}
\AtBeginDocument{%
  \providecommand\BibTeX{{%
    \normalfont B\kern-0.5em{\scshape i\kern-0.25em b}\kern-0.8em\TeX}}}
\usepackage{outlines}
\setlength{\headheight}{14.0pt}
\setlength{\footskip}{13.3pt}
\title{An Alternative to Pattern Matching, Inspired by Verse}
\author{Roger Burtonpatel}
\email{roger.burtonpatel@tufts.edu}
\affiliation{%
  \institution{Tufts University}
  \streetaddress{419 Boston Ave}
  \city{Medford}
  \state{Massachusetts}
  \country{USA}
  \postcode{02155}
}
\begin{document}



\section{Pattern Matching and equations}
\label{pmandequations}

In this section, I expand on the definitions, forms, and tradeoffs of pattern
matching and equations. These tradeoffs inform the compromises I make in
\VMinus down the line in Section~\ref{vminus}.

\subsection{Pattern matching}
\label{pmoverobservers}

% \1 What is pattern matching 

Pattern matching lets programmers examine and deconstruct data by matching them
against patterns. When a pattern $p$ matches with a value $v$, it can produce
bindings for any sub-values of $v$. For example, pattern $x::xs$ matches any 
application of the value constructor \it{cons} (\it{::}), and binds the first 
element of the cons cell to $x$ and the second to $xs$. 

Why use pattern matching? What could programmers use instead? One tool a
programmer might use to deconstruct data is \it{observers}
\citep{liskov:abstraction}: functions that explicitly inquire a piece of data's
structure and extract its components. Examples of observers in functional
programming languages include Scheme's \tt{null?}, \tt{car}, and \tt{cdr}, and
ML's \tt{null}, \tt{hd}, and \tt{tl}. But many functional programmers favor
pattern matching over observers. I demonstrate with an example and a claim. 

Consider a \it{shape} datatype in Standard ML, which represents shapes by their
dimensions: 

\medskip 
\begin{minipage}[t]{\textwidth}
    \begin{verbatim}
        datatype shape = SQUARE of real 
                       | TRIANGLE of real * real 
                       | TRAPEZOID of real * real * real
\end{verbatim}
\end{minipage}
\medskip 

I define an \tt{area} function on \it{shape}s, with this type and these
algebraic laws: 

\medskip 
\begin{minipage}[t]{\textwidth}
    \begin{verbatim}
        area : shape -> real 
        (* area (SQUARE s)              == s * s 
           area (TRIANGLE (w, h))       == 0.5 * w * h
           area (TRAPEZOID (b1, b2, h)) == 0.5 * b1 * b2 * h
        *)
\end{verbatim}
\end{minipage}
\medskip 

Now consider two implementations of \tt{area}, one with observers and one with
pattern matching, in Figure~\ref{fig:area}.

    \begin{figure}[H]
      \begin{minipage}[t]{0.7\textwidth}
        \begin{verbatim}
fun area sh =
  if isSquare sh
  then sqSide sh * sqSide sh
  else if isTriangle sh 
  then 0.5 * triW sh * triH sh
  else 0.5 * traB1 sh * traB2 sh * traH sh
            \end{verbatim}
            \Description{An implementation of area observers}
        \subcaption{\tt{area} with observers}
            \label{fig:observerarea} 
      \end{minipage}
      \vfill
      \begin{minipage}[t]{0.7\textwidth}
        % \centering 
        \begin{verbatim}
fun area sh =
  case sh 
    of SQUARE s              => s * s
     | TRIANGLE (w, h)       => 0.5 * w * h
     | TRAPEZOID (b1, b2, h) => 0.5 * b1 * b2 * h
                \end{verbatim}
       \Description{An implementation of area using implicit deconstruction via
       patterns.}
       \vspace{2.2em}
       \subcaption{\tt{area} with pattern matching}
       \label{fig:pmarea}
      \end{minipage}
      \caption{Implementing \tt{area} using observers is tedious, and the code
      doesn't look like the algebraic laws. Using pattern matching makes an
      equivalent implementation more appealing.}
      \label{fig:area}
    \end{figure}

    Implementing the observers \tt{isSquare}, \tt{isTriangle}, \tt{sqSide},
    \tt{triW}, \tt{traB1}, \tt{traB2}, and \tt{traH} is left as an
    (excruciating) exercise to the reader. 

    If that prospect doesn't convince you that pattern matching avoids a lot of
    the problems that observers have, I'll show five general reasons why
    programmers prefer pattern matching over observes. I refer to these as
    Nice~Properties for the rest of the paper. They are broken into two groups:
    Group~A, which contains properties of pattern matching that programmers
    enjoy in general, and Group~B,  which contains properties strictly to do
    with pattern matching's specific strengths over observers. 

      \begin{enumerate}
        \item [\textbf{A.}]
        \begin{enumerate}[label=\arabic*]
          \item With pattern matching, code more closely resembles algebraic laws. 
          \label{p1}
          \item With pattern matching, it's easier to avoid duplicating code.
          \label{p2}
          \item With pattern matching, a compiler may be able to tell if the
          code omits an important case through \it{exhaustiveness analysis}.
          \label{p5}
      \end{enumerate}
        
        \item [\textbf{B.}]
        \begin{enumerate}[start=4, label=\arabic*]
          \item Pattern matching plays nicer with constructed data. 
          \label{p3}
          \item With pattern matching, important intermediate values are always
          given a name. 
          \nolinebreak
          \label{p4}
        \end{enumerate}
      \end{enumerate}

    Nice Properties \ref{p1}~and~\ref{p2} are the most important of these: they
    allow programmers to write code that looks like what they write at the
    whiteboard, with flexible laws and minimal duplication. I show in
    Section~\ref{extensions} that extensions to pattern matching largely exist
    to uphold these two properties. 
    
    Let's see how each of our Nice Properties holds up in \tt{area}: 
    
    \begin{enumerate}
      \item [\textbf{A.}]
      \begin{enumerate}[label=\arabic*]
        \item[\ref{p1}] \ref{fig:pmarea}, which uses pattern matching, more
        closely resembles the algebraic laws for \tt{area}. 
        \item[\ref{p2}] \ref{fig:observerarea} had to call observers like
        \tt{squareSide} multiple times, and each observer needs \tt{sh} as an
        argument. \ref{fig:pmarea} was able to extract the \tt{shape}s' internal
        values with a single pattern, and the name \tt{sh} is not duplicated
        anywhere in its body. 
        \item[\ref{p5}] If the user adds another value constructor to
        \tt{shape}--- say, \tt{CIRCLE}, \ref{fig:observerarea} will not cause
        the compiler to complain, and if it's passed a \tt{CIRCLE} at runtime,
        the program will likely crash! In \ref{fig:pmarea}, the compiler will
        warn the user of the possibility of a \tt{Match} exception, and even
        tell them that they must add a pattern for \tt{CIRCLE} to rule out this
        possibility. 
    \end{enumerate}
      
    \item [\textbf{B.}]
      \begin{enumerate}[start=4, label=\arabic*]
        \item[\ref{p3}] Where did \tt{isSquare}, \tt{sqSide}, and all the other
        observers come from? To even \it{implement} \ref{fig:observerarea}, a
        programmer has to define a whole new set of observers for
        \tt{shape}s!\footnote{Sometimes the compiler throws programmers a bone:
        with some constructed data, i.e., Scheme's records, the compiler
        provides observers automatically. In others, like with algebraic
        datatypes in ML, it does not.} Most programmers find this tiresome
        indeed. \ref{fig:pmarea} did not have to do any of this.
        \item[\ref{p4}] To extract the internal values, \ref{fig:pmarea} had to
        name them, and their names serve as documentation. 
      \end{enumerate}
    \end{enumerate}


    Having had the chance to compare Figures \ref{fig:observerarea}
    and~\ref{fig:pmarea}, if you moderately prefer the latter, that's good: most
    functional programmers--- in fact, most \it{programmers}--- likely do as
    well. 

    Figure~\ref{fig:pmarea} provides an opportunity to introduce a few terms
    that are common in pattern matching. \ref{fig:pmarea}~is a classic example
    of where pattern matching most commonly occurs: within a \it{case}
    expression. A \it{case} expression tests a \it{scrutinee} (\tt{sh}
    in~\ref{fig:pmarea}) against a list of patterns (\tt{SQUARE s}, etc.). When
    the result of evaluating the scrutinee matches with a pattern, the program
    evaluates the right-hand side of the respective branch (\tt{s * s}, etc.)
    within the \it{case} expression.\footnote{OCaml, which you'll see in future
    sections, calls \it{case} \tt{match}. Some literature~\citep{guardproposal}
    calls this a \it{head expression}. I follow the example of~\citet{bpc} and
    \citet{maranget} in calling the things \it{case} and \it{scrutinee}. Any of
    these terms does the job.} 
% \1 Pattern matching has these properties, observers do not. 

\subsection{Popular extensions to pattern matching}
\label{extensions}

    Extensions to pattern matching simplify cases that are otherwise
    troublesome. Specifically, extensions help restore Nice Properties
    \ref{p1}~and~\ref{p2} in cases where pattern matching fails to satisfy them. 
    
    In this section, I illustrate several such instances of exactly this, and
    demonstrate how extensions help programmers write code that adheres to the
    Nice Properties. The three extensions I describe are those commonly found in
    the literature and implemented in compilers: side conditions, pattern
    guards, and or-patterns. 
    
    For the sake of comparison, I coin the term \it{bare pattern matching} to
    denote pattern matching \it{without} extensions: in bare pattern matching,
    the only syntactic forms of patterns are names and applications of value
    constructors to zero or more patterns. 


% Subject: Extensions to pattern matching. 

\subsubsection{Side conditions}

    First, I illustrate why programmers want \it{side conditions}, an extension
    to pattern matching common in most popular functional programming languages,
    including OCaml, Erlang, Scala, and~Haskell\footnote{I use the term \it{side
    conditions} to refer to a pattern followed by a boolean expression. Some
    languages call this a \it{guard}, which I use to describe a different, more
    powerful extension to pattern matching in Section~\ref{guards}.  
    Technically, Haskell \it{only} has guards, but they subsume side conditions,
    so I hand-wavingly say that it does have side conditions.}. 
    
    I'll define a (rather silly) function \tt{exclaimTall} in OCaml on
    \tt{shape}s. I'll have to translate our \tt{shape} datatype to OCaml, and
    while I'm at it, I'll write the type and algebraic laws for
    \tt{exclaimTall}:

    \begin{minipage}[t]{\textwidth}        
        \centering 
        \begin{verbatim}

type shape = Square of float 
           | Triangle of float * float
           | Trapezoid of float * float * float

exclaimTall : shape -> string 
(*
exclaimTall (Square s)              == "Wow! That's a tall square!", 
                                        where s > 100.0
exclaimTall (Triangle (w, h))       == "Goodness! Towering triangle!",
                                        where h > 100.0
exclaimTall (Trapezoid (b1, b2, h)) == "Zounds! Tremendous trapezoid!", 
                                        where h > 100.0
exclaimTall sh                      == "Your shape is small.", 
                                        otherwise
*)
    \end{verbatim}
    \end{minipage}

    Armed with pattern matching, I'll try to implement \tt{exclaimTall} in OCaml
    (Figure~\ref{fig:ifexclaimtall}).

    \begin{figure}[ht]
        \begin{verbatim}
            let exclaimTall sh =
            match sh with 
              | Square s -> if s > 100.0 
                            then "Wow! That's a tall square!"
                            else "Your shape is small." 
              | Triangle (_, h) -> 
                            if h > 100.0 
                            then "Goodness! Towering triangle!"
                            else "Your shape is small." 
              | Trapezoid (_, _, h) -> 
                            if h > 100.0
                            then "Zounds! Tremendous trapezoid!"
                            else "Your shape is small." 
            \end{verbatim}    
        \caption{An invented function \tt{exclaimTall} in OCaml combines pattern
        matching with an \tt{if} expression, and is not very pretty.}   
        \Description{An implementation of a function exclaimTall in OCaml, which
        uses pattern matching and an if statement.}
        \label{fig:ifexclaimtall}
    \end{figure}
    
    Here, I'm using the special variable \tt{\_}--- that's the underscore
    character, a wildcard pattern--- to indicate that I don't care about the
    bindings of a pattern. 

    I (and hopefully you, the reader) am not thrilled with this code. It gets
    the job done, but it fails to adhere to Nice Properties \ref{p1}
    and~\ref{p2}: the code does not look like the algebraic laws, and it
    duplicates right-hand side, \tt{"Your shape is small"}, three times. I find
    the code unpleasant to read, too: the actual “good” return values of the
    function, the exclamatory strings, are gummed up in the middle of the
    \tt{if-then-else} expressions.
    
    Fortunately, this code can be simplified by using the \tt{shape} patterns
    with a \it{side condition}, i.e., a syntactic form for “match a pattern
    \it{and} a boolean condition.” The \tt{when} keyword in OCaml provides such
    a form, as seen in Figure~\ref{fig:whenexclaimtall}.
        
        \begin{figure}[]
            \begin{verbatim}
                let exclaimTall sh =
                  match sh with 
                    | Square s when s > 100.0 ->
                            "Wow! That's a tall square!"
                    | Triangle (_, h) when h > 100.0 ->
                            "Goodness! Towering triangle!"
                    | Trapezoid (_, _, h) when h > 100.0 -> 
                            "Zounds! Tremendous trapezoid!"
                    | _ ->  "Your shape is small." 
                \end{verbatim}
            \caption{With a side condition, \tt{exclaimTall} in OCaml becomes
            simpler and more adherent to the Nice Properties.} 
            \Description{An implementation of a function exclaimTall in OCaml,
            which uses pattern matching and a side condition.}
            \label{fig:whenexclaimtall}
        \end{figure}

    A side condition streamlines pattern-and-boolean cases and minimize
    overhead, restoring Nice Properties \ref{p1}~and~\ref{p2}. And a side
    condition can exploit bindings that emerge from the preceding pattern match.
    For instance, the \tt{when} clauses in Figure~\ref{fig:whenexclaimtall}
    exploit names \tt{s} and \tt{h}, which are bound in the match of \tt{sh} to
    \tt{Square s}, \tt{Triangle (\_, h)}, and \tt{Trapezoid (\_, \_, h)},
    respectively. 

    Importantly, side conditions come at a cost: their inclusion means that
    keeping Nice Property~\ref{p5} becomes an NP-hard problem, because the
    compiler must now perform exhaustiveness analysis not only on patterns, but
    on arbitrary expressions. Modern compilers give a weaker form of
    exhaustiveness that only deals with patterns, and side conditions are worth
    the tradeoff for restoring the two most important of the Nice Properties:
    minimal code duplication and ease of translation from laws to code
    (\ref{p1}, \ref{p2}).

    A side condition can incorporate an extra “check”- in this case, a boolean
    expression--- within a pattern. But side conditions have a limitation. The
    check can make a decision based off of an expression evaluating to \tt{true}
    or \tt{false}. But it can't make a decision based off an arbitrary pattern
    match, and it can't bind names for the programmer to use in the right-hand
    side. In the next section, I showcase when this limitation matters, and how
    another extension addresses it. 

    \subsubsection{Pattern guards}
    \label{guards}

    To highlight a common use of pattern guards to address such a limitation, I
    modify an example from \citet{guardproposal}, the proposal for pattern
    guards in GHC. Suppose I have an OCaml abstract data type of finite maps,
    with a lookup operation: 

    \begin{minipage}[t]{\textwidth}
        \centering 
        \begin{verbatim}
            lookup : finitemap -> int -> int option
        \end{verbatim}
    \end{minipage}
    Let's say I want to perform three successive lookups, and call a “fail”
    function if \it{any} of them returns \tt{None}. Specifically, I want a
    function with this type and algebraic laws: 

    \begin{minipage}[t]{\textwidth}
        \centering 
        \begin{verbatim}
          

            tripleLookup : finitemap -> int -> int

              tripleLookup rho x == z, where 
                                        lookup rho x == Some w
                                        lookup rho w == Some y
                                        lookup rho y == Some z
              tripleLookup rho x == handleFailure x, otherwise
            
              handleFailure : int -> int 
              (* handleFailure's implementation is unimportant.
                handleFailure (x : int) = ... some error-handling ... -> x *)  

        \end{verbatim}
    \end{minipage}

    To express this computation succinctly, the program needs to make decisions
    based on how successive computations match with patterns, but neither bare
    pattern matching nor side conditions don't give that flexibility. 
    
    Side conditions don't appear to help here, so I'll try with bare pattern
    matching. Figure~\ref{fig:pmtriplelookup} shows how I might implement
    \tt{tripleLookup} as such. 

    \begin{figure}[ht]
        \begin{verbatim}

          let tripleLookup (rho : finitemap) (x : int) =
              match lookup rho x with Some w -> 
                (match lookup rho w with Some y -> 
                  (match lookup rho y with Some z -> z
                  | _ -> handleFailure x)
                | _ -> handleFailure x)
              | _ -> handleFailure x
            \end{verbatim}
        \caption{\tt{tripleLookup} in OCaml with bare pattern matching breaks
                    Nice Property~\ref{p2}: avoiding duplicated code. } 
                    
        \Description{An implementation of the tripleLookup with three levels of
        nested pattern matching.}
        \label{fig:pmtriplelookup}
    \end{figure}

    Once again, the code works, but it's lost Nice Properties \ref{p2}
    and~\ref{p1} by duplicating three calls to \tt{handleFailure} and stuffing
    the screen full of syntax that distracts from the algebraic laws.
    Unfortunately, it's not obvious how a side condition could help us here,
    because we need pattern matching to extract and name internal values from
    constructed data.

    To restore the Nice Properties, I'll introduce a more powerful extension to
    pattern matching: \it{pattern guards}, a form of “smart pattern” in which
    intermediate patterns bind to expressions within a single branch of a
    \tt{case}. Pattern guards can make \tt{tripleLookup} appear \it{much}
    simpler, as shown in Figure~\ref{fig:guardtriplelookup}--- which, since
    pattern guards aren't found in OCaml, is written in Haskell.

    \begin{figure}
        \begin{center}
        \begin{verbatim}
                        tripleLookup rho x
                           | Just w <- lookup rho x
                           , Just y <- lookup rho w
                           , Just z <- lookup rho y
                           = z
                        tripleLookup _ x = handleFailure x
        \end{verbatim}
        \end{center}    
    \caption{Pattern guards swoop in to restore the Nice Properties, and all is
    right again.} 
    \Description{An implementation of nameOf using pattern guards.}
    \label{fig:guardtriplelookup}
    \end{figure}

    Guards appear as a comma-separated list between the \tt{|} and the \tt{=}.
    On the left-hand side of the \tt{<-} is a pattern, and on the right is an
    expression. At runtime, the program processes a guard by evaluating the
    expression and testing if it matches with the pattern. If it does, it
    processes the next guard with any bindings introduced by processing guards
    before it. If it fails, the program skips evaluating the rest of the branch
    and falls through to the next one. As a bonus, a guard can simply be a
    boolean expression which the program tests the same way it would a side
    condition, so guards subsume side conditions! 
    
    If you need further convincing on why programmers want for guards, look no
    further than Erwig \& Peyton Jones's \it{Pattern Guards and Transformational
    Patterns}~\citep{guardproposal}, the proposal for pattern guards in GHC: the
    authors show several other examples where guards drastically simplify
    otherwise-maddening code. 
    
    The power of pattern guards lies in the ability for expressions within
    guards to utilize names bound in preceding guards, enabling imperative
    pattern-matched steps with expressive capabilities akin to Haskell's \tt{do}
    notation. It should come as no surprise that pattern guards are built in to
    GHC. \raggedbottom
\subsubsection{Or-patterns}
    I conclude our tour of extensions to pattern matching with or-patterns,
    which are built in to OCaml. Let's consider a final example. I have a type
    \tt{token} which represents a token in a video game and how much fun it is,
    and need to quickly know what game it's from and how much fun I'd have
    playing it. To do so, I'm going to write a function \tt{game\_of\_token} in
    OCaml. Here are the \tt{token} type and the type and algebraic laws for
    \tt{game\_of\_token}. 
  
\begin{minipage}[t]{\textwidth}
% \begin{figure}[t]
    \centering 
    \begin{footnotesize}
      \begin{verbatim}  
type funlevel = int

type token = BattlePass  of funlevel | ChugJug    of funlevel | TomatoTown of funlevel
           | HuntersMark of funlevel | SawCleaver of funlevel
           | MoghLordOfBlood of funlevel | PreatorRykard of funlevel
           ... other tokens ...
                    

game_of_token : token -> string * funlevel

game_of_token t == ("Fortnite", f),  where t is any of 
                                      BattlePass f, 
                                      ChugJug f, or
                                      TomatoTown f
game_of_token t == ("Bloodborne", 2 * f), 
                                      where t is any of 
                                        HuntersMark f or 
                                        SawCleaver f
game_of_token t == ("Elden Ring", 3 * f), 
                                      where t is any of 
                                        MoghLordOfBlood f or  
                                        PreatorRykard f
game_of_token _ == ("Irrelevant", 0), otherwise
      \end{verbatim}
    \end{footnotesize}
    \Description{Type and laws for game_of_token.} 
  \end{minipage}  
%     \caption{Type and laws for \tt{game\_of\_token}, which make great use of
%     "where."}
%     \ref{fig:gottypelaws}
% \end{figure}        
        
        I can write code for \tt{game\_of\_token} in OCaml using bare patterns
        (Figure~\ref{fig:baregot}), but I'm dissatisfied with how it fails the
        first three (\ref{p1}, \ref{p2}, \ref{p3}) of the Nice Properties: it
        has many duplicated right-hand sides, it is visually different from the
        algebraic laws, and the function, even though it uses pattern matching,
        doesn't really play nice with my custom type; deconstructing it is
        tedious and redundant.         
        
        \begin{figure}
            \begin{center}
                \begin{verbatim}
        let game_of_token token = match token with 
            | BattlePass f      -> ("Fortnite", f)
            | ChugJug f         -> ("Fortnite", f)
            | TomatoTown f      -> ("Fortnite", f)
            | HuntersMark f     -> ("Bloodborne", 2 * f)
            | SawCleaver f      -> ("Bloodborne", 2 * f)
            | MoghLordOfBlood f -> ("Elden Ring", 3 * f)
            | PreatorRykard f   -> ("Elden Ring", 3 * f)
            | _                 -> ("Irrelevant", 0)
                \end{verbatim}
            \end{center}    

        \caption{\tt{game\_of\_token}, with redundant right-hand sides,
        should raise a red flag.} 
        \Description{An implementation of game_of_token using bare patterns.}
        \label{fig:baregot}
        \end{figure}

        I could try to use a couple of helper functions to reduce clutter, and
        write something like Figure~\ref{fig:helpergot}. It looks ok, but I'm
        still hurting for Nice Properties \ref{p2} and \ref{p3}, and the
        additional calls hurt performance. 

        \begin{figure}
            \begin{center}
                \begin{verbatim}
                  let fortnite   f = ... complicated   ... in
                  let bloodborne f = ... complicated'  ... in
                  let eldenring  f = ... complicated'' ... in
                  match token with
                  | BattlePass f -> fortnite f
                  ... and so on ...                
                \end{verbatim}
            \end{center}    
        \caption{\tt{game\_of\_token} with helpers is somewhat better, but I'm
        not satisfied with it.} 
        \Description{An implementation of
        game_of_token using helper functions.}
        \label{fig:helpergot}
        \end{figure}

        Thankfully, an extension once again comes to the rescue.
        \it{Or-patterns} condense multiple patterns which share a right-hand
        side, and when any one of the patterns matches with the scrutinee, the
        right-hand side is evaluated with the bindings created by that pattern.
        I exploit or-patterns in Figure~\ref{fig:orgot} to restore the Nice
        Properties and eliminate much of the uninteresting code that appeared in
        \ref{fig:baregot}~and~\ref{fig:helpergot}. 

    \begin{figure}
    \begin{center}
    \begin{verbatim}
let game_of_token token = match token with 
    | BattlePass f | ChugJug f | TomatoTown f  -> ("Fortnite", f)
    | HuntersMark f | SawCleaver f             -> ("Bloodborne", 2 * f)
    | MoghLordOfBlood f | PreatorRykard f      -> ("Elden Ring", 3 * f)
    | _                                        -> ("Irrelevant", 0)
    \end{verbatim}
    \end{center}    
    \caption{Or-patterns condense \tt{game\_of\_token}
    significantly, and it is easier to read line-by-line.} 
    \Description{An
    implementation of game_of_token using or-patterns.}
    \label{fig:orgot}
    \end{figure}

    In addition to the inherent appeal of brevity, or-patterns serve to
    concentrate complexity at a single juncture and create single points of
    truth. And by eliminating helper functions, like the ones in
    Figure~\ref{fig:helpergot}, or-patterns actually boost performance.
      
    \subsubsection{Wrapping up pattern matching and extensions}
    
    I have presented three popular extensions that make pattern matching more
    expressive and how to use them effectively. Earlier, though, you might have
    noticed a problem. Say I face a decision-making problem that obliges me to
    use all of these extensions in unison. When picking a language to do so, I
    am stuck! Indeed, no major functional language has all three of these
    extensions. Remember when I had to switch from OCaml to Haskell to use
    guards, and back to OCaml for or-patterns? The two extensions are mutually
    exclusive in Haskell and OCaml, and also Scala, Erlang/Elixir, Rust, F\#,
    and Agda~\citep{haskell, ocaml, scala, erlang, elixir, rust, fsharp, agda}. 


    I find the extension story somewhat unsatisfying. At the very least, I want
    to be able to use pattern matching, with the extensions I want, in a single
    language. Or, I want an alternative that gives me the expressive power of
    pattern matching with these extensions. 

    
    \subsection{Verse's equations}
    \label{verseoverobservers}

    An intriguing alternative to pattern matching exists in \it{equations}, from
    the Verse Calculus (\VC), a core calculus for the functional logic
    programming language \it{Verse}~\citep{antoy2010functional,
    hanus2013functional, verse}. (For the remainder of this paper, I use “\VC”
    and “Verse” interchangeably.)

    Equations present a different, yet powerful, way to write code that makes
    decisions and deconstructs data. In this section, I will introduce you to
    equations similarly to I how I introduced to you pattern matching. Once
    you're familiar with equations, you'll be ready to compare their strengths
    and weaknesses with those of pattern matching, and judge the compromise I
    propose. 

    Even if you read~\citet{verse}, \VC's equations and the paradigms of
    functional logic programming might look unfamiliar. To help ease you into
    familiarity with equations, I'll ground explanations and examples in how
    they relate to pattern matching. 

    \VC uses \it{equations} instead of pattern matching to test for structural
    equality and create bindings. Like pattern matching, equations scrutinize
    and deconstruct data at runtime by testing for structural equality and
    unifying names with values. Unlike pattern matching, \VC's equations can
    unify names on both left--- \it{and} right-hand sides. 

    Every equation in Verse takes the form \it{x = e}, where \it{x} is a name
    and \it{e} is an expression. During runtime, \VC relies on a process called
    \it{unification} to attempt to bind \it{x} and any unbound names in \it{e}
    to values. Unification is the process of finding a substitution that makes
    two different logical atomic expressions identical. Much like pattern
    matching, unification can fail if the runtime attempts to bind incompatible
    values or structures (i.e., finds no substitution). 

    Equations offer a form of binding that looks like a single pattern match.
    What about a list of many patterns and right-hand sides, as in a case
    expression? For this, \VC has \it{choice} (\choice). The full semantics of
    choice are too complex to cover here, but you can get away with knowing that
    choice, when combined with the \tt{one} operator, has a very similar
    semantics to case; that is, “proceed and create bindings if any one of these
    computations succeed.” 

    Let's look at what equations, \tt{one}, and choice look like in Verse. I've
    written our \tt{area} function in \VC extended with a \tt{float} type and a
    multiplication operator \tt{*} in Figure~\ref{fig:versearea}. 

    \begin{figure}[]
        \verselst
        \begin{lstlisting}[numbers=none]
$\exists$ area. area = $\lambda$ sh. 
  one {  $\exists$ s. sh   = $\langle$SQUARE, s$\rangle$; s * s
       | $\exists$ w h. sh = $\langle$TRIANGLE, w, h$\rangle$; 
              0.5 * w * h
       | $\exists$ b1 b2 h. sh = $\langle$TRAPEZOID, b1, b2, h$\rangle$; 
              0.5 * b1 * b2 * h}
        \end{lstlisting}
    \caption{\tt{area} in \VC uses existentials and equations rather than
    pattern matching.} 
    \Description{An implementation of area in Verse.}
    \label{fig:versearea}
    \end{figure}

    Like in the pattern-matching example in~\ref{fig:pmarea}, the right-hand
    sides of \tt{area} are \it{guarded} by a “check;” now, the check is
    successful unification in an equation rather than a successful match on a
    pattern. Similarly, \tt{one} with a list of choices represents matching on
    any \it{one} pattern to evaluate a single right-hand side. 

    Why use equations? I begin with a digestible claim: \VC's equations are
    preferable to observers. This claim mirrors my argument for pattern
    matching, and to support it I appeal to the Nice Properties: 

    \begin{enumerate}
      \item \tt{area} using equations looks like the algebraic laws, with the
      addition of the explicit $\exists$. It relies more on mathematical
      notation, but that might not be a bad thing: though it doesn't resemble
      the algebraic laws a programmer would write, it likely resembles the
      equations that a mathematician would. 
      \item \tt{area} using equations does not duplicate virtually any code. 
      \item \tt{area} using equations deconstructs user-defined types as easily
      as \ref{fig:pmarea} does with pattern matching. 
      \item \tt{area} using equations has all important internal values named
      very explicitly.
      \item This Property may not hold, because \VC on its own is untyped.
      Without a type system, a compiler cannot help me with non-exhaustive
      cases. However, there is no published compiler, type system or no, for
      Verse, and only when one is made available can I make this assertion. For
      this reason, and for the sake of focusing on more important details, I
      choose to proceed in my analysis of equations in \VC without considering
      this Property. 
    \end{enumerate}

    If you still believe these Properties to be desirable, you understand why I
    claim programmers prefer equations to observers. Now I'll make a stronger
    claim: equations are \it{at least as good as} pattern matching with popular
    extensions. How can I claim this? By appealing again to the Nice Properties!
    In~Section~\ref{pmoverobservers}, I demonstrated how pattern matching had to
    resort to extensions to regain the Properties when challenging examples
    stole them away. In Figure~\ref{fig:verseextfuncs}, I've implemented those
    examples in \VC (this time extended with strings, floats, and \tt{*}) using
    choice and equations. Take a look for yourself! 

    \begin{figure}[ht] 
        \begin{minipage}[h]{0.54\linewidth}
          \verselst
          \begin{lstlisting}[numbers=none, basicstyle=\tiny, xleftmargin=.2em,
                            showstringspaces=false,
                            frame=single]
$\exists$ exclaimTall. exclaimTall = $\lambda$ sh. 
  one { 
      $\exists$ s. sh = $\langle$Square s$\rangle$; 
      s > 100.0; "Wow! That's a tall square!"
    | $\exists$ w h. sh = $\langle$Triangle, w, h$\rangle$; 
      h > 100.0; "Goodness! Towering triangle!"
    | $\exists$ b1 b2 h. sh = $\langle$Trapezoid, b1, b2, h$\rangle$;
      h > 100.0; "Zounds! Tremendous trapezoid!"
    | "Your shape is small." }
            \end{lstlisting}
                \Description{exclaimTall}
        \subcaption{\tt{exclaimTall} in \VC }
            \label{fig:verseexclaimtall} 
        %   \vspace{4ex}
        \end{minipage}%%
        \begin{minipage}[h]{0.5\linewidth}
          \verselst
          \begin{lstlisting}[numbers=none, basicstyle=\tiny, xleftmargin=2em,
                        frame=single]
$\exists$ tripleLookup. tripleLookup = $\lambda$ rho x. 
    one { $\exists$ w. lookup rho x = $\langle$Just w$\rangle$; 
          $\exists$ y. lookup rho w = $\langle$Just y$\rangle$; 
          $\exists$ z. lookup rho y = $\langle$Just z$\rangle$;
          z
        | handleFailure x }
          \end{lstlisting}
                \Description{exclaimTall}
            \subcaption{\tt{tripleLookup} in \VC }
            \label{fig:versetriplelookup} 
        \vspace{4ex}
        \end{minipage} 
        \begin{minipage}[h]{\linewidth}
          \verselst
          \begin{lstlisting}[numbers=none, basicstyle=\tiny, xleftmargin=9em, 
                            frame=single, showstringspaces=false]
$\exists$ game_of_token. game_of_token = $\lambda$ token. 
  $\exists$ f. one {
         token = one { $\langle$BattlePass, f$\rangle$ | $\langle$ChugJug, f$\rangle$ | $\langle$TomatoTown, f$\rangle$}; 
           $\langle$"Fortnite", f$\rangle$
       | token = one { $\langle$HuntersMark, f$\rangle$ | $\langle$SawCleaver, f$\rangle$}; 
           $\langle$"Bloodborne", 2 * f$\rangle$
       | token = one { $\langle$MoghLordOfBlood, f$\rangle$ | $\langle$PreatorRykard, f$\rangle$}; 
           $\langle$"Elden Ring", 3 * f$\rangle$
       |   $\langle$"Irrelevant", 0$\rangle$ }
          \end{lstlisting}
                \Description{game_of_token in \VC}
        \subcaption{\tt{game\_of\_token} in \VC }
            \label{fig:versegot} 
        \vspace{4ex}
        \end{minipage}%% 
    \caption{Code for the \ref{pmoverobservers} functions with equations looks
    similar, and it doesn't need extensions.}
    \label{fig:verseextfuncs}
      \end{figure}
        
    The code in Figure~\ref{fig:verseextfuncs} has all the Nice Properties
    (again, disregarding the ambiguous 5th.) This is promising for \VC. If it
    rivals pattern matching with popular extensions in desirable properties, and
    \VC does everything using only equations and choice, it seems like the
    language is a strong option for writing code! 

    \subsection{\VC has a challenging cost model}
    \label{vcbadcost}

    So what's the catch? Programmers everywhere have not thrown up their hands,
    renounced pattern matching, and adopted a puritanical dogma of equations. 
    Sadly, this is not merely a matter of preference. 

    \VC's equations appear to be comparably pleasing to pattern matching in
    their brevity and expressiveness. However, full Verse allows computations
    that are problematic, cost-wise. In \VC, names (logical variables) are
    \it{values}, and can just as easily be the result of evaluating an
    expression as an integer or tuple. To bind these names, \VC, like other
    functional logic languages, relies on \it{unification} of its logical
    variables and \it{search} at runtime to meet a set of program
    constraints~\citep{antoy2010functional, hanus2013functional}. Combining
    unifying logical variables with search at runtime classically requires
    backtracking, which can lead to exponential runtime
    cost~\citep{hanus2013functional, wadler1985replace, clark1982introduction}. 

    % Todo: An example would be nice. Backtracking/evaluating backwards, then
    % done. 

    Pattern matching, by contrast, has a desirable cost model. \citet{maranget}
    showed that pattern matching can be compiled to a \it{decision tree}, a data
    structure that enforces linear runtime performance by guaranteeing no part
    of the scrutinee will be examined more than once. A decision tree, however,
    forbids backtracking by nature: once the program makes a decision based on
    the form of a value, it can't re-test it later with new information. 
    

\end{document}
\documentclass[]{article}
\usepackage{vmlmacros}
\usepackage{syntax}
\usepackage{relsize}
% \usepackage{palatino} % I don't love this, to be honest. 
\usepackage{amsmath}
\usepackage{booktabs}
\usepackage{simplebnf}
\setcounter{secnumdepth}{1}


% \setlength{\grammarparsep}{20pt plus 1pt minus 1pt} % increase separation between rules
\setlength{\grammarindent}{10em} % increase separation between LHS/RHS
\setlength{\parindent}{0cm}
\title{Translation from {\PPlus} to {\VMinus}}
\author{Roger Burtonpatel}
\begin{document}

\maketitle

\subsection{Domains}
I give the names and domains of the translation functions: 

\begin{align*}
    &\PtoVTran: \PPlus\ Exp\; ->\; \VMinus\ Exp \\
    &\PTran: Pattern\; ->\; Name\; ->\; Name\ list\ *\ Guard\ list \\
    % &\mathcal{B}: Pattern\; ->\; Name\; Set \\
    \end{align*}

The translation functions \PtoVTran\ and \PTran\ are defined case by case: 

% \subsection{Binding names}

% \begin{align*}
%     &\Bindings{x} = \bracketed{x} \\ 
%     &\Bindings{K} = \bracketed{} \\
%     &\Bindings{K\; p_{1} {\dots} p_{n}} = \Bindings{p} \cup {\dots} \cup \Bindings{p_{n}} \\
%     &\Bindings{\porp} = \Bindings{p_{1}} \cap \Bindings{p_{2}} \\
%     &\Bindings{\pcommap} = \Bindings{p} \cup \Bindings{p'} \\
%     &\Bindings{\parrowe} = \Bindings{p} \\
%     &\Bindings{\whenexpr} = \bracketed{}
% \end{align*}

\subsection{Translating Expressions}

\newcommand\btran[1]{\mathcal{B}[\![#1]\!]}

\begin{align*}
    &\ptov[exp=x, result=x] \\
    &\ptov[exp={K\; \expr[1] {\dots} \expr[n]}, result={K\; \ptovtran{\expr[1]} {\dots} \ptovtran{\expr[n]}}] \\
    &\ptov[exp={\lambda x.\; \expr}, result={\lambda x.\; \ptovtran{\expr}}] \\
    &\ptov[exp={\expr[1]\; \expr[2]}, result={\ptovtran{\expr[1]}\; \ptovtran{\expr[2]}}] \\
    % &\ptov[exp={\tt{case}\; \expr\;  \emptyseq}, result={{\iffitt{\vexists{x}\; x = \ptovtran{\expr};\; \iffitt{}}}}]\; \rm{, $x$ fresh }   \\
    &\ptovtran{\tt{case}\; \expr\;  p_{1}\; \expr[1] \vert {\dots} \vert p_{n}\; \expr[n]} \rightsquigarrow \\
    & \rm{$\forall i.\; 1 \leq i \leq n:$} \\
    & \tt{if } {\vexists{x}\; x = \ptovtran{\expr};}\; \\
    & \rm{ let } (\mathit{ns}_{1}, \mathit{gs}_{1}) {\dots} (\mathit{ns}_{i}, \mathit{gs}_{i}) = \ptran{p_{1}}x\; \cdot\; {\dots} \cdot\; \ptran{p_{i}}x \rm { in } \\
    & \iffitt{\vexists{\mathit{ns}_{1}}\; {\mathit{gs}_{1}} \rightarrow \ptovtran{\expr[1]};\;
    \square\; {\dots} \square\; \vexists {\mathit{ns}_{i}}\; {\mathit{gs}_{i}} \rightarrow \ptovtran{\expr[i]}} \\
    & \tt{fi} \\
    &\rm{, $x$ fresh }
\end{align*}

\rab{Questions: What notes do you have on the formatting? How might I
differentiate between the two kinds of equal sign (object language equations and
metalanguage \it{let})? How is my use of oxford brackets? Do you see any obvious
places to insert or remove them?}

\subsection{Translating Patterns}

\begin{align*}
    &\pattov[pat=y, result={(y, [x = y])}] \\
    &\pattov[pat=K, result={([], [x = K])}] \\
    &\ptran{K\; p_{1}\; {\dots}\; p_{n}}x \rightsquigarrow \\
    & \rm{$\forall i.\; 1 \leq i \leq n:$} \\
    & \rm{ let } y_{i} \rm{ be a fresh name, }  \\
    & (\mathit{ns}_{1}, \mathit{gs}_{1}) {\dots} (\mathit{ns}_{i}, \mathit{gs}_{i}) = \ptran{p_{1}}y_{1} \cdot {\dots} \cdot \ptran{p_{i}}y_{i} \\
    & \rm{ in } \\
    & ({\mathit{ns}_{1} \cdot {\dots} \cdot \mathit{ns}_{i}} \cdot {y_{1} {\dots} y_{i}}, x = K\; y_{1}\; {\dots}\; y_{i} \cdot \; \mathit{gs}_{1} \cdot {\dots} \cdot \mathit{gs}_{i}) \\
    &\pattov[pat=\mathit{when}\; e, result={([], [\ptovtran{e}])}] \\
    &\pattov[pat=\pcommap, 
    result={\rm{let } 
    {(\mathit{ns}_{1}, \mathit{gs}_{1}) = \ptran{p}x}\; ; 
    {(\mathit{ns}_{2}, \mathit{gs}_{2}) = \ptran{p'}x} \rm{ in }
    (\mathit{ns}_{1} \cdot \mathit{ns}_{2}, \mathit{gs}_{1} \cdot \mathit{gs}_{2})}] \\
    &\pattov[pat=\porp, 
    result={\rm{let } (\mathit{ns}_{1}, \mathit{gs}_{1}) = \ptran{p}x\; ;
    (\mathit{ns}_{2}, \mathit{gs}_{2}) = \ptran{p'}x \rm{ in }
    (\mathit{ns}_{1} \cdot \mathit{ns}_{2}, [\mathit{gs}_{1} \choice \mathit{gs}_{2}])}]
\end{align*}


% \rab{The vconapp rule is a formatting disaster. How do I fix it.}

Let's make some claims: 

\begin{itemize}
    \item Equational translation of everything outside of case is purely equivalent 
    \item Inductive hypothesis could be over patterns? 
    \item Semantics: when \prun, \vmrung
\end{itemize}

\newcommand\translatesto\rightsquigarrow

\newcommand\ep{\ensuremath{e_{p}}}
\newcommand\ev{\ensuremath{e_{v}}}
\newcommand\nsgs{\ensuremath{(ns, gs)}}


Let \ep\ be a \PPlus\ expression, \ev\ be a \VMinus\ expression, and \Rho\ be a
specific but arbitrarily chosen environment. If: 
\begin{enumerate}
    \item $\ptov[exp=\ep,result=\ev]$
    \item \prun[exp=\ep, value=v_{1}]
    \item \vmrung[guard=\ev, result=v_{2}]
\end{enumerate}

then $v_{1} = v_{2}$


Let $p$ be a \PPlus\ pattern, \nsgs be a list of names and \VMinus\ guards,
\Rho\ be a specific but arbitrarily chosen environment, \context\ be a specific
but arbitrarily chosen context, and $x$ be a name that stands for the scrutinee
of a case expression $v$ such that $\Rho(x) = v$. If: 
\begin{enumerate}
    \item $\pattov[pat=p, result=\nsgs]$
    \item \pmatch[newenv=r_{1}]
    \item \vmgs[envext={\bracketed{\forall n \in ns, n \mapsto
    \bot}},result=r_{2}] 
\end{enumerate}

then $r_{1} = r_{2}$

NEXT STEPS:


Case: 

Name 

(1) \pattov[pat=y, result={(y, [x = y])}] (2) \pmatch[pat=y, newenv=r_{1}] (3) \vmgs[envext=\bracketed{y \mapsto \rho(x)}, guards={[x = y]}, result=r_{2}]

By rule BLAH, $r_{1}$ = BLAH1. 
By rule BLAH2, $r_{2}$ = BLAH2. 

Therefore, $r_{1} = r_{2}$. 

\vmsemantics


\end{document}

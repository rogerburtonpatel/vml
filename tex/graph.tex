\documentclass{article}
\usepackage{vmlmacros}

\title{Using graphs to prove properties about guarded expressions}

\begin{document}

\maketitle

A guarded expression is composed of parts. 

\bigskip

\it{Note: we could call these "equations," but intermediate expressions like 
"x" or "3" in} 

\tt{\vexists{x} x = 3; x; 3; 2 + x}
\it{aren't really equations. What do you think?}

\bigskip

A part can be solved if, considering an environment $\rho$, 
all names in the part are known within $\rho$. (A name is 
\it{known} in an environment if it exists in that environment 
and has a binding to a value in that environment. A \it{value} 
is simply either an integer or a value constructor applied 
to one or more arguments.). 

\bigskip

Let us consider a guarded expression $GE$ and the graph $G$. $G$ is 
composed of nodes which represent the parts of $GE$. The edges in $G$
are formed with this rule:

\it{If }


\end{document}
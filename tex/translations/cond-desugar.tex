\documentclass[]{article}
\usepackage{vmlmacros}
\begin{document}

A desugaring of Scheme \tt{cond} to \textit{V}: 

\begin{align*}
\tt{(cond)} \;\; & \triangleq \;\; \tt{wrong} \\
\tt{(cond [} e_g, \; e_a \tt{] \dots)} \;\; 
& \triangleq \;\; \tt{if} \;\; e_g  \;\; \tt{then} \;\; e_a \;\; 
  \tt{else (cond \dots)} \\
\end{align*}

where $e_g$ is a \it{guarded-exp} and $e_a$ is an \it{exp} on the 
right-hand side. 

This translation desugars \tt{cond} into the Verse \tt{if-then-else} form, 
which itself is syntactic sugar for \bf{one}. 

\medskip

Here is a translation from \tt{cond} to \bf{one}. 

$$ \tt{(cond)} \;\;  \triangleq \;\; \tt{wrong} $$

$$ \tt{(cond [} e_{g1} \; e_{a1}\tt{]} \; \tt{[}e_{g2} \; e_{a2} \tt{] \dots [}e_{gn} \; e_{an}\tt{] )}
  \triangleq \;\; 
$$

$$ (\bf{one} \; \{(e_{g1} ; \; \lambda\langle\rangle.\; e_{a1}) \; \choice \; 
(e_{g2} ; \;\lambda\langle\rangle.\; e_{a2}) \; \choice \; \dots \; \choice \; 
(e_{gn} ; \;\lambda\langle\rangle.\; e_{an}) \; 
\choice \; \bf{wrong} \})\langle\rangle  \\
$$  

Question: is it the above, or is the \bf{wrong} in a lambda, like
$$
(\bf{one} \; \{(e_{g1} ; \; \lambda\langle\rangle.\; e_{a1}) \; \choice \; 
(e_{g2} ; \;\lambda\langle\rangle.\; e_{a2}) \; \choice \; \dots \; \choice \; (e_{gn} ; \;\lambda\langle\rangle.\; e_{an}) \; 
\choice \; \lambda\langle\rangle.\;\bf{wrong} \})\langle\rangle  \\
$$  

? 

Either way, getting to the last branch produces wrong- if reaching \bf{wrong} is
a run-time error, it needen't be in the lambda, but if it's any sort of value, 
maybe evaluating it with he same semantics as the rest of the branches, as the 
result of a applying the returned lambda to $\langle\rangle$, could be best. 

\bigskip
Also, I suspect there's a better symbol to use than $\triangleq$ for 
translations. What have you used? 

% TODO: Build macros as you go. Translations, evaluations, judgements, rules, 
%       and all other forms will become clear as you write. 

% Example small-step:
% \[
% \hstate[fuel=F,queue=\parthreads{R'} Q]
% \goesto{\text{context switch}}
% \hstate[running=\running{R'},queue=\parthreads{Q}{R}]
% \]


% \begin{mathpar}
% \inferrule*[Right=\textsc{AndShort}]
%           {\evalr{e_1}  \Downarrow \state{v_1}{\xi'}{\phi}{\rho'}
%            \and 
%            v_1 = 0}
%            {\evalr{\textsc{and}(e_1, e_2)}
%                 \Downarrow
%             \state{v_1}{\xi'}{\phi}{\rho'}}
% \end{mathpar}

% This document doesn't add anything new yet. 

% \parskip=0.8\baselineskip plus 2pt
% \parindent=0pt


% Introduce an abstract machine semantics? uscheme in chapter 3. 

% First, talk about syntactic forms and forms of judgement. 

% 1 for verse, one for PM. 

% Forms of judgement 

% First, the base translation: 

% \textit{Translation from P to V-style if-then-else:}

% \hfill \break
% \textit{case} \tt{v} \textit{of} \_ $\rightarrow$ \textit{e}
% \hfill \break
% $\triangleq$
% \hfill \break
% \tt{if (True) e e}
% \hfill \break
% ------------------------
% \vspace{-10pt}

% TODO FIX if true e e to if true e impossible 


% \begin{mathpar}
%     \inferrule*[Left=\textsc{Verse-IfTrueBindings}]
%     {\ }
%     {{\textsc{Verse-If}(\tt{True} \;e\; e)}  \rightarrowtail {\{\}}
%     }
% \end{mathpar}


% \begin{mathpar}
%     \inferrule*[Left=\textsc{Verse-IfTrueEval}]
%     {\ }
%     {{\textsc{Verse-If}(\tt{True} \;e\; e)}  \rightarrowtail e
%     }
% \end{mathpar}

% Maybe want these as $\textsc{Verse-IfLiteralBindings}$?


% \begin{mathpar}
% \inferrule*[Left=\textsc{TranslateWildcardBindings}]
% {{\textsc{Verse-If}(\tt{True} \;e\; e)}  \rightarrowtail {\{\}}
% }
%   {{\textsc{case}(\textsc{Wildcard}, v)}
%    \rightarrowtail {\{\}} 
%   }
% \end{mathpar}



% \begin{mathpar}
%     \inferrule*[Left=\textsc{TranslateWildcardEval}]
%     {\ }
%       {{\textsc{case}(\textsc{Wildcard}, v, e)}
%        \rightarrowtail {\textsc{Verse-If}(\tt{True} \;e\; e)}
%       }
%     \end{mathpar}

%     \begin{mathpar}
%         \inferrule*[Left=\textsc{TranslateWildcardEval'}]
%         {{\textsc{Verse-If}(\tt{True} \;e\; e)}  \rightarrowtail e
%         }
%           {{\textsc{case}(\textsc{Wildcard}, v, e)}
%            \rightarrowtail e 
%           }
%         \end{mathpar}

% % Help wanted formalizing. Add evaluation? With bindings? 

% Moving on to variables: 

% \hfill \break
% \textit{case} \tt{v} \textit{of} $x \rightarrow$ \textit{e}
% \hfill \break
% $\triangleq$
% \hfill \break
% \tt{if ($\exists x. \; x = v$) e e}
% \hfill \break
% ------------------------

% \begin{mathpar}
%     \inferrule*[Left=\textsc{Verse-IfBindings}]
%     {\ }
%     {{\textsc{Verse-If} (\tt{($\exists x. \; x = v$)} e \; e)}  \rightarrowtail e{\{x \longmapsto v\}}
%     }
% \end{mathpar}

% \begin{mathpar}
%     \inferrule*[Left=\textsc{Verse-IfEval}]
%     {\ }
%     {{\textsc{Verse-If}(\tt{True} \;e\; e)}  \rightarrowtail e
%     }
% \end{mathpar}

% \begin{mathpar}
%     \inferrule*[Left=\textsc{TranslateVarBindings}]
%     {{\textsc{Verse-If} (\tt{($\exists x. \; x = v$)} \; e \; e)}  \rightarrowtail {\{x \longmapsto v\}}
%     }
%       {{\textsc{case}(x, v, e)}
%        \rightarrowtail {\{x \longmapsto v\}}
%       }
%     \end{mathpar}

%     \begin{mathpar}
%         \inferrule*[Left=\textsc{TranslateVarEval}]
%         {{\textsc{Verse-If} (\tt{($\exists x. \; x = v$)} \; e \; e)}  \rightarrowtail e{\{x \longmapsto v\}}
%         }
%           {{\textsc{case}(x, v, e)}
%            \rightarrowtail e{\{x \longmapsto v\}}
%           }
%         \end{mathpar}
    



\end{document}
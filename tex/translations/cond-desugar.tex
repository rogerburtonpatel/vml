\documentclass[]{article}
\usepackage{vmlmacros}
\begin{document}

$\mathcal{D} =$ Desugar.

\bigskip

A fixed desugaring of Scheme \tt{cond} to \textit{\v}'s \texttt{if-then-else}: 

\begin{align*}
  \desugar{\tt{(cond)}}\; ; & =\; ; \tt{wrong} \\
\desugar{\tt{(cond [} e_g\; e_a \tt{] \dots)}}\; ; 
& =\; ; \tt{if}\; ; \desugar{e_g} \; ; \tt{then}\; ; \desugar{e_a}\; ; 
  \tt{else}\; ; \desugar{\tt{(cond \dots)}} \\
\end{align*}

where $e_g$ and $e_a$ are \it{exp}s.

\bigskip

% mm this probably isn't so. but saving the idea. 
% NOTE: I beleive the translation of $e_g$ may need to introduce existentials.
% As such, though both $e_g$ and $e_a$ are \it{exp}s on the LHS, 
% I think the translation of $e_g$ looks more like 
% $\exists\; \texttt{fv}(e_g).\; e_g$

This translation desugars \tt{cond} into the Verse \tt{if-then-else} form, 
which itself is syntactic sugar for \bf{one}. 

\medskip

Here is a translation from \tt{cond} to \bf{one}. 

$$ \desugar{\tt{(cond)}}\; ;  =\; ; \tt{wrong} $$

$$ \desugar{\tt{(cond [} e_{g1}\; e_{a1}\tt{]}\; \tt{[}e_{g2}\; e_{a2} 
   \tt{] \dots [}e_{gn}\; e_{an}\tt{] )}}
  =\; ; 
$$

% $$ (\bf{one}\; \{(\desugar{e_{g1}} ;\; \lambda\langle\rangle.\; \desugar{e_{a1}})\; \choice\; 
% (\desugar{e_{g2}} ;\; lambda\langle\rangle.\; \desugar{e_{a2}})\; \choice\; \dots\; \choice\; 
% (\desugar{e_{gn}} ;\; lambda\langle\rangle.\; \desugar{e_{an}})\; 
% \choice\; \bf{wrong} \})\langle\rangle  \\
% $$  

$$
(\bf{one}\; \{(\desugar{e_{g1}} ;\; \lambda\langle\rangle.\; \desugar{e_{a1}})\; \choice\; 
(\desugar{e_{g2}} ;\; lambda\langle\rangle.\; \desugar{e_{a2}})\; \choice\; \dots\; \choice\; (\desugar{e_{gn}} ;\; lambda\langle\rangle.\; \desugar{e_{an}})\; 
\choice\; \lambda\langle\rangle.\; bf{wrong} \})\langle\rangle  \\
$$  


Inductively: 

$$
\desugar{\tt{(cond [} e_g\; e_a \tt{] \dots)}} = \; ; 
$$

$$
(\bf{one}\; \{(\desugar{e_{g}} ;\; \lambda\langle\rangle.\; \desugar{e_{a}})\; ; \choice\; ;
\desugar{\tt{(cond \dots)}}\})\langle\rangle  \\
$$  


% TODO: Build macros as you go. Translations, evaluations, judgements, rules, 
%       and all other forms will become clear as you write. 

% Example small-step:
% \[
% \hstate[fuel=F,queue=\parthreads{R'} Q]
% \goesto{\text{context switch}}
% \hstate[running=\running{R'},queue=\parthreads{Q}{R}]
% \]


% \begin{mathpar}
% \inferrule*[Right=\textsc{AndShort}]
%           {\evalr{e_1}  \Downarrow \state{\v[1]}{\xi'}{\phi}{\rho'}
%            \and 
%            \v[1] = 0}
%            {\evalr{\textsc{and}(e_1, e_2)}
%                 \Downarrow
%             \state{\v[1]}{\xi'}{\phi}{\rho'}}
% \end{mathpar}

% This document doesn't add anything new yet. 

% \parskip=0.8\baselineskip plus 2pt
% \parindent=0pt


% Introduce an abstract machine semantics? uscheme in chapter 3. 

% First, talk about syntactic forms and forms of judgement. 

% 1 for verse, one for PM. 

% Forms of judgement 

% First, the base translation: 

% \textit{Translation from P to \v-style if-then-else:}

% \hfill \break
% \textit{case} \tt{\v} \textit{of} \_ $\rightarrow$ \textit{\expr}
% \hfill \break
% $\triangleq$
% \hfill \break
% \tt{if (True) \expr \expr}
% \hfill \break
% ------------------------
% \vspace{-10pt}

% TODO FIX if true \expr \expr to if true \expr impossible 


% \begin{mathpar}
%     \inferrule*[Left=\textsc{Verse-IfTrueBindings}]
%     {\ }
%     {{\textsc{Verse-If}(\tt{True}\; expr\; \expr)}  \rightarrowtail {\{\}}
%     }
% \end{mathpar}


% \begin{mathpar}
%     \inferrule*[Left=\textsc{Verse-IfTrueEval}]
%     {\ }
%     {{\textsc{Verse-If}(\tt{True}\; expr\; \expr)}  \rightarrowtail \expr
%     }
% \end{mathpar}

% Maybe want these as $\textsc{Verse-IfLiteralBindings}$?


% \begin{mathpar}
% \inferrule*[Left=\textsc{TranslateWildcardBindings}]
% {{\textsc{Verse-If}(\tt{True}\; expr\; \expr)}  \rightarrowtail {\{\}}
% }
%   {{\textsc{case}(\textsc{Wildcard}, \v)}
%    \rightarrowtail {\{\}} 
%   }
% \end{mathpar}



% \begin{mathpar}
%     \inferrule*[Left=\textsc{TranslateWildcardEval}]
%     {\ }
%       {{\textsc{case}(\textsc{Wildcard}, \v, \expr)}
%        \rightarrowtail {\textsc{Verse-If}(\tt{True}\; expr\; \expr)}
%       }
%     \end{mathpar}

%     \begin{mathpar}
%         \inferrule*[Left=\textsc{TranslateWildcardEval'}]
%         {{\textsc{Verse-If}(\tt{True}\; expr\; \expr)}  \rightarrowtail \expr
%         }
%           {{\textsc{case}(\textsc{Wildcard}, \v, \expr)}
%            \rightarrowtail \expr 
%           }
%         \end{mathpar}

% % Help wanted formalizing. Add evaluation? With bindings? 

% Moving on to variables: 

% \hfill \break
% \textit{case} \tt{\v} \textit{of} $x \rightarrow$ \textit{\expr}
% \hfill \break
% $\triangleq$
% \hfill \break
% \tt{if ($\vexists{x}\; x = \v$) \expr \expr}
% \hfill \break
% ------------------------

% \begin{mathpar}
%     \inferrule*[Left=\textsc{Verse-IfBindings}]
%     {\ }
%     {{\textsc{Verse-If} (\tt{($\vexists{x}\; x = \v$)} \expr\; \expr)}  \rightarrowtail \expr{\{x \longmapsto \v\}}
%     }
% \end{mathpar}

% \begin{mathpar}
%     \inferrule*[Left=\textsc{Verse-IfEval}]
%     {\ }
%     {{\textsc{Verse-If}(\tt{True}\; expr\; \expr)}  \rightarrowtail \expr
%     }
% \end{mathpar}

% \begin{mathpar}
%     \inferrule*[Left=\textsc{TranslateVarBindings}]
%     {{\textsc{Verse-If} (\tt{($\vexists{x}\; x = \v$)}\; \expr\; \expr)}  \rightarrowtail {\{x \longmapsto \v\}}
%     }
%       {{\textsc{case}(x, \v, \expr)}
%        \rightarrowtail {\{x \longmapsto \v\}}
%       }
%     \end{mathpar}

%     \begin{mathpar}
%         \inferrule*[Left=\textsc{TranslateVarEval}]
%         {{\textsc{Verse-If} (\tt{($\vexists{x}\; x = \v$)}\; \expr\; \expr)}  \rightarrowtail \expr{\{x \longmapsto \v\}}
%         }
%           {{\textsc{case}(x, \v, \expr)}
%            \rightarrowtail \expr{\{x \longmapsto \v\}}
%           }
%         \end{mathpar}
    



\end{document}
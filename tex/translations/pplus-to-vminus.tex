\documentclass[]{article}
\usepackage{vmlmacros}
\usepackage{syntax}
\usepackage{relsize}
% \usepackage{palatino} % I don't love this, to be honest. 
\usepackage{amsmath}
\usepackage{booktabs}
\usepackage{simplebnf}
\setcounter{secnumdepth}{1}

\DeclareMathOperator{\dom}{dom}


% \setlength{\grammarparsep}{20pt plus 1pt minus 1pt} % increase separation between rules
\setlength{\grammarindent}{10em} % increase separation between LHS/RHS
\setlength{\parindent}{0cm}
\title{Translation from \PPlus to \VMinus}
\author{Roger Burtonpatel}
\date{February 18th, 2024}
\begin{document}

\maketitle

\subsection{Binding names}

\begin{align}
    &\Bindings{x} = \bracketed{x} \\ 
    &\Bindings{K} = \bracketed{} \\
    &\Bindings{K \; p_{1} \dots p_{n}} = \Bindings{p} \cup \dots \cup \Bindings{p_{n}} \\
    &\Bindings{\por} = \Bindings{p_{1}} \cap \Bindings{p_{2}} \\
    &\Bindings{\pcommap} = \Bindings{p} \cup \Bindings{p'} \\
    &\Bindings{\parrowe} = \Bindings{p} \\
    &\Bindings{\whenexpr} = \bracketed{}
\end{align}

\subsection{Translating Expressions}


\begin{align}
    &\ptov[exp=x, result=x] \\
    &\ptov[exp={K \; \expr[1] \dots \expr[n]}, result={K \etran{\expr[1]} \dots \etran{\expr[n]}}] \\
    &\ptov[exp={\lambda x.\; \expr}, result={\lambda x.\; \etran{\expr}}] \\
    &\ptov[exp={\expr[1] \; \expr[2]}, result={\etran{\expr[1]}\; \etran{\expr[2]}}] \\
    &\ptov[exp={\tt{case} \; \expr\;  \emptyseq}, result={{\iffitt{\exists x. \; x = \etran{\expr}; \; \iffitt{}}}}] \; \rm{, $x$ fresh }   
\end{align}
\[
    \ptov[exp={\tt{case} \; \expr\;  p_{1}\;\expr[1] \vert \dots \vert p_{n}\;\expr[n]}, 
    result={{\iffitt{\exists x. \; x = \etran{\expr}; \; 
            \iffitt{x = \ptran{p_{1}}; \rightarrow \etran{\expr[1]} 
            [] \dots [] x = \ptran{p_{n}}; \rightarrow \etran{\expr[n]}}}}}] \; \rm{, $x$ fresh }    
\]

\rab{how do you format this so it's not terrible}

\subsection{Translating Patterns}


Translating a name with the context of a name -> name = name 
Translating a vcon applied to zero patterns with the context of a name -> 
Vcon = name 
Translating a vcon applied to one or more patterns with the context of a name ->
Introduce a fresh name per pattern. Say name = K applied to fresh names,
then translate each sub-pattern with the context of its corresponding name. 

Translating (when e) with the context of a name -> e 
Translating (p <- e) with the context of a name -> 
Introduce a fresh name, bind it to e, bind it to p. 
Translating (p, p') with the context of a name -> translate 
p and p' with the context of that name, seperated by a semicolon 

Translating (p1 | p2) with the context of a name -> 
This is the hard one. Likely need to duplicate context. 

\begin{align*}
    &\pattov[pat=y, result={x = y}] \\
    &\pattov[pat=K, result={x = K}] \\
    &\ptran{K p_{1} \dots p_{n}}xe \rightsquigarrow \\
    \exists y_{1} \dots y_{n}. \; &\ptran{p_{1}}y_{1}e; \dots; \ptran{p_{n}}y_{n}e;\;
    x = K y_{1} \dots y_{n}, \\
    &\rm{$\forall i.\; 1 \leq i \leq n:\; y_{i}$ fresh} \\
    &\pattov[pat=\mathit{when}\; e', result=e'] \\
    &\pattov[pat=\pcommap, result=\ptran{p}xe;\;\ptran{p'}xe]
\end{align*}
        %     &\ptov[exp={K \; p_{1} \dots p_{n}}, result={K \; \ptran{p_{1}} \dots \ptran{\expr[n]}}] \\
% &\ptov[exp={p \; \tt{when} \; \expr}, result={\ptran{p}\tt{;} \; \etran{\expr}}] \\
%     &\ptov[exp={p_{1} \vert p_{2}}, result={\ptran{p_{1}} \vert \ptran{p_{2}}}] \\
%     &\ptov[exp={p,  p_{1} \leftarrow \expr[1], \dots, p_{n} \leftarrow \expr[n]}, 
%     result={\ptran{p}\tt{;} \ptran{p_{1}} = \etran{\expr[1]}\tt{;} \dots\tt{;} 
%     \ptran{p_{n}} = \etran{\expr[n]}}] \\


%     \section{Sequences}
    
%     The trivial sequence is \emptyseq. Sequences can be concatenated with infix 
% $\cdot$. In an appropriate context, a value like $x$ stands for 
% the singleton sequence containing $x$. 

% \begin{align*}
%     \emptyseq \cdot \ys &\equiv \ys \\
%     \ys \cdot \emptyseq &\equiv \ys \\
%     (\xs \cdot \ys) \cdot \zs &\equiv \xs \cdot (\ys \cdot \zs)
% \end{align*}

% \section{Rules (Big-step Operational Semantics) for \PPlus:}
    
% \subsection{Evaluating Expressions}


% \subsubsection{Evaluating expressions other than \tt{case}}

% \[
% \inferrule*[Left=\textsc{ (Eval-Vcon-Empty) }]
%     {\ }
%     {\veval{K}{K}}
% \]

% \[
% \inferrule*[Left=\textsc{ (Eval-Vcon-Multi) }]
%     {\inferrule* {}
%     {
%     \veval{\expr[i]}{v_{i}}
%     \and 
%     1 \leq i \leq n
%     }}
%     {\veval{K(\expr[1], \dots \expr[n])}{K(v_{1}, 
%     \dots v_{i})}}
% \]

% \[
% \inferrule*[Left=\textsc{ (Eval-Name) }]
%     {x \in \dom \rho 
%     \\\\
%     \rho(x) = \v}
%     {\prun[exp={\textsc{name} \;x}]}
% \]

% \[
% \inferrule*[Left=\textsc{ (Eval-Lambda-Decl) }]
%     {\ }
%     {\prun[exp={\lambda x.\expr}, value={\lambda x.\expr}]}
% \]


% \[
% \inferrule*[Left=\textsc{ (Eval-Funapp) }]
%     {\veval{\expr[1]}{\lambda x.\expr}
%     \\\\
%     \veval{\expr[2]}{\v'}
%     \\\\
%     \prun[envext=\bracketed{x \mapsto \v'}]}
%     {\veval{\expr[1]\; \expr[2]}{\v}}
% \]

% \[
% \inferrule*[Left=\textsc{ (CaseScrutinee) }]
%     {\inferrule* {}
%     {\prun[value=\v]}
%     \and 
%     \prun[exp=
%         {\textsc{case} \; (\textsc{literal}(\v), 
%         [p_{1} \; \expr[1]], \dots, [p_{n} \; \expr[n]])},
%         value=\v']}    
%     {\prun[exp=
%     {\textsc{case} \; (\expr, 
%     [p_{1} \; \expr[1]], \dots, [p_{n} \; \expr[n]])},
%     value=\v']}
% \]

% \[
% \inferrule*[Left=\textsc{ (CaseMatch) }]
%     {\inferrule* {}
%     {\pmatch[pat={p_{1}}, newenv=\rho']}
%     \and 
%     \prun[exp=\expr[1], env=\rho + \rho']}    
%     {\prun[exp=
%     {\textsc{case} \; (\textsc{literal}(\v), 
%     [p_{1} \; \expr[1]], \dots, [p_{n} \; \expr[n]])},
%     value=\v']}
% \]

% \[
% \inferrule*[Left=\textsc{ (CaseFail) }]
%     {\inferrule* {}
%     {\pmatch[pat={p_{1}}, newenv=\dagger]}
%     \and 
%     \prun[exp={\textsc{case} \; (\textsc{literal}(\v), 
%     [p_{2} \; \expr[2]], \dots, [p_{n} \; \expr[n]])},
%     value=\v']}    
%     {\prun[exp=
%     {\textsc{case} \; (\textsc{literal}(\v), 
%     [p_{1} \; \expr[1]], \dots, [p_{n} \; \expr[n]])},
%     value=\v']}
% \]



% \subsection{Rules for pattern matching}

% \subsubsection{Atoms}

% \[
% \inferrule*[Left=\textsc{ (MatchVcon) }]
%     {\inferrule* {}
%     {
%     \pmatch[pat=a_{i}, value=v_{i}, newenv=r_{i}], \and 1 \leq i \leq m
%     \\\\
%     r = r_{1} \uplus \dots \uplus r_{m}
%     }}
%     {\pmatch[pat={K\; a_{1}, \dots 
%             a_{m}}, value={\textsc{vcon} \; (K, [v_{1}', \dots v_{m}'])},
%             newenv=r]}
% \]

% \[
% \inferrule*[Left=\textsc{ (FailVcon) }]
%     {\v \rm{ does not have the form } \textsc{vcon} \; (K, [v_{1}', \dots v_{m}'])}
%     {\pmatch[pat={K\; a_{1}, \dots 
%             a_{m}}, 
%             newenv=\dagger]}
% \]

% \[
% \inferrule*[Left=\textsc{ (MatchBareVcon) }]
%     {\ }
%     {\pmatch[pat={K}, value={\textsc{vcon} \; (K, [])},
%             newenv=\bracketed{}]}
% \]

% \[
% \inferrule*[Left=\textsc{ (FailBareVcon) }]
%     {\v \neq {\textsc{vcon} \; (K, [])}}
%     {\pmatch[pat={K},
%             newenv=\dagger]}
% \]

% \[
% \inferrule*[Left=\textsc{ (MatchVar) }]
%     {\ }
%     {\pmatch[pat={x},
%             newenv=\bracketed{x \mapsto \v}]}
% \]

% \[
% \inferrule*[Left=\textsc{ (MatchWhen) }]
%     {\prun[value=\v'] \and \v' \neq \tt{false}}
%     {\pmatch[pat={\tt{when}\; \expr},
%             newenv=\bracketed{}]}
% \]

% \[
% \inferrule*[Left=\textsc{ (FailWhen) }]
% {\prun[value=\v'] \and \v' = \tt{false}}
% {\pmatch[pat={\tt{when}\; \expr},
%             newenv=\dagger]}
% \]

% \subsubsection{Factors}

% \[
% \inferrule*[Left=\textsc{ (AtomExpMatch) }]
% {\prun[value=\v'] \and \pmatch[pat=a, value=\v', newenv=r]}
% {\pmatch[pat={a \leftarrow \expr},
%             newenv=r]}
% \]

% \[
% \inferrule*[Left=\textsc{ (Atom) }]
% {\pmatch[pat=a, value=\v, newenv=r]}
% {\pmatch[pat=a,
%             newenv=r]}
% \]

% \subsubsection{Terms}

% \[
% \inferrule*[Left=\textsc{ (FactorSingle) }]
% {\pmatch[pat=r, value=\v, newenv=r]}
% {\pmatch[pat=f,
%             newenv=r]}
% \]

% \[
% \inferrule*[Left=\textsc{ (FactorMultiFail) }]
% {\pmatch[pat=f, newenv=\dagger]}
% {\pmatch[pat={f \cdot fs},
%             newenv=\dagger]}
% \]


% \[
% \inferrule*[Left=\textsc{ (FactorMultiResult) }]
% {\inferrule* {}
% {\pmatch[pat=f, newenv=\Rhoprime]}
% \\\\
% \pmatch[env=\Rho \uplus \Rhoprime, pat=fs, newenv=r]}
% {\pmatch[pat={f \cdot fs}, newenv=r]}
% \]

% \subsubsection{Patterns}

% \[
% \inferrule*[Left=\textsc{ (TermSucc) }]
%     {\pmatch[pat=t]}
%     {\pmatch[pat={t \cdot ts}]}
% \]

% \[
% \inferrule*[Left=\textsc{ (TermRec) }]
%     {\pmatch[pat=p_{1}, newenv=\dagger]
%     \and 
%     \pmatch[pat={ts}, newenv=r]}
%     {\pmatch[pat={t \cdot ts}, newenv=r]}
% \]


% \[
% \inferrule*[Left=\textsc{ (TermSingle) }]
%     {\pmatch[pat=t, newenv=r]}
%     {\pmatch[pat={t \cdot \emptyseq}, newenv=r]}
% \]




% % \section{Rules for matching Top-level Patterns}

% % \[
% % \inferrule*[Left=\textsc{ (OrSucc) }]
% %     {\pmatch[pat=p_{1}]}
% %     {\pmatch[pat={\textsc{or} \; (p_{1}, \dots p_{n})}]}
% % \]

% % \[
% % \inferrule*[Left=\textsc{ (OrRec) }]
% %     {\pmatch[pat=p_{1}, newenv=\dagger]
% %     \and 
% %     \pmatch[pat={\textsc{or} \; (p_{2}, \dots p_{n})}]}
% %     {\pmatch[pat={\textsc{or} \; (p_{1}, \dots p_{n})}]}
% % \]


% % \[
% % \inferrule*[Left=\textsc{ (OrFail) }]
% %     {\ }
% %     {\pmatch[pat={\textsc{or} \; (\emptyseq)}, newenv=\dagger]}
% % \]


% % \[
% % \inferrule*[Left=\textsc{ (WhenMatch) }]
% %     {\pmatch
% %     \\\\
% %     \prun[env=\Rho \uplus \Rhoprime, value=\v' \and \v' \neq \tt{false}]}
% %     {\pmatch[pat={\textsc{when} \; (p, \expr)}]}
% % \]

% % \[
% % \inferrule*[Left=\textsc{ (WhenNoMatch) }]
% %     {\pfail}
% %     {\pmatch[pat={\textsc{when} \; (p, \expr)}, newenv=\dagger]}
% % \]

% % \[
% % \inferrule*[Left=\textsc{ (WhenFalse) }]
% %     {\pmatch
% %     \\\\
% %     \prun[env=\Rho \uplus \Rhoprime, value=\v']  \and \v' = \tt{false}}
% %     {\pmatch[pat={\textsc{when} \; (p, \expr)}, newenv=\dagger]}
% % \]

% % \[
% % \inferrule*[Left=\textsc{ (GuardNoMatch) }]
% %     {\pmatch[newenv=\dagger]}
% %     {\pmatch[pat={\textsc{guard} \; (p, gs)}, newenv=\dagger]}
% % \]

% % \[
% % \inferrule*[Left=\textsc{ (GuardEmpty) }]
% %     {\pmatch[newenv=r]}
% %     {\pmatch[pat={\textsc{guard} \; (p, \emptyseq)}, newenv=r]}
% % \]

% % \[
% % \inferrule*[Left=\textsc{ (GuardFail) }]
% %     {\inferrule* {}
% %     {\pmatch[newenv=\Rhoprime]}
% %     \\\\
% %     \prun[env=\Rho \uplus \Rhoprime, value=\v']
% %     \\\\
% %     \pmatch[value=\v', newenv=\dagger]
% %     }
% %     {\pmatch[pat={\textsc{guard} \; (p, (p <- \expr) \cdot gs)}, newenv=\dagger]}
% % \]


% % \[
% % \inferrule*[Left=\textsc{ (GuardRec) }]
% %     {\inferrule* {}
% %     {\pmatch[newenv=\Rhoprime]}
% %     \\\\
% %     \prun[env=\Rho \uplus \Rhoprime, value=\v']
% %     \\\\
% %     \pmatch[value=\v', newenv=\rho'']
% %     \\\\
% %     \pmatch[env=\Rho \uplus \rho'', pat={\textsc{guard} \; (p, gs)},
% %     newenv=r]
% %     }
% %     {\pmatch[pat={\textsc{guard} \; (p, (p <- \expr) \cdot gs)}, newenv=r]}
% % \]


\end{document}

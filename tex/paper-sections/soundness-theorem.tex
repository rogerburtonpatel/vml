\documentclass[]{article}
\usepackage{vmlmacros}

\begin{document}
% The big-step operational semantics $S$ we provide is sound iff for all 
% Verse terms $t \in V$ and all equivalent \Vminus terms $t^{-} \in \Vminus$, 
% there exist evaluation orders $O_{1}$ and $O_{2}$ s.t. applying Verse 
% rewrite rules $R$ in order $O_{1}$ and applying our big-step rules in order 
% $O_{2}$ to $t$ and $t^{-}$ respectively yeilds equivalent results. 

English: Whenever $\vgctx \Downarrow \valpha$, and $g$ is equivalent to 

In mathspeak: 
% \[\forall t \in V,\; t^{-} \in \Vminus. \; 
% \exists \; O_{1} \; O_{2}. \; \rm{s.t.} \;  O_{1}R(t) = O_{2}S(t^{-})
% \]

% Ordered subset
% exists subset s.t. applying them successively to this term gives ... 
\[
    \exists r_{1} \dots r_{n} \in R, s_{1} \dots s_{m} \in S .\;
  r_{n} \circ \dots \circ r_{1}(t) = s_{m} \circ \dots \circ s_{1}(t^{-})
  \]
% Make t and t- equivalent explicity 

% Let O_1(t) be the function that applies rules in R to t in 
% the order \mathcal{O}_1
% Think of O1, O2 as permutations 
% R = \{r_{1}, ..., r_{n}\}
% S = \{s_{1}, ..., s_{m}\}
\end{document}
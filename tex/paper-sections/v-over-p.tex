\documentclass{article}
\usepackage{vmlmacros}
\begin{document}
One interesting aspect- in ways, a piece of found or emergent behavior- 
of Verse equations' flexibility with names is their ability to double bind a 
pattern name. Consider this code: 

\begin{verbatim}
    case (1, 1)
        of (x, x) => "I wish"
         | _ => "Naturally, this will not even compile"
    
\end{verbatim}

Indeed, we would need a side condition: 
\begin{verbatim}
    case (1, 1)
        of (x, x') when x = x' => "This works"
         | _ => "Now I won't even exist"
    
\end{verbatim}

Or a pattern guard: 

\begin{verbatim}
    case (1, 1)
        of (x, x')
            , x = x' => "This works"
        | _ => "Now I won't even exist"

    
\end{verbatim}
\VMinus (and subsequently Verse) actually has no trouble with this: 
\[
    \exists x.\; \mathit{PAIR}\; 1\; 1\; = \mathit{PAIR}\; x\; x; -> x 
\]

Because binding is done one bit at a time, x is first bound to 1, then 
\textit{checked} if it is bound to 1. Binding and assertion being a 
single, unified construct has its advantages. 

\end{document}
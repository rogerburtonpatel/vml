\documentclass[]{article}
\usepackage{vmlmacros}
\usepackage{syntax}
\usepackage{relsize}
% \usepackage{palatino} % I don't love this, to be honest. 
\usepackage{amsmath}
\usepackage{booktabs}
\usepackage{simplebnf}[2023-11-25]
% \usepackage{listings}
\usepackage{color}
\definecolor{eclipseBlue}{RGB}{42,0.0,255}
\definecolor{eclipseGreen}{RGB}{63,127,95}

\newcommand\smllst
{\lstset {
  basicstyle=\small\ttfamily,
  captionpos=b,
  tabsize=2,
  columns=fixed,
  breaklines=true,
  frame=l,
  numbers=left,
  numberstyle=\small\ttfamily,
  morekeywords= {
    EQUAL, GREATER, LESS, NONE, SOME, abstraction, abstype, and, andalso, array, as, before, bool, case, char, datatype, do, else, end, eqtype, exception, exn, false, fn, fun, functor, handle, if, in, include, infix, infixr, int, let, list, local, nil, nonfix, not, o, of, op, open, option, orelse, overload, print, raise, real, rec, ref, sharing, sig, signature, string, struct, structure, substring, then, true, type, unit, val, vector, where, while, with, withtype, word
  },
  morestring=[b]",
  morecomment=[s]{(*}{*)},
  stringstyle=\color{black},
  identifierstyle=\color{eclipseBlue},
  keywordstyle=\color{red},
  commentstyle=\color{eclipseGreen}
}}

\newcommand\verselst
{\lstset {
  basicstyle=\small\ttfamily,
  captionpos=b,
  tabsize=2,
  columns=fixed,
  breaklines=true,
  frame=l,
  numbers=left,
  numberstyle=\small\ttfamily,
  morekeywords= {
    EQUAL, GREATER, LESS, NONE, SOME, raise, $\exists$, if, then, else, =
  },
  morestring=[b]",
  morecomment=[s]{(*}{*)},
  stringstyle=\color{black},
  identifierstyle=\color{eclipseBlue},
  keywordstyle=\color{red},
  commentstyle=\color{eclipseGreen},
  mathescape
}}
\usepackage{xcolor}
\setcounter{secnumdepth}{1}

\DeclareMathOperator{\dom}{dom}



% \setlength{\grammarparsep}{20pt plus 1pt minus 1pt} % increase separation between rules
\setlength{\grammarindent}{10em} % increase separation between LHS/RHS
\setlength{\parindent}{0cm}
\title{""Having Your Pattern Matching Cake and Eating it Too in Verse""}
\author{Roger Burtonpatel}
\date{January 4th, 2023}
\begin{document}

\maketitle

CCS Concepts: 

\bigskip

Additional Key Words and Phrases: 

\bigskip

\section{Introduction}

The Verse language looks more expressive than pattern matching. 
Consider the following segment from our source code for \VMinus, implemented
in SML:

\smllst 

\begin{lstlisting}
    case \expr of ...

    | NAME n => if (Env.binds (rho, n))
                then (case Env.find (n, rho) 
                        of SOME \v => \v 
                    |      NONE   => raise NameNotBound n)
                else raise NameNotBound n 

    ...
\end{lstlisting}

Between the duplicate brances and awkward nesting, this code is stuffy: it says
much for a simple computation of "if \tt{n} is bound to \tt{SOME \v} in \tt{rho},
give me \tt{\v}, otherwise throw an error." We might turn to a monad in this
situation, but if our code is not already monadic, this presents significant
refactoring cost. 

\bigskip
If the reader might suggest collapsing duplicate cases with if and
\tt{isSome}/\tt{valOf}, we will present them with similarly stuffy code: 
\begin{lstlisting}
    case \expr of ...

    | NAME n => if not (Env.binds (rho, n))
                   orelse not (isSome (Env.find (n, rho)))
                then raise NameNotBound n 
                else valOf (Env.find (n, rho))

    ...
\end{lstlisting}

Here, we've introduced a new sin: duplicate function calls. Furthermore, the
code is no more easy to read (arguably more so with the doubly-negated
conditional). Removing the negation and flippling the branches mitigates the 
second problem, but does not solve the first. 

\bigskip 

The key cost here is the conflict between \tt{case} and \tt{if}: nesting and 
interleaving them is awkward, albeit frequently necessary in functional code. 
Extensions to pattern matching like pattern guards and side conditions can 
mitigate this- ""this will be discussed in a further section."" (How do we lead 
our reader to be curious about Verse without discounting pattern matching 
extensions?)

\bigskip

Now consider equivalent code in Verse: 
\verselst

\begin{lstlisting}
    $\exists$ n. \expr = NAME n; if (Env.binds(rho, n); $\exists$ \v. SOME \v = Env.find (n, rho)) 
                     then \v
                     else raise NameNotBound n
\end{lstlisting}

\bigskip

We've managed to condense the entire idea into a single \tt{if} condition, 
with one branch for success and one for failure. (Note Verse does not currently
support exceptions; we use an imaginary \tt{raise} keyword to illustrate
equivalence.)

Indeed, Verse's use of a single notion of success or failure, producable via 
both expression evaluation and attempts at unifying values with \tt{=}, is 
highly expressive: even this toy example, maddengly inelegant in the land of 
patterns, is both simpler and (we argue) more legible in Verse. 

The key to this example is that either \tt{Env.binds(rho, n)}  or \tt{SOME \v =
Env.find (n, rho)} can fail, and their failure is identical: it produces the
empty sequence of values. Verse's \tt{if} construct, and its desugaring to
\tt{one}, can test for this failure and act accordingly. This gives Verse a 
special kind of expressiveness reserved typically for (languages like?) Erlang, 
but Verse can express more than pattern matching in even those languages. 



% match the variable $x$ with a list of length
% 37 whose first element is \tt{"Hello"} and return \tt{"World"}, or match 
% $x$ with a 

But a full implementation of Verse must use logical variables at runtime, which
can be costly. To illustrate Verse's expressiveness while attempting to mitigate
costs, we have identified a subset of the Verse Calculus that subsumes pattern
matching but does not require logical variables at runtime. Furthermore, we
demonstrate how full Verse can be written to ""match the semantics"" of this
subset and subsequently enjoy its performance benefits. 

We make three main contributions. \bf{Our first contribution} is a syntax,
semantics, and implementation of a subset of Verse, which we call \VMinus
(pronounced \v-minus), that imposes a few key restrictions that remove the
possiblity of logical variables at runtime. We go on
to show that \VMinus is just Verse with syntactic restrictions, and we provide a
static analysis which identifies if full Verse programs abide by those
restrictions. 


\bf{Our second contribution} is a comparison language, \Pplus (P-plus), which
has pattern matching with popular extensions. We provide  a translation from
\Pplus to \VMinus and one from \VMinus to ""an intermediate representation
involving decision trees"", and a proof that the transformations are
semantics-preserving. 

\bf{Our third contribution} is less formal: we present an argument that Verse,
beyond being as expressive as \Pplus, is in many cases able to produce more elegant
programs. ""We also show how some extensions to \Pplus, when held alone, cannot
express what Verse can.""

\section{Restricting Verse: \VMinus}

We present a syntax of \VMinus, our subset of Verse: 

\subsubsection{\VMinus:}
\it{Note: I beleive we should change this to the ${\mu}$Verse syntax}
\begin{center}
    \begin{bnf}
    $\ealpha$ : \textsf{$\alpha$-Expressions} ::=
    | $\alpha$ : terminating alpha
    | $x, y, z$ : names
    % Question: ebnf braces vs. 
    | $\tt{if} \; \tt{[}\; \galpha \; \bracketed{[] \galpha} \;\tt{]} \; \tt{fi}$ : if-fi 
    | $K \bracketed{\ealpha}$ : value constructor application 
    | $\ealpha[1] \; \ealpha[2]$ : function application 
    ;;
    $\galpha$ : \textsf{Guarded Expressions} ::=  
    $\boldsymbol{\rightarrow}\ealpha$ : terminating $\ealpha$ 
    | $\ealpha; \; \galpha$ : intermediate expression 
    | $\exists \bracketed{x} \tt{.} \galpha$ : existential 
    | $\ealpha[1] = \ealpha[2]; \; \galpha$ : equation 
    % ;;
    \end{bnf}
\end{center}


    Do semantics go here? 

    Hole: Discuss- how it is different from Verse, how it is the same, and
    foreshadow static analysis tool. 
\section{Comparing with Pattern Matching: \Pplus}

\subsubsection{\Pplus:}
\begin{center}
    \begin{bnf}
$\expr$ : \textsf{Expressions} ::=
    | $\ttbraced{\tt{case} \; \expr \; \bracketed{\tt{[} p \; \expr \tt{]}}}$ : case expression 
    ;;
    $p$ : \textsf{Patterns} ::= $x$ : name 
    | $K$ : value constructor 
    | $\ttbraced{K \; \bracketed{p}}$ : value constructor application 
    % | $\ttbraced{p \; \tt{when} \; \expr }$ : side condition
    | $\ttbraced{\tt{oneof} \; p_{1} \;, p_{2} }$ : or-pattern 
    | $\ttbraced{p \tt{;} \bracketed{\expr \vert \ttbraced{p  <- \expr}}}$ : pattern guard
    \end{bnf}
\end{center}
    Hole: what extensions we tackle and why, a note on pattern matching
     efficiency: decision trees

\section{Translations}

\begin{itemize}
    \item \Pplus $->$ \VMinus 
    \item \VMinus $->$ \D
    \item "why we don't need \Pplus to \D"
\end{itemize}

\section{"The point"}


\begin{itemize}
    \item \VMinus, and therefore Verse, is as expressive as \Pplus
    \item \VMinus does not incur the cost of unifying logical variables at runtime 
    \item Hole: Discussion: Value generation in choice 
\end{itemize}

% \section{Comparing with Pattern Matching: \Pplus}

\bigskip

% % I attempted to use the grammar environment you provided. But there was either
% % something missing or something I overlooked in the example code and it would
% % not compile, despite many reducing changes I made. So I went with the
% % simplebnf package, which I quite like.  
% \begin{center}
%     \begin{bnf}
%     $P$ : \textsf{Programs} ::=
%     $\bracketed{d}$ : definition
%     ;;
%     $d$ : \textsf{Definitions} ::=
%     | $\tt{val} \; x \; \tt{=} \; \expr$ : bind name to expression
%     ;;
%     $\expr$ : Core expressions ::= 
%     | $x, y, z$ : names
%     | $K\bracketed{\expr}$ : value constructor application 
%     | $\tt{if} \; \expr[1] \; \tt{then} \; \expr[2] \; \tt{else} \; e_{3} $ : if
%     | $\lambda x. \; \expr$ : lambda 
%     | $\expr[1] \; \expr[2]$ : function application 
%     ;;
%     $\v$ : Values ::= $K\bracketed{\v}$ : value constructor application 
%     ;;
%     $K$ : \textsf{Value Constructors} ::=
%     % \cons : cons 
%     % | \tt{[]} : empty list 
%     | \tt{true} $\vert$ \tt{false} : booleans
%     | $\tt{\#}x$ : name beginning with \tt{\#}
%     | \tt{A-Z}$x$ : name beggining with capital letter
%     | $[\tt{-}\vert\tt{+}](\tt{0}-\tt{9})+$ : signed integer literal 

%     \end{bnf}
% \end{center}


% A \it{name} is any token that is not an integer literal, 
% does not contain whitespace, a bracket, or parenthesis, 
% and is not a value constructor name or a reserved word.


% We then present three language extensions that build off of this core: 
% \Pplus, the language of patterns, \VMinus, the language of 
% verse-like equations, and $D$, the language of decision trees. 

% \subsection{Three Language Extensions}

% \subsubsection{\VMinus:}

% \begin{center}
%     \begin{bnf}
%     $\ealpha$ : \textsf{$\alpha$-Expressions} ::=
%     | $\alpha$ : terminating alpha
%     | $x, y, z$ : names
%     % Question: ebnf braces vs. 
%     | $\tt{if} \; \tt{[}\; \galpha \; \bracketed{[] \galpha} \;\tt{]} \; \tt{fi}$ : if-fi 
%     | $K \bracketed{\ealpha}$ : value constructor application 
%     | $\ealpha[1] \; \ealpha[2]$ : function application 
%     ;;
%     $\galpha$ : \textsf{Guarded Expressions} ::=  
%     $\boldsymbol{\rightarrow}\ealpha$ : terminating $\ealpha$ 
%     | $\ealpha; \; \galpha$ : intermediate expression 
%     | $\exists \bracketed{x} \tt{.} \galpha$ : existential 
%     | $\ealpha[1] = \ealpha[2]; \; \galpha$ : equation 
%     % ;;
%     \end{bnf}
% \end{center}

% \bigskip 

% \subsubsection{\Pplus:}
% \begin{center}
%     \begin{bnf}
% $\expr$ : \textsf{Expressions} ::=
%     | $\ttbraced{\tt{case} \; \expr \; \bracketed{\tt{[} p \; \expr \tt{]}}}$ : case expression 
%     ;;
%     $p$ : \textsf{Patterns} ::= $x$ : name 
%     | $K$ : value constructor 
%     | $\ttbraced{K \; \bracketed{p}}$ : value constructor application 
%     % | $\ttbraced{p \; \tt{when} \; \expr }$ : side condition
%     | $\ttbraced{\tt{oneof} \; p_{1} \;, p_{2} }$ : or-pattern 
%     | $\ttbraced{p \tt{;} \bracketed{\expr \vert \ttbraced{p  <- \expr}}}$ : pattern guard
%     \end{bnf}
% \end{center}


% \bigskip 

% \subsubsection{$D$:}

% \begin{center}
%     \begin{bnf}
%         \Dalpha : \textsf{Decision Tree} ::= 
%         $\tt{case} \; x \; \tt{of} \; 
%         \bracketed{\vert \; K\bracketed{x} \; \tt{=>} \; \Dalpha}
%         [\vert \; x \; \tt{=>} \Dalpha]$ : test node 
%         | $\alpha$ : match node 
%         | $\tt{if} \; x \; \tt{then} \; \Dalpha \; \tt{else} \; \Dalpha$ : condition with two children 
%         | $\tt{let} \; x \; \tt{=} \; \expr \; \tt{in} \; \Dalpha$ : let-bind a name
%         ;;
%         $\expr$ : \textsf{Expressions} ::=
%         | $\mathcal{D}_{\expr}$ : decision trees 
%     \end{bnf}
% \end{center}

        
% \section{Refinement ordering on environments}

% \begin{align*}
% \rho \subseteq \Rhoprime \text{ when }&\dom\rho  \subseteq \dom \Rhoprime\\
% \text{ and } &\forall x \in \dom \rho: \rho(x) \subseteq \Rhoprime(x)
% \end{align*}



% \vfilbreak



% \section{Forms of Judgement for $\VMinus$:}
% \begin{tabular}{ll}
% \toprule
%     \multicolumn2{l}{\emph{Metavariables}} \\
% \midrule
%     % $\v, \; \v'$& value \\
%     \valpha& a metavalue produced from evaluating $\alpha$. \\
%     $eq$& equation \\ 
%     % $\tempstuck$& a temporarily-stuck equation \\
%     $\reject$& equation rejection \\
%     $\result$& $\vartheta \mid$ \reject : a result of \valpha \; or
%     rejection\\
%     \Rho& environment: $name \rightarrow \mathcal{\v}_{\bot}$ \\
%     $\rho\{ x \mapsto y \} $& environment extended with name $x$ mapping to $y$ \\
%     $\mathcal{T}$& Context of all temporarily stuck equations (a sequence) \\ 
%     $\ealpha$& An expression \\ 
%     $\galpha$& A guarded expression \\
%     % \uppsidown& Inability to compile to a decision tree; a compile time error \\
% \bottomrule
% \end{tabular}    

% \bigskip

% \begin{tabular}{ll}
%     \toprule
%         \multicolumn2{l}{\emph{Sequences}} \\
%     \midrule
%         $\emptyseq$& the empty sequence \\
%         $S_1 \cdot S_2 $&  Concatenate sequence $S_1$ and sequence $S_2$ \\
%         $x \cdot S_2 $& Cons $x$ onto sequence $S_2$ \\
%     \bottomrule
%     \end{tabular}    
    
%     \medskip
    
%     % \mkjudgementcmd{EquationSuccess}{\vtuple{\rho, eq}}{\rhohat REMOVE THIS}
%     % \mkjudgementcmd{EquationTempStuck}{\vtuple{\rho, eq}}{\tempstuck}
%     % \mkjudgementcmd{EquationReject}{\vtuple{\rho, eq}}{reject}

%     % \showvjudgement{EquationSuccess}{\EquationSuccess}
%     % \showvjudgement{EquationTempStuck}{\EquationTempStuck}
%     % \showvjudgement{EquationReject}{\EquationReject}
    
    
    
    
%     % Success only refines the environment; that~is, when
%     % ${\vtuple{\rho, \expr}} \rightarrowtail{} {\Rhoprime}$, we~expect $\rho \subseteq \Rhoprime$.
    
    

    
%     \subsection{Expressions}
    
%     \newcommand\GNoTree{\vmrun \rightsquigarrow \uppsidown} An expression in
%     core Verse evaluates to produce possibly-empty sequence of values. In
%     \VMinus, values depend on the form of abstract expression $\alpha.$ If
%     $\alpha$ is a Verse-like expression, \valpha ~will be a value sequence. If
%     it is an ML-like expression, it will be a single value. 
    
%     A guarded expression evaluates to produce a \bf{result}. A result is either
%     a metavalue \valpha ~or reject. 
    
%     \[\it{r} \; \rm{::=} \; \vartheta \;|\; \reject \]
    
%     \showvjudgement{Eval-Alpha}{\veval{\alpha}{\valpha}}
%     \showvjudgement{Eval-AlphaExpr}{\veval{\ealpha}{\valpha}}
%     \showvjudgement{Eval-Guarded-Expr}{\vmrun}
    
%     % \bigskip
%     % If a guarded expression cannot be evaluated without producing logical 
%     % variables at runtime, it cannot be expressed as a decision tree. 
%     % This notation indicates this failure (think of \uppsidown as a fallen 
%     % tree), which results in a compile-time error. 
%     % \showvjudgement{NoTree}{\GNoTree}
    
%     \bigskip
% \subsection{Equations}

% In \VMinus, we solve equations (intermediate computations in a guarded
% expression of the form \\$\ealpha[1] = \ealpha[2]$) in a similar way to the
% authors of the original Verse paper: we pick one, attempt to solve for it, and
% move on. 

% \rab{How do we express that \VMinus semantics preclude backtracking/logical 
% variables at runtime?}

% Given an environment from names to metavalues {\valpha}s \Rho, an equation \eq
% ~will either refine the environment ($\Rhoprime$) or lead to failure. We use
% the metavariable $\dagger$ to represent failure, and an equation failing will
% cause the top-level guarded expression to evaluate to \reject. 

% \showvjudgement{Eq-Refine}\eqrefine
% \showvjudgement{Eq-Fail}\eqfail
    
    
%     \section{Sequences}
    
%     The trivial sequence is \emptyseq. Sequences can be concatenated with infix 
% $\cdot$. In an appropriate context, a value like $x$ stands for 
% the singleton sequence containing $x$. 

% \begin{align*}
%     \emptyseq \cdot \ys &\equiv \ys \\
%     \ys \cdot \emptyseq &\equiv \ys \\
%     (\xs \cdot \ys) \cdot \zs &\equiv \xs \cdot (\ys \cdot \zs)
% \end{align*}

% \section{Rules (Big-step Operational Semantics) for $\VMinus$:}
    
% \subsection{Evaluating Guarded Expressions}


% \subsubsection{Evaluating simple parts of guarded expressions}

% \[
% \inferrule*[Left=\textsc{ (Eval-ArrowExpr) }]
%     {\vmrun[context=\emptyseq,term=\alpha,result=\vartheta]}
%     {\vmrun[context=\emptyseq, term=\rightarrow \alpha,result=\vartheta]}
% \]

% \[
% \inferrule*[Left=\textsc{ (Eval-Exists) }]
%     {\vmrun[env=\rho\{x \mapsto \bot \}]}
%     {\vmrun[term=\exists x.\; \galpha]}
% \]

% \[
% \inferrule*[Left=\textsc{ (Eval-Expseq) }]
%     {\inferrule* {}
%     {
%     \veval{\ealpha}{\valpha}
%     \and
%     \veval{\galpha}{\result}
%     }}
%     {\vmrun[term=\ealpha;\; \galpha]}
% \]
% \subsubsection{Shifting an equation to the context}
% \[
% \inferrule*[Left=\textsc{ (G-Move-To-Ctx) }]
%     {\inferrule*{\vmrun[context=eq \cdot \context]}
%     {}
%     }
%     {\vmrun[term=eq; \; \galpha]}
% \]

% \subsubsection{Choosing and solving an equation}

% \[
% \inferrule*[Left=\textsc{ (Eval-Eq-Refine) }]
%     {\eqrefine
%     \and
%     \vmrun[env=\Rhoprime,context=\TT]}
%     {\vmrun[context=\TeqT{\eq},result=\vartheta]}
% \]

% \[
% \inferrule*[Left=\textsc{ (Eval-Eq-Fail) }]
%     {\eqfail}
%     {\vmrun[context=\TeqT{\eq},result=\reject]}
% \]
% \subsubsection{Properties of equations}

% \[
% \inferrule*[Left=\textsc{ (Single-Eq-Assoc) }]
%     {\vmrun[context=\TeqT{\ealpha[2] = \ealpha[1]}]}
%     {\vmrun[context=\TeqT{\ealpha[1] = \ealpha[2]}]}
% \]

% \[
% \inferrule*[Left=\textsc{ (Multi-Eq-Assoc) }]
%     {\vmrun[context=\TeqT{\ealpha[2] = \ealpha[1] \cdot \ealpha[1] = \ealpha[2]}]}
%     {\vmrun[context=\TeqT{\ealpha[1] = \ealpha[2] \cdot \ealpha[2] = \ealpha[1]}]}
% \]

% \subsubsection{Desugaring of Complex Equations}
% \[
% \inferrule*[Left=\textsc{ (Desugar-EqExps) }]
%     {\inferrule* {}
%     {
%     x,\;y \; \rm{are distinct and fresh}
%     \\\\
%     \vmrun[envext=\bracketed{x \mapsto \bot,\; y \mapsto \bot},
%           context=\eqTT{x = \ealpha[1] \cdot y = \ealpha[2] \cdot x = y}]}}
%     {\vmrun[context=\TeqT{\ealpha[1] = \ealpha[2]}]}
% \]

% \[
% \inferrule*[Left=\textsc{ (Desugar-Vcon-Multi) }]
%     {
%     \vmrun[context=\TeqT{\lbrack \ealpha[i]=\ealpha[i]' \; 
%            \vert \; 1 \leq i \leq n \rbrack}]}
%     {\vmrun[context=\TeqT{K(\ealpha[1], \dots 
%             \ealpha[n]) = K(\ealpha[1]', \dots \ealpha[n]')}]}
% \]

% \subsubsection{Refinement with different types of equations}

% \[
% \inferrule*[Left=\textsc{ (G-NameExp-Bot) }]
%     {\inferrule* {}
%     {
%     x \in \dom \rho
%     \\\\
%     \veval{\ealpha}{\valpha}
%     \\\\
%     \rho(x) = \bot
%     }}
%     {\eqrefine[eq={x = \ealpha},newenv=\Rho\bracketed{x \mapsto \valpha}]}
% \]

% \[
% \inferrule*[Left=\textsc{ (G-NameExp-Eq) }]
%     {\inferrule* {}
%     {
%     x \in \dom \rho
%     \\\\
%     \veval{\ealpha}{\valpha}
%     \\\\
%     \rho(x) = \valpha
%     }}
%     {\eqrefine[eq={x = \ealpha},newenv=\Rho\bracketed{x \mapsto \valpha}]}
% \]

% \[
% \inferrule*[Left=\textsc{ (G-NameExp-Fail) }]
%     {\inferrule* {}
%     {
%     x \in \dom \rho
%     \\\\
%     \veval{\ealpha}{\valpha}
%     \\\\ 
%     \rho(x) = \valpha'
%     \\\\
%     \valpha \neq \valpha'
%     }}
%     {\eqrefine[eq={x = \ealpha},newenv=\dagger]}
% \]

% \[
% \inferrule*[Left=\textsc{ (G-NameNotFound-Fail) }]
%     {\inferrule* {}
%     {
%     x \not\in \dom \rho
%     }}
%     {\eqrefine[eq={x = \ealpha},newenv=\dagger]}
% \]

% \[
% \inferrule*[Left=\textsc{ (G-EqNames-Vals-Succ) }]
%     {\inferrule* {}
%     {
%     x,\;y \in \dom \rho
%     \\\\
%     \rho(x) = \valpha, \; \rho(y) = \valpha
%     }}
%     {\eqrefine[eq={x = y},newenv=\rho]}
% \]

% \[
% \inferrule*[Left=\textsc{ (G-EqNames-Vals-Fail) }]
%     {\inferrule* {}
%     {
%     x,\;y \in \dom \rho
%     \\\\
%     \rho(x) = \valpha, \; \rho(y) = \valpha'
%     \\\\
%     \valpha \neq \valpha'}}
%     {\eqrefine[eq={x = y},newenv=\dagger]}
% \]

% % \[
% % \inferrule*[Left=\textsc{ (G-EqNames-Bots-Fail) }]
% %     {\inferrule* {}
% %     {
% %     x,\;y \in \dom \rho
% %     \\\\
% %     \rho(x) = \bot, \; \rho(y) = \bot
% %     \\\\
% %     x,\;y \; \rm{do not appear in} \; \context, \; \context'}}
% %     {\vmrun[context=\TeqT{x = y}] 
% %     \rightsquigarrow \uppsidown}
% % \]

% \[
% \inferrule*[Left=\textsc{ (G-EqNames-BotVal-Succ) }]
%     {\inferrule* {}
%     {
%     x,\;y \in \dom \rho
%     \\\\
%     \rho(x) = \bot, \; \rho(y) = \valpha
%     }}
%     {\eqrefine[eq={x = y},newenv=\rho\bracketed{x \mapsto \valpha}]}
% \]

% \[
% \inferrule*[Left=\textsc{ (G-Vcon-Single-Fail) }]
%     {K \neq K'}
%     {\eqrefine[eq={K = K'}, newenv=\dagger]}
% \]

% \[
% \inferrule*[Left=\textsc{ (G-Vcon-Single-Succ) }]
%     {\ }
%     {\eqrefine[eq={K = K}, newenv=\Rho]}
% \]


% \[
% \inferrule*[Left=\textsc{ (G-Vcon-Multi-Fail) }]
%     {K \neq K'}
%     {\eqrefine[eq={K(\ealpha[1], \dots 
%             \ealpha[n]) = K'(\ealpha[1]', \dots \ealpha[n]')},
%             newenv=\dagger]}
% \]

% \[
% \inferrule*[Left=\textsc{ (G-Vcon-Multi-Arity-Fail) }]
%     {n \neq m}
%     {\eqrefine[eq={K(\ealpha[1], \dots 
%             \ealpha[n]) = K(\ealpha[1]', \dots \ealpha[m]')},
%             newenv=\dagger]}
% \]


% \subsection{Evaluating General Expressions}


% % \[
% % \inferrule*[Left=\textsc{ (If-Fi-Fail) }]
% %     {\ }
% %     {\veval{\iffi{\ }}{\emptyseq}}
% % \]

% \[
% \inferrule*[Left=\textsc{ (If-Fi-Success) }]
%     {\vmrun[] \Downarrow \vartheta}
%     {\veval{\iffi{\galpha \; \square \; \dots}}{\vartheta}}
% \]

% \[
% \inferrule*[Left=\textsc{ (If-Fi-Reject) }]
%     {\inferrule* {}
%     {\vmrun[result=\reject]}
%     \and 
%     \veval{\iffi{\dots}}{\vartheta}}
%     {\veval{\iffi{\galpha \; \square \; \dots }}{\vartheta}}
% \]

% \[
% \inferrule*[Left=\textsc{ (Vcon-Empty) }]
%     {\ }
%     {\veval{K}{K}}
% \]

% \[
% \inferrule*[Left=\textsc{ (Vcon-Multi) }]
%     {\inferrule* {}
%     {
%     \veval{\ealpha[i]}{\vartheta_{i}}
%     \and 
%     1 \leq i \leq n
%     }}
%     {\veval{K(\ealpha[1], \dots \ealpha[n])}{K(\vartheta_{1}, 
%     \dots \vartheta_{i})}}
% \]

% \[
% \inferrule*[Left=\textsc{ (Eval-Name) }]
%     {x \in \dom \rho 
%     \\\\
%     \rho(x) = \valpha}
%     {\veval{x}{\valpha}}
% \]

\end{document}

\documentclass[]{article}
\usepackage{amsmath}
\usepackage{amsthm}
\usepackage{amssymb}
\usepackage{tikz}
\usepackage{graphicx}
\usepackage{ebproof}
\usepackage{alltt}
\usepackage{verbatim}
\usepackage{hyperref}
\usepackage{amsmath}
\usepackage{qtree}
\usepackage{semantic}
\usepackage{mathpartir}

\newcommand{\br}[1]{\langle #1 \rangle}
\newcommand{\state}[4]{\langle {#1,#2,#3,#4} \rangle}
\newcommand{\evalr}[2][{}]{\state{#2}{\xi#1}{\phi}{\rho#1}}




\begin{document}

\parskip=0.8\baselineskip plus 2pt
\parindent=0pt


Here's some shitty-first draft bad opsem. My goal for today is to make good 
opsem. 

My questions are: 

Do we want opsem to formally notate the evaluation of $\expr$ with the new bindings?

I'm putting the verse rules on top to represent how they create no bindings 
as well. How kosher is this?

I assume we need to prove Verse-if true \expr \expr is \expr and creates no new bindings to 
logical variables in its scope. Am I right about this?

Introduce an abstract machine semantics? uscheme in chapter 3. 


First, talk about syntactic forms and forms of judgement. 

1 for verse, one for PM. 

Forms of judgement 

First, the base translation: 


\textit{Translation from P to \v-style if-then-else:}

\hfill \break
\textit{case} \texttt{\v} \textit{of} \_ $\rightarrow$ \textit{\expr}
\hfill \break
$\triangleq$
\hfill \break
\texttt{if (True) \expr \expr}
\hfill \break
------------------------
\vspace{-10pt}

TODO FIX if true \expr \expr to if true \expr impossible 


\begin{mathpar}
    \inferrule*[Left=\textsc{Verse-IfTrueBindings}]
    {\ }
    {{\textsc{Verse-If}(\texttt{True}\; expr\; \expr)}  \rightarrowtail {\{\}}
    }
\end{mathpar}


\begin{mathpar}
    \inferrule*[Left=\textsc{Verse-IfTrueEval}]
    {\ }
    {{\textsc{Verse-If}(\texttt{True}\; expr\; \expr)}  \rightarrowtail \expr
    }
\end{mathpar}

Maybe want these as $\textsc{Verse-IfLiteralBindings}$?


\begin{mathpar}
\inferrule*[Left=\textsc{TranslateWildcardBindings}]
{{\textsc{Verse-If}(\texttt{True}\; expr\; \expr)}  \rightarrowtail {\{\}}
}
  {{\textsc{case}(\textsc{Wildcard}, \v)}
   \rightarrowtail {\{\}} 
  }
\end{mathpar}



\begin{mathpar}
    \inferrule*[Left=\textsc{TranslateWildcardEval}]
    {\ }
      {{\textsc{case}(\textsc{Wildcard}, \v, \expr)}
       \rightarrowtail {\textsc{Verse-If}(\texttt{True}\; expr\; \expr)}
      }
    \end{mathpar}

    \begin{mathpar}
        \inferrule*[Left=\textsc{TranslateWildcardEval'}]
        {{\textsc{Verse-If}(\texttt{True}\; expr\; \expr)}  \rightarrowtail \expr
        }
          {{\textsc{case}(\textsc{Wildcard}, \v, \expr)}
           \rightarrowtail \expr 
          }
        \end{mathpar}

% Help wanted formalizing. Add evaluation? With bindings? 

Moving on to variables: 

\hfill \break
\textit{case} \texttt{\v} \textit{of} $x \rightarrow$ \textit{\expr}
\hfill \break
$\triangleq$
\hfill \break
\texttt{if ($\vexists{x}\; x = \v$) \expr \expr}
\hfill \break
------------------------

\begin{mathpar}
    \inferrule*[Left=\textsc{Verse-IfBindings}]
    {\ }
    {{\textsc{Verse-If} (\texttt{($\vexists{x}\; x = \v$)} \expr\; \expr)}  \rightarrowtail \expr{\{x \longmapsto \v\}}
    }
\end{mathpar}

\begin{mathpar}
    \inferrule*[Left=\textsc{Verse-IfEval}]
    {\ }
    {{\textsc{Verse-If}(\texttt{True}\; expr\; \expr)}  \rightarrowtail \expr
    }
\end{mathpar}

\begin{mathpar}
    \inferrule*[Left=\textsc{TranslateVarBindings}]
    {{\textsc{Verse-If} (\texttt{($\vexists{x}\; x = \v$)}\; \expr\; \expr)}  \rightarrowtail {\{x \longmapsto \v\}}
    }
      {{\textsc{case}(x, \v, \expr)}
       \rightarrowtail {\{x \longmapsto \v\}}
      }
    \end{mathpar}

    \begin{mathpar}
        \inferrule*[Left=\textsc{TranslateVarEval}]
        {{\textsc{Verse-If} (\texttt{($\vexists{x}\; x = \v$)}\; \expr\; \expr)}  \rightarrowtail \expr{\{x \longmapsto \v\}}
        }
          {{\textsc{case}(x, \v, \expr)}
           \rightarrowtail \expr{\{x \longmapsto \v\}}
          }
        \end{mathpar}
    


% \pagebreak

% \textbf{Problem C.}

% First, the rules for \&\&:
% \begin{mathpar}
% \inferrule*[Right=\textsc{AndShort}]
%           {\evalr{e_1}  \Downarrow \state{\v[1]}{\xi'}{\phi}{\rho'}
%            \and 
%            \v[1] = 0}
%            {\evalr{\textsc{and}(e_1, e_2)}
%                 \Downarrow
%             \state{\v[1]}{\xi'}{\phi}{\rho'}}
% \end{mathpar}
% \begin{mathpar}
% \inferrule*[Right=\textsc{AndLong}]
%           {\evalr{e_1}  \Downarrow \state{\v[1]}{\xi'}{\phi}{\rho'}
%            \and 
%            \v[1] \neq 0
%            \and
%            \state{e_2}{\xi'}{\phi}{\rho'} \Downarrow
%               \state{\v[2]}{\xi''}{\phi}{\rho''}}
%            {\evalr{\textsc{and}(e_1, e_2)}
%                 \Downarrow
%             \state{\v[2]}{\xi''}{\phi}{\rho''}}
% \end{mathpar}

% \vspace{20pt}
% And here are the optional rules for $\vert\vert$:
% \begin{mathpar}
% \inferrule*[Right=\textsc{OrShort}]
%           {\evalr{e_1}  \Downarrow \state{\v[1]}{\xi'}{\phi}{\rho'}
%            \and 
%            \v[1] \neq 0}
%            {\evalr{\textsc{or}(e_1, e_2)}
%                 \Downarrow
%             \state{\v[1]}{\xi'}{\phi}{\rho'}}
% \end{mathpar}
% \begin{mathpar}
% \inferrule*[Right=\textsc{OrLong}]
%           {\evalr{e_1}  \Downarrow \state{\v[1]}{\xi'}{\phi}{\rho'}
%            \and 
%            \v[1] = 0
%            \and
%            \state{e_2}{\xi'}{\phi}{\rho'} \Downarrow
%               \state{\v[2]}{\xi''}{\phi}{\rho''}}
%            {\evalr{\textsc{or}(e_1, e_2)}
%                 \Downarrow
%             \state{\v[2]}{\xi''}{\phi}{\rho''}}

% \end{mathpar}

% \hfill \break
% \textit{case} \texttt{scrutinee} \textit{of} \_ $\rightarrow$ \textit{\expr}
% \hfill \break
% $\triangleq$
% \hfill \break
% \texttt{if (True) \expr}
% \hfill \break

\end{document}
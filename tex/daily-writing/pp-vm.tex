%%
\documentclass[manuscript,screen,review, 12pt]{acmart}
\let\Bbbk\relax % Fix for amssymb clash 
\usepackage{vmlmacros}
\AtBeginDocument{%
  \providecommand\BibTeX{{%
    \normalfont B\kern-0.5em{\scshape i\kern-0.25em b}\kern-0.8em\TeX}}}

\begin{document}
\title{A Replacement for Pattern Matching, Inspired by Verse}

\author{Roger Burtonpatel}
\email{roger.burtonpatel@tufts.edu}
\affiliation{%
  \institution{Tufts University}
  \streetaddress{419 Boston Ave}
  \city{Medford}
  \state{Massachusetts}
  \country{USA}
  \postcode{02155}
}

\author{Norman Ramsey}
\email{nr@cs.tufts.edu}
\affiliation{%
\institution{Tufts University}
\streetaddress{419 Boston Ave}
\city{Medford}
\state{Massachusetts}
\country{USA}
\postcode{02155}
}

\author{Milod Kazerounian}
\email{milod.mazerouniantufts.edu}
\affiliation{%
\institution{Tufts University}
\streetaddress{419 Boston Ave}
\city{Medford}
\state{Massachusetts}
\country{USA}
\postcode{02155}
}

\renewcommand{\shortauthors}{Burtonpatel et al.}

\begin{abstract}
  Pattern matching is nice and has an appealing cost model, but people keep
  extending it to make it more expressive. Verse [cite-Verse] looks very simple
  and is surprisingly expressive, but the cost model is a challenge. Why not
  look for a compromise? We introduce two small functional langauges that
  attempt to address the limitations of pattern matching in popular functional
  languages.
  \end{abstract}

\maketitle

\section{Introduction}
Pattern matching is a well-established and researched construct in functional
programming langauges [cite, appeal to authority]. It shines in its ability to
break down constructed data implicity, which is often much preferred to using
explict deconstructors like Scheme's \tt{car} and \tt{cdr}. But when deciding to
perform a terminal computation based on prior checks that don't involve a single
value matching on a pattern, pattern matching struggles to retain its brevity.
Consider the following algorithm on an abstract type \tt{FiniteMap}, involving a
lookup function. This examples is a modificaiton of one borrowed from SPJ
proposal [cite]. 

\begin{center}
\tt{lookup :: FiniteMap -> Int -> Maybe Int}
\end{center}

"The lookup returns \tt{Nothing} if the supplied key is not in the domain of the
mapping, and (\tt{Just v}) otherwise, where v is the value that the key maps to.
Now consider the following definition:"

\begin{verbatim}
    clunky env varl var2 | ok1 && 0k2 = val1 + val2 
                         | otherwise = var1 + var2 
    where m1 = lookup env var1 
          m2 = lookup env var2
          ok1 = isJust m1 
          ok2 = isJust m2 
          Just val1 = m1 
          Just val2 = m2    
\end{verbatim}

% \begin{enumerate}
%     \item Given two ints, look the first up in an environment which maps \tt{Int -> Maybe Int}.
%     \item If the result of the first lookup is a \tt{Maybe Int val1}, look up the second name. 
%     \item If the result of the second lookup is a \tt{Maybe Int val2}, return the sum of \tt{val1} and \tt{val2}. 
%     \item If any step fails, return the sum of the inputs. 
% \end{enumerate}

The authors show how explicit pattern matching with \tt{case} may mitigate the
verbosity of this problem:

\begin{verbatim}
    clunky env var1 var2 = 
      case lookup env var1 of 
        Nothing -> fail 
        Just val1 -> 
          case lookup env var2 of 
            Nothing -> fail 
            Just val2 -> val1 + val2
    where 
        fail = var1 + var2
\end{verbatim}

They say: 

"This is a bit shorter, but hardly better. Worse, if this was just one equation
of clunky, with others that follow, then the thing would not work at all. " 

The authors' suggested mitigation is pattern gaurds, which lead to the following
solution: 
\begin{center}
\begin{verbatim}
    clunky env var2 var2    
    | Just vall <- lookup env var1
    , Just val2 <- lookup env var2
    = val1 + val2
\end{verbatim}
\it{... other equations for clunky}
\end{center}    

\it{This is obviously better. But now consider something where you need 
or-patterns; this fails. }

We make a contribution to this problem in the form of two small functional
languages. One, which we call \PPlus, is an attempt to unify extensions to
pattern matching under a single, flexible language. The other, which we call
\VMinus, takes inspiration from Verse and sidesteps the general structure of the
\tt{case} expression entirely, opting to take a more equation-based approach in
its decision making. We show that \PPlus and \VMinus have all the expressiveness
of pattern matching with no additional significant runtime overhead, and we
propose a form of writing Verse code that could lead to efficient compilation
based on our heuristics. 

% This is obviously better. But consider a new step to the algorithm, 
% given the following function signature: 

% \begin{center}
%     \tt{fileHandleLookup :: Int -> File}
% \end{center}

% where 

% \begin{center}
%     \tt{data File = Readable Int | }
% \end{center}

% (5) 

    % \begin{enumerate}
    %     \item If a list is non-empty, take its first element. 
    %     \item If that element is an even integer, take its SHA256 hash. 
    %     \item If the last digit of the hash is 0, return SOME of the hash. 
    %     \item If any step fails, return NONE. 
    % \end{enumerate}
    
    % The algorithm is implemented with the following Standard ML code: 
    
    % \begin{lstlisting}
    % case input 
    %     of []      => NONE 
    %        | x::xs => if even x 
    %                   then let val hash = sha256 x 
    %                         in case last_digit hash 
    %                              of 0 => hash 
    %                               | _ => NONE 
    %                         end 
    %                    else NONE 
    % \end{lstlisting}
    
% \begin{lstlisting}
% case input 
%     of []      => NONE 
%        | x::xs => if even x 
%                   then let val hash = sha256 x 
%                         in case last_digit hash 
%                              of 0 => hash 
%                               | _ => NONE 
%                         end 
%                    else NONE 
% \end{lstlisting}

\bigskip 
\it{The introduction will have a motivating example. Its big idea is: Pattern
matching is popular and well-established, but there are situations in which it
is clunky, and where Verse may shine.}
\bigskip 

% \begin{itemize}
%     \item \bf{P1}: 
%     \item Programmers use pattern matching (cite, appeal to authority). But
%     pattern matching is not good for everything. 
%     \item Consider Example- spj example. 
%     \item Explain example. 
%     \item Attempted mitigations in the past: extensions, SPJ example. 
%     \item Need to find something where pattern guards are not good enough. 
%     \item We go futher. 
%     \item \bf{P2}: 
%     \item Contribution: 2 languages that attempt to mitigate this problem. 
%     \item Our goal: One simple, expressive, efficient decision-making construct.
%     \item One is \PPlus, the other is \VMinus. 
%     \item Example of \VMinus in action on prior example. 
% \end{itemize}

% \begin{itemize}
%     \item \bf{P1}: 
%     \item Pattern matching uses this for that. 
%     \item Example. 
%     \item Explain example. 
%     \item \bf{P2}: 
%     \item Programmers use pattern matching. But pattern matching is not good for
%     everything. 
%     \item Consider Example. 
%     \item Explain example. 
%     \item Attempted mitigations in the past: extensions, SPJ example. 
%     \item We go futher. 
%     \item \bf{P3}: 
%     \item Contribution: 2 languages that attempt to mitigate this problem. 
%     \item Our goal: One simple, expressive, efficient decision-making construct.
%     \item One is \PPlus, the other is \VMinus. 
%     \item Example of \VMinus in action on prior example. 
% \end{itemize}


\section{Our languages, Informally}

We begin by informally introducing two small functional programming languages to
address the issue of inflexibility of patterns: \PPlus, a language with several
of the popular extensions to pattern matching and minimal syntactic
restrictions, and \VMinus, which replaces pattern matching with equations like
those in Verse [cite-Verse]. 

To facilitate comparisons and proofs, the languages are each a subset of a 
single language \U (\cref{fig:U}). Asterisks (${}^{*}$) indicate components of \U which 
belong to the sub-languages. 

The keen reader will notice that the syntax of \U includes a forms for
decision trees, which we have not yet discussed. Decision trees will be crucial
in proving properties of efficiency of our languages, and will we be discussed
in full in a later section. 

\begin{figure}[h!p]
    \small
    \begin{flushleft}
        \begin{bnf}
        $P$ : \textsf{Programs} ::=
        $\bracketed{d}$ : definition
        ;;
        $d$ : \textsf{Definitions} ::=
        | $\mathit{val} \; x \; \mathit{=} \; \expr$ : bind name to expression
        ;;
        $\expr$ : Expressions ::= 
        | $x, y, z$              : names
        | $K\bracketed{\expr}$   : value constructor application 
        | $\lambda x. \; \expr$  : lambda declaration  
        | $\expr[1] \; \expr[2]$ : function application 
    
        | $\mathit{case} \; \expr \; \bracketed{p \; \rightarrow \; \expr}$ : $\rm{case expression}^{*}$
        | $\mathit{if} \; \mathit{[}\; g \; 
            \bracketed{[] g} \;\mathit{]} \; \mathit{fi}$                   : $\rm{if-fi}^{**}$
        | $\dt$ : $\rm{decision tree}^{***}$
            ;;
            $p$ : $\textsf{Patterns}^{*}$ ::= 
            $x$ : name 
            | $K \; \bracketed{p}$          : value constructor application 
            | $\mathit{when}\; \expr$       : side condition
            | $p, p'$                       : pattern guard 
            | $p \; \leftarrow \; \expr$    : pattern from explicit expression  
            | $p_{1}\pbar p_{2}$            : or-pattern
            ;;
            $g$ : $\textsf{Guarded Expressions}^{**}$ ::=  
            $\boldsymbol{\rightarrow}\expr$ : terminating experession
            | $\expr; \; g$                 : intermediate expression 
            | $\exists x \mathit{.} g$      : existential 
            | $x = \expr; \; g$             : equation 
            ;;
            \dt : $\textsf{Decision Trees}^{***}$ ::= 
            | $\mathit{case} \; x \; \mathit{of} \; 
                \bracketed{\vert \; K\bracketed{x} \; \mathit{=>} \; \dt} 
                [\vert \; x \; \mathit{=>} \dt]$                                 : $\rm{test node }^{***}$
            | $\expr$                                                           : $\rm{match node }^{***}$
            | $\mathit{if} \; x \; \mathit{then} \; \dt \; \mathit{else} \; \dt$  : $\rm{condition with two children }^{***}$
            | $\mathit{let} \; x \; \mathit{=} \; \expr \; \mathit{in} \; \dt$   : $\rm{let-bind a name}^{***}$
        ;;
        $\v$ : Values ::= 
          $K\bracketed{\v}$     : value constructor application 
        | $\lambda x. \; \expr$ : lambda value 
        ;;
        $K$ : \textsf{Value Constructors} ::=
        | $\mathit{true}\; \vert\; \mathit{false}$ : booleans
        | $\mathit{\#}x$                           : name beginning with $\mathit{\#}$
        | $\mathit{A-Z}x$                          : name beggining with capital letter
        | $[\mathit{-}\vert\mathit{+}]
            (\mathit{0}-\mathit{9})+$              : signed integer literal 
        \end{bnf}
        \medskip
        
        
        \it{Concrete Syntax}: "$\lambda$" and "$\exists$" each scope as far to
        the right as possible.
        
        $\hskip 8em$ For example, $(\lambda y.\> \exists x.\> x = 1;\> x + y)$ means 
        $(\lambda y.\> (\exists x.\> ((x = 1);\> (x + y))))$.
        
        Parentheses may be used freely to aid readability and override default precedence.

        A \it{name} is any token that is not an integer literal, does not
        contain whitespace, a bracket, or parenthesis, and is not a value
        constructor name or a reserved word.
        
        \medskip

        % \bf{Desugaring of Extended Expressions}

        ${}^{*}$ Indicates forms within \PPlus

        ${}^{**}$ Indicates forms within \VMinus
        
        ${}^{***}$ Indicates forms within \D

    \end{flushleft}
    
    \medskip

    

    \caption{\U, a decision-making language}
    \Description{A BNF grammar for \U, the universal decision-making language.
                 It includes patterns, guarded expressions, and decision trees.}
    \label{fig:U}
\end{figure}


Most forms in \U are present in all of its subsets: Values, Value Constructors,
Definition Forms, and most forms of expression are shared. Indeed, the
sub-langauge, like in Verse, is just the the lambda calculus with a few extended
forms (this time favoring value constructors over tuples). Like in Verse, every
Lambda Calculus program is a valid \U program. 

On top of this core, three languages are defined as subsets of \U. \PPlus is the
subset of \U including Patterns and the $\mathit{case} \; \expr \; \bracketed{p
\rightarrow \expr}$ form of case expression. \VMinus is the subset of \U
including Guarded Expressions and \it{if-fi}. Finally, \D is the subset of \U
that includes decision tree syntax, including the reduced case expression,
$\mathit{case} \; x \; \mathit{of} \; \bracketed{\vert \; K\bracketed{x} \;
\mathit{=>} \; \dt} [\vert \; x \; \mathit{=>} \dt]$. Importantly, the three are
mutually exclusive with respect to exactly these extensions; all subsets share
the sub-langauge and no subset has more than one of the expanded \it{case},
\it{if-fi}, or the decision tree forms. 

% \PPlus provides or-patterns, side-conditions, and pattern guards, whose
% combination does not appear in Haskell, OCaml, Standard ML, or any other major
% functional language. 
% \bigskip
% Next steps: Finish introduction of language table, say where semantics are. 


\it{The following sections and paragraphs are written:}

\begin{itemize}
    \item \bf{Subsection: General Evaluation}
    \item Big-step opsem with environments. 
    \item \bf{Subsection: \PPlus}
    \item Patterns, case, environment + disjoint union 
\end{itemize}

\it{The following sections and paragraphs are in progress:}
\begin{itemize}
    \item when, or-patterns, pattern guards 
    \item \bf{Subsection: \VMinus}
    \item Guarded expressions, new type of environment
    \item Evaluation stragety 
\end{itemize}

\it{Finally, much has been commented out below because I am still determining
    the order in which their appearance is most clear. After your review of this
    initial format, I will begin to include them.}


% Prior work has introduced extesions to pattern matching [cite, including
% SPJ proposal for pattern guards.]

% \PPlus provides or-patterns, side-conditions, and pattern guards, whose
% combination does not appear in Haskell, OCaml, Standard ML, or any other major
% functional language. 

% \bigskip
% interesting thoughts: 

% Having a strategem for verse to decision tree is analagous to tail recursion. 
% You have to write your code in a way that allows the optimization. 


% Pattern matching is a well-established paradigm within functional languages, and
% """has been the subject of significant discourse.""" \it{Appeal to authority here.}

% Without pattern matching, it can be tiresome to deconstruct data using manual 
% accessor functions. Consider the following Standard ML code: 
% \smllst

% \begin{lstlisting}
% val rec len = fn ys => 
%     if null ys 
%     then 0 
%     else let xs = tl ys 
%          in len xs 
%          end 
% \end{lstlisting}
% \it{An implementation of List.len in Standard ML that does not use pattern 
%     matching.}

% Manual checking and deconstruction, with built-in functions like \tt{null} and
% \tt{tl}, can be both error-prone (\bf{say more on this?}) and verbose. Most
% functional programmers likely prefer the follwing implementation of \tt{len}:

% \begin{lstlisting}
%     val rec len = fn ys => 
%         case ys 
%           of [] => 0
%            | _::xs => 1 + len xs
%     \end{lstlisting}

% \it{An equivalent implementation that uses pattern matching.}

% Indeed, pattern matching is quite appealing when deconstructing data, especially 
% data that may be represented with an algebraic data type, is the primary programming problem: 



% \begin{lstlisting}
% val len = \ys. case ys 
%                  of [] -> 0
%                   | _::xs -> 1 + len xs
% \end{lstlisting}


% Figure 1 illustrates an example in which pattern matching is an elegant solution
% to a problem (compare to equivalent Standard ML code that manually deconstructs
% a list:)


% Most functional programmers likely prefer the first example 

% However, language designers continue to extend pattern matching 

% \section{Pattern Matching as it is Now}
% \it{Pattern matching} is defined as "checking and locating of specific sequences
% of data of some pattern among raw data or a sequence of tokens." We will return
% to the notion of "checking" often in this paper: pattern matching answers the
% question "when I'm checking to see if a piece of data (called a \it{scrutinee})
% is of the same form as a certain pattern, does that match succeed or fail?"

% \it{Example}.

% In addition to this checking and locating, pattern matching serves as
% \it{assignment}: it can bind fresh variables based on the form of data and use
% those bindings in subsequent expressions. 

% \it{Example}.

% Here, "checking" means "do the data match what I expect them to." Because
% pattern matching is inherently built to match a scrutinee (pure data) with a
% pattern, pattern matching is quite expressive in these cases. 

% Because of this, most modern functional languages, especially \it{data
% dependency languages} like Haskell or ML, \it{(is this right?)} employ pattern
% matching as their main way to deconstruct data (citation?). 

% We here explore pattern matching through the lens of \PPlus, an invented
% language that has pattern matching along with several of its popular extensions.
% (The examples above are written in \PPlus.) 
% \subsection{Strengths}

% - "Checking" and assignment- nice! No car, cdr 

% - Nested patterns - powerful 

% - Literal patterns let you mix names and values 

% \subsection{Weaknesses and Proposed Mitigations}

% We return to the concept of unifying "checking" with assignment, i.e. "match if
% the data take this form, and give them names." Pattern matching succeeds here
% when checking means "is the form of data the way I expect"; in fact, as we know,
% assignment in general \it{is} pattern matching (figure/example?). But when
% checking means "does this computation succeed" or "does this binding conflict
% with a prior binding," pattern matching is at a loss, where Verse succeeds. 

% Example: 


% Pattern matching's extensions get closer to unifying "checking" and binding.
% Here, Verse enjoys a different suite of advantages. 

% First, its "or" operation (`one` with `choice`) allows for more than patterns to
% appear as a top-level "choose this or that" construct in a match sequence; you
% can also include arbitrary expressions. You can't do this in pattern matching's
% version, which is an or-pattern. Simply put, saying "does this pattern match or
% is this expression true" is easy in Verse and clunky if you use patterns. I'll
% show examples at our meeting. 

% Second, Verse can express operations out of order, letting important checks
% appear higher up even if they are executed later. This helps program legibility.
% Again, I have examples from the chapter I wrote today. 

% Third, in a pattern-match clause, the initial data must still match an initial
% pattern in order to enter a guard; in Verse there is no restriction. This is
% minor, because you could simply match the data to a variable, and then enter a
% guard-- but again, all of these advantages are in elegance and brevity, and
% elegant that solution is not. 

% Finally, mingling pattern guards with other extensions to pattern matching
% (especially or-patterns) is a murky subject. Haskell has pattern guards and side
% conditions, but no or-patterns. OCaml has side conditions and or-patterns, but
% no guards. Mixing all three is (according to some readings) simply difficult for
% implementers- including, interestingly, those of parsers. In Verse, having `one`
% and `choice` closely tied in with the simple `e1 = e2` equation form, which by
% itself subsumes pattern matching, side-conditions, and pattern guards, means
% that integrating options is easy. A key theme that arises of this: Verse has
% fewer constructs, and they are more expressive.  

% \section{A Proposal, Inspired by the Verse Calculus}

% \subsection{Verse Flexibility}
% \subsection{Something else}
% \subsection{A third thing}

% \section{Verse's Equations Subsume Pattern Matching}

% \subsection{Claim}
% \subsection{Proof}
% \subsection{Translations}

% \section{(Maybe) Writing Efficient Verse Code}

% \subsection{Claim}
% \subsection{Proof}
% \subsection{Translations}


% \section{Citations and Bibliographies}

\end{document}
\endinput
%%

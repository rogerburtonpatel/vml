%%
\documentclass[manuscript,screen,review]{acmart}
\usepackage{../syntax-and-judgement/vmlmacros}
\AtBeginDocument{%
  \providecommand\BibTeX{{%
    \normalfont B\kern-0.5em{\scshape i\kern-0.25em b}\kern-0.8em\TeX}}}

\begin{document}

%%
%% The "title" command has an optional parameter,
%% allowing the author to define a "short title" to be used in page headers.
\title{A Replacement for Pattern Matching, Inspired by Verse}

%%
%% The "author" command and its associated commands are used to define
%% the authors and their affiliations.
%% Of note is the shared affiliation of the first two authors, and the
%% "authornote" and "authornotemark" commands
%% used to denote shared contribution to the research.
\author{Roger Burtonpatel}
\email{roger.burtonpatel@tufts.edu}
\affiliation{%
  \institution{Tufts University}
  \streetaddress{419 Boston Ave}
  \city{Medford}
  \state{Massachusetts}
  \country{USA}
  \postcode{02155}
}

\author{Norman Ramsey}
\email{nr@cs.tufts.edu}
\affiliation{%
\institution{Tufts University}
\streetaddress{419 Boston Ave}
\city{Medford}
\state{Massachusetts}
\country{USA}
\postcode{02155}
}

\author{Milod Kazerounian}
\email{milod.mazerouniantufts.edu}
\affiliation{%
\institution{Tufts University}
\streetaddress{419 Boston Ave}
\city{Medford}
\state{Massachusetts}
\country{USA}
\postcode{02155}
}



%%
%% By default, the full list of authors will be used in the page
%% headers. Often, this list is too long, and will overlap
%% other information printed in the page headers. This command allows
%% the author to define a more concise list
%% of authors' names for this purpose.
\renewcommand{\shortauthors}{Burtonpatel et al.}

%%
%% The abstract is a short summary of the work to be presented in the
%% article.
\begin{abstract}
  Pattern matching is nice and has an appealing cost model, but people keep
  extending it to make it more expressive. Verse looks very simple and is
  surprisingly expressive, but the cost model is a challenge. Why not look for a
  compromise?

  \rab{CCS goes here. I know we'll do this later; I'd like help when we do.}
\end{abstract}

%%
%% The code below is generated by the tool at http://dl.acm.org/ccs.cfm.
%% Please copy and paste the code instead of the example below.
%%

% \begin{CCSXML}
% <ccs2012>
%  <concept>
%   <concept_id>00000000.0000000.0000000</concept_id>
%   <concept_desc>Do Not Use This Code, Generate the Correct Terms for Your Paper</concept_desc>
%   <concept_significance>500</concept_significance>
%  </concept>
%  <concept>
%   <concept_id>00000000.00000000.00000000</concept_id>
%   <concept_desc>Do Not Use This Code, Generate the Correct Terms for Your Paper</concept_desc>
%   <concept_significance>300</concept_significance>
%  </concept>
%  <concept>
%   <concept_id>00000000.00000000.00000000</concept_id>
%   <concept_desc>Do Not Use This Code, Generate the Correct Terms for Your Paper</concept_desc>
%   <concept_significance>100</concept_significance>
%  </concept>
%  <concept>
%   <concept_id>00000000.00000000.00000000</concept_id>
%   <concept_desc>Do Not Use This Code, Generate the Correct Terms for Your Paper</concept_desc>
%   <concept_significance>100</concept_significance>
%  </concept>
% </ccs2012>
% \end{CCSXML}

% \rab{NR: can you help with these?}
% \ccsdesc[500]{Do Not Use This Code~Generate the Correct Terms for Your Paper}
% \ccsdesc[300]{Do Not Use This Code~Generate the Correct Terms for Your Paper}
% \ccsdesc{Do Not Use This Code~Generate the Correct Terms for Your Paper}
% \ccsdesc[100]{Do Not Use This Code~Generate the Correct Terms for Your Paper}

%%
%% Keywords. The author(s) should pick words that accurately describe
%% the work being presented. Separate the keywords with commas.
\keywords{Pattern Matching, Decision Trees, choice operator, functional
programming, lambda calculus,  logic programming, logical variables, normal
forms, rewrite rules, big-step operational semantics, substitution, unification,
Verse calculus, Verse language}

%% A "teaser" image appears between the author and affiliation
%% information and the body of the document, and typically spans the
%% page.
% \begin{teaserfigure}
%   \caption{Seattle Mariners at Spring Training, 2010.}
%   \Description{Enjoying the baseball game from the third-base
%   seats. Ichiro Suzuki preparing to bat.}
%   \label{fig:teaser}
% \end{teaserfigure}

\received{20 February 2007}
\received[revised]{12 March 2009}
\received[accepted]{5 June 2009}

%%
%% This command processes the author and affiliation and title
%% information and builds the first part of the formatted document.
\maketitle

\section{Introduction}

Single Contribution: Most of the expressiveness of verse and all that of Pattern
matching 

\section{Pattern Matching as it is Now}
\it{Pattern matching} is defined as "checking and locating of specific sequences
of data of some pattern among raw data or a sequence of tokens." We will return
to the notion of "checking" often in this paper: pattern matching answers the
question "when I'm checking to see if a piece of data (called a \it{scrutinee})
is of the same form as a certain pattern, does that match succeed or fail?"

\it{Example}.

In addition to this checking and locating, pattern matching serves as
\it{assignment}: it can bind fresh variables based on the form of data and use
those bindings in subsequent expressions. 

\it{Example}.

Here, "checking" means "do the data match what I expect them to." Because
pattern matching is inherently built to match a scrutinee (pure data) with a
pattern, pattern matching is quite expressive in these cases. 

Because of this, most modern functional languages, especially \it{data
dependency languages} like Haskell or ML, \it{(is this right?)} employ pattern
matching as their main way to deconstruct data (citation?). 

We here explore pattern matching through the lens of \Pplus, an invented
language that has pattern matching along with several of its popular extensions.
(The examples above are written in \Pplus.) 
\subsection{Strengths}

- "Checking" and assignment- nice! No car, cdr 

- Nested patterns - powerful 

- Literal patterns let you mix names and values 

\subsection{Weaknesses and Proposed Mitigations}

We return to the concept of unifying "checking" with assignment, i.e. "match if
the data take this form, and give them names." Pattern matching succeeds here
when checking means "is the form of data the way I expect"; in fact, as we know,
assignment in general \it{is} pattern matching (figure/example?). But when
checking means "does this computation succeed" or "does this binding conflict
with a prior binding," pattern matching is at a loss, where Verse succeeds. 

Example: 


Pattern matching's extensions get closer to unifying "checking" and binding.
Here, Verse enjoys a different suite of advantages. 

First, its "or" operation (`one` with `choice`) allows for more than patterns to
appear as a top-level "choose this or that" construct in a match sequence; you
can also include arbitrary expressions. You can't do this in pattern matching's
version, which is an or-pattern. Simply put, saying "does this pattern match or
is this expression true" is easy in Verse and clunky if you use patterns. I'll
show examples at our meeting. 

Second, Verse can express operations out of order, letting important checks
appear higher up even if they are executed later. This helps program legibility.
Again, I have examples from the chapter I wrote today. 

Third, in a pattern-match clause, the initial data must still match an initial
pattern in order to enter a guard; in Verse there is no restriction. This is
minor, because you could simply match the data to a variable, and then enter a
guard-- but again, all of these advantages are in elegance and brevity, and
elegant that solution is not. 

Finally, mingling pattern guards with other extensions to pattern matching
(especially or-patterns) is a murky subject. Haskell has pattern guards and side
conditions, but no or-patterns. OCaml has side conditions and or-patterns, but
no guards. Mixing all three is (according to some readings) simply difficult for
implementers- including, interestingly, those of parsers. In Verse, having `one`
and `choice` closely tied in with the simple `e1 = e2` equation form, which by
itself subsumes pattern matching, side-conditions, and pattern guards, means
that integrating options is easy. A key theme that arises of this: Verse has
fewer constructs, and they are more expressive.  

\section{A Proposal, Inspired by the Verse Calculus}

\subsection{Verse Flexibility}
\subsection{Something else}
\subsection{A third thing}

\section{Verse's Equations Subsume Pattern Matching}

\subsection{Claim}
\subsection{Proof}
\subsection{Translations}

\section{(Maybe) Writing Efficient Verse Code}

\subsection{Claim}
\subsection{Proof}
\subsection{Translations}


\section{Citations and Bibliographies}

\end{document}
\endinput
%%
%% End of file `sample-authordraft.tex'.

\documentclass[manuscript,screen,review, 12pt, nonacm]{acmart}
\let\Bbbk\relax % Fix for amssymb clash 
\usepackage{vmlmacros}
\AtBeginDocument{%
  \providecommand\BibTeX{{%
    \normalfont B\kern-0.5em{\scshape i\kern-0.25em b}\kern-0.8em\TeX}}}
\usepackage{outlines}
\setlength{\headheight}{14.0pt}
\setlength{\footskip}{13.3pt}
\title{An Alternative to Pattern Matching, Inspired by Verse}

\author{Roger Burtonpatel}
\email{roger.burtonpatel@tufts.edu}
\affiliation{%
\institution{Tufts University}
\streetaddress{419 Boston Ave}
  \city{Medford}
  \state{Massachusetts}
  \country{USA}
  \postcode{02155}
  }
\begin{document}
  
\section{A compromise}
  % To explore the design space of pattern matching, equations, and decision
  % trees,
    I have created and implemented a semantics for three core languages that aim
    to bridge the gap between pattern matching, equations, and decision trees.
    They are \PPlus, which has pattern matching with all three popular
    extensions, \VMinus, which has equations without backtracking or multiple
    results, and \D, which has decision trees. 

    In this section, I present the three languages. Their semantics appear in
    Figures X, Y, and Z, respectively, and in Appendix A. TODO In my design, I
    took inspiration from Verse: each of \PPlus, \VMinus, and \D\ has a
    conventional sub-language that is the lambda calculus extended with named
    value constructors:
    
    
    TODO FLUSH THIS OUT

    I chose named value constructors over \VC's tuples
    because they look more like patterns. 

    
    Because they share a core, and to facilitate comparisons and proofs, I have
    decided to present \VMinus\, \PPlus\, and \D\ as three subsets of a single
    unifying language, whose abstract syntax is given in Figure
    \ref{fig:unilang} TODO. Column “Unique To” indicates which components of
    this unifying language belong to the sub-languages. For all intents and
    purposes, the three languages are distinct; it is because they all share the
    same core of lambdas, value constructors (\it{K}), names, and function
    application that I have decided to house them under the same umbrella
    figure. 
    
    Like in \VC, every lambda calculus term is valid in our langauges and has
    the same semantics. Also like the lambda calculus and \VC, all three
    languages are untyped. Creating a type system for \PPlus\ and \VMinus\ is a
    worthy effort but is the subject of another paper. \D\ mainly exists as the
    target of translations, so typing it is neither particularly interesting nor
    worthwhile. 
    
    In the subsections below, I discuss each language in more detail. Section 7
    discusses futher how \PPlus\ and \VMinus\ relate to modern languages with
    pattern matching and \VC\ respectively. 

\subsection{Introducing \PPlus}

    \PPlus\ packages common extensions to pattern matching with a standard core.
    In addition to bare pattern matching— names and applications of value
    constructors— the language includes pattern guards, or-patterns, and side
    conditions. You may point out that pattern guards subsume side conditions
    (yes, they do!); we include side conditions and separate them from guards
    for purpose of study. Furthermore, \PPlus\ introduces a new form of pattern:
    \pcommap. This allows a pattern in \PPlus\ be a \it{sequence} of
    sub-patterns, allowing a programmer to stuff as many patterns as they want
    in the space of a single one. I discuss the implications of this in Section
    TODO.

    \bf{\PPlus\ concrete Syntax}

\begin{figure}
\begin{center}
\begin{bnf}
$P$ : \textsf{Programs} ::=
$\bracketed{d}$ : definition
;;
$d$ : \textsf{Definitions} ::=
| $\tt{val}\; x\; \tt{=}\; \expr$ : bind name to expression
;;
$\expr$ : Expressions ::= 
| $v$ : literal values 
| $x, y, z$ : names
| $K\bracketed{\expr}$ : value constructor application 
| $\ttbackslash x.\; \expr$ : lambda declaration  
| $\expr[1]\; \expr[2]$ : function application 
| $\tt{case}\; \expr\; \bracketed{p\; \ttrightarrow\; \expr}$ : case expression 
| \ttbraced{$\expr$}
;;
$p$ : \textsf{Patterns} ::= $p_{1}\pbar p_{2}$ : or-pattern
| $p \tt{,} p'$ : pattern guard 
| $p\; \tt{<-}\; \expr$ : pattern from explicit expression  
| $x$ : name 
| $\tt{\_}$ : wildcard 
| $K\; \bracketed{p}$ : value constructor application 
| $\tt{when}\; \expr$
| \ttbraced{$p$}
;;
$\v$ : Values ::= $K\bracketed{\v}$ : value constructor application 
| $\ttbackslash x.\; \expr$ : lambda value 
;;
$K$ : \textsf{Value Constructors} ::=
| $\tt{true}\; \vert\; \tt{false}$ : booleans
| $\tt{A-Z}x$ : name beginning with capital letter
% | $[\tt{-}\vert\tt{+}](\tt{0}-\tt{9})+$ : signed integer literal 
\end{bnf}
\end{center}
\Description{The concrete syntax of PPlus.}
\caption{\PPlus: Concrete syntax}
\end{figure}

  \subsection{Forms of Judgement for {\PPlus}:}
\begin{tabular}{ll}
\toprule
    \multicolumn2{l}{\emph{Metavariables}} \\
\midrule
    $\expr$& expression \\
    $\v,\; \v'$& value \\
    $K$& value constructor \\ 
    $p$& pattern \\ 
    $x, y$& names \\ 
    $g$& pattern guard \\ 
    $gs$& list of pattern guards \\ 
    $\dagger$& pattern match failure \\ 
    $s$& a solution, either $\v$ or $\dagger$ \\ 
    \Rho& environment: $name \rightarrow \mathcal{\v}$ \\
    \Rho\:+\:\Rhoprime& extension \\
    $\Rho \uplus \Rho'$& disjoint union \\
    $\{ x \mapsto y \} $& environment with name $x$ mapping to $y$ \\
\bottomrule
\end{tabular}    


\begin{align*}
    &\pmatch \quad   \rm{ Pattern $p$ matches value $\v$, 
                          producing bindings $\rho'$;} \\
    &\pfail  \quad\; \rm{ Pattern $p$ does not match value $\v$.} 
\end{align*}

If a pattern is bound to an expression in the form $\pmatch[pat = {p \leftarrow
\expr}, newenv = \solution]$, it will match if the expression $e$ evaluates to
value $v'$ and $p$ matches with $v'$. If a pattern is standalone, as in any
other form, it will match on $v$, the \it{original} scrutinee of the case
expression. For example, in $\pmatch[pat = {p_{1}, p_{2} \leftarrow \expr},
newenv = \solution]$, $p_{1}$ matches with $v$, and when $\ppeval[result=\v']$,
$p_{2}$ matches with $\v'$. 

Pattern matching is defined inductively: 
\begin{itemize}
    \item a name $x$ matches any value $\v$, and produces the environment 
    $\bracketed{x \mapsto \v}$. 
    \item a value constructor $K$ applied to atoms  matches 
    a value $\v$ if $\v$ is an application of $K$ to the same number of values,
    each of which matches the corresponding atom. Its match produces 
    the disjoint union of matching all internal atoms to all internal values. 
    \item a pattern \whenexpr\ matches when $\expr$ evaluates to a value other than 
    $\mathit{false}$, and produces the empty environment. 
    \item a pattern \parrowe\ matches when $\expr$ evaluates to 
          value $\v$, and $p$ matches $\v$. 
    \item a pattern \pcommap\ matches if both $p$ and $p_{2}$ match.
    \item a pattern \porp\ matches if either $p_{1}$ or $p_{2}$
    matches. 
\end{itemize}

When a pattern is of the form $K, p_{1}, \dots, p_{n}$, each sub-pattern
$p_{i}$ may introducing new variables during pattern matching. Bindings for
all these variables must be combined in a single environment. \it{Disjoint
union} is an operation on two environments:
TODO FROM BOOK 

%     TODO: include disjoint union section "Don't
% incorporate technical material by reference. You should explain disjoint union
% and +, but not necessarily the way I did. Your proximate audience is your
% thesis committee. Your ultimate audience is PL professionals."    


\subsection{Expressions in \PPlus}

    An expression in {\PPlus} evaluates to produce a single value. 
    \begin{align*}
        &\prun
    \end{align*}
    
    \ppsemantics 
    % TODO REINCLUDE 
      
      \bigskip 

    
    Figure~\ref{fig:ppfuncs} provides an example of how a programmer might utilize
    \PPlus\ to solve the previous problems: 

    \begin{figure}[ht] 
      \begin{minipage}[h]{0.54\linewidth}
        \pplst 
        \begin{lstlisting}[numbers=none, basicstyle=\tiny, xleftmargin=.2em,
          showstringspaces=false,
          frame=single]
val exclaimTall = \sh.
  case sh of Square s, when (> s) 100 -> 
            Wow! That's A Tall Square!  
    | Triangle w h, when (> s) 100 ->
            Goodness! Towering Triangle!
    | Trapezoid b1 b2 h, when (> s) 100 -> 
            Zounds! Tremendous Trapezoid!
    | _ ->  Your Shape Is Small
  \end{lstlisting}
      \Description{exclaimTall in PPlus}
      \subcaption{\tt{exclaimTall} in \PPlus}
          \label{fig:ppexclaimtall} 
      %   \vspace{4ex}
      \end{minipage}%%
      \begin{minipage}[h]{0.5\linewidth}
        \pplst 
        \begin{lstlisting}[numbers=none, basicstyle=\tiny, xleftmargin=2em,
                      frame=single]
  val tripleLookup = \ rho. \x.
    case x of 
      Some w <- (lookup rho) x
    , Some y <- (lookup rho) w
    , Some z <- (lookup rho) y -> z
    | _ -> handleFailure x

   \end{lstlisting}
        \Description{tripleLookup in PPlus}
        \subcaption{\tt{tripleLookup} in \PPlus}
            \label{fig:pptriplelookup} 
        \vspace{4ex}
      \end{minipage} 
      \begin{minipage}[h]{\linewidth}
        \pplst 
        \begin{lstlisting}[numbers=none, basicstyle=\tiny, xleftmargin=9em,
          showstringspaces=false,
          frame=single]
val game_of_token = \t. 
  case t of  
    BattlePass f  | (ChugJug f | TomatoTown f) -> P (Fortnite f)
  | HuntersMark f | SawCleaver f               -> P (Bloodborne ((* 2) f))
  | MoghLordOfBlood f | PreatorRykard f        -> P (EldenRing ((* 3) f))
  | _                                          -> P (Irrelevant 0)
\end{lstlisting}
      \Description{game_of_token in PPlus}
      \subcaption{\tt{game\_of\_token} in \PPlus}
          \label{fig:ppgot}
      \vspace{4ex}
      \end{minipage}%% 
      \caption{The functions in \PPlus\ have the same desirable implementation
      as before.}
  \label{fig:ppfuncs}
    \end{figure}        
    
    I've had to make some syntactic concessions: \PPlus doesn't have strings or
    tuples, so I've used masquerading value constructors, and the parser has no
    left recursion or notion of infix operators, so I've had to parenthesize a
    few expressions. Extending \PPlus\ with strings, left recursion, true
    tuples, and math operators is left as an exercise to the reader. 
    
\subsection{Addressing how \PPlus\ handles unusual pattern combinations}
    Given its minimal restrictions on what kind of pattern can appear where,
    \PPlus\ admits of strange looking patterns: consider \tt{Cons (when true)
    zs}. But these should not be alarming, because such syntactic forms reduce
    to normal forms by (direct) application of algebraic laws: 

    \begin{align}
      K (\whenexpr) \;p2\; \dots &=== K \;\_ \;p2\; \dots, \whenexpr \\
      K (\whenexpr, \;p2\;) \;p3\; \dots  &=== K \;p2\; \;p3\; \dots, \whenexpr \\
      K (p1\;, \whenexpr) \;p3\; \dots  &=== K \;p1\; \;p3\; \dots, \whenexpr \\
      K (\whenexpr \pbar p2\;) \;p3\; \dots &=== (K \;\_ \;p3\; \dots, \whenexpr) \pbar (K \;p2\; \;p3\; \dots) \\
      K (p1 \pbar \whenexpr) \;p3\; \dots &=== (K \;p2\; \;p3\; \dots) \pbar (K \;\_ \;p3\; \dots, \whenexpr)  \\
      \whenexpr \leftarrow e &=== \;\_ <- e, \whenexpr
    \end{align}
    
    
    Repeatedly applying these laws until the program reaches a fixed point
    normalizes placements of \it{when}. Laws (2) and (3) work because \PPlus\
    has no side effects and the laws assume all names are unique (the compiler
    takes care of this), so changing the order in which patterns match has no
    effect on semantics.         


\subsection{Introducing \VMinus\ }

        To fuel the pursuit of smarter decision-making, I now draw inspiration
        from \VC. I begin by trimming away the aforementioned culprits that
        tends to lead to unpredictable or costly outcomes during runtime:
        backtracking and multiple results. Removing these language traits strips
        much of the identity of \VC; however, I do so in order to study only
        \VC's \it{equations} while staying grounded in an otherwise-typical
        programming context of single results and no backtracking at runtime.
        Both backtracking and multiple results often manifest within \VC's
        choice operator, which tempt its removal altogether. However, I am drawn
        to harnessing the expressive potential of this operator, particularly
        when paired with \VC's 'one' keyword. When employed with choice as a
        condition, 'one' elegantly signifies 'proceed if any branch of the
        choice succeeds. 
        
        To this end, I imagine a new language, \VMinus, in which choice is
        permitted with several modifications:

        \begin{enumerate}
        \item Choice may only appear as a condition or 'guard', not as a result
        or the right-hand side of a binding.
        \item If any branch of the choice succeeds, the choice succeeds,
        producing any bindings found in that branch. The program examines the
        branches in a left-to-right order.
        \item The existential $\exists$ may not appear under choice.
        \end{enumerate}

        I introduce one more crucial modification to the \VC\ runtime: a name in
        \VMinus is an \it{expression} rather than a \it{value}. This alteration,
        coupled with my adjustments to choice, eradicates backtracking. Our
        rationale behind this is straightforward: if an expression returns a
        name, and subsequently, the program imposes a new constraint on that
        name, it may necessitate the reevaluation of the earlier expression—- a
        scenario I strive to avoid. An example in \VC\ illustrates this precise
        scenario:

        FIRST PAPER EXAMPLE 

        IMPORTANTLY, \VMinus REALLY ISN'T A FUNCTIONAL LOGIC PROGRAMMING LANGUAGE ANYMORE. 
        
         Doesn't backtrack
        
         No multiple results 
        
         Doesn't evaluate functions backwards, have top-level patterns like verse, the list goes on 

        % If you're well-versed in functional logic programming, may perceive that
        % imposing such restrictions on choice and names effectively strips away
        % much of Verse's essence as a functional logic programming language. With
        % these constraints enforced, there can be no backtracking, multiple
        % results, backward function evaluation, or top-level patterns, among
        % other classic functional logic programming features. But do not fear:
        % our intent is not to recklessly strip Verse of its meticulously crafted
        % core tenets. Instead, our aim is to extract a select few—namely, its
        % equations, existentials, and nondeterministic evaluation order—and
        % juxtapose them with pattern matching.    

        Indeed, having stripped out the functional logic programming elements of
        \VC, we are left with just the decision-making bits. To wrap these, we
        add a classic decision-making construct: guarded commands [cite-
        Dijkstra paper, others.] The result is \VMinus. \VMinus is defined in
        Figure~(TODO, whatever extension solution you choose). 

        % \bf{Choice forces "name knowing." Names must be known in all branches 
        % if bound to unknown variable or bound to known variable if any name 
        % is unknown}.
        
        With these modifications in place, alongside a few additional
        adjustments to the placement of existentials and the timing of choice
        evaluation (footnote: we discuss these later), we unveil a redefined decision-making core. To complete this
        transformation, we incorporate a venerable decision-making construct:
        Guarded commands [cite: Dijkstra]. The result is \VMinus. 

    \bf{Definition of \VMinus}

    
    \bf{\VMinus\ admits of many of the same pleasing computations as full Verse. }
    
    Even with multiple modifications, \VMinus\ still allows for many of the same
    pleasing computations as full Verse. Programmers can... 
        \begin{enumerate}
            \item Introduce a set of equations, to be solved in a nondeterministic order 
            \item Guard expressions with those equations 
            \item Flexibly express "proceed when any of these operations succeeds" with choice 
        \end{enumerate}

    Figure~TODO provides an example of how a programmer might utilize
    \VMinus\ to solve the previous problems: 
    
    \bf{Examples of programming in \VMinus }
    
    \bf{Show: beautiful examples. }
    
    Look! It looks just like Verse! 
   
    \bf{\VMinus\ is at least as good as \PPlus? Or just as comparable? }
        Or incomparable? (Why not both?)
    
    \subsection{\VMinus\ and \PPlus, side by side}

    \begin{itemize}
        \item We compare \VMinus\ with \PPlus\ as an exercise in comparing
        equations with pattern matching. 
        \item They definitely look similar: as expressive as pattern. Great! 
        \item{BIG PAYLOAD ALERT}
        \item Bonus 1: You don't need the scrutinee in \VMinus. 
        \item Bonus 2: Binding, checking are all in one construct: =. No
                $\leftarrow$ vs case. No guessing what value patterns match to. 
                And, unlike ML, Haskell, OCaml: no need for \tt{let}.
        \item Bonus 3: Names. Explicit name introduction means we can use a name
        from before and know if it's a pattern or an expression. This prevents
        common mistakes like the kind 105 students make. 
    \end{itemize}
    

    \bf{\bf{This is likely motivated strongly by examples of comparable code.}}

    Let's say 2-3 examples here


\bigskip


\begin{center}
    \begin{bnf}
    $P$ : \textsf{Programs} ::=
    $\bracketed{d}$ : definition
    ;;
    $d$ : \textsf{Definitions} ::=
    | $\tt{val}\; x\; \tt{=}\; \expr$ : bind name to expression
    ;;
    $\expr$ : \textsf{Expressions} ::=
    | $v$ : literal values 
    | $x, y, z$ : names
    | $\tt{if}\; \tt{[}\; G\; \bracketed{\square\; G}\; \tt{]}\; \tt{fi}$ : if-fi 
    | $K \bracketed{\expr}$ : value constructor application 
    | $\expr[1]\; \expr[2]$ : function application 
    | $\lambda x.\; \expr$ : lambda declaration 
    ;;
    $G$ : \textsf{Guarded Expressions} ::=  
    $[\vexists{\bracketed{x}}] \bracketed{g} \boldsymbol{\rightarrow}\expr$ : names, guards, and body
    ;;
    $g$ : \textsf{Guards} ::=  
    | $\expr$ : intermediate expression 
    | $x = \expr$ : equation 
    | $ g \bracketed{; g} \; \choice \; g \bracketed{; g}$ : choice 
    ;;
    $\v$ : Values ::= $K\bracketed{\v}$ : value constructor application 
    | $\lambda x.\; \expr$ : lambda value
    ;;
    \end{bnf}
\end{center}

A \it{name} is any token that is not an integer literal, 
does not contain whitespace, a bracket, or parenthesis, 
and is not a value constructor name or a reserved word.
        
\subsection{Refinement ordering on environments}

\begin{align*}
\rho \subseteq \Rhoprime \text{ when }&\dom\rho  \subseteq \dom \Rhoprime\\
\text{ and } &\forall x \in \dom \rho: \rho(x) \subseteq \Rhoprime(x)
\end{align*}



\vfilbreak



\subsection{Forms of Judgement for ${\VMinus}$:}
\begin{tabular}{ll}
\toprule
    \multicolumn2{l}{\emph{Metavariables}} \\
\midrule
    $\expr$& An expression \\ 
    $\v,\; \v'$& value \\
    $\fail$& expression failure \\
    $\result$& $\v \vert \fail$ : expressions produce \it{results}: values or
    failure. \\
    \Rho& environment: $name \rightarrow {\mathcal{V}}_{\bot}$ \\
    $\rho\{ x \mapsto y \} $& environment extended with name $x$ mapping to $y$ \\
    $g$& A guard \\
    $eq$& equation \\ 
    % $\tempstuck$& a temporarily-stuck equation \\
    $\dagger$& when solving guards is rejected \\
    $\result$& $\rhohat \mid \dagger$ : guards produce \it{solutions}: a
    refined environment \rhohat\; or rejection\\
    $\mathcal{T}$& Context of all temporarily stuck guards (a sequence) \\ 
    $G$& A guarded expression \\
    % \uppsidown& Inability to compile to a decision tree; a compile time error \\
\bottomrule
\end{tabular}    

\bigskip

  
    
    \medskip
    
    Success only refines the environment; that~is, when
    ${\vtuple{\rho, \expr}} \rightarrowtail{} {\Rhoprime}$, 
    we~expect $\rho \subseteq \Rhoprime$.
    
    \subsection{Expressions}
    
    \newcommand\GNoTree{\vmrung \rightsquigarrow \uppsidown} 
    
    An expression in core Verse evaluates to produce possibly-empty sequence of
    values. In {\VMinus}, expressions evaluate to a single value. 
    % We will explore a Verse variant in which expressions may evaluate to
    % multiple values. 

    An expression evaluates to produce a \bf{result}. A result is either a
    value $\v$ or \fail. 
    
    \[\result\; \rm{::=}\; \rhohat\; \vert \; \fail \]
    
    \showvjudgement{Eval}{\vmeval}
    \showvjudgement{Process-Guards}{\vmgs}
    
    \bigskip
\subsection{Guards}

In {\VMinus}, we solve guards similarly to how the authors of the original Verse
paper solve equations: we pick one, attempt to solve for it, and move on. 

In our semantics, we do this with a \it{list} of guards: we pick a guard,
attempt to solve it to refine our environment or fail, and repeat. We only pick
a guard out of the context \context\ that we know we can solve. If we can't pick
a guard and there are guards left over, the semantics gets stuck.

Given an environment $\rho$ from names to values {\v}s plus $\bot$, solving a
guard \g\ will either refine the environment ($\Rhoprime$) or be \bf{rejected}.
We use the metavariable $\dagger$ to represent rejection, and a guard producing
$\dagger$ will cause the top-level list of guards to also produce $\dagger$. 

    \showvjudgement{Guard-Refine}{\grefine[result=\rho']}
    \showvjudgement{Guard-reject}\gfail

TODO Uncomment for semantics here
\vmsemantics

\end{document}
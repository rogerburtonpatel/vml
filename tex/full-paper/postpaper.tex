\documentclass[manuscript,screen,review, 12pt, nonacm]{acmart}
\let\Bbbk\relax % Fix for amssymb clash 
\usepackage{vmlmacros}
\AtBeginDocument{%
  \providecommand\BibTeX{{%
    \normalfont B\kern-0.5em{\scshape i\kern-0.25em b}\kern-0.8em\TeX}}}
\usepackage{outlines}
\setlength{\headheight}{14.0pt}
\setlength{\footskip}{13.3pt}
\title{An Alternative to Pattern Matching, Inspired by Verse}

\author{Roger Burtonpatel}
\email{roger.burtonpatel@tufts.edu}
\affiliation{%
\institution{Tufts University}
\streetaddress{419 Boston Ave}
  \city{Medford}
  \state{Massachusetts}
  \country{USA}
  \postcode{02155}
  }
\begin{document}

\section{Acknowledgements}

This thesis would not have been possible without the infinitely generous time
and support of my advisors, Norman Ramsey and Milod Kazerounian. Norman's
offhand comment of “I wonder if Verse's equations subsume pattern matching” was
the entire basis of this work, and his generosity in agreeing to advise a full
thesis to answer his question will always be profoundly appreciated. During the
academic year, Norman provided me with materials on both the technical story and
on how to write about it well. He gave me regular feedback and has helped
improve my research skills, my technical writing, and my understanding of
programming languages in general tremendously. I especially appreciate how he
has guided me in-person at the end of my undergraduate when his book~\citep{bpc}
got me started down the path of PL at the beginning of it. Finally, Norman is
also fantastically fun to pair program with. 

From the beginning, Milod provided me with encouraging mentorship that kept
me enthusiastic and determined to complete the project. He was exceptionally
patient as I gave him whirlwind tour after whirlwind tour of the changing
codebase and problems, and kept me grounded in the problems at hand. He 
sent me helpful examples of his research to aid me in my proofs, and gave me
some particularly encouraging words towards the end of the project that I 
will not soon forget. 

My undergraduate advisor, Mark Sheldon, has always been both supportive and
kind. I have enjoyed many long talks in his office, and I am deeply grateful
that he is on my committee. 

Alva Couch was the advisor of my original thesis idea, which was to compile
programming languages with music. Ultimately, we decided that I should pursue
this project instead, and I am grateful to him for his mentorship in that moment
and onwards. 

My family- my mother, Jennifer Burton, my father, Aniruddh Patel, and my
sister, Lilia Burtonpatel, have all given me support, encouragement, and
(arguably most importantly) food. My gratitude for them is immeasurable. 

My gratitude towards my friends is also without limit. In particular, and in
no order, Liam Strand, Annika Tanner, Max Stein, Cecelia Crumlish, and
Charlie Bohnsack all showed specific interest in the work and encouraged me
as I moved forward. Rachael Clawson was subjected to much rubber-ducking,
and endured valiantly. Aliénor Rice and Marie Kazibwe were as steadfast
thesis buddies as I could ever hope for. Jasper Geer, my PL partner in
crime, was always one of my favorite people to talk to about my thesis. His
pursuits in research inspire mine. 

Skylar Gilfeather, my unbelievable friend. Yours is support that goes
beyond words; care, food, silent and spoken friendship, late night rides to
anywhere, laughs and tears are some that can try to capture it. I am so, so
grateful for how ceaselessly you've encouraged me on this journey. Every one
of your friends is lucky to have you in their life, and I am blessed that 
you are such a core part of mine. 

Anna Quirós, I am writing these words as you sleep behind me. Your support
and love have been immeasurable. I will always be grateful to you for this
year \it{increíble}. I could not have smiled through it all without you. 

Thank you all. 

\section{References}
\bibliographystyle{authordate3}
\bibliography{sources}

\renewcommand\thesection{\Alph{section}}
\setcounter{section}{0}
\section{Proofs}
\begin{outline}
\1 \bf{Proof: Translation from \VMinus to \D\ preserves semantics }
\1 \bf{Proof: Translation from \PPlus\ to \VMinus preserves semantics }
% \1 \bf{Proof: Translation from \VMinus to Verse preserves semantics}
\end{outline}

\section{Is \VMinus\ a true subset of \VC?}
\VMinus\ certainly appears to relate to \VC\ semantically. Translating
\it{if-fi} and choice in \VMinus to \bf{one} and choice in \VC\ is likely a
sufficient embedding. Formalizing this translation and, more importantly,
proving that our semantics of \VMinus are consistent with Augutsson et al.'s
\VC\ is an excellent goal for future work. 

\section{Formal Definitions of all languages}
\label{languagedefs}

\utable
\ppsemantics
\vmsemantics

% \begin{figure}
% \begin{center}
%     \begin{bnf}
%     $P$ : \textsf{Programs} ::=
%     $\bracketed{d}$ : definition
%     ;;
%     $d$ : \textsf{Definitions} ::=
%     | $\tt{val}\; x\; \tt{=}\; \expr$ : bind name to expression
%     ;;
%     $\expr$ : \textsf{Expressions} ::=
%     | $v$ : literal values 
%     | $x, y, z$ : names
%     | $\tt{if}\; [\; G\; \bracketed{\tt{[]}\; G}\;]\; \tt{fi}$ : if-fi 
%     | $K \bracketed{\expr}$ : value constructor application 
%     | $\expr[1]\; \expr[2]$ : function application 
%     | $\ttbackslash x\tt{.}\; \expr$ : lambda declaration 
%     ;;
%     $G$ : \textsf{Guarded Expressions} ::=  
%     $[\tt{E }{\bracketed{x}}{\tt{.}}] \bracketed{g} \;\ttrightarrow\; \expr$ : names, guards, and body
%     ;;
%     $g$ : \textsf{Guards} ::=  
%     | $\expr$ : intermediate expression 
%     | $x \;\tt{=}\; \expr$ : equation 
%     | $ g \bracketed{\tt{;} g} \; \pbar \; g \bracketed{\tt{;} g}$ : choice 
%     ;;
%     $\v$ : Values ::= $K\bracketed{\v}$ : value constructor application 
%     | $\ttbackslash x\tt{.}\; \expr$ : lambda value
%     \end{bnf}
% \end{center}
% \Description{The concrete syntax of PPlus.}
%     \caption{\VMinus: Concrete syntax}
%     \label{fig:vmsyntax}
%     \end{figure}

% \begin{figure}
%     \begin{center}
%     \begin{bnf}
%     $P$ : \textsf{Programs} ::=
%     $\bracketed{d}$ : definition
%     ;;
%     $d$ : \textsf{Definitions} ::=
%     | $\tt{val}\; x\; \tt{=}\; \expr$ : bind name to expression
%     ;;
%     $\expr$ : Expressions ::= 
%     | $v$ : literal values 
%     | $x, y, z$ : names
%     | $K\bracketed{\expr}$ : value constructor application 
%     | $\ttbackslash x\tt{.}\; \expr$ : lambda declaration  
%     | $\expr[1]\; \expr[2]$ : function application 
%     | $\tt{case}\; \expr\; \bracketed{p\; \ttrightarrow\; \expr}$ : case expression 
%     | \ttbraced{$\expr$}
%     ;;
%     $p$ : \textsf{Patterns} ::= $p_{1}\pbar p_{2}$ : or-pattern
%     | $p \tt{,} p'$ : pattern guard 
%     | $p\; \tt{<-}\; \expr$ : pattern from explicit expression  
%     | $x$ : name 
%     | $\tt{\_}$ : wildcard 
%     | $K\; \bracketed{p}$ : value constructor application 
%     | $\tt{when}\; \expr$
%     | \ttbraced{$p$}
%     ;;
%     $\v$ : Values ::= $K\bracketed{\v}$ : value constructor application 
%     | $\ttbackslash x\tt{.}\; \expr$ : lambda value 
%     ;;
%     $K$ : \textsf{Value Constructors} ::=
%     | $\tt{true}\; \vert\; \tt{false}$ : booleans
%     | $\tt{A-Z}x$ : name beginning with capital letter
%     % | $[\tt{-}\vert\tt{+}](\tt{0}-\tt{9})+$ : signed integer literal 
%     \end{bnf}
%     \end{center}
%     \Description{The concrete syntax of PPlus.}
%     \caption{\PPlus: Concrete syntax}
%     \end{figure}

\end{document}

%%
\documentclass[manuscript,screen,review, 12pt]{acmart}
\let\Bbbk\relax % Fix for amssymb clash 
\usepackage{vmlmacros}
\AtBeginDocument{%
  \providecommand\BibTeX{{%
    \normalfont B\kern-0.5em{\scshape i\kern-0.25em b}\kern-0.8em\TeX}}}
\usepackage{outlines}
\setlength{\headheight}{14.0pt}
\setlength{\footskip}{13.3pt}
\begin{document}
\title{A Replacement for Pattern Matching, Inspired by Verse}

\author{Roger Burtonpatel}
\email{roger.burtonpatel@tufts.edu}
\affiliation{%
  \institution{Tufts University}
  \streetaddress{419 Boston Ave}
  \city{Medford}
  \state{Massachusetts}
  \country{USA}
  \postcode{02155}
}

\author{Norman Ramsey}
\email{nr@cs.tufts.edu}
\affiliation{%
\institution{Tufts University}
\streetaddress{177 College Ave}
\city{Medford}
\state{Massachusetts}
\country{USA}
\postcode{02155}
}

\author{Milod Kazerounian}
\email{milod.mazerouniantufts.edu}
\affiliation{%
\institution{Tufts University}
\streetaddress{177 College Ave}
\city{Medford}
\state{Massachusetts}
\country{USA}
\postcode{02155}
}

\renewcommand{\shortauthors}{Burtonpatel et al.}

\begin{abstract}
    \it{WILL BE REVISED.}
  Pattern matching is nice and has an appealing cost model, but seems to lack in
  expressiveness: people keep extending it to make it more expressive. Verse
  [cite-Verse] doesn't use pattern matching and instead favors equations, which
  look very simple and surprisingly expressive, but the language's cost model is
  a challenge. Why not look for a compromise? 
  
%   We introduce two small functional programming languages that
%   attempt to address the limitations of pattern matching in popular functional
%   languages.

  We explore the realm of pattern matching in common functional programming
  languages and compare it with Verse's equation-based approach. Recognizing the
  balance between expressiveness and efficiency, we introduce two toy languages
  aimed at addressing the limitations of prevalent pattern matching techniques.
  We show how a subset of Verse can be compiled to decision trees. 

  \end{abstract}

\maketitle

\section{Introduction}
\it{We will add this at the end.}

Pattern matching is popular and well-researched. However, it often needs to be
augmented with extensions. Is this the most efficient thing to be doing? Can we
take inspiration from Verse and use equations instead? Do Verse's equations
subsume pattern matching with popular extensions? Can we get them at no
additional runtime cost? 
% We consider a toy language with many of the popular
% extensions to pattern matching. 
% \VMinus can be compiled to efficient decision
% trees. 


\section{Pattern Matching and Equations}
\subsection{Pattern matching as an upgrade to observer functions}
    
    
    % extensively discussed in abundant literature
    Pattern matching is a well-established and researched construct in
    functional programming languages, and the topic of much rich literature
    [cite, appeal to authority]. A primary appeal of pattern matching is its
    dominance over observer functions\footnote{In functional programming,
    observer functions manually scrutinize and deconstruct data. Examples
    include Scheme's \tt{null?}, \tt{car}, and \tt{cdr}, and ML's \tt{null},
    \tt{hd}, and \tt{tl}.}. Consider implementation of \tt{List.length} in Standard
    ML, using observers in Figure~\ref{fig:observerlen}. \rab{Should I say
    'observers' to avoid repeating myself, or 'observer functions' to use the
    same term for the same thing?}
    
    % Figure 1 illustrates an example in which pattern matching is an elegant solution
    % to a problem (compare to equivalent Standard ML code that manually deconstructs
    % a list:)
    
    
    % Most functional programmers likely prefer the first example 
    
    
    % Pattern matching is a well-established and researched construct in functional
    % programming langauges [cite, appeal to authority]. It shines in its ability to
    % break down constructed data implicity, which is often much preferred to using
    % explict deconstructors like Scheme's \tt{car} and \tt{cdr}. But when deciding to
    % perform a terminal computation based on prior checks that don't involve a single
    % value matching on a pattern, pattern matching struggles to retain its brevity.
    % Consider the following algorithm on an abstract type~\tt{FiniteMap}, involving a
    % lookup function. This examples is a modificaiton of one borrowed from SPJ
    % proposal [cite]. 
    

    \begin{figure}[ht!]
        \smllst
        \captionsetup{justification=centering}
        \begin{verbatim}
            fun length ys =
                if null ys 
                then 0 
                else let xs = tl ys 
                     in length xs 
                     end 
        \end{verbatim}
    \caption{Implementing \tt{List.length} using observer functions is tedious\\\hspace{\textwidth}
    and doesn't look very functional.}
    % \caption{An implementation of \tt{List.length} in Standard ML that does not use 
    % pattern matching.}
    \Description{An implementation of length using null, hd, and tl}
    \label{fig:observerlen}
    \end{figure}
    
    Manual case analysis and deconstruction with observer functions can be both
    error-prone (\rab{say more on this or cite?}) and verbose (\rab{can I make
    this claim?}). Most functional programmers would likely prefer the
    implementation of \tt{length} found in Figure~\ref{fig:pmlen}.
    \begin{figure}[htt]
    \begin{verbatim}
            fun length ys =
                case ys 
                of []    => 0
                 | x::xs => 1 + length xs
        \end{verbatim}

    \caption{Using pattern matching makes an equivalent implementation 
             more appealing.}

    \Description{An implementation of length using implicit deconstruction via
    patterns.}
    \label{fig:pmlen}
    \end{figure}

    Figure~\ref{fig:pmlen} is an example of the most common occurrence of where
    pattern matching occurs: within a \tt{case} expression. \tt{case}
    expressions test a \it{head expression} or \it{scrutinee} (\tt{ys} in
    Figure~\ref{fig:pmlen}) against a list of patterns (\tt{[]}, \tt{x::xs}).
    When the result of evaluating the scrutinee matches on a pattern, the
    program evaluates the right-hand side of the respective branch (\tt{0} or
    \tt{1 + length xs}) within the \it{case} expression. 

    As a final step towards simplicity, most functional programming languages
    allow programmers to merge pattern matching and function declaration in a
    form of syntactic sugar known as a \it{clausal definition} that allow the
    programmer to eliminate the top-level \tt{case} entirely.
    Figure~\ref{fig:pmclausallen} shows a final implementation of \tt{length}
    using pattern matching with a clausal definition. 

    \begin{figure}[ht]
    \smllst
    \begin{verbatim}
            fun length []      = 0
              | length (x::xs) = 1 + length xs
        \end{verbatim}
    \caption{Pattern matching with a clausal definition gives a simple and 
             elegant implementation of \tt{List.length}.}
    \Description{An implementation of length using pattern matching within a 
    clausal definition.}
    \label{fig:pmclausallen}
    \end{figure}
        

    
\subsection{Extensions as upgrades to pattern matching}

    Indeed, pattern matching serves as a cornerstone in functional programming
    languages as a way to make decisions [cite, appeal to authority]. However,
    in particular instances, pattern matching may prove cumbersome, driving
    functional programming languages to frequently employ \it{extensions} to mitigate such
    scenarios. 
    
    Below, we illustrate several such instances and demonstrate how popular
    functional programming languages employ extensions to streamline clunky or verbose code.
    For the sake of comparison, we coin the term \it{bare pattern matching} to
    denote pattern matching \it{without} extensions: in bare pattern matching,
    the only syntactic forms of patterns are names and applications of value
    constructors to zero or more patterns. 
    

    

\subsubsection{Side conditions}

    First, we illustrate the need for \it{side conditions}\footnote{We use the
    term \it{side conditions} to refer to a pattern followed by a boolean
    expression. Some languages call this a \it{guard}, which we use to describe
    a different, more powerful extension to pattern matching in
    Section~\ref{guards}.}, an extension to pattern matching common in most
    popular functional programming languages, including OCaml, Erlang, Scala,
    and~Haskell\footnote{Haskell uses guards to subsume side conditions. We
    discuss guards in Section~\ref{guards}.}. Consider
    Figure~\ref{fig:ifnameof}. 
    
    \begin{figure}[ht]
        \begin{verbatim}
            let nameOf rho e = match e with 
                  NAME n -> if binds rho n then n else raise NotFound
                | _      -> raise NotFound  

                (* where binds rho n == true, where n in dom(rho)
                         binds rho n == false, otherwise
                and NotFound is an exception 
                delcared earlier in the program *)
            \end{verbatim}    
        \caption{An invented function \tt{nameOf} combines pattern matching with
        an \tt{if} expression, and is not very pretty.}    
        \Description{An implementation of a function nameOf in OCaml, which
        uses pattern matching and an if statement.}
        \label{fig:ifnameof}
    \end{figure}
    
    
    duplicates a right-hand side, and the actual “good” return of the function,
    \tt{n}, is lost in the middle of the \tt{if} expression. The reader likely
    sees how the code could be simplified with a \it{side condition} to a
    pattern, i.e., a syntactic form for “match a pattern \it{and} a boolean
    condition.” Fortunately, the \tt{when} keyword in OCaml provides such a
    form, as seen in Figure~\ref{fig:whennameof}.
        
        \begin{figure}[ht]
            \begin{verbatim}
            let nameOf rho e = match e with     
                  NAME n when binds rho n -> n
                | _                       -> raise NotFound  
                \end{verbatim}
            \caption{With a side condition, \tt{nameOf} becomes simpler and more
            clear in its goal.}
            \Description{An implementation of a function nameOf in OCaml, which
            uses pattern matching and a side condition.}
            \label{fig:whennameof}
            \end{figure}

        
    
    
    

    Programmers can employ side conditions to streamline cases and minimize
    overhead. One notable advantage of side conditions is their capacity to
    exploit bindings that emerge from the preceding pattern match. For
    instance, the \tt{when} clause in Figure~\ref{fig:whennameof} utilizes \tt{n},
    which is bound in the match of \tt{e} to \tt{NAME n}.

    Side conditions serve to address the necessity of incorporating an extra
    “check”- in this case, a boolean expression- within a pattern. But what if
    the need arises to conduct additional pattern matches within such a check?
    Here, side conditions prove inadequate. Without additional tooling,
    programmers may resort to nested pattern matching as a potential solution;
    yet, this often leads to duplicated right-hand sides and convoluted code
    that is cumbersome to write and challenging to follow at first glance. We
    now consider a more powerful extension to pattern matching which addresses
    this problem. 

 \subsubsection{Pattern guards}
\label{guards}

    % 

    
    % 
    % \rab{Help here:
            % \begin{enumerate}
            % \item I'm quoting a chunk of another paper, including code blocks. 
            % My education up to now hasn't covered this. What's protocol here?
            % \item How can I force the figures to appear in the same order as in 
            %       the original paper? Using [h!] does not appear to be working. 
            % \item With correct formatting, I believe this borrowed example is
            % helpful as a clear example and an appeal to authority to illustrate
            % the expressive power of guards as extensions to pattern matching. Do
            % you agree? 
            % \end{enumerate}}


            % \rab{Better way to begin this next paragraph?}

            While pattern matching is known for its effectiveness in
            deconstructing individual values, its expressiveness diminishes when
            faced with sequential computations. What about when we wish to fire
            a right-hand side (\rab{should I just say "evaluate?" Many authors I read use "fire a
            right-hand side."}) only after \it{multiple
            sequential expressions} match with patterns? Consider the following
            example, borrowed from Erwig and Peyton-Jones (2001): 

    \begin{quote}
    
    \dots suppose we have an abstract data type of finite maps, with a lookup
    operation:

        \begin{verbatim}


lookup :: FiniteMap -> Int -> Maybe Int
        \end{verbatim}

        The lookup returns \tt{Nothing} if the supplied key is not in the domain of
    the mapping, and \tt{(Just v)} otherwise, where \tt{v} is the value that the
    key maps to. Now consider the following definition:


    % \begin{figure}[hbt!]
    \begin{minipage}{\textwidth}
        \begin{verbatim}
clunky env varl var2 | ok1 && 0k2 = val1 + val2 
                        | otherwise  = var1 + var2 
where 
    m1  = lookup env var1 
    m2  = lookup env var2
    ok1 = isJust m1 
    ok2 = isJust m2 
    Just val1 = m1 
    Just val2 = m2    
        \end{verbatim}        
    \end{minipage}
    % \caption{\tt{clunky} in Haskell with bare pattern matching} \Description{An
    % implementation of clunky using only simple pattern matching within a clausal
    % definition.}
    % \label{fig:whereclunky}
    % \end{figure}
        
    \dots This is certainly legal Haskell, but it is a tremendously verbose and
    un-obvious way to achieve the desired effect. Is it any better using case
    expressions?

        % The authors show how explicit pattern matching with \tt{case} may
        % mitigate the verbosity of this problem:
    
    % \begin{figure}[hbt!]
        \begin{minipage}{\textwidth}
            \begin{verbatim}


clunky env var1 var2 = 
    case lookup env var1 of 
        Nothing -> fail 
        Just val1 -> 
            case lookup env var2 of 
            Nothing -> fail 
            Just val2 -> val1 + val2
where 
    fail = var1 + var2
            \end{verbatim}
        \end{minipage}
    % \caption{\tt{clunky} with \tt{case}} 
    % \Description{An implementation of clunky using case.}
    % \label{fig:caseclunky}
    % \end{figure}
    
        This is a bit shorter, but hardly better. Worse, if this was just one
        equation of \tt{clunky}, with others that follow, then the thing would
        not work at all.

    \end{quote}

    (Source: Erwig, Martin, and Simon Peyton Jones. "Pattern guards and transformational patterns." Electronic Notes in Theoretical Computer Science 41.1 (2001): 2-3.)

        To mitigate the problem, the authors propose \it{pattern guards}, a form
        of “smart pattern” in which intermediate patterns bind to
        expressions within a single branch of a match. Their solution with
        pattern guards is shown in Figure~\ref{fig:guardclunky}. 

    \begin{figure}[hbt!]  
        \begin{center}
        \begin{verbatim}
                        clunky env var1 var2    
                           | Just vall <- lookup env var1
                           , Just val2 <- lookup env var2
                           = val1 + val2
        \end{verbatim}
        \it{… other equations for clunky}
        \end{center}    
    \caption{The authors' solution to \tt{clunky} with pattern guards is plainly
    the most elegant.} 
    \Description{An implementation of clunky using pattern
    guards.}
    \label{fig:guardclunky}
    \end{figure}

    Guards offer an elegant solution to the problem of matching on successive
    computations.\footnote{The authors of the proposal show several other
    examples where guards drastically simplify otherwise-maddening code.}
    Additionally, like side conditions, a further power of pattern guards lies
    in the ability for expressions within guards to utilize names bound in
    preceding guards. This enables imperative pattern-matched steps with
    expressive capabilities akin to Haskell's \tt{do} notation.

    \rab{two things: 
    \begin{enumerate}
    \item Do I try to cite this last claim, or come up with my own example? I
    came up with this idea, and have not been able to find prior evidence that
    it exists. "guards are like \tt{do}" is not a strong claim, so maybe I can
    get away with just saying it? Would like your guidance.
    \item {should this "additional power of pattern guards" be its own example?}
    \end{enumerate}}

        % \it{This is obviously better. But now consider something where you need 
        % or-patterns; this fails. }
        

\subsubsection{Or-patterns}

        In the preceding example, we observed how pattern guards facilitated a
        multi-step computation within a single match. However, what if the
        programmer's intention isn't to match on \it{all} parts of a pattern
        sequence, but instead to match a value on \it{any one} of such a
        sequence of patterns? Our example in Figure~\ref{fig:barepgot}
        illustrates this need. Examining \tt{parent\_game\_of\_token}, written
        in OCaml using bare patterns, many functional programmers likely sense
        the need for such functionality.
        
        
        \begin{figure}
            \begin{center}
                \begin{verbatim}
        let parent_game_of_token token = match token with 
            | BattlePass      -> "Fortnite"
            | ChugJug         -> "Fortnite"
            | TomatoTown      -> "Fortnite"
            | HuntersMark     -> "Bloodborne"
            | SawCleaver      -> "Bloodborne"
            | MoghLordOfBlood -> "Elden Ring"
            | PreatorRykard   -> "Elden Ring"
            | _               -> "Irrelevant"
                \end{verbatim}
            \end{center}    

        \caption{\tt{parent\_game\_of\_token}, with redundant right-hand sides,
        should raise a red flag.} 
        \Description{An implementation of
        parent_game_of_token using bare patterns.}
        \label{fig:barepgot}
        \end{figure}

Thankfully, an extension once again comes to our rescue. \it{Or-patterns}
condense multiple patterns which share right-hand sides, and programmers can
exploit them to eliminate much redundant code
(Figure~\ref{fig:orpgot}).\footnote{Or-patterns are a feature built into OCaml.}

    \begin{figure}
    \begin{center}
    \begin{verbatim}
        let parent_game_of_token token = match token with 
            | BattlePass | ChugJug | TomatoTown  -> "Fortnite"
            | HuntersMark | SawCleaver           -> "Bloodborne"
            | MoghLordOfBlood | PreatorRykard    -> "Elden Ring"
            | _                                  -> "Irrelevant"
    \end{verbatim}
    \end{center}    
    \caption{With or-patterns, \tt{parent\_game\_of\_token} condenses
    tremendously and is easier to read line-by-line.} 
    \Description{An
    implementation of parent_game_of_token using or-patterns.}
    \label{fig:orpgot}
    \end{figure}

    In addition to the inherent appeal of brevity, or-patterns serve to
    concentrate complexity at a single juncture. Consider a scenario where,
    instead of a string, \tt{parent\_game\_of\_token} yields the outcome of a
    complex computation. In the initial model, duplicating the right-hand sides
    across multiple patterns would necessitate the programmer to manage numerous
    redundant copies of this computation. Or-patterns empower programmers to
    centralize such maintenance at a single point of truth.


\end{document}
\endinput
%%

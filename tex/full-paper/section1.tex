\documentclass[manuscript,screen,review, 12pt, nonacm]{acmart}
\let\Bbbk\relax % Fix for amssymb clash 
\usepackage{vmlmacros}

\begin{document}

\section{Introduction}
% Subjects: Pattern matching and Equations 

Pattern matching is a beloved tool among functional programmers for examining
and deconstructing data. 
% It does so implicitly by matching constructed data
% directly against a number of possible forms.
% Maybe combine? 
Pattern matching is also an established and well-researched topic in
academy[cite]. In particular, it is lauded for its ability to be compiled to a
\it{decision tree}, a data structure that enforces linear runtime performance by
guaranteeing no part of the data will be examined more than once.[cite]
% Extensions should have a section # to indicate they're coming 
But pattern matching has seen a number of extensions across popular programming
languages. Why? 
% A little too coloquial
Because pattern matching cannot succinctly express certain
common computations, and, without extensions, it forces programmers who wish to
express these computations to duplicate code, nest case expressions, and create
multiple points of truth. 

Extensions are nice, but they aren't unified across programming languages.
Rather than continuing to extend pattern matching in different directions \it{ad
infinitum}, it might be nice to find an alternative that doesn't need extensions
to succinctly solve the problems programmers face. Last year, Verse[cite -
Verse] introduced a tempting possibility in a new way to implicitly deconstruct
data: equations. Equations look expressive, and it appears at a glance that they
can express everything that pattern matching can, including the popular
extensions. 
% Make it very clear that Verse has nothing to do with PM 
But a full implementation of Verse is complicated, cost-wise. Verse is a
functional logic programming language, and that can mean expressions can
backtrack at runtime and return multiple results, both of which are
hard to predict in their costs. 

What would really be excellent is a compromise.
% A little too coloquial
Can the expressiveness of Verse's equations be combined with the decision tree
property of patterns? 


\bf{My contribution in this thesis} is to show that \VMinus, a language that
uses Verse's equations, can be compiled to a decision tree. I also demonstrate
how \VMinus\ subsumes pattern matching with popular extensions. 

Add some more things to expect in the paper: 

Central contribution of this thesis is X. In order to achieve this, 
I also made the following contributions: 

Formalize subset of Verse into a core language \VMinus, bigstep opsem 

Formalize pattern matching in a core language \PPlus 

Formalize translation between them 

Give section for each of these! 

Implementation of each language 

\end{document}
%%
\documentclass[manuscript,screen,review, 12pt]{acmart}
\let\Bbbk\relax % Fix for amssymb clash 
\usepackage{vmlmacros}
\AtBeginDocument{%
  \providecommand\BibTeX{{%
    \normalfont B\kern-0.5em{\scshape i\kern-0.25em b}\kern-0.8em\TeX}}}
\usepackage{outlines}
\begin{document}
\title{A Replacement for Pattern Matching, Inspired by Verse}

\author{Roger Burtonpatel}
\email{roger.burtonpatel@tufts.edu}
\affiliation{%
  \institution{Tufts University}
  \streetaddress{419 Boston Ave}
  \city{Medford}
  \state{Massachusetts}
  \country{USA}
  \postcode{02155}
}

\author{Norman Ramsey}
\email{nr@cs.tufts.edu}
\affiliation{%
\institution{Tufts University}
\streetaddress{177 College Ave}
\city{Medford}
\state{Massachusetts}
\country{USA}
\postcode{02155}
}

\author{Milod Kazerounian}
\email{milod.mazerouniantufts.edu}
\affiliation{%
\institution{Tufts University}
\streetaddress{177 College Ave}
\city{Medford}
\state{Massachusetts}
\country{USA}
\postcode{02155}
}

\renewcommand{\shortauthors}{Burtonpatel et al.}

\begin{abstract}
    \it{WILL BE REVISED.}
  Pattern matching is nice and has an appealing cost model, but people keep
  extending it to make it more expressive. Verse [cite-Verse] looks very simple
  and is surprisingly expressive, but the cost model is a challenge. Why not
  look for a compromise? We introduce two small functional langauges that
  attempt to address the limitations of pattern matching in popular functional
  languages.
  \end{abstract}

\maketitle

\section{Introduction}
\it{We will add this at the end.}

\VMinus\ can be compiled to efficient decision trees. 


\section{Languages are Interesting}
    \begin{outline}[enumerate]
    \1 PM dominates observer functions 

    Pattern matching dominates observer functions. Consider the following
    implementation of \tt{List.len} in Standard ML, using observers:
    
    
    % Figure 1 illustrates an example in which pattern matching is an elegant solution
    % to a problem (compare to equivalent Standard ML code that manually deconstructs
    % a list:)
    
    
    % Most functional programmers likely prefer the first example 
    
    
    % Pattern matching is a well-established and researched construct in functional
    % programming langauges [cite, appeal to authority]. It shines in its ability to
    % break down constructed data implicity, which is often much preferred to using
    % explict deconstructors like Scheme's \tt{car} and \tt{cdr}. But when deciding to
    % perform a terminal computation based on prior checks that don't involve a single
    % value matching on a pattern, pattern matching struggles to retain its brevity.
    % Consider the following algorithm on an abstract type~\tt{FiniteMap}, involving a
    % lookup function. This examples is a modificaiton of one borrowed from SPJ
    % proposal [cite]. 
    
    
    
    \2 Example. 
    \smllst
    
    \begin{lstlisting}
    fun len yes =
        if null ys 
        then 0 
        else let xs = tl ys 
             in len xs 
             end 
    \end{lstlisting}
    \it{An implementation of List.len in Standard ML that does not use pattern 
        matching.}
    
    Manual checking and deconstruction, with built-in functions like \tt{null} and
    \tt{tl}, can be both error-prone (\bf{say more on this?}) and verbose. Most
    functional programmers likely prefer the follwing implementation of \tt{len}:
    
    \begin{lstlisting}
        fun len ys =
            case ys 
              of [] => 0
               | _::xs => 1 + len xs
        \end{lstlisting}
    
    \it{An equivalent implementation that uses pattern matching.}
    
    In most functional languages, the \tt{case} can be removed entirely in a form of
    syntactic sugar known as a \it{clausal definition}:
    \begin{lstlisting}
        fun len []      = 0
          | len (_::xs) = 1 + len xs
        \end{lstlisting}
    
    \it{A final, concise implementation that uses pattern matching.}

    \1 Common extensions simplify some special cases that would otherwise be awkward
        \2 Example: pat guards
        \2 Example: or-patterns
    \1 Verse Dominates observer functions 
\verselst
    \begin{lstlisting}
        $\exists$ len. len = \ys. 
          one {ys = $\langle \rangle$; 0
               |  $\exists$ x xs. ys = $\langle$x, xs$\rangle$; 1 + len xs}
    \end{lstlisting}
        \2 Example. 
    \1 Verse is comparably pleasing to PM 
    \1 Full Verse allows/encourages computations that are bad, cost-wise 
        \2 Hard to predict?
        \2 Exponential?
        \2 Multiple results?
        \2 Bad how? 
    \end{outline}
\section{Simplification for Study and Synthesis}
\begin{outline}[enumerate]
    \1 To study "all" the interesting/standard extensions to PM, we introduce
    \PPlus.
    \2 Definition of \PPlus
    \2 {\PPlus} packages common/standard extensions to PM
        \3 Examples of \PPlus. 
    \2 \PPlus admits of strange looking patterns, but these should not be alarming. 
    \2 -> Because they reduce to standard things by (direct) application of algebraic laws. 
        \3 Laws: float out vcon 
    \1 We want to study decision-making inspired by Verse. 
        \2 We remove obvious backtracking/multiple results (ask what the notecard says)
        \2 We add a classic decision-making construct: Guarded commands. 
        \2 The result: \VMinus. 
    \1 Definition of \VMinus
    \1 Continued Discussion:
    \2 {\VMinus} admits of many of the same pleasing computations as full Verse. 
    \2 Examples of programming in \VMinus 
    \2 {\VMinus} is at least as good as {\PPlus}? Or just as comparable? 
        Or incomparable? (Why not both?)
    \end{outline}

IN EITHER OF THESE SECTIONS: Semantics of \PPlus\ , \VMinus

\section{(PAYLOAD) {\VMinus} can be compiled to \D 
        [Evidence that \dots efficiency]}
\begin{outline}[enumerate]
    \1 Definition of \D 
    \1 Examples of Decision Trees 
    \1 D is a simple generalization of Maragnet, et al. 
    \1 Semantics of \D 
    \1 \D\ has a great cost model
    \1 Algorithm: Translation from \VMinus\ to \D\ : match compiler 
    \1 Theorem: Translation from \VMinus\ to \D\ preserves semantics 
    \2 Likely inductivey hypothesis. 1-4 sentences on proof max. 
\end{outline}

\section{\PPlus -> \VMinus is interesting (optional)}
\begin{outline}[enumerate]
    \1 \VMinus subsumes pattern matching 
    \1 Algorithm: Translation from \PPlus\ to \VMinus
    \1 Theorem: Translation from \PPlus\ to \VMinus preserves semantics 
    \1 Claim: Translation \PPlus\ -> \D\ is consistent with Maragnet and others
\end{outline}

\section{Related Work}
\section{Future Work}
\begin{outline}[enumerate]
    \1 Why we wish for $\alpha$s
    \1 What's up with \PPlus\ ? It's worth study in its own right- in future work.
    \2 \PPlus is interesting because it combines or-patterns with pattern 
        guards. No major functional language does this. 
    \1 Programs written in Verse using ideas from \VMinus\ might have a 
    friendlier cost model (depending on compiler)
    \1 \VMinus might give Verse programmers good ideas 
        (That is, how to solve problems in Verse)
    \2 Examples of programming in Verse in the style of \VMinus 
\end{outline}

\section{Conclusion, Discussion}

\renewcommand\thesection{\Alph{section}}
\setcounter{section}{0}
\section{Proofs}
\begin{outline}
    \1 Proof: Translation from \VMinus to \D\ preserves semantics 
    \1 Proof: Translation from \PPlus\ to \VMinus preserves semantics 
    \1 Proof: Translation from \VMinus to Verse preserves semantics     
\end{outline}

\section{Trust me on \VMinus (It's reduced to Verse)}
\begin{outline}
    \1 \VMinus\ has something to do with Verse, semantically 
    \1 Our semantics of Verse is consistent with ICFP's semantics of Verse 
    \1 Definition: Our semantics of Verse
    \1 Theorem: Translation from \VMinus to Verse preserves semantics     
\end{outline}


%     \item Programmers use pattern matching. But pattern matching is not good for
%     everything. 
%     \item Consider Example. 
%     \item Explain example. 
%     \item Attempted mitigations in the past: extensions, SPJ example. 
%     \item We go futher. 
%     \item \bf{P3}: 
%     \item Contribution: 2 languages that attempt to mitigate this problem. 
%     \item Our goal: One simple, expressive, efficient decision-making construct.
%     \item One is {\PPlus}, the other is {\VMinus}. 
%     \item Example of {\VMinus} in action on prior example. 
% \end{itemize}

% We begin by informally introducing two small functional programming languages to
% address the issue of inflexibility of patterns.

% {\PPlus} models standard pattern matching with common extensions. {\VMinus}
% models Verse without features that lead to unpredictable costs (backtracking and
% multiple results) [cite-Verse]. 

% To aid proofs of efficiency, we also introduce a third langauge, {\D}. {\D} is a
% language of decision trees to which both {\PPlus} and {\VMinus} can be
% translated. Targets of translation are efficient in the standard technical
% sense: no value is ever scrutinized more than once (Maranget 2008).

% To facilitate comparisons and proofs, the languages are each a subset of a
% single unifying language(\cref{fig:U}). Asterisks (${}^{*}$) indicate components
% of this unifying language which belong to the sub-languages. 

% % The keen reader will notice that our syntax includes a forms for decision trees,
% % which we have not yet discussed. Decision trees will be crucial in proving
% % properties of efficiency of our languages, and will we be discussed in full in a
% % later section. 

% \begin{figure}[h!p]
%     \small
%     \begin{flushleft}
%         \begin{bnf}
%         $P$ : \textsf{Programs} ::=
%         $\bracketed{d}$ : definition
%         ;;
%         $d$ : \textsf{Definitions} ::=
%         | $\mathit{val}\; x\; \mathit{=}\; \expr$ : bind name to expression
%         ;;
%         $\expr$ : Expressions ::= 
%         | $x, y, z$             : names
%         | $K\bracketed{\expr}$  : value constructor application 
%         | $\lambda x.\; \expr$  : lambda declaration  
%         | $\expr[1]\; \expr[2]$ : function application 
    
%         | $\mathit{case}\; \expr\; \bracketed{p\; \rightarrow\; \expr}$ : $\rm{case expression}^{*}$
%         | $\mathit{if}\; \mathit{[}\; g\; 
%             \bracketed{[] g}\; \mathit{]}\; \mathit{fi}$                : $\rm{if-fi}^{**}$
%         | $\dt$                                                         : $\rm{decision tree}^{***}$
%         ;;
%         $\v$ : Values ::= 
%           $K\bracketed{\v}$     : value constructor application 
%         | $\lambda x.\; \expr$  : lambda value 
%         ;;
%         $p$ : $\textsf{Patterns}^{*}$ ::= 
%         $x$ : name 
%         | $K\; \bracketed{p}$           : value constructor application 
%         | $\mathit{when}\; \expr$       : side condition
%         | $p, p'$                       : pattern guard 
%         | $p\; \leftarrow\; \expr$      : pattern from explicit expression  
%         | $p_{1}\pbar p_{2}$            : or-pattern
%         ;;
%         $g$ : $\textsf{Guarded Expressions}^{**}$ ::=  
%         $\boldsymbol{\rightarrow}\expr$ : terminating experession
%         | $\expr;\; g$                  : intermediate expression 
%         | $\vexists{x} g$      : existential 
%         | $x = \expr;\; g$              : equation 
%         ;;
%         \dt : $\textsf{Decision Trees}^{***}$ ::= 
%         | $\mathit{case}\; x\; \mathit{of}\; 
%             \bracketed{\vert\; K\bracketed{x}\; \mathit{=>}\; \dt} 
%             [\vert\; x\; \mathit{=>} \dt]$                              : $\rm{test node }^{***}$
%         | $\expr$                                                       : $\rm{match node }^{***}$
%         | $\mathit{if}\; x\; \mathit{then}\; \dt\; \mathit{else}\; \dt$ : $\rm{condition with two children }^{***}$
%         | $\mathit{let}\; x\; \mathit{=}\; \expr\; \mathit{in}\; \dt$   : $\rm{let-bind a name}^{***}$
%         % ;;
%         % $K$ : \textsf{Value Constructors} ::=
%         % | $\mathit{true}\; \vert\; \mathit{false}$ : booleans
%         % | $\mathit{\#}x$                           : name beginning with $\mathit{\#}$
%         % | $\mathit{A-Z}x$                          : name beggining with capital letter
%         % | $[\mathit{-}\vert\mathit{+}]
%         %     (\mathit{0}-\mathit{9})+$              : signed integer literal 
%         \end{bnf}
%         \medskip
        
        
%         \it{Concrete Syntax}: "$\lambda$" and "\exists" each scope as far to
%         the right as possible.
        
%         $\hskip 8em$ For example, $(\lambda y.\> \vexists{x}\> x = 1;\> x + y)$ means 
%         $(\lambda y.\> (\vexists{x}\> ((x = 1);\> (x + y))))$.
        
%         Parentheses may be used freely to aid readability and override default precedence.

%         A \it{name} is any token that is not an integer literal, does not
%         contain whitespace, a bracket, or parenthesis, and is not a value
%         constructor name or a reserved word.
        
%         \medskip

%         % \bf{Desugaring of Extended Expressions}

%         ${}^{*}$ Indicates forms within {\PPlus}

%         ${}^{**}$ Indicates forms within {\VMinus}
        
%         ${}^{***}$ Indicates forms within {\D}

%     \end{flushleft}
    
%     \medskip

    

%     \caption{\U, a decision-making language}
%     \Description{A BNF grammar for \U, the universal decision-making language.
%                  It includes patterns, guarded expressions, and decision trees.}
%     \label{fig:U}
% \end{figure}
% % \begin{table}[h]
% %     \centering
% %     \small
% %     \begin{tabular}{l l l}
% %         \textbf{Syntactic Forms} & \textbf{Cases} & \textbf{Belong to} \\

% %         $P$ : Programs & $\bracketed{d}$ & many definitions & \\
% %         $d$ : Definitions & $\mathit{val}\; x\; \mathit{=}\; \expr$ & bind name to expression & \\
% %         $\expr$ : Expressions & $x, y, z$ & names & \\
% %         & $K\bracketed{\expr}$ & value constructor applied to expressions & \\
% %         & $\lambda x.\; \expr$ & lambda declaration & \\
% %         & $\expr[1]\; \expr[2]$ & function application & \\
% %         & $\mathit{case}\; \expr\; \bracketed{p\; \rightarrow\; \expr}$ & $\rm{case expression}$ & \\
% %         & $\mathit{if}\; \mathit{[}\; g\; \bracketed{[] g}\; \mathit{]}\; \mathit{fi}$ & $\rm{if-fi}$ & \\
% %         & $\dt$ & $\rm{decision tree}$ & \\
% %         $\v$ : Values & $K\bracketed{\v}$ & value constructor applied to values & \\
% %         & $\lambda x.\; \expr$ & lambda value & \\
% %         $p$ : Patterns & $x$ & name & \\
% %         & $K\; \bracketed{p}$ & value constructor applied to patterns & \\
% %         & $\mathit{when}\; \expr$ & side condition & \\
% %         & $p, p'$ & pattern guard & \\
% %         & $p\; \leftarrow\; \expr$ & pattern from explicit expression & \\
% %         & $p_{1}\pbar p_{2}$ & or-pattern & \\
% %         $g$ : Guarded Expressions & $\boldsymbol{\rightarrow}\expr$ & terminating expression & \\
% %         & $\expr;\; g$ & intermediate expression & \\
% %         & $\vexists{x} g$ & existential & \\
% %         & $x = \expr;\; g$ & equation & \\
% %         \dt : Decision Trees & $\mathit{case}\; x\; \mathit{of}\; \bracketed{\vert\; K\bracketed{x}\; \mathit{=>}\; \dt} [\vert\; x\; \mathit{=>} \dt]$ & $\rm{test node}$ & \\
% %         & $\expr$ & $\rm{match node}$ & \\
% %         & $\mathit{if}\; x\; \mathit{then}\; \dt\; \mathit{else}\; \dt$ & $\rm{condition with two children}$ & \\
% %         & $\mathit{let}\; x\; \mathit{=}\; \expr\; \mathit{in}\; \dt$ & $\rm{let-bind a name}$ & \\
% %     \end{tabular}
% %     \caption{Example LaTeX Table}
% %     \label{tab:example}
% % \end{table}

% \begin{table}[h]
%     \centering
%     \small
%     \begin{tabular}{l l l}
%         \textbf{Syntactic Forms} & \textbf{Cases} & \textbf{Unique to} \\
%         \hline
%         $P$ : Programs & $\bracketed{d}$ & \\
%         $d$ : Definitions & $\mathit{val}\; x\; \mathit{=}\; \expr$ & \\
%         $\expr$ : Expressions & $x, y, z$ & \\
%         & $K\bracketed{\expr}$ &  \\
%         & $\lambda x.\; \expr$ & \\
%         & $\expr[1]\; \expr[2]$ & \\
%         & $\mathit{case}\; \expr\; \bracketed{p\; \rightarrow\; \expr}$ & \PPlus \\
%         & $\mathit{if}\; \mathit{[}\; g\; \bracketed{[] g}\; \mathit{]}\; \mathit{fi}$ & \VMinus \\
%         & $\dt$ & \D \\
%         $\v$ : Values & $K\bracketed{\v}$ & \\
%         & $\lambda x.\; \expr$ & \\
%         $p$ : Patterns & $x$ & \PPlus \\
%         & $K\; \bracketed{p}$ & \PPlus \\
%         & $\mathit{when}\; \expr$ & \PPlus \\
%         & $p, p'$ & \PPlus \\
%         & $p\; \boldsymbol{\leftarrow}\; \expr$ & \PPlus \\
%         & $p_{1}\pbar p_{2}$ & \PPlus \\
%         $g$ : Guarded Expressions & $\boldsymbol{\rightarrow}\expr$ & \VMinus \\
%         & $\expr;\; g$ & \VMinus \\
%         & $\vexists{x} g$ & \VMinus \\
%         & $x = \expr;\; g$ & \VMinus \\
%         \dt : Decision Trees & $\mathit{case}\; x\; \mathit{of}\; \bracketed{\vert\; K\bracketed{x}\; \mathit{=>}\; \dt} [\vert\; x\; \mathit{=>} \dt]$ & \D \\
%         & $\expr$ & \D \\
%         & $\mathit{if}\; x\; \mathit{then}\; \dt\; \mathit{else}\; \dt$ & \D \\
%         & $\mathit{let}\; x\; \mathit{=}\; \expr\; \mathit{in}\; \dt$ & \D \\
%     \end{tabular}
%     \caption{Example LaTeX Table}
%     \label{tab:example}
% \end{table}



% Most syntactic categories in \U are present in all of its subsets: Values, Value
% Constructors, Definition Forms, and most forms of expression are shared. Indeed,
% the sub-langauge, like in Verse, is just the the lambda calculus with a few
% extended syntactic categories (this time favoring value constructors over
% tuples). Like in Verse, every Lambda Calculus program is a valid \U program. 

% On top of this core, three languages are defined as subsets of \U. {\PPlus} is
% the subset of~\U including Patterns and the $\mathit{case}\; \expr\;
% \bracketed{p \rightarrow \expr}$ form of case expression. {\VMinus} is the
% subset of \U including Guarded Expressions and \it{if-fi}. Finally, {\D} is the
% subset of \U that includes decision tree syntax, including the reduced case
% expression, $\mathit{case}\; x\; \mathit{of}\; \bracketed{\vert\;
% K\bracketed{x}\; \mathit{=>}\; \dt} [\vert\; x\; \mathit{=>} \dt]$.
% Importantly, the three are mutually exclusive with respect to exactly these
% extensions; all subsets share the sub-langauge and no subset has more than one
% of the expanded \it{case}, \it{if-fi}, or the decision tree syntactic
% categories. 

% % {\PPlus} provides or-patterns, side-conditions, and pattern guards, whose
% % combination does not appear in Haskell, OCaml, Standard ML, or any other major
% % functional language. 
% % \bigskip
% % Next steps: Finish introduction of language table, say where semantics are. 


% \it{The following sections and paragraphs are written:}

% \begin{itemize}
%     \item \bf{Subsection: General Evaluation}
%     \item Big-step opsem with environments. 
%     \item \bf{Subsection: {\PPlus}}
%     \item Patterns, case, environment + disjoint union 
% \end{itemize}

% \it{The following sections and paragraphs are in progress:}
% \begin{itemize}
%     \item when, or-patterns, pattern guards 
%     \item \bf{Subsection: {\VMinus}}
%     \item Guarded expressions, new type of environment
%     \item Evaluation stragety 
% \end{itemize}

% \it{Finally, much has been commented out below because I am still determining
%     the order in which their appearance is most clear. After your review of this
%     initial format, I will begin to include them.}


% % Prior work has introduced extesions to pattern matching [cite, including
% % SPJ proposal for pattern guards.]

% % {\PPlus} provides or-patterns, side-conditions, and pattern guards, whose
% % combination does not appear in Haskell, OCaml, Standard ML, or any other major
% % functional language. 

% % \bigskip
% % interesting thoughts: 

% % Having a strategem for verse to decision tree is analagous to tail recursion. 
% % You have to write your code in a way that allows the optimization. 


% % Pattern matching is a well-established paradigm within functional languages, and
% % """has been the subject of significant discourse.""" \it{Appeal to authority here.}

% % Without pattern matching, it can be tiresome to deconstruct data using manual 
% % accessor functions. Consider the following Standard ML code: 
% % \smllst

% % \begin{lstlisting}
% % val rec len = fn ys => 
% %     if null ys 
% %     then 0 
% %     else let xs = tl ys 
% %          in len xs 
% %          end 
% % \end{lstlisting}
% % \it{An implementation of List.len in Standard ML that does not use pattern 
% %     matching.}

% % Manual checking and deconstruction, with built-in functions like \tt{null} and
% % \tt{tl}, can be both error-prone (\bf{say more on this?}) and verbose. Most
% % functional programmers likely prefer the follwing implementation of \tt{len}:

% % \begin{lstlisting}
% %     val rec len = fn ys => 
% %         case ys 
% %           of [] => 0
% %            | _::xs => 1 + len xs
% %     \end{lstlisting}

% % \it{An equivalent implementation that uses pattern matching.}

% % Indeed, pattern matching is quite appealing when deconstructing data, especially 
% % data that may be represented with an algebraic data type, is the primary programming problem: 



% % \begin{lstlisting}
% % val len = \ys. case ys 
% %                  of [] -> 0
% %                   | _::xs -> 1 + len xs
% % \end{lstlisting}


% % Figure 1 illustrates an example in which pattern matching is an elegant solution
% % to a problem (compare to equivalent Standard ML code that manually deconstructs
% % a list:)


% % Most functional programmers likely prefer the first example 

% % However, language designers continue to extend pattern matching 

% % \section{Pattern Matching as it is Now}
% % \it{Pattern matching} is defined as "checking and locating of specific sequences
% % of data of some pattern among raw data or a sequence of tokens." We will return
% % to the notion of "checking" often in this paper: pattern matching answers the
% % question "when I'm checking to see if a piece of data (called a \it{scrutinee})
% % is of the same form as a certain pattern, does that match succeed or fail?"

% % \it{Example}.

% % In addition to this checking and locating, pattern matching serves as
% % \it{assignment}: it can bind fresh variables based on the form of data and use
% % those bindings in subsequent expressions. 

% % \it{Example}.

% % Here, "checking" means "do the data match what I expect them to." Because
% % pattern matching is inherently built to match a scrutinee (pure data) with a
% % pattern, pattern matching is quite expressive in these cases. 

% % Because of this, most modern functional languages, especially \it{data
% % dependency languages} like Haskell or ML, \it{(is this right?)} employ pattern
% % matching as their main way to deconstruct data (citation?). 

% % We here explore pattern matching through the lens of {\PPlus}, an invented
% % language that has pattern matching along with several of its popular extensions.
% % (The examples above are written in {\PPlus}.) 
% % \subsection{Strengths}

% % - "Checking" and assignment- nice! No car, cdr 

% % - Nested patterns - powerful 

% % - Literal patterns let you mix names and values 

% % \subsection{Weaknesses and Proposed Mitigations}

% % We return to the concept of unifying "checking" with assignment, i.e. "match if
% % the data take this form, and give them names." Pattern matching succeeds here
% % when checking means "is the form of data the way I expect"; in fact, as we know,
% % assignment in general \it{is} pattern matching (figure/example?). But when
% % checking means "does this computation succeed" or "does this binding conflict
% % with a prior binding," pattern matching is at a loss, where Verse succeeds. 

% % Example: 


% % Pattern matching's extensions get closer to unifying "checking" and binding.
% % Here, Verse enjoys a different suite of advantages. 

% % First, its "or" operation (`one` with `choice`) allows for more than patterns to
% % appear as a top-level "choose this or that" construct in a match sequence; you
% % can also include arbitrary expressions. You can't do this in pattern matching's
% % version, which is an or-pattern. Simply put, saying "does this pattern match or
% % is this expression true" is easy in Verse and clunky if you use patterns. I'll
% % show examples at our meeting. 

% % Second, Verse can express operations out of order, letting important checks
% % appear higher up even if they are executed later. This helps program legibility.
% % Again, I have examples from the chapter I wrote today. 

% % Third, in a pattern-match clause, the initial data must still match an initial
% % pattern in order to enter a guard; in Verse there is no restriction. This is
% % minor, because you could simply match the data to a variable, and then enter a
% % guard-- but again, all of these advantages are in elegance and brevity, and
% % elegant that solution is not. 

% % Finally, mingling pattern guards with other extensions to pattern matching
% % (especially or-patterns) is a murky subject. Haskell has pattern guards and side
% % conditions, but no or-patterns. OCaml has side conditions and or-patterns, but
% % no guards. Mixing all three is (according to some readings) simply difficult for
% % implementers- including, interestingly, those of parsers. In Verse, having `one`
% % and `choice` closely tied in with the simple `e1 = e2` equation form, which by
% % itself subsumes pattern matching, side-conditions, and pattern guards, means
% % that integrating options is easy. A key theme that arises of this: Verse has
% % fewer constructs, and they are more expressive.  

% % \section{A Proposal, Inspired by the Verse Calculus}

% % \subsection{Verse Flexibility}
% % \subsection{Something else}
% % \subsection{A third thing}

% % \section{Verse's Equations Subsume Pattern Matching}

% % \subsection{Claim}
% % \subsection{Proof}
% % \subsection{Translations}

% % \section{(Maybe) Writing Efficient Verse Code}

% % \subsection{Claim}
% % \subsection{Proof}
% % \subsection{Translations}


% % \section{Citations and Bibliographies}

\end{document}
\endinput
%%

\documentclass[manuscript,screen,review, 12pt, nonacm]{acmart}
\let\Bbbk\relax % Fix for amssymb clash 
\usepackage{vmlmacros}
\AtBeginDocument{%
  \providecommand\BibTeX{{%
    \normalfont B\kern-0.5em{\scshape i\kern-0.25em b}\kern-0.8em\TeX}}}
\usepackage{outlines}
\setlength{\headheight}{14.0pt}
\setlength{\footskip}{13.3pt}
\title{An Alternative to Pattern Matching, Inspired by Verse}

\author{Roger Burtonpatel}
\email{roger.burtonpatel@tufts.edu}
\affiliation{%
\institution{Tufts University}
\streetaddress{419 Boston Ave}
  \city{Medford}
  \state{Massachusetts}
  \country{USA}
  \postcode{02155}
  }
\begin{document}
  

\section{{\VMinus} can be compiled to a decision tree}
% [Evidence that \dots efficiency]
\begin{outline}[enumerate]
    \1 In our semantics of \PPlus\ and \VMinus, we adhere to the strict
    invariant that no value is examined more than once at runtime. But we wish
    to go further in our assertion, so we compile \VMinus to a decision tree. 

    \1 \bf{Examples of Decision Trees - From Maragnet. }
    \1 For our compilation, we introduce \D, a language of decision trees. 
    \1 \bf{Definition of \D}
    \2 \bf{Indeed, we are consistent with Maragnet }
    \1 \bf{D is a simple generalization of Maragnet, et al. }
    \1 \bf{Semantics of \D: ML Runtime! Yippee!}
    \1 \bf{\D\ has a great cost model: }

    \bf{A decision tree can be exponential in size but never examines a word of
    the scrutinee more than once. }

    \1 \bf{Algorithm: Translation from \VMinus\ to \D\ : match compiler }
    \1 \bf{Theorem: Translation from \VMinus\ to \D\ preserves semantics }
    \2 \bf{Likely inductive hypothesis. 1-4 sentences on proof max. }
\end{outline}

\section{\PPlus -> \VMinus is interesting (optional)}
\begin{outline}[enumerate]
    \1 \bf{\VMinus subsumes pattern matching}
    \1 \bf{Algorithm: Translation from \PPlus\ to \VMinus}
    \1 \bf{Theorem: Translation from \PPlus\ to \VMinus preserves semantics}
    \1 \bf{Claim: Translation \PPlus\ -> \D\ is consistent with Maragnet and others}
\end{outline}

\section{Related Work}
\section{Future Work}
\begin{outline}[enumerate]
    \1 \bf{Why we wish for $\alpha$s}
    \1 \bf{What's up with \PPlus\ ? It's worth study in its own right- in future work.}
    \2 \bf{\PPlus is interesting because it combines or-patterns with pattern }
        guards. No major functional language does this. 
    \1 \bf{Programs written in Verse using ideas from \VMinus\ might have a }
    friendlier cost model (depending on compiler)
    \1 \bf{\VMinus might give Verse programmers good ideas }
        (That is, how to solve problems in Verse)
    \2 \bf{Examples of programming in Verse in the style of \VMinus }
\end{outline}

\section{Conclusion, Discussion}

\renewcommand\thesection{\Alph{section}}
\setcounter{section}{0}
\section{Proofs}
\begin{outline}
    \1 \bf{Proof: Translation from \VMinus to \D\ preserves semantics }
    \1 \bf{Proof: Translation from \PPlus\ to \VMinus preserves semantics }
    \1 \bf{Proof: Translation from \VMinus to Verse preserves semantics     }
\end{outline}

\section{Trust me on \VMinus (It's reduced to Verse)}
\begin{outline}
    \1 \bf{\VMinus\ has something to do with Verse, semantically }
    \1 \bf{Our semantics of Verse is consistent with ICFP's semantics of Verse }
    \1 \bf{Definition: Our semantics of Verse}
    \1 \bf{Theorem: Translation from \VMinus to Verse preserves semantics     }
\end{outline}

\begin{table}[ht]
    \centering
    \small
    \begin{tabular}{l l l}
        \textbf{Syntactic Forms} & \textbf{Cases} & \textbf{Unique to} \\
        \hline
        $P$ : Programs & $\bracketed{d}$ & \\
        $d$ : Definitions & $\mathit{val}\; x\; \mathit{=}\; \expr$ & \\
        $\expr$ : Expressions & $x, y, z$ & \\
        & $K\bracketed{\expr}$ &  \\
        & $\lambda x.\; \expr$ & \\
        & $\expr[1]\; \expr[2]$ & \\
        & $\mathit{case}\; \expr\; \bracketed{p\; \rightarrow\; \expr}$ & \PPlus \\
        & $\mathit{if}\; \mathit{[}\; g\; \bracketed{[] g}\; \mathit{]}\; \mathit{fi}$ & \VMinus \\
        & $\dt$ & \D \\
        $\v$ : Values & $K\bracketed{\v}$ & \\
        & $\lambda x.\; \expr$ & \\
        $p$ : Patterns & $x$ & \PPlus \\
        & $K\; \bracketed{p}$ & \PPlus \\
        & $\mathit{when}\; \expr$ & \PPlus \\
        & $p, p'$ & \PPlus \\
        & $p\; \boldsymbol{\leftarrow}\; \expr$ & \PPlus \\
        & $p_{1}\pbar p_{2}$ & \PPlus \\
        $g$ : Guarded Expressions & $\boldsymbol{\rightarrow}\expr$ & \VMinus \\
        & $\expr;\; g$ & \VMinus \\
        & $\vexists{x} g$ & \VMinus \\
        & $x = \expr;\; g$ & \VMinus \\
        \dt : Decision Trees & $\mathit{case}\; x\; \mathit{of}\; \bracketed{\vert\; K\bracketed{x}\; \mathit{=>}\; \dt} [\vert\; x\; \mathit{=>} \dt]$ & \D \\
        & $\expr$ & \D \\
        & $\mathit{if}\; x\; \mathit{then}\; \dt\; \mathit{else}\; \dt$ & \D \\
        & $\mathit{let}\; x\; \mathit{=}\; \expr\; \mathit{in}\; \dt$ & \D \\
    \end{tabular}
    \caption{Example LaTeX Table}
    \label{fig:unilang}
\end{table}

\end{document}
%%
\documentclass[manuscript,screen,review, 12pt]{acmart}
\let\Bbbk\relax % Fix for amssymb clash 
\usepackage{vmlmacros}
\AtBeginDocument{%
  \providecommand\BibTeX{{%
    \normalfont B\kern-0.5em{\scshape i\kern-0.25em b}\kern-0.8em\TeX}}}
\usepackage{outlines}
\usepackage{caption}
\usepackage{subcaption}
\setlength{\headheight}{14.0pt}
\setlength{\footskip}{13.3pt}
\begin{document}
\title{An Alternative to Pattern Matching, Inspired by Verse}

\author{Roger Burtonpatel}
\email{roger.burtonpatel@tufts.edu}
\affiliation{%
  \institution{Tufts University}
  \streetaddress{419 Boston Ave}
  \city{Medford}
  \state{Massachusetts}
  \country{USA}
  \postcode{02155}
}

% \author{Norman Ramsey}
% \email{nr@cs.tufts.edu}
% \affiliation{%
% \institution{Tufts University}
% \streetaddress{177 College Ave}
% \city{Medford}
% \state{Massachusetts}
% \country{USA}
% \postcode{02155}
% }

% \author{Milod Kazerounian}
% \email{milod.mazerouniantufts.edu}
% \affiliation{%
% \institution{Tufts University}
% \streetaddress{177 College Ave}
% \city{Medford}
% \state{Massachusetts}
% \country{USA}
% \postcode{02155}
% }

\renewcommand{\shortauthors}{Burtonpatel et al.}

\begin{abstract}
    \bf{Abstract}

    \it{WILL BE REVISED.}
  Pattern matching is nice and has an appealing cost model, but seems to lack in
  expressiveness: people keep extending it to make it more expressive. Verse
  [cite-Verse] doesn't use pattern matching and instead favors equations, which
  look very simple and surprisingly expressive, but the language's cost model is
  a challenge. Why not look for a compromise? 
  
%   We introduce two small functional programming languages that
%   attempt to address the limitations of pattern matching in popular functional
%   languages.

  I explore the realm of pattern matching in common functional programming
  languages and compare it with Verse's equation-based approach. Recognizing the
  balance between expressiveness and efficiency, I introduce two toy languages
  aimed at addressing the limitations of prevalent pattern matching techniques.
  I show how a subset of Verse can be compiled to decision trees. 

  \end{abstract}

\maketitle

\section{Introduction}
\it{We will add this at the end.}

Pattern matching is popular and well-researched. However, it often needs to be
augmented with extensions. Is this the most efficient thing to be doing? Can we
take inspiration from Verse and use equations instead? Do Verse's equations
subsume pattern matching with popular extensions? Can we get them at no
additional runtime cost? 
% We consider a toy language with many of the popular
% extensions to pattern matching. 
% \VMinus\ can be compiled to efficient decision
% trees. 


\section{Pattern Matching and Equations}
\label{pmandequations}
\subsection{Pattern matching as an upgrade to observers}
\label{pmoverobservers}

    
    

    \it{In this section, I introduce pattern matching as a way for programmers
    to deconstruct data. If you are familiar with pattern matching, you can skip
    to section~\ref{extensions}.}
    
    % extensively discussed in abundant literature
    Programmers write computer programs to process data. Processing involves
    inspecting the data, deconstructing them into parts, and making decisions
    based on their forms. One tool a functional programmer might use to
    deconstruct data is \it{observers}[cite- Liskov and Guttag]: functions that
    explicitly inquire a piece of data's structure and manually extract its
    components.\footnote {Examples of observers in functional programming
    languages include Scheme's \tt{null?}, \tt{car}, and \tt{cdr}, and ML's
    \tt{null}, \tt{hd}, and \tt{tl}.} But in the pursuit of succinct, resilient
    strategies to scrutinize and deconstruct data, many functional programmers
    favor \it{pattern matching}, which enables them to match data directly with
    against a number of possible forms. Pattern matching dominates observers in
    multiple ways. Let us address these by comparing two implementations of
    Standard ML's \tt{List.length}, which has these algebraic laws: 

    \begin{minipage}[t]{\textwidth}
        \begin{verbatim}
            length []      == 0
            length (x::xs) == 1 + length xs 
        \end{verbatim}
    \end{minipage}
    
    Let's consider two programmers, Pat and Obe. Pat favors pattern matching;
    Obe likes observer functions. Figures \ref{fig:observerlen}
    and~\ref{fig:pmlen} show side-by side examples of how they implement
    \tt{length}. 

    \begin{figure}[h]
        \centering
      \begin{minipage}[t]{0.4\textwidth}
        \begin{verbatim}
    fun length ys =
        if null ys 
        then 0 
        else let xs = tl ys 
             in length xs 
             end 
            \end{verbatim}
            \Description{An implementation of length using null, hd, and tl}
        \subcaption{Obe's \tt{length} with observers}
            \label{fig:observerlen} 
      \end{minipage}
      \quad
      \begin{minipage}[t]{0.4\textwidth}
        \begin{verbatim}
fun length ys =
    case ys 
      of []   => 0
          | x::xs => 1 + length xs
                \end{verbatim}
       \Description{An implementation of length using implicit deconstruction via
       patterns.}
       \vspace{2.2em}
       \subcaption{Pat's \tt{length} with pattern matching}
       \label{fig:pmlen}
      \end{minipage}
      \caption{Implementing \tt{List.length} using observers is tedious and doesn't
      look very functional. Using pattern matching makes an equivalent
      implementation more appealing.}
      \label{fig:len}
    \end{figure}
    
    Obe has written code that is obviously more verbose. In addition, Obe's use
    of manual case analysis and deconstruction with observers
    in~\ref{fig:observerlen} can be error-prone: Direct use of \tt{hd} or
    \tt{tl} can cause a program to halt with an uncaught \tt{Empty} exception, a
    possibility that the compiler cannot rule out. By contrast, pattern matching
    can fail only with a \tt{Match} exception, a possibility the compiler
    \it{can} rule out. 
    
    In addition to writing code that is more concise and safer to use, Pat has
    implemented \tt{length} in a way that is obviously similar to its algebraic
    laws. You need to squint a bit around the extra syntax- \tt{case}, \tt{of}-
    but compared to Obe's use of \tt{if-then-else} and \tt{let} expressions with
    calls to \tt{null}, \tt{hd}, and \tt{tl}, Pat has paid little syntactic
    cost. 

    Having had the chance to compare Figures \ref{fig:observerlen}
    and~\ref{fig:pmlen}, if you moderately prefer the latter, that's good: most
    functional programmers- indeed, most \it{programmers}- likely do as well. 

    Since you might be a bit happier about \ref{fig:pmlen}, I will use it to
    introduce a few terms common to pattern matching. \ref{fig:pmlen}~is a
    classic example of where pattern matching most commonly occurs: within a
    \tt{case} expression\footnote{Sometimes also called a \tt{match expression}.
    OCaml, which we'll see in future sections, calls \tt{case} \tt{match}.}.
    \tt{case} expressions test a \it{scrutinee}\footnote{Some literature (cite
    SPJ) calls this a \it{head expression}. I follow Maragnet's (cite) example
    in calling the expression the \it{scrutinee}. Either term does the job.}
    (\tt{ys} in~\ref{fig:pmlen}) against a list of patterns (\tt{[]},
    \tt{x::xs}). When the result of evaluating the scrutinee matches with a
    pattern, the program evaluates the right-hand side of the respective branch
    (\tt{0} or \tt{1 + length xs}) within the \tt{case} expression. 

    Now that we know that \tt{case} works, can we help Pat do better? It would
    indeed be nice if Pat could get rid of her extraneous \tt{case} syntax
    entirely. And thankfully, with another trick of pattern matching, she can:
    in most functional programming languages, Pat can simplify code with a
    top-level \tt{case} expression even further by using a form of syntactic
    sugar known as a \it{clausal definition}. She can use clausal definitions to
    merge pattern matching and function declaration, which eliminates the
    top-level \tt{case} entirely. Figure~\ref{fig:pmclausallen} shows a final
    implementation of \tt{length} using pattern matching with a clausal
    definition. 
    
    

    \begin{figure}[ht]
    \smllst
    \begin{verbatim}
            fun length []      = 0
              | length (x::xs) = 1 + length xs
        \end{verbatim}
    \caption{Pattern matching with a clausal definition gives a simple and 
             elegant implementation of \tt{length}.}
    \Description{An implementation of length using pattern matching within a 
    clausal definition.}
    \label{fig:pmclausallen}
    \end{figure}
    
    Now you don't even need to squint to see that Pat has written a \tt{length}
    that very obviously implements the laws! What's more, Pat's code, compared
    to~\ref{fig:observerlen}, is extremely concise, and the compiler can
    guarantee it will not raise an exception at runtime. Finally, Obe simply has
    no tool to keep up: he can't call \tt{null} on the left-hand side of the
    \tt{=}, so he can't reason about \tt{length}'s arguments nearly as
    succinctly using observers as Pat can using pattern matching. 


    

\subsection{Extensions as upgrades to pattern matching}
\label{extensions}
    \it{In this section, I introduce three popular extensions to pattern
    matching, along with examples in which they are useful. If you are familiar
    with side conditions, GHC's pattern guards, and or-patterns, you can skip to
    section~\ref{verseoverobservers}.}

    Hopefully, the evidence thus far has convinced you as to why programmers
    often favor pattern matching over observers. Indeed, pattern matching is a
    well-established and researched construct in functional programming, and the
    topic of much literature [cite, appeal to authority]. However, in particular
    instances like I'll show you below, pattern matching on its own can be as
    cumbersome as we saw observers to be in \ref{fig:observerlen}! But
    functional programmers like Pat like pattern matching (by now, you might
    too), so rather than give up on it, language designers frequently introduce
    \it{extensions} to pattern matching to mitigate scenarios where its
    expressiveness falls short. 
    
    Below, I illustrate several such instances, and demonstrate how programmers
    writing in popular functional programming languages employ extensions to
    streamline clunky or verbose code. The extensions I describe are those
    commonly found in the literature and implemented in common compilers.
    
    For the sake of comparison, I coin the term \it{bare pattern matching} to
    denote pattern matching \it{without} extensions: in bare pattern matching,
    the only syntactic forms of patterns are names and applications of value
    constructors to zero or more patterns. 
    
    

    

\subsubsection{Side conditions}

    First, I illustrate why programmers want \it{side conditions}\footnote{I
    use the term \it{side conditions} to refer to a pattern followed by a
    boolean expression. Some languages call this a \it{guard}, which I use to
    describe a different, more powerful extension to pattern matching in
    Section~\ref{guards}.}, an extension to pattern matching common in most
    popular functional programming languages, including OCaml, Erlang, Scala,
    and~Haskell\footnote{Haskell uses guards to subsume side conditions. I
    discuss guards in Section~\ref{guards}.}. Consider a function \tt{nameOf}
    with these algebraic laws: 
    % narrative description of nameOf
    % which tests if an expression is a name, and if that name is in the domain of 
    % an environment. If both of those conditions hold, \tt{nameOf} returns 
    % \tt{Some} of that name. Otherwise, it returns \tt{None}. \tt{nameOf} has 

    \begin{minipage}[t]{\textwidth}
        \centering 
        \begin{verbatim}
nameOf rho (Name n) == Some n, where binds rho n
nameOf rho _        == None 
        \end{verbatim}
    \end{minipage}

    Armed with pattern matching, Pat tries to implement \tt{nameOf}
    (Figure~\ref{fig:ifnameof}).

    \begin{figure}[ht]
        \begin{verbatim}
            let nameOf rho e = match e with 
                  Name n -> if binds rho n then Some n else None
                | _      -> None

                (* where binds rho n == true, where n in dom(rho)
                         binds rho n == false, otherwise *)
            \end{verbatim}    
        \caption{An invented function \tt{nameOf} combines pattern matching with
        an \tt{if} expression, and is not very pretty.}    
        \Description{An implementation of a function nameOf in OCaml, which
        uses pattern matching and an if statement.}
        \label{fig:ifnameof}
    \end{figure}
    
    Obe, who likes using \tt{if-then-else} expressions, is happy with this code,
    but Pat (and hopefully you, the reader) is not. The code gets the job done,
    but it duplicates a right-hand side, and the actual “good” return of the
    function, \tt{n}, is lost in the middle of the \tt{if-then-else} expression.
    Plainly, the code does not look like the algebraic laws. 
    
    Fortunately, this code can be simplified by using the pattern \it{Name n}
    with a \it{side condition}, i.e., a syntactic form for “match a pattern
    \it{and} a boolean condition.” The \tt{when} keyword in OCaml
    provides such a form, as seen in Figure~\ref{fig:whennameof}.
        
        \begin{figure}[ht]
            \begin{verbatim}
            let nameOf rho e = match e with     
                  Name n when binds rho n -> Some n
                | _                       -> None  
                \end{verbatim}
            \caption{With a side condition, \tt{nameOf} becomes simpler and more
            clear in its goal.}
            \Description{An implementation of a function nameOf in OCaml, which
            uses pattern matching and a side condition.}
            \label{fig:whennameof}
        \end{figure}

        
    
    
    

    Side conditions streamline pattern-and-boolean cases and minimize overhead.
    One notable advantage of side conditions is their capacity to exploit
    bindings that emerge from the preceding pattern match. For instance, the
    \tt{when} clause in Figure~\ref{fig:whennameof} exploits \tt{n}, which is
    bound in the match of \tt{e} to \tt{Name n}.

    Side conditions address the necessity of incorporating an extra “check”- in
    this case, a boolean expression- within a pattern. But what if the 
    arises to conduct additional pattern matches within such a check? In the
    next section, I'll showcase an example that highlights- and an extension
    that mitigates- this problem. 
    
    \subsubsection{Pattern guards}
    \label{guards}
    
    Say now we want \tt{nameOf} to perform a series of checks, described by
    these algebraic laws: 

    \begin{minipage}[t]{\textwidth}
        \centering 
        \begin{verbatim}
nameOf rho (Name n) == Some v, where 
                                 binds rho n, 
                                 lookup n rho == Some (x::xs),
                                 lookup x rho == Some v
nameOf rho _        == None 
        \end{verbatim}
    \end{minipage}


    % \begin{enumerate}
    %     \item First, to call a lookup function on \tt{n} in \tt{rho} if the
    %     \tt{when} condition succeeds; 
    %     \item then, if the lookup returns \tt{Some} of a nonempty list, to call
    %     \tt{lookup} on the first element of that list in \tt{rho};
    %     \item finally, if the second lookup returns \tt{Some} of a value, return \tt{Some}
    %     of that value.
    %     \item If any of the above steps fails, \tt{nameOf} returns \tt{None}.
    % \end{enumerate}

    % Our new \tt{nameOf} has these algebraic laws: 

    
    

    Woof! We need to match successive computations with patterns, but side
    conditions don't give us that flexibility. I shudder to ask Obe to help us
    here, but let's do it anyway. To help him, I'll give him side conditions,
    which play nice with his observer functions, and I'll ask him to stomach the
    top-level \tt{match}. Let's look at his implementation of the new
    \tt{nameOf} in Figure~\ref{fig:obenameof}.

    \begin{figure}[ht]
        \begin{verbatim}
let nameOf rho e = match e with     
      Name n   
         when binds rho n 
      andalso isSome (lookup n rho) 
      andalso not    (null (Option.get (lookup n rho))) 
      andalso isSome (lookup (hd (Option.get (lookup n rho))) rho) ->
            Option.get (lookup (hd (Option.get (lookup n rho))) rho)
    | _  -> None
            \end{verbatim}
        \caption{Obe's \tt{nameOf} is unsurprisingly horrifying.}
        \Description{An implementation of the new nameOf, with a lot of 
        manual deconstruction with side conditions.}
        \label{fig:obenameof}
    \end{figure}

    This is grisly. There are four redundant calls to \tt{lookup}, which we can
    only pray the compiler will optimize by putting their values into a register
    at runtime. And there are functions that can raise exceptions everywhere:
    four calls to \tt{Option.get} can each raise the \tt{Invalid\_argument}
    exception, and either of the calls to \tt{hd} can raise \tt{Failure}. 

    Let's ask Pat for help- her solution will likely bind the call to
    \tt{lookup} to a variable, and she'll use pattern matching to eschew
    exception-throwing functions. But her solution in Figure~\ref{fig:patnameof} has
    its own problems. 

    \begin{figure}[ht]
        \begin{verbatim}
        let nameOf rho e = match e with     
              Name n when binds rho n -> 
                            let r1 = lookup n rho in 
                              match r1 with 
                                Some (x::xs) -> 
                                  let r2 = lookup x rho in 
                                    match r2 with 
                                      Some v -> Some v   
                                    | _ -> None
                                | _ -> None
            | _ -> None 
            \end{verbatim}
        \caption{Pat's \tt{nameOf} duplicates many right-hand sides.}
        \Description{An implementation of a function nameOf in OCaml, which
        uses pattern matching and a side condition.}
        \label{fig:patnameof}
    \end{figure}

    Pat's program is not worse than Obe's, but it's not much better. It contains
    four duplicate right-hand sides, a lot of nesting of \tt{let} and
    \tt{match}, and our “good” return value is once again completely lost in the
    middle. Moreover, it isn't at all easy to tell Pat's program implements the
    algebraic laws.
    
    Pat and Obe agree that both side conditions and bare pattern matching prove
    inadequate here: both styles of solution involve convoluted code that is
    cumbersome to write and challenging to follow at first glance. 
    
    To help them, I'll introduce a more powerful extension to pattern matching
    which addresses this problem: \it{pattern guards}, a form of “smart pattern”
    in which intermediate patterns bind to expressions within a single branch of
    a \tt{case}. Let's look at what \tt{nameOf} looks like with pattern guards.
    Before we do, we'll have to transition briefly to Haskell to use pattern
    guards, since they are absent in OCaml. The two languages' syntaxes are
    similar enough that the example (Figure~\ref{fig:guardnameof}) should still
    be clear. 

    \begin{figure}[hbt!]  
        \begin{center}
        \begin{verbatim}
                        nameOf rho (Name n)
                           | binds rho n 
                           , Just (x:xs) <- lookup rho n
                           , Just v <- lookup rho x
                           = Just v
                        nameOf rho _ = None
        \end{verbatim}
        \end{center}    
    \caption{The solution to \tt{nameOf} with pattern guards is plainly the most
    elegant.} 
    \Description{An implementation of nameOf using pattern guards.}
    \label{fig:guardnameof}
    \end{figure}

    Guards offer an elegant solution to the problem of matching on successive
    computations. The first guard even subsumes a side condition! If you need
    further convincing on why programmers want for guards, look no further than
    [cite], the proposal for pattern guards in GHC: the authors show several
    other examples where guards drastically simplify otherwise-maddening code
    [cite- Erwig \& Peyton Jones]. 
    
    The power of pattern guards lies in the ability for expressions within
    guards to utilize names bound in preceding guards, enabling imperative
    pattern-matched steps with expressive capabilities akin to Haskell's \tt{do}
    notation. It should come as no surprise that pattern guards are built in to
    GHC. 

        

\subsubsection{Or-patterns}

        I conclude our tour of extensions to pattern matching with or-patterns,
        which are built in to OCaml. Let's consider a final example, where Pat
        tries to implement a function \tt{parent\_game\_of\_token} with these 
        algebraic laws: 

        \begin{minipage}[t]{\textwidth}
            \centering 
            \begin{verbatim}
parent_game_of_token f == "Fortnite",   where f is any of 
                                            BattlePass, 
                                            ChugJug, or
                                            TomatoTown
parent_game_of_token b == "Bloodborne", where b is any of 
                                            HuntersMark or 
                                            SawCleaver
parent_game_of_token e == "Elden Ring", where e is any of 
                                            MoghLordOfBlood or  
                                            PreatorRykard 
parent_game_of_token _ == "Irrelevant"
                                        \end{verbatim}
        \end{minipage}
    

        % In the preceding example, we observed how pattern guards facilitated a
        % multi-step computation within a single match. However, what if the
        % programmer's intention isn't to match on \it{all} parts of a pattern
        % sequence, but instead to match a value on \it{any one} of such a
        % sequence of patterns? Our example in Figure~\ref{fig:barepgot}
        % illustrates this need. 
        
        
        Pat can write \tt{parent\_game\_of\_token} in OCaml using bare patterns
        (Figure~\ref{fig:barepgot}), but she's dissatisfied with the duplicated
        right-hand sides and the length of the code. 
        
        
        \begin{figure}
            \begin{center}
                \begin{verbatim}
        let parent_game_of_token token = match token with 
            | BattlePass      -> "Fortnite"
            | ChugJug         -> "Fortnite"
            | TomatoTown      -> "Fortnite"
            | HuntersMark     -> "Bloodborne"
            | SawCleaver      -> "Bloodborne"
            | MoghLordOfBlood -> "Elden Ring"
            | PreatorRykard   -> "Elden Ring"
            | _               -> "Irrelevant"
                \end{verbatim}
            \end{center}    

        \caption{\tt{parent\_game\_of\_token}, with redundant right-hand sides,
        should raise a red flag.} 
        \Description{An implementation of
        parent_game_of_token using bare patterns.}
        \label{fig:barepgot}
        \end{figure}

        Obe tries to give her a hand by suggesting side conditions- after all,
        they might help the code look like the laws. Together, they come up with
        Figure~\ref{fig:sideconditionpgot}. Side conditions look to help at
        first, but the code quickly spirals into redundancy, and the additional
        calls hurt performance. 

        \begin{figure}
            \begin{center}
                \begin{verbatim}
        let parent_game_of_token token = 
        match token with 
        | f when from_fortnite   f -> "Fortnite"
        | b when from_bloodborne b -> "Bloodborne"
        | e when from_eldenring  e -> "Elden Ring"
        | _                        -> "Irrelevant"
        and from_fortnite t = match t with 
          | BattlePass -> true
          | ChugJug    -> true
          | TomatoTown -> true
          | _          -> false 
        and from_bloodborne t = match t with 
          | SawCleaver  -> true
          | HuntersMark -> true
          | _           -> false 
        and from_eldenring t = match t with 
          | MoghLordOfBlood -> true
          | PreatorRykard   -> true
          | _               -> false 
                \end{verbatim}
            \end{center}    

        \caption{\tt{parent\_game\_of\_token} with side conditions ends up 
        even more verbose.} 
        \Description{An implementation of
        parent_game_of_token using side conditions.}
        \label{fig:sideconditionpgot}
        \end{figure}

        Thankfully, an extension once again comes to her rescue.
        \it{Or-patterns} condense multiple patterns which share right-hand
        sides, and Pat can exploit them to eliminate much redundant code in
        Figure~\ref{fig:orpgot}.

    \begin{figure}
    \begin{center}
    \begin{verbatim}
        let parent_game_of_token token = match token with 
            | BattlePass | ChugJug | TomatoTown  -> "Fortnite"
            | HuntersMark | SawCleaver           -> "Bloodborne"
            | MoghLordOfBlood | PreatorRykard    -> "Elden Ring"
            | _                                  -> "Irrelevant"
    \end{verbatim}
    \end{center}    
    \caption{With or-patterns, \tt{parent\_game\_of\_token} condenses
    tremendously and is easier to read line-by-line.} 
    \Description{An
    implementation of parent_game_of_token using or-patterns.}
    \label{fig:orpgot}
    \end{figure}

    In addition to the inherent appeal of brevity, or-patterns serve to
    concentrate complexity at a single juncture and create single points of
    truth. And in cases with nested functions, like in
    Figure~\ref{fig:sideconditionpgot}, or-patterns actually boost performance.
    
    % Consider a scenario where,
    % instead of a string, \tt{parent\_game\_of\_token} yields the outcome of a
    % complex computation. In the initial model, duplicating the right-hand sides
    % across multiple patterns would necessitate the programmer to manage numerous
    % redundant copies of this computation. Or-patterns empower programmers to
    % centralize such maintenance at a single point of truth.
    
    \subsubsection{Wrapping up pattern matching and extensions}
    Now, any pattern matching-loving functional programmer like Pat, myself, or
    (now, hopefully!) you knows about the extensions that make pattern matching
    more expressive and how to use them effectively. Earlier, though, you might
    have noticed a problem. Say we face a decision-making problem that obliges
    us to use all of our extensions in unison. When picking a language to do so,
    we are stuck! Indeed, no major functional language has all three of these
    extensions.\footnote{Remember when we had to switch from OCaml to Haskell to
    use guards, and back to OCaml for or-patterns? The two extensions are
    mutually exclusive in Haskell, OCaml, Scala, Erlang/Elixir, Rust, F\#, Agda,
    and more [cite].} I claim there's another, deeper problem: in each example,
    we keep having to extend pattern matching to meet our needs. This is not an
    imaginary issue: language designers implemented of these extensions in a
    real compiler to address real need. If we as designers build a language with
    all three of our extensions, or even as many extensions as we can imagine,
    how will we know it won't need to be extended again? \rab{This claim is
    cool, but I'm not sure if it holds water. I would like your advice on if this
    is a smart claim to make (i.e., bolsters my argument that extending pattern
    matching is tiresome, and an alternative would be nice) or a horrible trap
    in which C++ implementors wring my neck.}


    Keep this issue in your mind as I move us onwards. In the next section, I'll
    introduce you to \it{equations}, a different way for programmers to write
    code that makes decisions and deconstructs data. Once you're familiar with
    equations, you'll be ready to compare their strengths and weaknesses with
    those of pattern matching, and judge the compromise I propose.
    \rab{Likewise, I don't know if I like this bit at all. I want a nice
    transition to Verse, motivated by pattern matching's shortcomings. Does this
    do the trick, and sound ok? }
    
    \subsection{Verse's equations as an upgrade to observers}
    \label{verseoverobservers}
    
    
    
    I now shift focus away from pattern matching to examine a different strategy
    for making decisions. I focus our study on Verse [cite- Verse], a functional
    logic programming languages [cite- flps]. 

    Even if you and Pat have both read the Verse paper, the language might look
    unfamiliar to your pattern matching-accustomed brains. To help ease you into
    Verse, I'll ground explanations and examples in how they relate to pattern
    matching. 

    Verse uses \it{equations} instead of pattern matching to test for structural
    equality and create bindings. Like pattern matching, equations scrutinize
    and deconstruct data at runtime by testing for structural equality and
    unifying names with values. Unlike pattern matching, Verse's equations can
    unify names on both left- \it{and} right-hand sides. 
    
    Equations in Verse take the form \it{x = e}, where \it{x} is a name and
    \it{e} is an expression. During runtime, both \it{x} and any unbound names
    in \it{e} are unified with values. Much like pattern matching, unification
    can fail if the runtime attempts to bind incompatible values or structures. 

    Let's look at what equations look like in real, written Verse code. Pat is
    curious about exactly this, so she asks her friend Verv, a Verse enthusiast,
    to show how he would write \tt{length}, reminding him that she only knows
    pattern matching and doesn't have a background in functional logic
    programming. Verv comes up with Figure~\ref{fig:verselen}. 
    
    

    \begin{figure}[h]
        \verselst
        \begin{lstlisting}[numbers=none]
$\exists$ length. length = \ys. 
  one {ys = $\langle \rangle$; 0
       |  $\exists$ x xs. ys = $\langle$x, xs$\rangle$; 1 + length xs}
        \end{lstlisting}
    \caption{\tt{length} in Verse uses existentials and equations rather than
    pattern matching.} 
    \Description{An implementation of length in Verse.}
    \label{fig:verselen}
    \end{figure}

    Like in the pattern-matching examples in Figures~\ref{fig:pmlen}
    and~\ref{fig:pmclausallen}, the right-hand sides of \tt{length} are
    \it{guarded} by a “check;” now, the check is successful unification in an
    equation rather than a successful match on a pattern. 

    Obe watches sadly from a distance: Pat smiles and enjoys Verv's
    implementation of \tt{length}, preferring it to the one using observers (you
    might, as well). Verv's \tt{length} is a bit more math-y than hers, but that
    might not be a bad thing: though it doesn't resemble the algebraic laws a
    programmer would write, it likely resembles the equations that a
    mathematician would. 

    
    
    
    You and Pat notice that in Verse, unlike in our pattern matching languages,
    programmers must introduce names explicitly with the existential $\exists$.
    At runtime, equations will attempt to unify all these names with values.
    When unification succeeds, equations will bind the names to those values,
    just as pattern matching binds names to values when a match succeeds. On
    line~3 of the example, the existential $\exists$ introduces two fresh names,
    \tt{x} and \tt{xs}, before attempting to bind them to (by unifying them
    with) \tt{ys}. These fresh names correspond to the \tt{x} and \tt{xs} in
    Figure \ref{fig:pmlen}. 

    You and Pat see that despite these differences, Verse's equations appear
    comparable to pattern matching, and that the language enjoys several forms
    that solve the same problems as pattern matching's extensions. 

    

    Like patterns in pattern matching, equations in Verse always guard
    expressions; they cannot be returned. Verse enforces this restriction in the
    syntax: expressions may take the form \it{eq; e}, where \it{eq} is an
    equation or an expression. Importantly, either form of \it{eq} guards {e}.
    If \it{eq} fails (returns no results), regardless of whether it is an
    expression or an equation, \it{e} will not be evaluated. This means
    programmers can also express “perform a computation if a condition succeeds”
    with a boolean\footnote{Verse does not have true booleans. Instead, it
    replaces 'true' and 'false' with the notion of 'success' and 'failure.'}
    expression in the \it{eq} position. 

    

    

    
    any one succeeds, with bindings.” 

    % % programmers introduce names explicitly 

    % % leading to differences: 
    % % 105 mistake. 


    % INLINE EXAMPLES. 

    % Verse's equations comparably pleasing to pattern matching in their brevity
    % and expressiveness, though they differ in a few key ways. For one, in Verse,
    % the programmer \it{explicitly} introduces the names that appear inside an
    % equation with the existential $\exists$. These names will be bound to the
    % components of a value when a match succeeds. On line~3 of the example, we
    % see the existential $\exists$ introduce two fresh names, \tt{x} and \tt{xs},
    % before attempting to bind them to \tt{ys}. These fresh names correspond to
    % the \tt{x} and \tt{xs} in (\cref{fig:pmlen}). 

    % Also, in Verse's equations, names with prior bindings \it{retain those
    % bindings} during matching. This is different from pattern matching, where a
    % name in a pattern is always considered fresh and will always succeed in
    % binding to a value. This is best illustrated with an example (Figure~TODO). 
    % We discuss the implications of this difference in Section~(TODO which).

    % Todo: show "105 student mistake" example. 

    % Todo: write more. 

    % \rab{The transition here feels shaky at best. I'd like help connecting the
    % two points: 1. Verse is comparably pleasing to PM, albeit a bit different
    % and 2. However, Verse has an unpredictable cost model.}

    % BIOHAZARD END 

    % Example above illustrates this. Likely add more. 
    % 
    
    % 
    % 
    % 
    % 

    
    

    Verse's equations appear to be comparably pleasing to pattern matching
    in their brevity and expressiveness. However, full Verse allows/encourages
    computations that are problematic, cost-wise. In Verse, names (logical
    variables) are \tt{values}, and can just as easily be the result of
    evaluating an expression as an integer or tuple. To bind these names, Verse,
    like other functional logic languages, relies on unifying its logical
    variables at runtime to meet a set of program constraints [cite]. Unifying
    logical variables at runtime classically requires backtracking [cite], which
    can lead to exponential runtime cost [cite]. 
        
    Furthermore, rather than returning a single value, an expression in Verse
    evaluates to return a \it{sequence} of zero or more values. As such,
    unifying the results of expressions can quickly lead to """massive
    combinatorials""" whose size can be difficult to predict. (Todo: technical phrase for 'something massive.') 


    

    

    

    
\section{Simplification for Study and Synthesis}

We now present solutions to the problem posed above in the form of two small
functional programming languages. % which we informally introduce

\subsection{Introducing \PPlus\ }


    
    \PPlus. Its definition is found in \dots TODO 
    
    
    \begin{itemize}
        \item side conditions
        \item or-patterns 
        \item pattern guards 
    \end{itemize}

    \PPlus\ packages common and standard extensions to pattern matching. In
    addition to traditional patterns (maybe say bare patterns?)-- names and
    applications of value constructors-- the language includes pattern guards,
    or-patterns, and side conditions (Footnote: side conditions are subsumed by
    pattern guards [cite?]; they are included for purpose of study.)
    Furthermore, a pattern in \PPlus\ be a \it{sequence} of sub-patterns,
    allowing for combinations of arbitrary numbers of patterns. 
    
    

    Figure~\ref{fig:ppexs} provides an example of how a programmer might utilize
    \PPlus\ to solve the previous problems: 

    TODO fix line nums, add other examples 
    \begin{figure}
        \begin{center}
            \pplst 
            \begin{lstlisting}
        val parent_game_of_token = \t. 
            case t of  
                BattlePass | ChugJug | TomatoTown  -> Fortnite
            | HuntersMark | SawCleaver             -> Bloodborne
            | MoghLordOfBlood | PreatorRykard      -> Elden Ring
            | _                                    -> Irrelevant
        \end{lstlisting}
        \end{center}    
        \caption{\tt{parent\_game\_of\_token} in \PPlus has the same desirable
        implementation as before.} 
        \Description{An implementation of
        parent_game_of_token in \PPlus.}
        \label{fig:ppexs}
        \end{figure}
    
    
\subsection{Addressing how \PPlus\ handles unusual pattern combinations}




    Given its minimal restrictions on what kind of pattern can appear where,
    \PPlus\ admits of strange looking patterns: consider \tt{Cons (when true)
    zs}. But these should not be alarming, because such syntactic forms reduce
    to normal forms by (direct) application of algebraic laws: 

    
        
        

        
        
        
        
        
        
        

        
        These laws are applied repeatedly until we reach a fixed point. 
        Todo: add in revision 
\subsection{Introducing \VMinus\ }

        
        To fuel our pursuit of smarter decision-making, we now draw inspiration
        from Verse. We begin by trimming away the aforementioned culprits that
        tends to lead to unpredictable or costly outcomes during runtime:
        backtracking and multiple results. Footnote (Our removing these language
        traits strips much of the identity of Verse; however, we do so in order
        to study only Verse's \it{equations} while grounding ourselves in an
        otherwise-typical programming context of single results and no
        backtracking at runtime.) Both backtracking and multiple results often
        manifest within Verse's choice operator, tempting us to consider its
        removal altogether. However, we are drawn to harnessing the expressive
        potential of this operator, particularly when paired with Verse's 'one'
        keyword. When employed with choice as a condition, 'one' elegantly
        signifies 'proceed if any branch of the choice succeeds. 
        
        
        SAY WHAT CHOICE DOES. 

        To this end, we imagine a new language, \VMinus, in which we permit
        choice with several modifications:

        \begin{enumerate}
        \item Choice may only appear as a condition or 'guard', not as a result
        or the right-hand side of a binding.
        \item If any branch of the choice succeeds, the choice succeeds,
        producing any bindings found in that branch. The program examines the
        branches in a left-to-right order.
        \item The existential $\exists$ may not appear under choice.
        \end{enumerate}

        We introduce one more crucial modification to the Verse runtime: a name
        in \VMinus is an \it{expression} rather than a \it{value}. This
        alteration, coupled with our adjustments to choice, effectively
        \rab{remove 'effectively' here?} eradicates backtracking. Our rationale
        behind this is straightforward: if an expression returns a name, and
        subsequently, the program imposes a new constraint on that name, it may
        necessitate the reevaluation of the earlier expression—- a scenario we
        strive to avoid. An example in Verse illustrates this precise scenario:

        FIRST PAPER EXAMPLE 

        

        
        
        
        


        The discerning reader, particularly those well-versed in functional
        logic programming, may perceive that imposing such restrictions on
        choice and names effectively strips away much of Verse's essence as a
        functional logic programming language. With these constraints enforced,
        there can be no backtracking, multiple results, backward function
        evaluation, or top-level patterns, among other classic functional logic
        programming features. But do not fear: our intent is not to recklessly
        strip Verse of its meticulously crafted core tenets. Instead, our aim is
        to extract a select few—namely, its equations, existentials, and
        nondeterministic evaluation order—and juxtapose them with pattern
        matching.

        % The reader, especially if they are familiar with functional logic
        % programming, will by now have realized that such restrictions on choice
        % and names effectively eliminate much of the identity of Verse as a
        % functional logic programming language. With them in place, there can be
        % no backtracking, no multiple results, no evaluating functions backwards,
        % no top-level patterns: the list of curios goes on. But fear not: we do
        % not aim to thoughtlessly gut Verse of its meticulously-chosen core
        % features, but rather to extract a select few of them- namely, its
        % equations, existentials, and nondeterministic evaluation order- and
        % compare them with pattern matching. 
    

        Indeed, having stripped out the functional logic programming elements of
        Verse, we are left with just the decision-making bits. To wrap these, we
        add a classic decision-making construct: guarded commands [cite-
        Dijkstra paper, others.] The result is \VMinus. \VMinus is defined in 
        Figure~(TODO, whatever extension solution you choose). 

        % \bf{Choice forces "name knowing." Names must be known in all branches 
        % if bound to unknown variable or bound to known variable if any name 
        % is unknown}.
        
        
        
        
        With these modifications in place, alongside a few additional
        adjustments to the placement of existentials and the timing of choice
        evaluation (footnote: we discuss these later), we unveil a redefined decision-making core. To complete this
        transformation, we incorporate a venerable decision-making construct:
        Guarded commands [cite: Dijkstra]. The result is \VMinus. 

    

    To facilitate comparisons and proofs, \VMinus\ and \PPlus\ are each a subset
    of a single unifying language(\cref{fig:unilang}). Column “Unique To” indicates
    which components of this unifying language belong to the sub-languages. 
    \PPlus, \VMinus, and the decision tree language \D\ (Section~TODO) all share 
    the same core of lambdas, value constructors (\it{K}), names, and function 
    application. 


    

    
    
    
    



    \subsection{Maybe a subsection break here}
    
    
    
    Even with multiple modifications, \VMinus\ still allows for many of the same
    pleasing computations as full Verse. Programmers can... 
        \begin{enumerate}
            \item Introduce a set of equations, to be solved in a nondeterministic order 
            \item Guard expressions with those equations 
            \item Flexibly express "proceed when any of these operations succeeds" with choice 
        \end{enumerate}

    Figure~TODO provides an example of how a programmer might utilize
    \VMinus\ to solve the previous problems:

    
    
    
    
    Look! It looks just like Verse! 
   
    
        Or incomparable? (Why not both?)
    
    \subsection{\VMinus\ and \PPlus, side by side}

    \begin{itemize}
        \item We compare \VMinus\ with \PPlus\ as an exercise in comparing
        equations with pattern matching. 
        \item They definitely look similar: as expressive as pattern. Great! 
        \item{BIG PAYLOAD ALERT}
        \item Bonus 1: You don't need the scrutinee in \VMinus. 
        \item Bonus 2: Binding, checking are all in one construct: =. No
                $\leftarrow$ vs case. No guessing what value patterns match to. 
                And, unlike ML, Haskell, OCaml: no need for \tt{let}.
        \item Bonus 3: Names. Explicit name introduction means we can use a name
        from before and know if it's a pattern or an expression. This prevents
        common mistakes like the kind 105 students make. 
    \end{itemize}
    

    

    

    

IN EITHER OF THESE SECTIONS: Semantics of \PPlus, \VMinus

\section{(PAYLOAD) {\VMinus} can be compiled to \D 
        [Evidence that \dots efficiency]}

    
    strict invariant that no value is examined more than once at runtime. But we wish to go further in our assertion, 
    so we compile \VMinus to a decision tree. 
    
    
    
    
    
    
    

    \bf{A decision tree can be exponential in size but never examines a word of
    the scrutinee more than once. }

    
    
    


\section{\PPlus -> \VMinus is interesting (optional)}

    
    
    
    


\section{Related Work}
\section{Future Work}

    
    
    
        guards. No major functional language does this. 
    
    friendlier cost model (depending on compiler)
    
        (That is, how to solve problems in Verse)
    


\section{Conclusion, Discussion}

\renewcommand\thesection{\Alph{section}}
\setcounter{section}{0}
\section{Proofs}

    
    
    


\section{Trust me on \VMinus (It's reduced to Verse)}

    
    
    
    


\begin{table}[ht]
    \centering
    \small
    \begin{tabular}{l l l}
        \textbf{Syntactic Forms} & \textbf{Cases} & \textbf{Unique to} \\
        \hline
        $P$ : Programs & $\bracketed{d}$ & \\
        $d$ : Definitions & $\mathit{val}\; x\; \mathit{=}\; \expr$ & \\
        $\expr$ : Expressions & $x, y, z$ & \\
        & $K\bracketed{\expr}$ &  \\
        & $\lambda x.\; \expr$ & \\
        & $\expr[1]\; \expr[2]$ & \\
        & $\mathit{case}\; \expr\; \bracketed{p\; \rightarrow\; \expr}$ & \PPlus \\
        & $\mathit{if}\; \mathit{[}\; g\; \bracketed{[] g}\; \mathit{]}\; \mathit{fi}$ & \VMinus \\
        & $\dt$ & \D \\
        $\v$ : Values & $K\bracketed{\v}$ & \\
        & $\lambda x.\; \expr$ & \\
        $p$ : Patterns & $x$ & \PPlus \\
        & $K\; \bracketed{p}$ & \PPlus \\
        & $\mathit{when}\; \expr$ & \PPlus \\
        & $p, p'$ & \PPlus \\
        & $p\; \boldsymbol{\leftarrow}\; \expr$ & \PPlus \\
        & $p_{1}\pbar p_{2}$ & \PPlus \\
        $g$ : Guarded Expressions & $\boldsymbol{\rightarrow}\expr$ & \VMinus \\
        & $\expr;\; g$ & \VMinus \\
        & $\vexists{x} g$ & \VMinus \\
        & $x = \expr;\; g$ & \VMinus \\
        \dt : Decision Trees & $\mathit{case}\; x\; \mathit{of}\; \bracketed{\vert\; K\bracketed{x}\; \mathit{=>}\; \dt} [\vert\; x\; \mathit{=>} \dt]$ & \D \\
        & $\expr$ & \D \\
        & $\mathit{if}\; x\; \mathit{then}\; \dt\; \mathit{else}\; \dt$ & \D \\
        & $\mathit{let}\; x\; \mathit{=}\; \expr\; \mathit{in}\; \dt$ & \D \\
    \end{tabular}
    \caption{Example LaTeX Table}
    \label{fig:unilang}
\end{table}



%     \item Programmers use pattern matching. But pattern matching is not good for
%     everything. 
%     \item Consider Example. 
%     \item Explain example. 
%     \item Attempted mitigations in the past: extensions, SPJ example. 
%     \item We go futher. 
%     \item \bf{P3}: 
%     \item Contribution: 2 languages that attempt to mitigate this problem. 
%     \item Our goal: One simple, expressive, efficient decision-making construct.
%     \item One is {\PPlus}, the other is {\VMinus}. 
%     \item Example of {\VMinus} in action on prior example. 
% \end{itemize}

% We begin by informally introducing two small functional programming languages to
% address the issue of inflexibility of patterns.

% {\PPlus} models standard pattern matching with common extensions. {\VMinus}
% models Verse without features that lead to unpredictable costs (backtracking and
% multiple results) [cite-Verse]. 

% To aid proofs of efficiency, we also introduce a third langauge, {\D}. {\D} is a
% language of decision trees to which both {\PPlus} and {\VMinus} can be
% translated. Targets of translation are efficient in the standard technical
% sense: no value is ever scrutinized more than once (Maranget 2008).

% To facilitate comparisons and proofs, the languages are each a subset of a
% single unifying language(\cref{fig:U}). Asterisks (${}^{*}$) indicate components
% of this unifying language which belong to the sub-languages. 

% % The keen reader will notice that our syntax includes a forms for decision trees,
% % which we have not yet discussed. Decision trees will be crucial in proving
% % properties of efficiency of our languages, and will we be discussed in full in a
% % later section. 

% \begin{figure}[ht!p]
%     \small
%     \begin{flushleft}
%         \begin{bnf}
%         $P$ : \textsf{Programs} ::=
%         $\bracketed{d}$ : definition
%         ;;
%         $d$ : \textsf{Definitions} ::=
%         | $\mathit{val}\; x\; \mathit{=}\; \expr$ : bind name to expression
%         ;;
%         $\expr$ : Expressions ::= 
%         | $x, y, z$             : names
%         | $K\bracketed{\expr}$  : value constructor application 
%         | $\lambda x.\; \expr$  : lambda declaration  
%         | $\expr[1]\; \expr[2]$ : function application 
    
%         | $\mathit{case}\; \expr\; \bracketed{p\; \rightarrow\; \expr}$ : $\rm{case expression}^{*}$
%         | $\mathit{if}\; \mathit{[}\; g\; 
%             \bracketed{[] g}\; \mathit{]}\; \mathit{fi}$                : $\rm{if-fi}^{**}$
%         | $\dt$                                                         : $\rm{decision tree}^{***}$
%         ;;
%         $\v$ : Values ::= 
%           $K\bracketed{\v}$     : value constructor application 
%         | $\lambda x.\; \expr$  : lambda value 
%         ;;
%         $p$ : $\textsf{Patterns}^{*}$ ::= 
%         $x$ : name 
%         | $K\; \bracketed{p}$           : value constructor application 
%         | $\mathit{when}\; \expr$       : side condition
%         | $p, p'$                       : pattern guard 
%         | $p\; \leftarrow\; \expr$      : pattern from explicit expression  
%         | $p_{1}\pbar p_{2}$            : or-pattern
%         ;;
%         $g$ : $\textsf{Guarded Expressions}^{**}$ ::=  
%         $\boldsymbol{\rightarrow}\expr$ : terminating experession
%         | $\expr;\; g$                  : intermediate expression 
%         | $\vexists{x} g$      : existential 
%         | $x = \expr;\; g$              : equation 
%         ;;
%         \dt : $\textsf{Decision Trees}^{***}$ ::= 
%         | $\mathit{case}\; x\; \mathit{of}\; 
%             \bracketed{\vert\; K\bracketed{x}\; \mathit{=>}\; \dt} 
%             [\vert\; x\; \mathit{=>} \dt]$                              : $\rm{test node }^{***}$
%         | $\expr$                                                       : $\rm{match node }^{***}$
%         | $\mathit{if}\; x\; \mathit{then}\; \dt\; \mathit{else}\; \dt$ : $\rm{condition with two children }^{***}$
%         | $\mathit{let}\; x\; \mathit{=}\; \expr\; \mathit{in}\; \dt$   : $\rm{let-bind a name}^{***}$
%         % ;;
%         % $K$ : \textsf{Value Constructors} ::=
%         % | $\mathit{true}\; \vert\; \mathit{false}$ : booleans
%         % | $\mathit{\#}x$                           : name beginning with $\mathit{\#}$
%         % | $\mathit{A-Z}x$                          : name beggining with capital letter
%         % | $[\mathit{-}\vert\mathit{+}]
%         %     (\mathit{0}-\mathit{9})+$              : signed integer literal 
%         \end{bnf}
%         \medskip
        
        
%         \it{Concrete Syntax}: "$\lambda$" and "\exists" each scope as far to
%         the right as possible.
        
%         $\hskip 8em$ For example, $(\lambda y.\> \vexists{x}\> x = 1;\> x + y)$ means 
%         $(\lambda y.\> (\vexists{x}\> ((x = 1);\> (x + y))))$.
        
%         Parentheses may be used freely to aid readability and override default precedence.

%         A \it{name} is any token that is not an integer literal, does not
%         contain whitespace, a bracket, or parenthesis, and is not a value
%         constructor name or a reserved word.
        
%         \medskip

%         % \bf{Desugaring of Extended Expressions}

%         ${}^{*}$ Indicates forms within {\PPlus}

%         ${}^{**}$ Indicates forms within {\VMinus}
        
%         ${}^{***}$ Indicates forms within {\D}

%     \end{flushleft}
    
%     \medskip

    

%     \caption{\U, a decision-making language}
%     \Description{A BNF grammar for \U, the universal decision-making language.
%                  It includes patterns, guarded expressions, and decision trees.}
%     \label{fig:U}
% \end{figure}
% % \begin{table}[ht]
% %     \centering
% %     \small
% %     \begin{tabular}{l l l}
% %         \textbf{Syntactic Forms} & \textbf{Cases} & \textbf{Belong to} \\

% %         $P$ : Programs & $\bracketed{d}$ & many definitions & \\
% %         $d$ : Definitions & $\mathit{val}\; x\; \mathit{=}\; \expr$ & bind name to expression & \\
% %         $\expr$ : Expressions & $x, y, z$ & names & \\
% %         & $K\bracketed{\expr}$ & value constructor applied to expressions & \\
% %         & $\lambda x.\; \expr$ & lambda declaration & \\
% %         & $\expr[1]\; \expr[2]$ & function application & \\
% %         & $\mathit{case}\; \expr\; \bracketed{p\; \rightarrow\; \expr}$ & $\rm{case expression}$ & \\
% %         & $\mathit{if}\; \mathit{[}\; g\; \bracketed{[] g}\; \mathit{]}\; \mathit{fi}$ & $\rm{if-fi}$ & \\
% %         & $\dt$ & $\rm{decision tree}$ & \\
% %         $\v$ : Values & $K\bracketed{\v}$ & value constructor applied to values & \\
% %         & $\lambda x.\; \expr$ & lambda value & \\
% %         $p$ : Patterns & $x$ & name & \\
% %         & $K\; \bracketed{p}$ & value constructor applied to patterns & \\
% %         & $\mathit{when}\; \expr$ & side condition & \\
% %         & $p, p'$ & pattern guard & \\
% %         & $p\; \leftarrow\; \expr$ & pattern from explicit expression & \\
% %         & $p_{1}\pbar p_{2}$ & or-pattern & \\
% %         $g$ : Guarded Expressions & $\boldsymbol{\rightarrow}\expr$ & terminating expression & \\
% %         & $\expr;\; g$ & intermediate expression & \\
% %         & $\vexists{x} g$ & existential & \\
% %         & $x = \expr;\; g$ & equation & \\
% %         \dt : Decision Trees & $\mathit{case}\; x\; \mathit{of}\; \bracketed{\vert\; K\bracketed{x}\; \mathit{=>}\; \dt} [\vert\; x\; \mathit{=>} \dt]$ & $\rm{test node}$ & \\
% %         & $\expr$ & $\rm{match node}$ & \\
% %         & $\mathit{if}\; x\; \mathit{then}\; \dt\; \mathit{else}\; \dt$ & $\rm{condition with two children}$ & \\
% %         & $\mathit{let}\; x\; \mathit{=}\; \expr\; \mathit{in}\; \dt$ & $\rm{let-bind a name}$ & \\
% %     \end{tabular}
% %     \caption{Example LaTeX Table}
% %     \label{tab:example}
% % \end{table}

% Most syntactic categories in \U are present in all of its subsets: Values, Value
% Constructors, Definition Forms, and most forms of expression are shared. Indeed,
% the sub-langauge, like in Verse, is just the the lambda calculus with a few
% extended syntactic categories (this time favoring value constructors over
% tuples). Like in Verse, every Lambda Calculus program is a valid \U program. 

% On top of this core, three languages are defined as subsets of \U. {\PPlus} is
% the subset of~\U including Patterns and the $\mathit{case}\; \expr\;
% \bracketed{p \rightarrow \expr}$ form of case expression. {\VMinus} is the
% subset of \U including Guarded Expressions and \it{if-fi}. Finally, {\D} is the
% subset of \U that includes decision tree syntax, including the reduced case
% expression, $\mathit{case}\; x\; \mathit{of}\; \bracketed{\vert\;
% K\bracketed{x}\; \mathit{=>}\; \dt} [\vert\; x\; \mathit{=>} \dt]$.
% Importantly, the three are mutually exclusive with respect to exactly these
% extensions; all subsets share the sub-langauge and no subset has more than one
% of the expanded \it{case}, \it{if-fi}, or the decision tree syntactic
% categories. 

% % {\PPlus} provides or-patterns, side-conditions, and pattern guards, whose
% % combination does not appear in Haskell, OCaml, Standard ML, or any other major
% % functional language. 
% % \bigskip
% % Next steps: Finish introduction of language table, say where semantics are. 


% \it{The following sections and paragraphs are written:}

% \begin{itemize}
%     \item \bf{Subsection: General Evaluation}
%     \item Big-step opsem with environments. 
%     \item \bf{Subsection: {\PPlus}}
%     \item Patterns, case, environment + disjoint union 
% \end{itemize}

% \it{The following sections and paragraphs are in progress:}
% \begin{itemize}
%     \item when, or-patterns, pattern guards 
%     \item \bf{Subsection: {\VMinus}}
%     \item Guarded expressions, new type of environment
%     \item Evaluation stragety 
% \end{itemize}

% \it{Finally, much has been commented out below because I am still determining
%     the order in which their appearance is most clear. After your review of this
%     initial format, I will begin to include them.}


% % Prior work has introduced extesions to pattern matching [cite, including
% % SPJ proposal for pattern guards.]

% % {\PPlus} provides or-patterns, side-conditions, and pattern guards, whose
% % combination does not appear in Haskell, OCaml, Standard ML, or any other major
% % functional language. 

% % \bigskip
% % interesting thoughts: 

% % Having a strategem for verse to decision tree is analagous to tail recursion. 
% % You have to write your code in a way that allows the optimization. 


% % Pattern matching is a well-established paradigm within functional programming languages, and
% % """has been the subject of significant discourse.""" \it{Appeal to authority here.}

% % Without pattern matching, it can be tiresome to deconstruct data using manual 
% % accessor functions. Consider the following Standard ML code: 
% % \smllst

% % \begin{lstlisting}
% % val rec length = fn ys => 
% %     if null ys 
% %     then 0 
% %     else let xs = tl ys 
% %          in length xs 
% %          end 
% % \end{lstlisting}
% % \it{An implementation of List.length in Standard ML that does not use pattern 
% %     matching.}

% % Manual checking and deconstruction, with built-in functions like \tt{null} and
% % \tt{tl}, can be both error-prone (\bf{say more on this?}) and verbose. Most
% % functional programmers likely prefer the follwing implementation of \tt{length}:

% % \begin{lstlisting}
% %     val rec length = fn ys => 
% %         case ys 
% %           of [] => 0
% %            | _::xs => 1 + length xs
% %     \end{lstlisting}

% % \it{An equivalent implementation that uses pattern matching.}

% % Indeed, pattern matching is quite appealing when deconstructing data, especially 
% % data that may be represented with an algebraic data type, is the primary programming problem: 



% % \begin{lstlisting}
% % val length = \ys. case ys 
% %                  of [] -> 0
% %                   | _::xs -> 1 + length xs
% % \end{lstlisting}


% % Figure 1 illustrates an example in which pattern matching is an elegant solution
% % to a problem (compare to equivalent Standard ML code that manually deconstructs
% % a list:)


% % Most functional programmers likely prefer the first example 

% % However, language designers continue to extend pattern matching 

% % \section{Pattern Matching as it is Now}
% % \it{Pattern matching} is defined as "checking and locating of specific sequences
% % of data of some pattern among raw data or a sequence of tokens." We will return
% % to the notion of "checking" often in this paper: pattern matching answers the
% % question "when I'm checking to see if a piece of data (called a \it{scrutinee})
% % is of the same form as a certain pattern, does that match succeed or fail?"

% % \it{Example}.

% % In addition to this checking and locating, pattern matching serves as
% % \it{assignment}: it can bind fresh variables based on the form of data and use
% % those bindings in subsequent expressions. 

% % \it{Example}.

% % Here, "checking" means "do the data match what I expect them to." Because
% % pattern matching is inherently built to match a scrutinee (pure data) with a
% % pattern, pattern matching is quite expressive in these cases. 

% % Because of this, most modern functional programming languages, especially \it{data
% % dependency languages} like Haskell or ML, \it{(is this right?)} employ pattern
% % matching as their main way to deconstruct data (citation?). 

% % We here explore pattern matching through the lens of {\PPlus}, an invented
% % language that has pattern matching along with several of its popular extensions.
% % (The examples above are written in {\PPlus}.) 
% % \subsection{Strengths}

% % - "Checking" and assignment- nice! No car, cdr 

% % - Nested patterns - powerful 

% % - Literal patterns let you mix names and values 

% % \subsection{Weaknesses and Proposed Mitigations}

% % We return to the concept of unifying "checking" with assignment, i.e. "match if
% % the data take this form, and give them names." Pattern matching succeeds here
% % when checking means "is the form of data the way I expect"; in fact, as we know,
% % assignment in general \it{is} pattern matching (figure/example?). But when
% % checking means "does this computation succeed" or "does this binding conflict
% % with a prior binding," pattern matching is at a loss, where Verse succeeds. 

% % Example: 


% % Pattern matching's extensions get closer to unifying "checking" and binding.
% % Here, Verse enjoys a different suite of advantages. 

% % First, its "or" operation (`one` with `choice`) allows for more than patterns to
% % appear as a top-level "choose this or that" construct in a match sequence; you
% % can also include arbitrary expressions. You can't do this in pattern matching's
% % version, which is an or-pattern. Simply put, saying "does this pattern match or
% % is this expression true" is easy in Verse and clunky if you use patterns. I'll
% % show examples at our meeting. 

% % Second, Verse can express operations out of order, letting important checks
% % appear higher up even if they are executed later. This helps program legibility.
% % Again, I have examples from the chapter I wrote today. 

% % Third, in a pattern-match clause, the initial data must still match an initial
% % pattern in order to enter a guard; in Verse there is no restriction. This is
% % minor, because you could simply match the data to a variable, and then enter a
% % guard-- but again, all of these advantages are in elegance and brevity, and
% % elegant that solution is not. 

% % Finally, mingling pattern guards with other extensions to pattern matching
% % (especially or-patterns) is a murky subject. Haskell has pattern guards and side
% % conditions, but no or-patterns. OCaml has side conditions and or-patterns, but
% % no guards. Mixing all three is (according to some readings) simply difficult for
% % implementers- including, interestingly, those of parsers. In Verse, having `one`
% % and `choice` closely tied in with the simple `e1 = e2` equation form, which by
% % itself subsumes pattern matching, side-conditions, and pattern guards, means
% % that integrating options is easy. A key theme that arises of this: Verse has
% % fewer constructs, and they are more expressive.  

% % \section{A Proposal, Inspired by the Verse Calculus}

% % \subsection{Verse Flexibility}
% % \subsection{Something else}
% % \subsection{A third thing}

% % \section{Verse's Equations Subsume Pattern Matching}

% % \subsection{Claim}
% % \subsection{Proof}
% % \subsection{Translations}

% % \section{(Maybe) Writing Efficient Verse Code}

% % \subsection{Claim}
% % \subsection{Proof}
% % \subsection{Translations}


% % \section{Citations and Bibliographies}

\end{document}
\endinput
%%

\documentclass{article}
\usepackage[top=0.2in,bottom=0.3in]{geometry}
\usepackage{amsmath}

\makeatletter
\newcommand{\mono}[1]{%  Thanks Paul A. (Windfall Software)
  {\@tempdima = \fontdimen2\font
   \frenchspacing
   \texttt{\spaceskip = 1.1\@tempdima{}#1}}}
\newcommand\xmono[2][1.1]
  {\@tempdima = \fontdimen2\font
   \frenchspacing
   \texttt{\spaceskip = #1\@tempdima{}#2}}
\makeatother
\newcommand*\monobox[2][1.1]{\mbox{\upshape\xmono[#1]{#2}}}


% ★
\newcommand\slicemessage{\mbox{☆}}
\newcommand\slicecont{\ensuremath{K_{☆}}}
\newcommand\notrunning{\mbox{•}}

\newcommand\emptylist{\ensuremath{[\,]}}

\usepackage{fontspec}

  \setmainfont{STIX Two}[
                  UprightFont={* Math},
                  ItalicFont={* Text Italic},
                  BoldFont={* Text Bold},
                  BoldItalicFont={* Text Bold Italic},
                  ]

\usepackage{keyval}


\makeatletter

\newcommand\defcomponent[2]{%
  \define@key{state}{#1}{\@nameuse{setmyval@#1}{##1}}%
  \@namedef{setmyval@#1}##1{\@namedef{myval@#1}{##1}}%
  \@nameuse{setmyval@#1}{\ensuremath{#2}}%
}

\newcommand\component[1]{\@nameuse{myval@#1}}

\makeatother

\defcomponent{fuel}{F}
\defcomponent{running}{R}
\newcommand\norunning{\ensuremath{\mathord{\bullet}}}
\defcomponent{queue}{Q}
\defcomponent{heap}{H}
\defcomponent{jsheap}{J}
\defcomponent{mc}{M}
\defcomponent{slicemessage}{\slicemessage}
%\defcomponent{cont}{K}
\defcomponent{slicecont}{\slicecont}
\newcommand\running[1]{\ensuremath{\ast#1}}

\newcommand\anystate[1][]{%
  {\tracingall\setkeys{state}{#1}%
   \let\c=\component
   \ensuremath{\langle \c{fuel}, \c{running}, \c{queue}, \c{heap}; 
               \c{mc}, \c{jsheap} \rangle}}}
\newcommand\hstate[1][]{\anystate[running=\running{R},mc=K,#1]}
\newcommand\jstate[1][]{\anystate[fuel=0,running=\notrunning,mc=\running{M},#1]}

\newcommand\goesto[1]{\xrightarrow{#1}} % {\stackrel{#1}{\longrightarrow}}

\newcommand\cons[2]{\ensuremath{#1\mathbin:#2}}
\newcommand\parthreads[2]{\ensuremath{#1\mathbin\mid#2}}

\begin{document}

\title{Interaction of GHC Runtime with JavaScript Runtime}
\author{Norman Ramsey\\Cheng Shao}

\maketitle




\section{Semantics of concurrency}\label{concurrency-sketch}

To~describe how GHC's run-time system must interact with JavaScript,
we~present a small-step operational semantics using labeled
transitions.
The~semantics is in the same style as the
semantics in Simon PJ's tutorial \emph{Tackling the Awkward Squad},
only not as thorough and careful.

\subsection{Metavariables and state}

The semantics describes transitions of a state machine whose
components model the states of \emph{both} Haskell and JavaScript runtimes.
\begin{itemize}
\item
  $F$ (``fuel'') is the number of ticks a Haskell thread can
  execute before returning control to JavaScript. This component is
  always nonnegative, and it is
  nonzero only when Haskell code is running.
\item
  $R$ (``running'') is either the currently running Haskell
  thread, or if no Haskell code is currently running, it is \notrunning\ (``nothing'')
\item
  $Q$ (``run queue'') is a collection of runnable Haskell threads.
  For the moment we model it using the parallel-composition
  operator~$\mid$ from the Awkward Squad tutorial.
  So a queue with $n$ runnable threads might have the form $R_1 \mid
  R_2 \mid \cdots \mid R_n$.
  An~empty queue is written like an empty list, thus:~\emptylist.
\item
  $H$ (``heap'') is the Haskell heap, which may contain
  \texttt{MVar}s and threads that are blocked on them.
\item
$M$ (``message'') is a JavaScript message.
Note that in JavaScript,  \emph{message} is a term of~art.
The message is the atomic unit of concurrency.
JavaScript works by taking a message from its ``message queue,''
running that message to completion (atomically),
then repeating.
While JavaScript is executing a message~$M$, that message will show up
as a component of the state machine.
\item
$K$ (``continuation'') is a JavaScript continuation.
This component represents the state of a JavaScript message that is
suspended waiting for Haskell code to return a value (most commonly a
\texttt{Promise}). 
The~continuation is resumed when a value is supplied, so
$K\,v$ is a message.
\item
$J$ is the JavaScript heap, including the message queue, which we
don't model separately.
To~add a message~$M$ to the message queue, we write
$J+M$, which stands for the JavaScript heap that is like~$J$, except
$M$~is added to the message queue.
\item
\slicemessage\ is the special message that, when executed, causes JavaScript
to allocate a time slice to Haskell.
When GHC's runtime system wants to be sure it has a future opportunity
to run Haskell code, it~will add this message \slicemessage\ to
JavaScript's message queue.
\item
\slicecont\ is a continuation that JavaScript is suspended on when
it has granted a time slice to Haskell.
This~continuation is unique in that it doesn't expect a meaningful
return value. 
It~is resumed by passing it the empty tuple, so $\slicecont()$ is a message.
\item
$v$ is a value (JavaScript value or Haskell WHNF).
\item
$\expr$ is an unevaluated Haskell expression, thunk, or closure (all words for the same
thing).
JavaScript calls into Haskell by constructing a closure, then asking
GHC's run-time system to evaluate~it.
\end{itemize}

\noindent
A~complete machine state includes both Haskell components and
JavaScript components.
The two sets of components are separated by a semicolon.
Common forms of machine state include the following:
\begin{itemize}
\item
In state \hstate[mc=\slicecont], Haskell code is running in thread $R$ with fuel~$F$.
JavaScript is suspended on continuation~$\slicecont$ waiting for
Haskell to give back the~CPU.
\item
In state \hstate[mc=K], Haskell code is running in thread $R$ with fuel~$F$.
JavaScript called Haskell to get a \texttt{Promise}, and it is
suspended on continuation~$K$ waiting for that \texttt{Promise} to be delivered.
\item
In state \jstate, JavaScript has the~CPU and is running code
associated with message~$M$.
Haskell is suspended and not running.
Typically $F=0$, but the value of~$F$ is not relevant.
\end{itemize}
Components $R$, $Q$, $H$, and~$J$ are all meant to be stored in global
variables in their respective run-time systems.

It~would be lovely to have a good invariant describing when the
message \slicemessage\ can appear in the JavaScript message queue, and
how many times.
I~think we like ``exactly when Haskell code is runnable'' and
 ``at most once.''

\subsection{Most interesting state transitions}


The machine will enjoy a set of labeled transitions such as are
described in Simon PJ's paper on the ``Awkward Squad.'' Call these the
``standard transitions.'' (The awkward-squad machine state is a single
term, formed by the parallel composition of $R$ with all the
threads of~$Q$ and all the MVars of~$H$. The awkward squad
doesn't care about order, but we~do.)

An~important nonstandard transition is made when the Haskell engine
runs out of fuel and returns control to JavaScript. 
The~currently running thread is placed on the run queue.
\[
\hstate[fuel=0,mc=\slicecont] 
\goesto{\text{\texttt{return} from \texttt{rts\_grant\_slice}}}
\jstate[queue=\parthreads R Q,mc=\slicecont(),jsheap=J+\slicemessage]
\]
JavaScript starts running $\slicecont()$, that is, it resumes the
message that was suspended waiting for Haskell.
Notice that before returning, Haskell puts message \slicemessage\ on
JavaScript's message queue---thereby ensuring that at some future
point, it~will get more~CPU.
That's accomplished by a \texttt{setImmediate} call into the
JavaScript runtime, whose function will call \texttt{rts\_grant\_slice}
in the Haskell runtime.

Haskell will also return control to JavaScript if there are no Haskell
threads available to run.
In~this case, it~doesn't ask for future access to the~CPU.
\[
\hstate[running=\notrunning,queue={\emptylist},mc=\slicecont] 
\goesto{\text{\texttt{return} from \texttt{rts\_grant\_slice}}}
\jstate[queue={\emptylist},mc=\slicecont()]
\]

When JavaScript sees the special message~\slicemessage, it~grants CPU
to Haskell by calling \texttt{rts\_grant\_slice($F_0$)}, where
$F_0$~is the amount of fuel to grant.
(In~practice, this policy will likely be set elsewhere, not passed
from JavaScript.)
\[
\jstate[queue=\parthreads R Q,mc=\slicemessage]
\goesto{\mathtt{rts\_grant\_slice}(F_0)}
\hstate[running=R,fuel=F_0,mc=\slicecont] 
\]

When JavaScript wants to call Haskell asynchronously, it first constructs a
closure~$\expr$, then calls \texttt{eval\_async($\expr$)}, which forks a new
Haskell thread.
\[
\begin{array}{l@{}l}
\multicolumn2{l}{%
  \jstate[mc=K(\expr)]
  \goesto{\mathtt{eval\_async}(\expr)\text{ and \texttt{return} $p$}}
  \jstate[mc=K(p),queue=\parthreads R Q,jsheap=J+\slicemessage]
}
\\[5pt]
\qquad \text{where}~
&
p = \text{a fresh promise}\\
&
R = \text{\texttt{deliver} $p$ (\texttt{try} $\expr$)}\\
&
\text{\texttt{deliver p (Left exn)}} = \text{\texttt{p.fails exn}}\\
&
\text{\texttt{deliver p (Right v)~}} = \text{\texttt{p.succeeds v}}
\\
%
%\begin{minipage}{0.5\linewidth}
\end{array}
\]
This transition specifies the following sequence of actions:
\begin{enumerate}
\item
Allocate a fresh promise~$p$.
\item
Fork a new thread~$R$, which will evaluate~$\expr$.
(When the evaluation of~$\expr$ completes, $R$~will deliver the result to promise~$p$.)
\item
Ask JavaScript to grant future CPU (so that $R$ can be run).
\item
Return $p$ to JavaScript \emph{immediately}.
\end{enumerate}


When Haskell wants to call JavaScript asynchronously, it~expects a
\texttt{Promise} in return, and it suspends the current thread waiting
on the \texttt{Promise}.  
The run-time system already handles this case as part of its support
for \texttt{MVar}s and exceptions; no special state transition is required.
To~demonstrate,
let's assume we define an exception constructor something like this:
\begin{verbatim}
data JSFault = JSFault JSValue
instance Exception JSFault where ...
\end{verbatim}
The Haskell equivalent of a call \monobox{f \expr} to a foreign JavaScript
function with argument~\texttt{\expr} can look something like this:
\begin{verbatim}
  do let v = haskellToJS \expr  -- evaluates \expr, converts result to JavaScript
     p <- g v               -- call to JavaScript returns a `Promise`
     m <- newEmptyMVar
     myself <- threadId
     p.then(putMVar m, throwTo myself . JSFault) -- attach continuations to p
     takeMVar m             -- block waiting for async call to complete
\end{verbatim}
To~make this code work, the two continuations that contain references
to~\texttt{m} and \texttt{myself} will have to be wrapped in
JavaScript lambdas, something like this:
\begin{verbatim}
  (v => rts_eval_async(⟦putVar m⟧))
\end{verbatim}
where \texttt{⟦putVar m⟧} denotes the run-time representation of the
Haskell closure in JavaScript. 

Notes: Calls \monobox{g v} and \monobox{p.then(...)} are both
\emph{synchronous}.



\subsection{Other state transitions}

Computation is modeled by evolving the running thread from
$R$~to~$R'$.
Computation may also change the Haskell heap and the queue of runnable
threads, as described in the Awkward Squad paper.
Computation uses fuel, which is always
nonnegative.
\[
\hstate[fuel=F+1]
\goesto{\text{computation}}
\hstate[running=\running{R'},queue=Q',heap=H']
\]

The Haskell scheduler can decide to grant the CPU to a different
Haskell thread.
\[
\hstate[fuel=F,queue=\parthreads{R'} Q]
\goesto{\text{context switch}}
\hstate[running=\running{R'},queue=\parthreads{Q}{R}]
\]

\subsection{Scaling up to POSIX}

What if a user types Control-C?

Unix:
\begin{itemize}
\item
Computation doesn't use fuel.
\item
Add component to state machine: \emph{why} does $F=0$?
Maybe because: a~user hit Control-C.
\end{itemize}



\end{document}


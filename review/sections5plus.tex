\documentclass[manuscript,screen,review, 12pt, nonacm]{acmart}
\let\Bbbk\relax % Fix for amssymb clash 
\usepackage{vmlmacros}
\AtBeginDocument{%
  \providecommand\BibTeX{{%
    \normalfont B\kern-0.5em{\scshape i\kern-0.25em b}\kern-0.8em\TeX}}}
\usepackage{outlines}
\setlength{\headheight}{14.0pt}
\setlength{\footskip}{13.3pt}
\title{An Alternative to Pattern Matching, Inspired by Verse}

\author{Roger Burtonpatel}
\email{roger.burtonpatel@tufts.edu}
\affiliation{%
\institution{Tufts University}
\streetaddress{419 Boston Ave}
  \city{Medford}
  \state{Massachusetts}
  \country{USA}
  \postcode{02155}
  }
\begin{document}
\section{Equations subsume pattern matching with popular extensions}
\label{pplustovminus}
    In my introduction I stated that \VMinus can be compiled to a decision
    tree, and that \VMinus subsumes pattern matching with popular extensions.
    Having proved the former, I now prove the latter. I do so by showing an
    algorithm \PtoVTran\ which translates \PPlus to \VMinus, and proving that
    the translation preserves semantics. 

    \subsection{Domains}

    I give the names and domains of the translation functions: 
    
    \begin{align*}
        &\PtoVTran: \PPlus Exp\; \rightarrow\; \VMinus Exp \\
        &\PTran: Pattern\; \rightarrow\; Name\; \rightarrow\; Name\ list\ *\ Guard\ list \\
        % &\mathcal{B}: Pattern\; ->\; Name\; Set \\
    \end{align*}
    
    The translation functions \PtoVTran\ and \PTran\ are defined case by case: 
    
    % \subsection{Binding names}
    
    % \begin{align*}
        %     &\Bindings{x} = \bracketed{x} \\ 
        %     &\Bindings{K} = \bracketed{} \\
        %     &\Bindings{K\; p_{1} {\dots} p_{n}} = \Bindings{p} \cup {\dots} \cup \Bindings{p_{n}} \\
        %     &\Bindings{\porp} = \Bindings{p_{1}} \cap \Bindings{p_{2}} \\
        %     &\Bindings{\pcommap} = \Bindings{p} \cup \Bindings{p'} \\
        %     &\Bindings{\parrowe} = \Bindings{p} \\
        %     &\Bindings{\whenexpr} = \bracketed{}
        % \end{align*}
        
        \subsection{Translating Expressions}
        
        \newcommand\btran[1]{\mathcal{B}[\![#1]\!]}
        
        \begin{align*}
            &\ptov[exp=x, result=x] \\
            &\ptov[exp={K\; \expr[1] {\dots} \expr[n]}, result={K\; \ptovtran{\expr[1]} {\dots} \ptovtran{\expr[n]}}] \\
            &\ptov[exp={\lambda x.\; \expr}, result={\lambda x.\; \ptovtran{\expr}}] \\
            &\ptov[exp={\expr[1]\; \expr[2]}, result={\ptovtran{\expr[1]}\; \ptovtran{\expr[2]}}] \\
            % &\ptov[exp={\tt{case}\; \expr\;  \emptyseq}, result={{\iffitt{\vexists{x}\; x = \ptovtran{\expr};\; \iffitt{}}}}]\; \rm{, $x$ fresh }   \\
            &\ptovtran{\tt{case}\; \expr\;  p_{1}\; \expr[1] \vert {\dots} \vert p_{n}\; \expr[n]} \rightsquigarrow \\
            &\hspace{2em} \rm{$\forall i.\; 1 \leq i \leq n:$} \\
            &\hspace{2em} \tt{if } {\vexists{x}\; x \tt{ = } \ptovtran{\expr};}\; \\
            &\hspace{2em} \rm{ let } (\mathit{ns}_{1}, \mathit{gs}_{1}) {\dots} (\mathit{ns}_{i}, \mathit{gs}_{i}) = \ptran{p_{1}}x\; \cdot\; {\dots} \cdot\; \ptran{p_{i}}x \rm { in } \\
            &\hspace{2em} \iffitt{\vexists{\mathit{ns}_{1}}\; {\mathit{gs}_{1}} \rightarrow \ptovtran{\expr[1]};\;
                       \square\; {\dots} \square\; \vexists {\mathit{ns}_{i}}\; {\mathit{gs}_{i}} \rightarrow \ptovtran{\expr[i]}} \\
            &\hspace{2em} \tt{fi} \\
            &\hspace{2em} \rm{, $x$ fresh }
        \end{align*}
        
        \subsection{Translating Patterns}
        
        \begin{align*}
            &\pattov[pat=y, result={(y, [x = y])}] \\
            &\pattov[pat=K, result={([], [x = K])}] \\
            &\ptran{K\; p_{1}\; {\dots}\; p_{n}}x \rightsquigarrow \\
            &\hspace{2em} \rm{$\forall i.\; 1 \leq i \leq n:$} \\
            &\hspace{2em} \rm{ let } y_{i} \rm{ be a fresh name, }  \\
            &\hspace{2em} (\mathit{ns}_{1}, \mathit{gs}_{1}) {\dots} (\mathit{ns}_{i}, \mathit{gs}_{i}) = \ptran{p_{1}}y_{1} \cdot {\dots} \cdot \ptran{p_{i}}y_{i} \\
            &\hspace{2em} \rm{ in } \\
            &\hspace{2em} ({\mathit{ns}_{1} \cdot {\dots} \cdot \mathit{ns}_{i}} \cdot {y_{1} {\dots} y_{i}}, x = K\; y_{1}\; {\dots}\; y_{i} \cdot \; \mathit{gs}_{1} \cdot {\dots} \cdot \mathit{gs}_{i}) \\
            &\pattov[pat=\mathit{when}\; e, result={([], [\ptovtran{e}])}] \\
            &\pattov[pat=\pcommap, 
            result={\rm{let } 
            {(\mathit{ns}_{1}, \mathit{gs}_{1}) = \ptran{p}x}\; , 
            {(\mathit{ns}_{2}, \mathit{gs}_{2}) = \ptran{p'}x} \rm{ in }
            (\mathit{ns}_{1} \cdot \mathit{ns}_{2}, \mathit{gs}_{1} \cdot \mathit{gs}_{2})}] \\
            &\pattov[pat=\porp, 
            result={\rm{let } (\mathit{ns}_{1}, \mathit{gs}_{1}) = \ptran{p}x\; ,
            (\mathit{ns}_{2}, \mathit{gs}_{2}) = \ptran{p'}x \rm{ in }
            (\mathit{ns}_{1} \cdot \mathit{ns}_{2}, [\mathit{gs}_{1} \choice \mathit{gs}_{2}])}]
        \end{align*}
        

        \rab{Questions: What notes do you have on the formatting? How is my use
        of oxford brackets? Do you see any obvious places to insert or remove
        them?}

    To compile \it{case} expressions to decision trees like Maranget does,
    translate \PPlus to \D using $(\DTran\; o\; \PtoVTran)$.
    
    Finally, I claim that the translation from \it{case} expressions to decision
    trees, $(\DTran\; o\; \PtoVTran)$, is consistent with Maranget and
    others~\citep{maranget,scottramsey}. Proving this claim is a good goal for
    future work; it is not the main focus of this paper. 

    \section{Related Work}

    The dual foundations of this paper are Augustsson et al.'s
    Verse~\citep{verse} and Maranget's decision trees~\citep{maranget}.
    Augustsson et al. give the formal rewrite semantics for \VC; Maranget gives
    an elegant formalism of decision trees. An excellent example of language
    designers appealing to extensions in pattern matching is Erwig \&
    Peyton Jones~\citep{guardproposal}. Compiling Pattern
    Matching~\citep{augustsson1985compiling} by Augustsson gives a foundation in
    exactly what it says. Ramsey and Scott have a crisp example of a
    match-compilation algorithm (pattern matching to decision trees) in When Do
    Match-Compilation Heuristics Matter?~\citep{scottramsey}. 
    
    \section{Future Work}        
    \label{futurework}
        \subsubsection{Typing \PPlus and \VMinus}
        \label{typingppandvm}

        I have described how a type system for both \PPlus and \VMinus is a
        worthwhile effort. Notably, a type system can help restore Nice Property
        5: with a type system, the compiler can warn programmers of a missing or
        extraneous match condition in a \it{case} expression, and potentially an
        \it{if-fi}. Owing to its significantly more flexible structure, however,
        \it{if-fi} may prove trickier to analyze for missing match conditions
        than \it{case}.

        \subsubsection{Type-agnostic decision-making: the $\alpha$}
        \label{alphas}

        The three languages look similar: they each have value constructors and
        a 'decision-making construct' to deal with constructed data. In \PPlus, the
        construct is pattern-matching; in \VMinus, it is the guarded expression; in \D,
        it is the decision tree. 

        Because of this, it might be possible to make all three languages
        \it{higher order} in right-hand sides; that is, to parameterize the
        expressions that occur \it{after} all decision-making. Imagine an
        abstract expression $\alpha$ that appears on the right-hand side of a
        \it{case} branch, the right-hand side of a guarded expression, or in a
        \it{match} node. The abstract syntax of the new \it{case}, \it{if-fi},
        and decision tree might look like this: 
        \begin{center}
            \begin{bnf}
                $\VMinus_{\alpha}$ : \VMinus with $\alpha$ ::=
                $\mathit{if}\; \mathit{[}\; g\; \bracketed{[] g}\; \mathit{]}\; \mathit{fi}$ : if-fi with $\alpha$
                ;;
                $G_{\alpha}$ : Guarded Expressions with $\alpha$ ::=
                $[\vexists{\bracketed{x}}] \bracketed{g} \boldsymbol{\rightarrow}\alpha$ : 
                ;;
                $\PPlus_{\alpha}$ : \PPlus with $\alpha$::=
                $\tt{case}\; \expr\; \bracketed{p\; \ttrightarrow\; \alpha}$ : \it{case} expression with $\alpha$ 
                ;;
                $t_{\alpha}$ : Decision Trees with $\alpha$ ::= 
                | \dots : other forms of tree 
                | $\alpha$ : match node 
            \end{bnf}
        \end{center}

        Why would one want to do this? Well, recall that \VMinus had to be
        stripped of multiple results and other \VC-like constructs in order to
        retain its desirable efficiency properties. $\alpha$ lets \VMinus and
        the other languages do efficient decision-making without worrying about
        the form of the result. If the result is a complex multi-value or a
        computation that involves backtracking, it will be agnostic of the
        decision-making. $\alpha$ makes right-hand sides polymorphic and
        abstract, so a programmer could potentially insert expressions from \VC
        in their place and know that the decision-making before the $\alpha$
        will still be efficient. This would allow for fuller interoperability
        between \VC and \VMinus. Section~\ref{vminusandvc} further describes
        why bridging the gap between the two languages might be a worthwhile
        exercise. 

        However, the language designer must take special care to ensure no
        $\alpha$ finds its way into the decision-making itself, or the whole
        idea falls apart. I am developing an implementation that enforces this
        invariant, and may include it in a future publication. 
        
        % Other alpha material: 
% The decision-making construct that gets us there. Whether
% it's a single value (ML-style) a sequence of values (Verse-style), or
% even something else, the $\alpha$ represents \it{any} ultimate result of
% "making a decision," and it's the ways in which we make decisions that
% we truly care about examining. By making the return result both
% polymorphic and abstract, we eschew the need to worry about its type and
% compatibility with other results of otherwise-equivalent trees. 

% An expression in core Verse evaluates to produce possibly-empty sequence of
% values. In \VMinus, values depend on the form of abstract expression $\alpha.$
% If $\alpha$ is a Verse-like expression, \valpha\ will be a value sequence. If it
% is an ML-like expression, it will be a single value. 
            
%         A guarded expression evaluates to produce a \bf{result}. A result is either
%         a metavalue \valpha\ or reject. 
            
%         \[\it{r}\; \rm{::=}\; \vartheta\; \vbar \; \reject \]
            
%         \showvjudgement{Eval-Alpha}{\veval{\alpha}{\valpha}}

% Of note in both \VMinus and \D is that the 'decision-making construct'
% is annotated with an $\alpha$. This annotation gives us type flexibility on the
% right-hand side of the \it{terminating} case for each construct
% (\tt{$\rightarrow$ exp} in \VMinus and the match node in \D.) 


        \subsubsection{Studying \PPlus independently}
        \label{pplusindependently}
        \PPlus has side conditions, guards, and or-patterns. No major
        functional language has all three of these extensions. Back
        Section~\ref{extensions}'s examples, I had to switch from OCaml to
        Haskell to use guards, and back to OCaml for or-patterns. The two
        extensions are mutually exclusive in Haskell, OCaml, Scala,
        Erlang/Elixir, Rust, F\#, Agda.~\citep{haskell, ocaml, scala, erlang,
        elixir, rust, fsharp, agda}

        I have yet to encounter a substantial justification for this. I have
        several theories: one, reengineering the Haskell parser to integrate
        or-patterns into the language may be considered too great an effort;
        two, the lesser popularity of functional programming in comparison to
        other paradigms has meant there are not enough voices in any one
        language's community claiming that theirs needs all known extensions to
        pattern matching; three, the most efficient algorithms for compiling the
        three extensions are somehow incompatible. Future work may explore if
        efficient compilation of \PPlus is possible; such a study may answer this
        question. 

        \subsubsection{Using \VMinus to inform programming in Verse}
        \label{vminusandvc}
        
        At ICFP last year, Tim Sweeney said that he wanted Verse to be an
        accessible programming language to write a scalable, collaborative
        metaverse~\citep{timtalk}. Can \VMinus be aid in this goal? I can imagine
        two ways in which it might:
        
        \begin{enumerate}
            \item \VMinus could be  a tool to help ease programmers who are
            more familiar with pattern matching into the realm of functional
            logic programming with equations. 
            \item Programs written in Verse using ideas from \VMinus might have
            a friendlier cost model (depending on the compiler)
        \end{enumerate}
        
        To point~1, \VMinus sits both syntactically and semantically in between
        \PPlus and \VC, which might help a new programmer to Verse bridge the
        conceptual gap between pattern matching and equations. Also, \PtoVTran,
        the $\PPlus \rightarrow \VMinus$ translation, could help a programmer 
        who wishes to write code using pattern matching see how their ideas 
        can be expressed in Verse. 

        To point~2, \DTran\ and the proof that \DTran\ preserves semantics help
        show that certain computations that use equations for decision-making
        can be compiled to efficient code. My hope is that, using these ideas,
        both the Verse programmer and language designer might make any discovery
        that allows them to increase the efficiency of full-Verse programs. 

    \section{Conclusion}

    I have introduced the languages \PPlus and \VMinus to explore the design
    space of pattern matching and equations, and \D, \PtoVTran, and \DTran\ to
    show how these languages can be compiled to a decision tree. With these
    languages and transformations, I aim to bridge the gaps between pattern
    matching, equations, and efficient compilation. 

    The languages are made for use and experimentation: they are syntactically
    simple and have conceptually accessible operational semantics. I hope that
    programmers will explore and develop their own opinions of these languages,
    which are publically available at
    \url{https://github.com/rogerburtonpatel/vml}. 

    Finally, and in particular with \VMinus, I hope to have paved a small
    segment of the path that the curious programmer or language enthusiast who
    wishes to better understand Verse will take. Be they transitioning to
    equations from pattern matching to equations or curious about how those
    equations might be compilable to decision trees, I hope they find the
    languages, and this document, illuminating. 
    
\end{document}